\documentclass[12pt, twoside]{report}
\usepackage{libertine}

% General packages:
\usepackage[utf8]{inputenc}    % Input encoding
\usepackage[T1]{fontenc}       % Font encoding
\usepackage[british]{babel}    % Naming of figures and such
\usepackage[british]{isodate}  % Formatting of dates
\cleanlookdateon               % Show dates cleanly.
%\usepackage{xurl}              % Allow line breaks anywhere in URLs.
\usepackage{minted}            % Needs to be loaded before `csquotes`.
\usepackage[
    citestyle    = authoryear,
    bibstyle     = authoryear,
    giveninits   = true,
    uniquename   = init,
    uniquelist   = false,
    sortlocale   = en_GB,
    backend      = biber,
    backref      = true,
    maxbibnames  = 100,
    maxcitenames = 1,
    dashed       = false,
    % Needed to get sorting right for tussenvoegsels.
    %sortcites    = true,       
    ]{biblatex}                % Bibliography
\usepackage[
    style=american
    ]{csquotes}                % Required for BiBLaTeX
\usepackage{import}            % Importing subdocuments
\usepackage{standalone}        % Compilable subdocuments
% `setspace` needs to be loaded before `hyperref`.
\usepackage{setspace}          % Set line spacing
\usepackage{hyperref}          % Clickable references
\usepackage{microtype}         % Nice typography
\usepackage{makeidx}           % Make an index
\usepackage{silence}           % Silence warnings and errors
\WarningFilter{remreset}{The remreset package}
\WarningFilter*[parskip]{latex}{Command}

% Set line spacing.
\setstretch{1.25}

% Page layout:
\usepackage{tocloft}         % Control ToC
\usepackage[
    a4paper,
    bottom     = 1.1in,
    top        = 1.1in,
    left       = 1.1in,
    right      = 1.1in,
    headheight = 15pt,
    footskip   = 1.5\baselineskip
    ]{geometry}              % Margins
\usepackage{fancyhdr}        % Header
\usepackage{lastpage}        % Page numbers in footer
\usepackage{setspace}        % Line spacing
\usepackage{ragged2e}        % Better line endings
\ActivateWarningFilters[parskip]
\usepackage[
    parfill
    ]{parskip}               % Newlines instead of indentation
\DeactivateWarningFilters[parskip]
\usepackage{titlesec}        % Size of sections
\usepackage{titlecaps}       % Automatic capitalisation of titles
\usepackage[
    hang,
    bottom
    ]{footmisc}              % Configure footnotes
\usepackage{afterpage}       % Include stuff after the current page
\usepackage{placeins}        % Things like `\FloatBarrier`
\usepackage{pdflscape}       % Turn pages sideways
\usepackage{emptypage}       % Remove pagenumbers on empty pages

% Utility:
\usepackage[table]{xcolor}   % Colours
\usepackage[
    separate-uncertainty=true,
    per-mode=symbol
    ]{siunitx}               % Display units
\usepackage{enumitem}        % Better enumeration
\usepackage{xifthen}         % If statements
\usepackage{xpatch}          % Patch things
\usepackage{lipsum}          % Dummy text
\usepackage{listings}        % Listings
\usepackage{xparse}          % Better arguments for \newcommand
\usepackage{xfrac}           % Better fractions
\usepackage{etoolbox}        % Patch stuff
\usepackage[
    notquote
    ]{hanging}               % Hanging paragraphs
\usepackage{scalerel}        % Scale objects
\usepackage{soul}            % Highlighting

% Figures:
\usepackage{pgfplots}        % Plots
\pgfplotsset{compat=newest}
\usepackage{tikz}            % TikZ figures
\usepackage{float}           % Control floating of figures
\usepackage{multirow}        % Cells spanning multiple rows
\usepackage{graphicx}        % Graphics
\usepackage{caption}         % Subfigures and captions
\usepackage{subcaption}      % Subfigures and captions
\usepackage{booktabs}        % Nice looking tables
\usepackage{tabularx}        % Extended functionality for tables

% Math:
\usepackage{amsmath}         % Math
\usepackage{amssymb}         % Math symbols
\usepackage{mathtools}       % Math tools
\usepackage{amsthm}          % Theorems
\usepackage{thmtools}        % More math theorems
\usepackage{bm}              % Bold math symbols
\usepackage{bbm}             % More bold math symbols
\usepackage{cancel}          % Cancel equations
\usepackage{upgreek}         % Upright greek symbols
\usepackage[
    artemisia
    ]{textgreek}             % Greek symbols in text

% This package should be loaded last.
\usepackage[
    noabbrev,
    capitalize,
    nameinlink]{cleveref}    % Automatic referencing

% Add a comma in the cite style.
\renewcommand*{\nameyeardelim}{\addcomma\space}

% Make sure that every `\citeyear` also prints the disambiguating `extradate`.
\DeclareCiteCommand{\citeyearlabel}
    {\usebibmacro{prenote}}
    {\printfield{year}\printfield{extradate}}
    {\multicitedelim}
    {\usebibmacro{postnote}}
\let\citeyearold\citeyear
\let\citeyear\citeyearlabel
\let\citeyearnolabel\citeyearold

% Define a standard for citing within theorem statements.
\newcommand{\theoremcite}[1]{\citeauthor{#1}, \citeyear{#1}}

\newcommand{\fulltextcite}[1]{%
    \AtNextCite{\AtEachCitekey{\defcounter{maxnames}{999}}}%
    \textcite{#1}%
}
\newcommand{\fullparencite}[1]{%
    \AtNextCite{\AtEachCitekey{\defcounter{maxnames}{999}}}%
    \parencite{#1}%
}

% Remove the colon after "In" in the bibliograhpy.
\renewbibmacro*{in:}{\bibstring{in} }

% Tune the electronic print field.
\DeclareFieldFormat{eprint}{%
  \iffieldundef{eprinttype}
    {Electronic print}
    {\thefield{eprinttype}}%
  \addcolon\space
  \ifhyperref
    {\url{#1}}
    {\nolinkurl{#1}}%
  \iffieldundef{eprintclass}
    {}
    {\addspace\mkbibparens{\thefield{eprintclass}}}\printtext{.}}

% Tune URL.
\DeclareFieldFormat{url}{\mkbibacro{URL}\addcolon\space\url{#1}\printtext{.}}

% Tune DOI.
\DeclareFieldFormat{doi}{%
  \mkbibacro{DOI}\addcolon\space
  \ifhyperref
    {\href{https://doi.org/#1}{\nolinkurl{#1}}}
    {\nolinkurl{#1}}\printtext{.}}

% In the below, we will redefine `\finentry` to not print a period if
%` pageref` exists. The period is then added back by `pageref`, but before the
% parentheses. Problems occur when `urldate` is also defined. In that case, 
% the extra period by `pageref` will add a period after `urldate`s
% parentheses, which we do not want. We use the toggle `printperiod` to detect
% this case.

% Tune "visited on".
\DeclareFieldFormat{urldate}{\mkbibparens{\bibstring{urlseen}\space#1\printtext{.}}}
\newtoggle{printperiod}
\toggletrue{printperiod}
\AtBeginBibliography{\renewbibmacro*{urldate}{\printurldate\togglefalse{printperiod}}}

% Tune "cited on".
\renewbibmacro*{finentry}{\iflistundef{pageref}{}{\renewcommand{\finentrypunct}{}}\finentry}
\renewbibmacro*{pageref}{%
  \iflistundef{pageref}
    {}
    {\iftoggle{printperiod}{\setunit{\adddot\addspace}}{}\toggletrue{printperiod}\printtext[parens]{%
       \ifnumgreater{\value{pageref}}{1}
         {\bibstring{backrefpages}\ppspace}
         {\bibstring{backrefpage}\ppspace}%
          \printlist[pageref][-\value{listtotal}]{pageref}.}}}

% Add Oxford comma.
\DefineBibliographyExtras{british}{\def\finalandcomma{\addcomma}}

% Don't abbreviate the backreferences.
\DefineBibliographyStrings{english}{%
    backrefpage = {Cited on page},
    backrefpages = {Cited on pages}
}

% Get tussenvoegsels right. This is not a perfect solution, because the
% in-text citations will be sorted as if the tussenvoegsel is part of the
% last name.
\makeatletter
\AtBeginDocument{\toggletrue{blx@useprefix}}
\AtBeginBibliography{\togglefalse{blx@useprefix}}
\makeatother

% Make an index.
\makeindex

% Show the page number in the bottom center.
\fancypagestyle{plain}{
    \fancyhf{}                          % Clear all header and footers.
    \renewcommand{\headrulewidth}{0pt}  % Remove the header rule.
    \cfoot{\thepage}                    % Show page number in the center.
}
\pagestyle{plain}

% Chapter styling from https://texblog.org/2012/07/03/fancy-latex-chapter-styles/
\newcommand{\hsp}{\hspace{20pt}}
\titleformat{\chapter}[hang]
    {\Huge\bfseries}
    {\thechapter\hsp\textcolor{gray75}{|}\hsp}
    {0pt}
    {\Huge\bfseries}
\titleformat{\section}
    {\normalfont\Large\bfseries\raggedright}
    {\thesection}
    {1em}
    {}
\titleformat{\subsection}
    {\normalfont\large\bfseries\raggedright}
    {\thesubsection}
    {1em}
    {}

% Set spacing after chapter equal to spacing right before section. Then
% things look nicely aligned.
\titlespacing{\chapter}{0pt}{3.5ex}{3.5ex}

% Define the styling of paragraphs.
\renewcommand{\paragraph}[1]{\textbf{#1.}}

% Redefine abstract environment.
\renewenvironment{abstract}
    {\begingroup
    \setstretch{1.25}
    \begin{center}
        \textbf{Abstract}
    \end{center}}
    {\par
    \endgroup}

% Kill any use of the \cite command.
\renewcommand{\cite}[1]{\PackageError{thesis}{use either parencite or authorcite}{}}

% Configure captions.
\WarningFilter{caption}{Unused \captionsetup}
\captionsetup[table]{font=small}
\captionsetup[figure]{font=small}

% Shortcuts for writing.
\newcommand{\ie}{\textit{i.e.}}
\newcommand{\Ie}{\textit{I.e.}}
\newcommand{\eg}{\textit{e.g.}}
\newcommand{\Eg}{\textit{E.g.}}
\newcommand{\cf}{\textit{c.f.}}
\newcommand{\Cf}{\textit{C.f.}}

% \ifempty command:
\newcommand{\ifempty}[3]{\ifthenelse{\equal{#1}{}}{#2}{#3}}

% Style footnotes.
\setlength{\skip\footins}{\baselineskip}
\setlength{\footnotesep}{.75\baselineskip}
\renewcommand{\footnotelayout}{\setstretch{1.0}}
\setlength{\footnotemargin}{1em}

% Typewriter font:
\renewcommand*{\ttdefault}{pcr}  % Courier
\newcommand{\code}[1]{{\small\texttt{#1}}}

% Define some colours.
\definecolor{darkblue} {rgb} {0.0 , 0.0 , 0.65}
\definecolor{darkred}  {rgb} {0.80, 0.0 , 0.0 }
\definecolor{redaccent}{HTML}{E64C66}
\definecolor{darkgreen}{rgb} {0.0 , 0.50, 0.0 }
\definecolor{gray75}   {gray}{0.75}

% Define shortcuts for colours.
\newcommand{\red}[1]{{\color{red} #1}}
\newcommand{\blue}[1]{{\color{blue} #1}}
\newcommand{\green}[1]{{\color{green} #1}}
\newcommand{\orange}[1]{{\color{orange} #1}}
\newcommand{\darkred}[1]{{\color{darkred} #1}}
\newcommand{\darkblue}[1]{{\color{darkblue} #1}}
\newcommand{\darkgreen}[1]{{\color{darkgreen} #1}}
\newcommand{\magenta}[1]{{\color{magenta} #1}}
\newcommand{\grey}[1]{{\color{gray} #1}}

% Use colours to define checkmarks and crosses.
\usepackage{pifont}
\newcommand{\xmark}{\text{\ding{55}}}
\newcommand{\cmark}{\text{\ding{51}}}
\newcommand{\good}{\darkgreen{$\checkmark$}}
\newcommand{\mediumgood}{\orange{$\checkmark$}}
\newcommand{\mediumbad}{\orange{\xmark}}
\newcommand{\bad}{\darkred{\xmark}}

% Load TikZ libraries.
\usetikzlibrary{
    calc,
    positioning,
    fit,
    tikzmark,
    arrows.meta,
    shapes,
    decorations.pathreplacing,
    intersections,
    through
}
% Graphical models:
\tikzset{
    line/.style = {
        thick,
        ->,
        > = {
            Triangle[length=1.5mm, width=1.5mm]
        }
    },
    arrow/.style = {
        line
    },
    % Invisible node:
    hidden node/.style = {
        circle,
        minimum size = 1cm,
        draw = white,
        thick
    },
    % Latent variable:
    latent node/.style = {
        hidden node,
        draw = black,
    },
    % Latent variable:
    factor node/.style = {
        hidden node,
        rectangle,
        draw = black,
    },
    % Observed variable:
    observed node/.style = {
        latent node,
        fill = gray!15
    },
    % Plate:
    plate/.style = {
        draw,
        label={[anchor=north west]south west:#1},
        rounded corners=2pt,
        shape=rectangle,
        inner sep=10pt,
        thick
    }
}

% Circled number:
\newcommand{\ballnumber}[1]{%
    \tikz[baseline=(n.base)] \node[circle,fill=.,inner sep=1pt,text=white] (n) {\normalshape\bfseries\footnotesize #1};%
}
\newcommand{\itemballnumber}[1]{%
    \raisebox{.5pt}{\ballnumber{#1}}%
}

% Style enumerate and lists. Do not use `label=(\arabic*)` because equation
% numbering already uses the same style.
\setenumerate{topsep=.25\baselineskip, itemsep=0pt}
\setlist{topsep=.25\baselineskip, itemsep=0pt}
% \setlist[enumerate]{
%     label=(\arabic*),
%     itemsep=-0.25\baselineskip,
%     topsep=0.25\baselineskip,
%     after={\vspace*{0.25\baselineskip}}
% }
% \setlist[enumerate,2]{
%     topsep=-0.25\baselineskip,
%     itemsep=0pt,
%     label=(\alph*)
% }
% \setlist[enumerate,3]{
%     topsep=-0.25\baselineskip,
%     itemsep=0pt,
%     label=(\roman*)
% }
% \setlist[itemize]{
%     label=\textbullet,
%     itemsep=-0.25\baselineskip,
%     topsep=0.25\baselineskip
% }
% \setlist[itemize,2]{
%     topsep=-0.25\baselineskip,
%     itemsep=0pt
% }
% \setlist[itemize,3]{
%     topsep=-0.25\baselineskip,
%     itemsep=0pt
% }

% Set the default float placement correctly.
\floatplacement{figure}{tbp}
\floatplacement{table}{tbp}

% Set spacing between figures and text.
\setlength{\textfloatsep}{30pt plus 1.0pt minus 2.0pt}
\setlength{\floatsep}{30pt plus 1.0pt minus 2.0pt}
\setlength{\intextsep}{30pt plus 1.0pt minus 2.0pt}

% Adjust line spacing in captions.
\captionsetup{font={stretch=1.25}}

% Newline for in a title.
\newcommand{\tnl}{\texorpdfstring{\\}{}}

% Define outline tools.
\newcommand{\outline}[1]{
    \parbox{\linewidth}{
        \setstretch{1.0}
        \color{darkgreen}
        #1
    }\vspace{\parskip}
}
\newlist{outlinelist}{enumerate}{1}
\setlist[outlinelist]{
    label=\arabic*.,
    noitemsep,
    topsep=0pt,
    parsep=0pt,
    partopsep=0pt
}
\newcommand{\note}[1]{{\color{darkred}#1}}

% Left-justified text in tabularx environment:
\newcolumntype{L}{>{\RaggedRight\arraybackslash}X}

% Hyperlink setup.
\hypersetup{
    colorlinks,
    citecolor = black,
    filecolor = black,
    linkcolor = black,
    urlcolor  = black
}

% Landscape figures:
\newcommand{\landscapefloat}[1]{
    \afterpage{
        \begin{landscape}
            #1
        \end{landscape}
    }
}

% Hanging full citation:
\newcommand{\hangcite}[1]{\hangpara{\bibhang}{1}\AtNextCite{\defcounter{maxnames}{99}}\fullcite{#1}}

% Define \charfusion. From https://tex.stackexchange.com/a/52673.
\makeatletter
\def\moverlay{\mathpalette\mov@rlay}
\def\mov@rlay#1#2{\leavevmode\vtop{%
   \baselineskip\z@skip \lineskiplimit-\maxdimen
   \ialign{\hfil$\m@th#1##$\hfil\cr#2\crcr}}}
    \newcommand{\charfusion}[3][\mathord]{
        #1{\ifx#1\mathop\vphantom{#2}\fi
        \mathpalette\mov@rlay{#2\cr#3}}
        \ifx#1\mathop\expandafter\displaylimits\fi
    }
\makeatother

% Sha symbol.
%   Source: https://tex.stackexchange.com/questions/124738/i-just-want-to-write-sha-without-ruining-everything
\DeclareFontFamily{U}{wncy}{}
\DeclareFontShape{U}{wncy}{m}{n}{<->wncyr10}{}
\DeclareSymbolFont{mcy}{U}{wncy}{m}{n}
\DeclareMathSymbol{\Sha}{\mathord}{mcy}{"58}

% Define a bigger \cdot.
%   Source: https://tex.stackexchange.com/questions/235118/making-a-thicker-cdot-for-dot-product-that-is-thinner-than-bullet
\makeatletter
\newcommand*\bigcdot{\mathpalette\bigcdot@{.5}}
\newcommand*\bigcdot@[2]{\mathbin{\vcenter{\hbox{\scalebox{#2}{$\m@th#1\bullet$}}}}}
\makeatother

% Bold math symbols from text greeks:
\newcommand{\mathbfup}[1]{\mathord{\textnormal{\textbf{#1}}}}

% Bold characters:
\renewcommand{\vec}[1]{\boldsymbol{#1}}
\newcommand{\vecu}[1]{\hat{\vec{#1}}}
\newcommand{\mat}[1]{\vec{#1}}

% Predefined matrices and vectors:
\newcommand{\vnull}{\mathbf{0}}

\newcommand{\va}{\mathbf{a}}
\newcommand{\vb}{\mathbf{b}}
\newcommand{\vc}{\mathbf{c}}
\newcommand{\vd}{\mathbf{d}}
\newcommand{\ve}{\mathbf{e}}
\newcommand{\vf}{\mathbf{f}}
\newcommand{\vg}{\mathbf{g}}
\newcommand{\vh}{\mathbf{h}}
\newcommand{\vi}{\mathbf{i}}
\newcommand{\vj}{\mathbf{j}}
\newcommand{\vk}{\mathbf{k}}
\newcommand{\vl}{\mathbf{l}}
\newcommand{\vm}{\mathbf{m}}
\newcommand{\vn}{\mathbf{n}}
\newcommand{\vo}{\mathbf{o}}
\newcommand{\vp}{\mathbf{p}}
\newcommand{\vq}{\mathbf{q}}
\newcommand{\vr}{\mathbf{r}}
\newcommand{\vs}{\mathbf{s}}
\newcommand{\vt}{\mathbf{t}}
\newcommand{\vu}{\mathbf{u}}
\newcommand{\vv}{\mathbf{v}}
\newcommand{\vw}{\mathbf{w}}
\newcommand{\vx}{\mathbf{x}}
\newcommand{\vy}{\mathbf{y}}
\newcommand{\vz}{\mathbf{z}}

% Must be enclosed in another group.
\newcommand{\vmu}{{\mathbfup{\textmu}}}
\newcommand{\vtheta}{{\bm{\uptheta}}}
\newcommand{\vphi}{{\mathbfup{\textphi}}}
\newcommand{\vtau}{{\mathbfup{\texttau}}}
\newcommand{\vep}{{\mathbfup{\textepsilon}}}
\newcommand{\vth}{{\mathbfup{\texttheta}}}

\newcommand{\mA}{\mathbf{A}}
\newcommand{\mB}{\mathbf{B}}
\newcommand{\mC}{\mathbf{C}}
\newcommand{\mD}{\mathbf{D}}
\newcommand{\mE}{\mathbf{E}}
\newcommand{\mF}{\mathbf{F}}
\newcommand{\mG}{\mathbf{G}}
\newcommand{\mH}{\mathbf{H}}
\newcommand{\mI}{\mathbf{I}}
\newcommand{\mJ}{\mathbf{J}}
\newcommand{\mK}{\mathbf{K}}
\newcommand{\mL}{\mathbf{L}}
\newcommand{\mM}{\mathbf{M}}
\newcommand{\mN}{\mathbf{N}}
\newcommand{\mO}{\mathbf{O}}
\newcommand{\mP}{\mathbf{P}}
\newcommand{\mQ}{\mathbf{Q}}
\newcommand{\mR}{\mathbf{R}}
\newcommand{\mS}{\mathbf{S}}
\newcommand{\mT}{\mathbf{T}}
\newcommand{\mU}{\mathbf{U}}
\newcommand{\mV}{\mathbf{V}}
\newcommand{\mW}{\mathbf{W}}
\newcommand{\mX}{\mathbf{X}}
\newcommand{\mY}{\mathbf{Y}}
\newcommand{\mZ}{\mathbf{Z}}

\newcommand{\mSigma}{\mathbfup{\textSigma}}

% Letters commonly used in blackboard bold font:
\newcommand{\Ab}{\mathbb{A}}
\newcommand{\Bb}{\mathbb{B}}
\newcommand{\Db}{\mathbb{D}}
\newcommand{\C}{\mathbb{C}}
\newcommand{\E}{\mathbb{E}}
\newcommand{\Hb}{\mathbb{H}}
\newcommand{\Jb}{\mathbb{J}}
\newcommand{\K}{\mathbb{K}}
\newcommand{\Lb}{\mathbb{L}}
\newcommand{\Mb}{\mathbb{M}}
\newcommand{\N}{\mathbb{N}}
\renewcommand{\P}{\mathbb{P}}
\newcommand{\Q}{\mathbb{Q}}
\newcommand{\R}{\mathbb{R}}
\newcommand{\eR}{\overline{\mathbb{R}}}
\newcommand{\Sb}{\mathbb{S}}
\newcommand{\Tb}{\mathbb{T}}
\newcommand{\V}{\mathbb{V}}
\newcommand{\Z}{\mathbb{Z}}

% Letters commonly used in calligraphic font:
\newcommand{\A}{\mathcal{A}}
\newcommand{\B}{\mathcal{B}}
\newcommand{\Cc}{\mathcal{C}}
\newcommand{\D}{\mathcal{D}}
\newcommand{\Ec}{\mathcal{E}}
\newcommand{\F}{\mathcal{F}}
\newcommand{\G}{\mathcal{G}}
\let\Haccent\H
\renewcommand{\H}{\mathcal{H}}
\newcommand{\J}{\mathcal{J}}
\newcommand{\Jrat}{\mathcal{J}_{\text{rat}}}
\newcommand{\Kc}{\mathcal{K}}
\renewcommand{\L}{\mathcal{L}}
\newcommand{\Li}{\mathcal{L}{}^1}
\newcommand{\eLi}{\overline{\mathcal{L}}{}^1}
\newcommand{\m}{\text{m}}
\newcommand{\M}{\mathcal{M}}
\newcommand{\eM}{\overline{\mathcal{M}}}
\newcommand{\Nc}{\mathcal{N}}
\newcommand{\Oc}{\mathcal{O}}
\newcommand{\Pc}{\mathcal{P}}
\newcommand{\Qc}{\mathcal{Q}}
\newcommand{\Rc}{\mathcal{R}}
\renewcommand{\S}{\mathcal{S}}
\newcommand{\Tc}{\mathcal{T}}
\newcommand{\U}{\mathcal{U}}
\newcommand{\Vc}{\mathcal{V}}
\newcommand{\W}{\mathcal{W}}
\newcommand{\X}{\mathcal{X}}
\newcommand{\Y}{\mathcal{Y}}
\newcommand{\Zc}{\mathcal{Z}}

% Letters sometimes used with a wide tilde:
\newcommand{\tpi}{\widetilde{\pi}}
\newcommand{\ttau}{\widetilde{\tau}}
\newcommand{\tvtau}{\widetilde{\vtau}}
\newcommand{\tD}{\widetilde{\D}}
\newcommand{\tX}{\widetilde{\X}}
\newcommand{\tI}{\widetilde{I}}
\newcommand{\tQc}{\widetilde{\Qc}}

% Letters commonly used in sans serif font:
\newcommand{\T}{\text{\normalshape\textsf{T}}}
\newcommand{\Ht}{\text{\normalshape\textsf{H}}}
\renewcommand{\c}{\text{\normalshape\textsf{c}}}

% Symbols:
\newcommand{\es}{\varnothing}
\newcommand{\e}{\varepsilon}
\newcommand{\sub}{\subseteq}
\renewcommand{\d}{\partial}
\renewcommand{\th}{\theta}
\newcommand{\Th}{\Theta}

% Substitute l for \ell throughout.
%   Source: https://tex.stackexchange.com/questions/1975/substituting-character-l-with-ell-throughout-math-mode
\mathcode`l="8000
\begingroup
\makeatletter
\lccode`\~=`\l
\DeclareMathSymbol{\lsb@l}{\mathalpha}{letters}{`l}
\lowercase{\gdef~{\ifnum\the\mathgroup=\m@ne \ell \else \lsb@l \fi}}%
\endgroup

% Convergence symbols:
\newcommand{\oto}[1]{\overset{#1}{\to}}
\newcommand{\uto}[1]{\underset{#1}{\to}}
\newcommand{\outo}[2]{\overset{#1}{\underset{#2}{\to}}}
\newcommand{\Lto}[1]{\oto{\L^{#1}}}
\newcommand{\weakto}{\rightharpoonup}
\newcommand{\distto}{\oto{\text{d}}}
\newcommand{\weakstarto}{\overset{\ast}{\rightharpoonup}}
% Big-O notation:
\renewcommand{\O}{O}

% Equality symbols:
\newcommand{\disteq}{\overset{\text{\normalshape d}}{=}}

% Operators:
\DeclareMathOperator*{\argmax}{arg\,max}
\DeclareMathOperator*{\argmin}{arg\,min}
\let\max\relax\DeclareMathOperator*{\max}{max}
\let\min\relax\DeclareMathOperator*{\min}{min}
\let\sup\relax\DeclareMathOperator*{\sup}{sup}
\let\inf\relax\DeclareMathOperator*{\inf}{inf\vphantom{p}}
\let\limsup\relax\DeclareMathOperator*{\limsup}{lim\,sup}
\let\liminf\relax\DeclareMathOperator*{\liminf}{lim\,inf\vphantom{p}}
\DeclareMathOperator*{\esssup}{ess\,sup}
\DeclareMathOperator*{\essinf}{ess\,inf}
\newcommand{\AC}{\operatorname{AC}}
\newcommand{\atan}{\operatorname{atan}}
\newcommand{\Breg}{\operatorname{D}}
\newcommand{\BV}{\operatorname{BV}}
\newcommand{\card}{\#}
\newcommand{\cconv}{\circledast}
\newcommand{\chol}{\operatorname{chol}}
\newcommand{\ch}{\operatorname{ch}}
\newcommand{\cliques}{\operatorname{cliques}}
\newcommand{\cl}{\operatorname{cl}}
\newcommand{\col}{\operatorname{col}}
\newcommand{\comp}{\circ}
\newcommand{\contains}{\supseteq}
\newcommand{\conv}{\ast}
\newcommand{\cov}{\operatorname{cov}}
\newcommand{\var}{\operatorname{var}}
\renewcommand{\deg}{\operatorname{deg}}
\renewcommand{\det}{\operatorname{det}}
\newcommand{\diag}{\operatorname{diag}}
\renewcommand{\dim}{\operatorname{dim}}
\newcommand{\diam}{\operatorname{diam}}
\renewcommand{\div}{\operatorname{div}}
\newcommand{\dist}{\operatorname{dist}}
\newcommand{\dom}{\operatorname{dom}}
\newcommand{\dotcup}{\charfusion[\mathbin]{\cup}{\cdot}}
\newcommand{\dotunion}{\charfusion[\mathop]{\bigcup}{\cdot}}
\newcommand{\dsum}{\oplus}
\newcommand{\Dsum}{\bigoplus}
\newcommand{\erf}{\operatorname{erf}}
\newcommand{\epi}{\operatorname{epi}}
\newcommand{\had}{\odot}
\newcommand{\Hell}{\operatorname{D}_{\text{Hell}}}
\newcommand{\hypo}{\operatorname{hypo}}
\newcommand{\im}{\operatorname{im}}
\renewcommand{\Im}{\operatorname{Im}}
\newcommand{\ind}{\mathbbm{1}}
\newcommand{\intersection}{\bigcap}
\newcommand{\join}{\lor}
\newcommand{\KL}{\operatorname{KL}}
\newcommand{\KOT}{\operatorname{KOT}}
\newcommand{\kron}{\otimes}
\newcommand{\len}{\operatorname{len}}
\newcommand{\Lip}{\operatorname{Lip}}
\newcommand{\meet}{\land}
\newcommand{\MOT}{\operatorname{MOT}}
\newcommand{\nullspace}{\operatorname{null}}
\newcommand{\Per}{\operatorname{Per}}
\newcommand{\push}{_\#}
\newcommand{\poly}{\operatorname{poly}}
\newcommand{\rank}{\operatorname{rank}}
\renewcommand{\Re}{\operatorname{Re}}
\newcommand{\row}{\operatorname{row}}
\newcommand{\schur}{\operatorname{Schur}}
\newcommand{\sign}{\operatorname{sign}}
\newcommand{\sinth}{\operatorname{\sin\!\text{-}\Theta}}
\newcommand{\Sinth}{\operatorname{\text{Sin-}\Theta}}
\newcommand{\supp}{\operatorname{supp}}
\newcommand{\symdiff}{\bigtriangleup}
\newcommand{\Tan}{\operatorname{Tan}}
\newcommand{\tensor}{\otimes}
\newcommand{\TV}{\operatorname{TV}}
\newcommand{\tr}{\operatorname{tr}}
\newcommand{\union}{\bigcup}
\newcommand{\vecspan}{\operatorname{span}}
\newcommand{\vect}{\operatorname{vec}}
\newcommand{\vol}{\operatorname{vol}}

% Probability distributions:
\newcommand{\Ber}{\operatorname{Ber}}
\newcommand{\Beta}{\operatorname{Beta}}
\newcommand{\Bin}{\operatorname{Bin}}
\newcommand{\BM}{\operatorname{BM}}
\newcommand{\Cat}{\operatorname{Cat}}
\newcommand{\Cauchy}{\operatorname{Cauchy}}
\newcommand{\Dir}{\operatorname{Dir}}
\newcommand{\Exp}{\operatorname{Exp}}
\newcommand{\Gam}{\operatorname{Gamma}}
\newcommand{\Geom}{\operatorname{Geom}}
\newcommand{\GP}{\mathcal{GP}}
\newcommand{\InvWishart}{\mathcal{W}^{-1}}
\newcommand{\Laplace}{\operatorname{Laplace}}
\newcommand{\LogNormal}{\log\mathcal{N}}
\newcommand{\Mult}{\operatorname{Mult}}
\newcommand{\NegBin}{\operatorname{NegBin}}
\newcommand{\Normal}{\mathcal{N}}
\newcommand{\Poisson}{\operatorname{Poisson}}
\newcommand{\Rad}{\operatorname{Rad}}
\newcommand{\Unif}{\operatorname{Unif}}
\newcommand{\Wishart}{\mathcal{W}}
\newcommand{\WN}{\operatorname{WN}}

\newcommand{\simiid}{\overset{\text{i.i.d.}}{\sim}}

% Probability commands:
\newcommand{\Var}{\V}
\newcommand{\Lik}{\L}
\newcommand{\iidsim}{\overset{\text{\tiny{i.i.d.}}}{\sim}}

% Special math commands:
\newcommand{\cond}{\,|\,}                  % Conditioning
\newcommand{\middlecond}{\,\middle|\,}     % Conditioning between a \left and \right
\newcommand{\divsep}{\,\|\,}               % Separator in divergences
\newcommand{\middledivsep}{\,\middle\|\,}  % Separator in divergences ebtween a \left and a \right

\newcommand{\sd}[1]{\mathrm{d} #1}         % Straight 'd'
\newcommand{\isd}[1]{\, \mathrm{d} #1}     % Straight 'd' in integral (with spacing)
% Straight 'd' in integral with spacing and absolute value around it
\newcommand{\asd}[1]{\, |\mathrm{d} #1|}

\newcommand{\idf}{\text{\textsf{id}}}            % Identity function
\newcommand{\sce}{\text{\sc{e}}}                 % Scientific notation
\newcommand{\vardot}{\,\bigcdot\,}               % Variable dot
\let\ssorig\ss
\renewcommand{\ss}[1]{_{\text{\normalshape #1}}} % Text subscripts
\newcommand{\us}[1]{^{\text{\normalshape #1}}}   % Text superscripts

% Hard to type words:
\newcommand{\cadlag}{c\`adl\`ag}
\newcommand{\Cadlag}{C\`adl\`ag}
\newcommand{\ito}{It\^o}
\newcommand{\Ito}{\ito}
\newcommand{\levy}{L\`evy}
\newcommand{\Levy}{\levy}

% Subscripts
\newcommand{\loc}{_{\text{loc}}}
\newcommand{\lip}{_{\text{Lip}}}

% Half rectangle:
\newcommand{\halfrect}[2]{[\![#1,#2)\hspace{-1pt}\!)}
\newcommand{\Rint}{\text{(R)}\int}

% Paired delimiters:
\DeclarePairedDelimiter\parens{(}{)}             % Parentheses
\DeclarePairedDelimiter\sbrac{[}{]}              % Square brackets
\DeclarePairedDelimiter\cbrac{\{}{\}}            % Curly braces
\DeclarePairedDelimiter\set{\{}{\}}              % Set
\DeclarePairedDelimiter\lra{\langle}{\rangle}    % Angle brackets
\DeclarePairedDelimiter\dlra
    {\langle\!\langle}{\rangle\!\rangle}         % Double angle brackets
\DeclarePairedDelimiter\floor{\lfloor}{\rfloor}  % Floor
\DeclarePairedDelimiter\ceil{\lceil}{\rceil}     % Ceil
\DeclarePairedDelimiter\norm{\|}{\|}             % Norm
\DeclarePairedDelimiter\abs{|}{|}                % Absolute value

% Shortcuts for angle brackets:
\newcommand{\la}{\langle}
\newcommand{\ra}{\rangle}
\newcommand{\lla}{\left\langle}
\newcommand{\rra}{\right\rangle}

% Other commands:
\newcommand{\resp}[1]{[#1]}                  % Respective statement
\newcommand{\bs}{\textbackslash}             % Blackslash in text
% Use \cref in a title.
\newcommand{\creftitle}[2]{\texorpdfstring{\cref{#2}}{#1 \ref{#2}}}
\newcommand{\mcheckthis}{{}^{[\checkmark]}}  % Checkmark in math
\newcommand{\checkthis}{$\mcheckthis$}       % Checkmark in text

% Compatibility with old commands:
\newcommand{\id}[1]{\sd{#1}}
\newcommand{\sargmax}[1]{\argmin_{#1}}
\newcommand{\sargmin}[1]{\argmax_{#1}}
\newcommand{\lrset}[1]{\set*{#1}}
\newcommand{\middleCond}{\middlecond}
\newcommand{\rel}[2]{($(#1) + (#2)$)}
\newcommand{\blackLinks}{}
\newcommand{\Ac}{\A}

\newcommand{\notheorems}{}  % We'll configure our own numbering below.
% Patch \listoftheorems.
%   Source: https://tex.stackexchange.com/questions/249963/remove-repeated-theorem-in-the-list-of-theorems
\makeatletter
\patchcmd\thmt@mklistcmd
    {\thmt@thmname}
    {\check@optarg{\thmt@thmname}}
    {}{}
\patchcmd\thmt@mklistcmd
    {\thmt@thmname\ifx}
    {\check@optarg{\thmt@thmname}\ifx}
    {}{}
\protected\def\check@optarg#1{%
    \@ifnextchar\thmtformatoptarg\@secondoftwo{#1}%
}
\makeatother

% Define lists of things commands.
\let\oldlistoftheorems\listoftheorems
\newcommand{\listofmodels}{
    \renewcommand{\listtheoremname}{List of Models}
    \oldlistoftheorems[ignoreall, show={model}]
}
\newcommand{\listofstatements}{
    \renewcommand{\listtheoremname}{List of Mathematical Statements}
    \oldlistoftheorems[ignoreall, show={theorem,corollary,proposition,lemma,fact}]
}
\renewcommand{\listoftheorems}{
    \renewcommand{\listtheoremname}{List of Theorems}
    \oldlistoftheorems[ignoreall, show={theorem}]
}

% Define environments.
\newlength{\thmtopsep}\setlength{\thmtopsep}{\topsep}
\newlength{\thmbotsep}\setlength{\thmbotsep}{\topsep}
\newtheoremstyle{theoremstyle}
    {\thmtopsep}{\thmbotsep}
    {}           % Body font
    {}           % Indent amount
    {\bfseries}  % Theorem head font
    {.}          % Punctuation after theorem head
    {.5em}       % Space after theorem head
    {}           % Theorem head spec
\theoremstyle{theoremstyle}

\ifcsname notheorems\endcsname
\else
    \newtheorem{theorem}{Theorem}[section]
    \newtheorem{proposition}{Proposition}[section]
    \newtheorem{corollary}{Corollary}[section]
    \newtheorem{fact}{Fact}[section]
    \newtheorem{lemma}{Lemma}[section]

    \newtheorem{assumption}{Assumption}[section]
    \newtheorem{definition}{Definition}[section]
    \newtheorem{question}{Question}[section]
    \newtheorem{example}{Example}[section]
    \newtheorem{model}{Model}[section]
    \newtheorem{remark}{Remark}[section]
\fi

% Set referencing formats.
\crefname{assumption}{Assumption}{Assumptions}
\Crefname{assumption}{Assumption}{Assumptions}
\crefname{corollary}{Corollary}{Corollaries}
\Crefname{corollary}{Corollary}{Corollaries}
\crefname{definition}{Definition}{Definitions}
\Crefname{definition}{Definition}{Definitions}
\crefname{example}{Example}{Examples}
\Crefname{example}{Example}{Examples}
\crefname{fact}{Fact}{Facts}
\Crefname{fact}{Fact}{Facts}
\crefname{lemma}{Lemma}{Lemmas}
\Crefname{lemma}{Lemma}{Lemmas}
\crefname{model}{Model}{Models}
\Crefname{model}{Model}{Models}
\crefname{proposition}{Proposition}{Propositions}
\Crefname{proposition}{Proposition}{Propositions}
\crefname{question}{Question}{Questions}
\Crefname{question}{Question}{Questions}
\crefname{remark}{Remark}{Remarks}
\Crefname{remark}{Remark}{Remarks}
\crefname{theorem}{Theorem}{Theorems}
\Crefname{theorem}{Theorem}{Theorems}

% Referentiable list items in environments
\newlist{asslist}{enumerate}{1}
\setlist[asslist]{
    ref=\theassumption.(\arabic*),
    label=(\arabic*),
    % topsep=0pt,
}
\crefname{asslisti}{Assumption}{Assumptions}
\Crefname{asslisti}{Assumption}{Assumptions}
\newlist{corlist}{enumerate}{1}
\setlist[corlist]{
    ref=\thecorollary.(\arabic*),
    label=(\arabic*),
    % topsep=0pt,
}
\crefname{corlisti}{Corollary}{Corollaries}
\Crefname{corlisti}{Corollary}{Corollaries}
\newlist{deflist}{enumerate}{1}
\setlist[deflist]{
    ref=\thedefinition.(\arabic*),
    label=(\arabic*),
    % topsep=0pt,
}
\crefname{deflisti}{Definition}{Definitions}
\Crefname{deflisti}{Definition}{Definitions}
\newlist{exlist}{enumerate}{1}
\setlist[exlist]{
    ref=\theexample.(\arabic*),
    label=(\arabic*),
    % topsep=0pt,
}
\crefname{exlisti}{Example}{Examples}
\Crefname{exlisti}{Example}{Examples}
\newlist{factlist}{enumerate}{1}
\setlist[factlist]{
    ref=\thefact.(\arabic*),
    label=(\arabic*),
    % topsep=0pt,
}
\crefname{factlisti}{Fact}{Facts}
\Crefname{factlisti}{Fact}{Facts}
\newlist{lemlist}{enumerate}{1}
\setlist[lemlist]{
    ref=\thelemma.(\arabic*),
    label=(\arabic*),
    % topsep=0pt,
}
\crefname{lemlisti}{Lemma}{Lemmas}
\Crefname{lemlisti}{Lemma}{Lemmas}
\newlist{modlist}{enumerate}{1}
\setlist[modlist]{
    ref=\themodel.(\arabic*),
    label=(\arabic*),
    % topsep=0pt,
}
\crefname{modlisti}{Model}{Models}
\Crefname{modlisti}{Model}{Models}
\newlist{proplist}{enumerate}{1}
\setlist[proplist]{
    ref=\theproposition.(\arabic*),
    label=(\arabic*),
    % topsep=0pt,
}
\crefname{proplisti}{Proposition}{Propositions}
\Crefname{proplisti}{Proposition}{Propositions}
\newlist{qlist}{enumerate}{1}
\setlist[qlist]{
    ref=\theremark.(\arabic*),
    label=(\arabic*),
    % topsep=0pt,
}
\crefname{qlisti}{Question}{Questions}
\Crefname{qlisti}{Question}{Questions}
\newlist{remlist}{enumerate}{1}
\setlist[remlist]{
    ref=\theremark.(\arabic*),
    label=(\arabic*),
    % topsep=0pt,
}
\crefname{remlisti}{Remark}{Remarks}
\Crefname{remlisti}{Remark}{Remarks}
\newlist{thmlist}{enumerate}{1}
\setlist[thmlist]{
    ref=\thetheorem.(\arabic*),
    label=(\arabic*),
    % topsep=0pt,
}
\crefname{thmlisti}{Theorem}{Theorems}
\Crefname{thmlisti}{Theorem}{Theorems}

% Reference numbers in the list.
\newcommand{\listnum}[1]{(#1)}
\newcommand{\listimp}[2]{\listnum{#1} $\Rightarrow$ \listnum{#2}}
\newcommand{\listeq}[2]{\listnum{#1} $\Leftrightarrow$ \listnum{#2}}

% Backward compatibility:
\newcommand{\listimpl}[2]{\listnum{#1} $\Rightarrow$ \listnum{#2}:}


% We actually will use coloured links.
\definecolor{aquamarine}{HTML}{218274}
\hypersetup{
    colorlinks,
    citecolor = aquamarine,
    filecolor = aquamarine,
    linkcolor = aquamarine,
    urlcolor  = aquamarine,
}

% Use italic font in bodies for clarity.
\newtheoremstyle{theoremstyle}
    {\thmtopsep}{\thmbotsep}
    {\itshape}   % Body font
    {}           % Indent amount
    {\bfseries}  % Theorem head font
    {.}          % Punctuation after theorem head
    {.5em}       % Space after theorem head
    {}           % Theorem head spec
\theoremstyle{theoremstyle}

% Set numbering of theorems right.
\newtheorem{theorem}{Theorem}[chapter]
\newtheorem{proposition}[theorem]{Proposition}
\newtheorem{corollary}[theorem]{Corollary}
\newtheorem{fact}[theorem]{Fact}
\newtheorem{lemma}[theorem]{Lemma}

\newtheorem{assumption}[theorem]{Assumption}
\newtheorem{definition}[theorem]{Definition}
\newtheorem{procedure}[theorem]{Procedure}

\newtheorem{question}[theorem]{Question}
\newtheorem{example}[theorem]{Example}
\newtheorem{model}[theorem]{Model}
\newtheorem{remark}[theorem]{Remark}

% Hack `\listoftheorem` to remove the title.
\makeatletter
\renewcommand\listoftheorems[1][]{%
    \begingroup
    \setlisttheoremstyle{#1}%
    \let\listfigurename\listtheoremname
    \def\contentsline##1{%
        \csname thmt@contentsline@##1\endcsname{##1}%
    }%
    \@for\thmt@envname:=\thmt@allenvs\do{%
        \thmtlo@newentry
    }%
    \let\thref@starttoc\@starttoc
    \def\@starttoc##1{\thref@starttoc{loe}}%
    \@fileswfalse
    \AtEndDocument{%
        \if@filesw
        \@ifundefined{tf@loe}{%
            \expandafter\newwrite\csname tf@loe\endcsname
            \immediate\openout \csname tf@loe\endcsname \jobname.loe\relax
        }{}%
        \fi
    }%
    \@starttoc{lof}
    \endgroup
}
\makeatother

% Fix spacing around proofs.
\xpatchcmd{\proof}{\topsep6\p@\@plus6\p@\relax}{}{}{}
\BeforeBeginEnvironment{proof}{\vspace{-0.5em}}
\AfterEndEnvironment{proof}{\vspace{-0.5em}}

% Commands specific for thesis:
\newcommand{\AR}{\operatorname{AR}}
\newcommand{\enc}{\mathsf{enc}}
\newcommand{\dec}{\mathsf{dec}}
\newcommand{\mult}{\operatorname{mult}}

\newcommand{\missingcite}{\note{(CITATION MISSING)}}

% Listings:
\usepackage{tcolorbox}
\tcbuselibrary{minted,skins,breakable}

\definecolor{solarized-light-bg}{HTML}{fdf6e3}
\definecolor{solarized-light-fg}{HTML}{586e75}
\newtcblisting{pythoncode}[2]{
    listing engine = minted,
    listing only,
    minted style = solarized-light,
    minted language = python,
    minted options = {
        fontsize = #1,
        escapeinside = ||,
        mathescape = true,
        highlightlines = #2,
        highlightcolor = red,
    },
    colback = solarized-light-bg,
    colframe = solarized-light-bg,
    toprule = 0pt,
    left = 5pt,
    left = 5pt,
    leftrule = 0pt,
    rightrule = 0pt,
    bottomrule = 0pt,
    arc = 0pt,
    frame hidden,
    breakable,
}
% Disable italics.
\AtBeginEnvironment{pythoncode}{\let\itshape\relax}

% Override the textwriter font.
%\usepackage[scaled=0.8]{beramono}
\usepackage[scaled=0.95]{inconsolata}
\renewcommand{\code}[1]{\colorbox{solarized-light-bg}{\small\color{solarized-light-fg}\texttt{#1}}}

% We never want footnotes to break across pages.
\interfootnotelinepenalty=10000

% Allow restatable environments.
\newenvironment{manual}[3][]{%
    \def\savedarg{#2}%
    \expandafter\renewcommand\csname the#2\endcsname{#3}%
    \ifempty{#1}{\csname #2\endcsname}{\csname #2\endcsname[#1]}%
}{%
    \csname end\savedarg\endcsname%
    \addtocounter{\savedarg}{-1}%
}
\newcommand{\statement}[1]{
    \begingroup
        \subimport{}{#1}
        \ifempty{\statementoption}{
            \csname\statementtype\endcsname
        }{
            \expandafter\csname\statementtype\endcsname[\statementoption]%
        }
        \label{\statementlabel}
        \statementcontent
        \csname end\statementtype\endcsname
    \endgroup
}
\newcommand{\restatement}[1]{
    \begingroup
        \subimport{}{#1}
        \ifempty{\statementoption}{
            \manual{\statementtype}{\ref{\xrprefix{\statementlabel}}}
        }{
            \manual[\statementoption]{\statementtype}{\ref{\xrprefix{\statementlabel}}}
        }
        \newcommand{\insiderestatement}{}
        \renewcommand{\label}[1]{}
        \statementcontent
        \endmanual
    \endgroup
}

%\newcommand{\ballnumber}[1]{\tikz[baseline=(myanchor.base)] \node[circle,fill=.,inner sep=1pt] (myanchor) {\color{-.}\normalshape\bfseries\footnotesize #1};}

\newcommand{\highlight}{\textcolor{}{}}

\addbibresource{../../bibliography.bib}

\usepackage{xr}
\externaldocument[xr-]{../../main}
\newcommand{\xrprefix}[1]{xr-#1}

\begin{document}

\chapter
    {Proofs for Chapter \ref{\xrprefix{chap:repr_theorems}}}
\label{app:proofs_repr_theorems}

\section{Proofs for Section \ref{\xrprefix{sec:repr_theorems:functions}}}
\label{sec:proofs_repr_theorems:functions}

In what follows, $D,E,F \in \D$ will denote generic data sets, and $A \sub \D$ will denote a generic collection of data sets.
For $\e > 0$, denote the open balls of radius $\e$ by
\begin{equation}
    \Bb_\e(D) = \set{E \in \D : d_\D(D, E) < \e},
    \quad
    \Bb_\e([D]) = \set{E \in \D : d_{[\D]}([D], [E]) < \e}.
\end{equation}
Let $\Sb = \union_{N \ge 0} \Sb^N$ be the collection of permutations of any size.


\restatement{../../chapters/reprtheorems/statements/properties.tex}
\begin{proof}
    \listnum{1}:
    To begin with, observe that $d_{\D}(D, E) = d_{\D}(\sigma D, \sigma E)$ for all $\sigma \in \Sb^{\abs{D}}$.
    We call this property \emph{permutation invariance} of $d_\D$.
    We first show that $d_{[\D]}$ is well defined.
    To this end, consider $D, D', E, E' \in \D$ such that $D = \sigma_D D'$ and $E = \sigma_E E'$ for some $\sigma_D \in \Sb^{\abs{D'}}$ and $\sigma_E \in \Sb^{\abs{E'}}$.
    In particular, this means that $\abs{D} = \abs{D'}$ and $\abs{E} = \abs{E'}$.
    Then
    \begin{align}
        d_{[\D]}([D], [E])
        &= \inf_{\sigma_2 \in \Sb^{\abs{E}}} d_\D(D, \sigma_2 E) \\
        &= \inf_{\sigma_2 \in \Sb^{\abs{E'}}} d_\D(\sigma_D D', \sigma_2 \sigma_E E') \\
        &= \inf_{\sigma_2 \in \Sb^{\abs{E'}}} d_\D(D', \sigma_D^{-1} \sigma_2 \sigma_E E') \\
        &= \inf_{\sigma_2 \in \Sb^{\abs{E'}}} d_\D(D', \sigma_2 E') \\
        &= d_{[\D]}([D'], [E']).
    \end{align}
    We conclude that $d_{[\D]}$ does not depend on the choice of the representative, so it is well defined.
    It remains to show that $d_{[\D]}$ is a metric.
    It is clear that $d_{[\D]}([D], [E]) = d_{[\D]}([E], [D])$ and that $d_{[\D]}([D], [E]) = 0$ if and only if $[D] = [E]$.
    To show the triangle inequality,
    let $D, E, F \in \D$.
    We will show that
    \begin{equation}
        d_{[\D]}([D], [F])
        \le
        d_{[\D]}([D], [E])
        + d_{[\D]}([E], [F]).
    \end{equation}
    If either of the terms on the right-hand side is infinite, then nothing remains to be shown.
    Hence, assume that both are finite.
    In that case, $\abs{D} = \abs{E} = \abs{F}$.
    Let $\sigma_2, \sigma'_2 \in \Sb^{\abs{D}}$.
    Then, by the triangle inequality of $d_\D$,
    \begin{equation}
        d_\D(D, \sigma_2 \sigma_2' F) \le d_\D(D, \sigma_2 E) + d_\D(\sigma_2 E, \sigma_2 \sigma_2' F)
        = d_\D(D, \sigma_2 E) + d_\D(E, \sigma_2' F),
    \end{equation}
    using permutation invariance of $d_\D$.
    Take the infimum over $\sigma_2 \in \Sb^{\abs{D}}$ then $\sigma_2' \in \Sb^{\abs{D}}$ to conclude.

    \listnum{2}:
    Let $[D] \in [A]$.
    Then there exists a $\sigma \in \Sb$ such that $\sigma D \in A$.
    Hence, there exists a $\e > 0$ such that $B_\e(\sigma D) \sub A$.
    We claim that $\Bb_\e([D]) \sub [A]$, which shows the result.
    To show the claim, let $[E] \in \Bb_\e([D])$.
    Then $d_{[\D]}([D], [E]) < \e$, so $d_\D(D, \sigma' E) < \e$ for some $\sigma' \in \Sb$,
    meaning that $d_\D(\sigma D, \sigma \sigma' E) < \e$.
    Hence $\sigma \sigma' E \in \Bb_\e(\sigma D) \sub A$, so $[E] = [\sigma \sigma' E] \in [A]$.

    \listnum{3}:
    Let $([D_i])_{i \ge 1} \sub [A]$ be convergent to some $[D] \in [\D]$.
    Then $d([D_i], [D]) \to 0$,
    so there exists a sequence $(\sigma_i)_{i \ge 1}$ such that $d(\sigma_i D_i, D) \to 0$.
    Since $A$ is permutation invariant, all $\sigma_i D_i \in A$.
    Hence $D \in A$, because $A$ is closed.
    We conclude that $[D] \in [A]$, so $[A]$ is also closed.
    
    \listnum{4}:
    Denote $p\colon \D \to [\D]$, $p(D) = [D]$.
    Note that $p$ is $d_\D$--$d_{[\D]}$ continuous.
    Let $[A] \sub [\D]$.
    If $[A]$ is open in the quotient topology, then, by definition, $p^{-1}([A])$ is open.
    Hence, by \listnum{3}, $p(p^{-1}([A]))$ is open in $d_{[\D]}$.
    Conversely, assume that $[A]$ is open in $d_{[\D]}$.
    Then, by $d_\D$--$d_{[\D]}$ continuity of $p$, $p^{-1}([A])$ is open in $d_\D$.
    Hence, $[A]$ is open in the quotient topology.
\end{proof}

\restatement{../../chapters/reprtheorems/statements/equivalence.tex}
\begin{proof}
    That $[f]$ is well defined follows from permutation invariance of $f$,
    and that $[f]$ is continuous follows from continuity of $f$.
    Using continuity of the quotient map $D \mapsto [D]$, the converse is immediate.
\end{proof}

\section{Proofs for Section \ref{\xrprefix{sec:repr_theorems:conv_deep_sets}}}
\label{sec:proofs_repr_theorems:conv_deep_sets}

Call a space $\F$ of functions on $\X$ \emph{interpolating} if, for every $N \in \N$, $\vx \in \X^N$, and $\vy \in \R^N$, there exists an $f \in \F$ such that $f(x_n) = y_n$ for all $n = 1, \ldots, N$.
Note that every reproducing kernel Hilbert space associated to a strictly-positive-definite kernel is interpolating.
Throughout, we assume that $\Y$ is compact.

\begin{lemma} \label{lem:compact_injection}
    Let $[\D'_{N}] \sub [\D_N]$ have multiplicity $K$.
    Set
    \begin{equation}
        \phi\colon \mathcal{Y} \to \R^{K+1}, \quad
        \phi(y) = (y^0, y^1, \cdots, y^K).
    \end{equation}
    Let $k\colon \X \times \X \to \R$ be a continuous strictly-positive-definite kernel.
    Denote the reproducing kernel Hilbert space of $k$ by $\Hb$.
    Endow $\Hb^{K+1}$ with the inner product $\lra{f, g}_{\Hb^{K+1}} = \sum_{i=1}^{K+1} \lra{f_i, g_i}_{\Hb}$.
    Define
    \begin{equation}
        \Hb_{N} = \set*{
            \sum_{n=1}^N \phi(y_n) k(\vardot, x_n) :
            [(x_n, y_n)_{n=1}^N] \sub [\D'_N]
        }
        \subseteq \Hb^{K+1},
    \end{equation}
    Then the embedding
    \begin{equation}
        \enc_N \colon [\D'_{N}] \to \Hb_{N}, \quad
        \enc_N([(x_1, y_1), \ldots, (x_N, y_N)])
        = \sum_{n=1}^N \phi(y_n) k(\vardot, x_n)
    \end{equation}
    is injective, hence invertible, and continuous. 
\end{lemma}
\begin{proof}
    First, we show that $\enc_N$ is injective.
    Suppose that
    \begin{equation}
        \sum_{n=1}^N \phi(y_n) k(\vardot, x_n)
        = \sum_{n=1}^{N} \phi(y'_n) k(\vardot, x'_n).
    \end{equation}
    Denote $\vx = (x_1, \ldots, x_N)$ and $\vy = (y_1, \ldots, y_N)$, and denote $\vx'$ and $\vy'$ similarly.
    Taking the inner product with any $f \in \Hb$ on both sides and using the reproducing property of $k$ implies that
    \begin{equation}
        \sum_{n=1}^N \phi(y_n) f(x_n) = \sum_{n=1}^{N} \phi(y'_n) f(x'_n)
        \quad
        \text{for all $f \in \Hb$}.
    \end{equation}
    In particular, by construction $\phi_1(\vardot) = 1$, so
    \begin{equation}
        \sum_{n=1}^N f(x_n) = \sum_{n=1}^{N} f(x'_n)
    \end{equation}
    for all $f \in \Hb$.
    Using that $\Hb$ is interpolating, choose an element $\hat x$ of $\vx$ or $\vx'$, and let $f \in \Hb$ be such that $f(\hat x) = 1$ and $f(\vardot) = 0$ at all other elements of $\vx$ and $\vx'$.
    Then
    \begin{equation}
        \sum_{n\,:\,x_n = \hat x} 1
        = \sum_{n\,:\,x'_n = \hat x} 1,
    \end{equation}
    so the number of $\hat x$ in $\vx$ and the number of $\hat x$ in $\vx'$ are the same.
    Since this holds for every $\hat x$, $\vx$ is a permutation of $\vx'$: $\vx = \sigma \vx'$ for some permutation $\sigma \in \Sb^N$.
    Plugging in the permutation, we can write 
    \begin{align}
        \sum_{n=1}^N \phi(y_n) f(x_n)
        &= \sum_{n=1}^{N} \phi(y'_n) f(x'_n) \\
        &= \sum_{n=1}^{N} \phi(y'_n) f(x_{\pi^{-1}(n)}) && \text{$(\vx' = \sigma^{-1}\vx)$} \\
        &= \sum_{n=1}^{N} \phi(y'_{\pi(n)}) f(x_{n}). && \text{$(n \gets \sigma^{-1}(n))$}
    \end{align}
    Then, by a similar argument, for any particular $\hat x$, 
    \begin{equation} \label{eq:injectivity_phi}
        \sum_{n\,:\,x_n = \hat x} \phi(y_n)
        = \sum_{n\,:\,x_n = \hat x} \phi(y'_{\sigma(n)}).
    \end{equation}
    Let the number of terms in each sum equal $S$.
    Since $[\D_{N}']$ has multiplicity $K$, $S \leq K$.
    Consider the first $S$ dimensions of $\phi$ in \eqref{eq:injectivity_phi}.
    Then, using the definition of $\phi$, Lemma 4 by \textcite{Zaheer:2017:Deep_Sets} shows that
    \begin{equation} \label{eq:permutation}
        (y_n)_{n\,:\,x_n = \hat x}
        \quad\text{is a permutation of}\quad
        (y'_{\sigma(n)})_{n\,:\,x_n = \hat x}.
    \end{equation}
    %Note that $x_n = \hat x$ for all $n$ such that $x_n = \hat x$.
    By $\sigma \vx = \vx'$, note that $x'_{\sigma(n)} = x_n = \hat x$ for all these $n$.
    Therefore, all these $x_n$ and $x'_{\sigma(n)}$ are equal.
    Using \eqref{eq:permutation},
    adjust the permutation $\sigma$ to obtain $y_n = y'_{\sigma(n)}$ for all these $n$.
    Since all these $x_n$ and $x'_{\sigma(n)}$ are equal,
    still $\vx = \sigma \vx'$ after the adjustment.
    Performing this adjustment for all $\hat x$, we find that $\vy = \sigma \vy'$ and $\vx = \sigma \vx'$.
    
    Second, we show that $E_N$ is continuous.
    % Recall that for a product of Hilbert spaces $\Hb^{k}$ we can specify, without loss of generality,
    % $
    % \| f \|_{\Hb^k}^2 = \sum_{i=1}^k \| f \|_{\Hb}^2.
    % $\wpbnote{I wouldn't say that we ``have that'' but rather ``choose that''.}
    Compute 
    \begin{align}
        &\left\|
            \sum_{n=1}^N \phi(y_n) k(\vardot, x_n)
            - \sum_{n=1}^{N} \phi(y'_n) k(\vardot, x'_n)
        \right\|^2_{\Hb^{K+1}}
        \\ & \qquad =
            \sum_{i=1}^{K+1} \parens{
                \phi^\T(\vy) k(\vx, \vx) \phi_i(\vy)
                - 2 \phi^\T(\vy) k(\vx, \vx') \phi_i(\vy')
                + \phi^\T(\vy') k(\vx', \vx') \phi_i(\vy')
            }, \nonumber
    \end{align}
    which, by continuity of $k$, goes to zero if $[(\vx', \vy')] \to [(\vx, \vy)]$.
\end{proof}

%Having established the injection, we now show that this mapping is a homeomorphism, i.e.~that the inverse is continuous. This is formalized in the following lemma.

\begin{lemma} \label{lem:closed_injection}
    Consider \cref{lem:compact_injection}.
    Suppose that $[\D'_N]$ is also closed and that $k$ also satisfies
    (1) $k(x, x) = \sigma^2 > 0$,
    (2) $k \ge 0$, and
    (3) $k(x, x') \to 0$ as $\abs{x} \to \infty$.
    Then $\Hb_N$ is closed in $\Hb^{K+1}$.
    Moreover, $\enc_N$ is a homeomorphism.
\end{lemma}
\begin{proof}
    Define 
    \begin{equation}
        [\D_J]
        =
        [([-J, J] \times \mathcal{Y})^N]
        \cap [\D'_N],
    \end{equation}
    which is compact in $[\D_N]$ as a closed subset of the compact set $[([-J, J] \times \mathcal{Y})^N]$.
    We aim to show that $\Hb_N$ is closed in $\Hb^{K+1}$ and that $\enc_N^{-1}$ is continuous.
    To this end, consider a convergent sequence 
    \begin{equation}
        f^{(m)} = {\sum_{n=1}^N} \phi(y^{(m)}_n) k(\vardot, x^{(m)}_n)
        \to f \in \Hb^{K+1}.
    \end{equation}
    Denote $\vx^{(m)} = (x^{(m)}_1, \ldots, x^{(m)}_N)$ and $\vy^{(m)} = (y^{(m)}_1, \ldots, y^{(m)}_N)$.
    Claim: $(\vx^{(m)})_{m\ge1}$ is a bounded sequence, so $(\vx^{(m)})_{m\ge1} \sub [-J, J]^{N}$ for $J$ large enough, which means that $[(\vx^{(m)}, \vy^{(m)})_{m\ge1}] \sub [\D_J]$.
    
    First, assuming the claim, we show that $\Hb_N$ is closed.
    By boundedness of $(\vx^{(m)}, \vy^{(m)})_{m\ge1}$, $(f^{(m)})_{m\ge1}$ is in the image of $\enc_N|_{[\D_J]} \colon [\D_J] \to \Hb_N$.
    By continuity of $\enc_N|_{[\D_J]}$ and compactness of $[\D_J]$, the image of $\enc_N|_{[\D_J]}$ is compact and hence closed.
    Therefore, the image of $\enc_N|_{[\D_J]}$ contains the limit $f$.
    Since the image of $\enc_N|_{[\D_J]}$ is included in $\Hb_N$, we have that $f \in \Hb_N$, which shows that $\Hb_N$ is closed.

    Second, assuming the claim, we prove that $\enc_N^{-1}$ is continuous.
    Consider $\enc_N|_{[\D_J]} \colon [\D_J] \to \enc_N([\D_J])$ restricted to its image.
    Then $(\enc_N|_{[\D_J]})^{-1}$ is continuous, because every continuous bijection from a compact space to a Hausdorff space is a homeomorphism \parencite[Theorem 26.6;][]{Munkres:2000:Topology}.
    Therefore,%
    \begin{equation}
        \enc_N^{-1}(f^{(m)})
        = [(\vx^{(m)}, \vy^{(m)})]
        = (\enc_N|_{[\D_J]})^{-1}(f^{(m)})
        \to (\enc_N|_{[\D_J]})^{-1}(f).
    \end{equation}
    Denote $(\enc_N|_{[\D_J]})^{-1}(f) = [(\vx, \vy)]$.
    By continuity and invertibility of $\enc_N$, then $f^{(m)} \to \enc_N([(\vx, \vy)])$, which means that $[(\vx, \vy)] = \enc_N^{-1}(f)$  by uniqueness of limits.
    We conclude that $\enc_N^{-1}(f^{(m)}) \to \enc_N^{-1}(f)$, so $\enc_N^{-1}$ is continuous.
    
    It remains to show the claim. Let $f_1$ denote the first element of $f$, \ie~the \emph{density channel}.
    Using the reproducing property of $k$,
    \begin{align}
        |f^{(m)}_1(x) - f_1(x)|
        &= |\langle k(x, \vardot), f^{(m)}_1 - f_1\rangle_\Hb| \\
        &\le \|k(x, \vardot)\|_{\Hb} \|f_1^{(m)} - f_1\|_{\Hb} \\
        &= \sigma \|f_1^{(m)} - f_1\|_{\Hb},
    \end{align}
    so $f_1^{(n)} \to f_1$ in $\Hb$ means that it does so uniformly.
    Hence, we can let $M \in \N$ be such that $m \ge M$ implies that $|f^{(m)}_1(x) - f_1(x)| < \tfrac13 \sigma^2$ for all $x \in \X$.
    Let $R$ be such that $\displaystyle|k(x, x^{(M)}_n)| < \tfrac13 \sigma^2 / N$ for $\abs{x} \ge R$ and all $n \in \set{1, \ldots, N}$. 
    Then, for $\abs{x} \ge R$,
    \begin{equation}
        |f_1^{(M)}(x)|
        \le {\sum_{n=1}^N} |k(x, x^{(M)}_n)|
        < \tfrac13 \sigma^2,
    \end{equation}
    which implies that, for all $\abs{x} \ge R$,
    \begin{equation}
        |f_1(x)| \le |f^{(M)}_1(x)| + |f^{(M)}_1(x) - f_1(x)| < \tfrac23 \sigma^2.
    \end{equation}
    At the same time, by pointwise nonnegativity of $k$, we have that
    \begin{equation}
        f_1^{(m)}(x^{(m)}_{n})
        = {\sum_{n'=1}^N} k(x^{(n)}_{n}, x^{(n)}_{n'})
        \ge k(x^{(m)}_n, x^{(m)}_n)
        = \sigma^2.
    \end{equation}
    Towards contradiction, suppose that $(\vx^{(m)})_{m\ge1}$ is unbounded. 
    Then $\displaystyle(x^{(m)}_n)_{m\ge1}$ is unbounded for some $n \in \set{1, \ldots, N}$.
    Therefore, $\displaystyle|x_n^{(m)}| \ge R$ for some $m \ge M$, so
    \begin{equation}
        \tfrac23 \sigma^2
        > |f_1(x_n^{(m)})|
        \ge |f^{(m)}_1(x_n^{(m)})| - |f^{(m)}_1(x_n^{(m)}) - f_1(x_n^{(m)})|
        \ge \sigma^2 - \tfrac13 \sigma^2
        = \tfrac23 \sigma^2,
    \end{equation}
    which is a contradiction.
\end{proof}

\begin{lemma}
    \label{lem:encoding_varying_size}
    Suppose that $[\D'] \sub [\D]$ is closed, has multiplicity $K$, and has maximum data set size $N\ss{max} < \infty$.
    Set
    \begin{equation}
        \phi\colon \mathcal{Y} \to \R^{K+1}, \quad
        \phi(y) = (y^0, y^1, \cdots, y^K).
    \end{equation}
    Let $k\colon \X \times \X \to \R$ be a continuous strictly-positive-definite kernel such that
    (1) $k(x, x) = \sigma^2 > 0$,
    (2) $k \ge 0$, and
    (3) $k(x, x') \to 0$ as $\abs{x} \to \infty$.
    Denote the reproducing kernel Hilbert space of $k$ by $\Hb$.
    Endow $\Hb^{K+1}$ with the inner product $\lra{f, g}_{\Hb^{K+1}} = \sum_{i=1}^{K+1} \lra{f_i, g_i}_{\Hb}$.
    For $N \in \set{0, \ldots, N\ss{max}}$, denote $[\D'_N] = [\D'] \cap [\D_N]$ and define
    \begin{equation}
        \Hb_{N} = \set*{
            \sum_{n=1}^{N} \phi(y_n) k(\vardot, x_n) :
            [(x_n, y_n)_{n=1}^N] \sub [\D'_N]
        }
        \subseteq \Hb^{K+1}.
    \end{equation}
    % For every $m \in [N]$, consider a collection $\D'_m \subseteq \D_{m}$ that
    % \begin{itemize}
    %     \item has multiplicity $K$,
    %     \item is topologically closed, and
    %     \item is closed under permutations.
    % \end{itemize}
    % \begin{equation}
    %     [\D'_{\le N}] = \bigcup_{n=1}^N [\D'_n]
    %     \quad\text{and}\quad
    % \end{equation}
    Then $(\Hb_{N})_{N=0}^{N\ss{max}}$ are pairwise disjoint.
    Denote $\Hb' = \bigcup_{N=0}^{N\ss{max}} \Hb_{N}$.
    Then
    \begin{equation}
        \enc\colon
            [\D']
            \to
            \Hb',
        \quad
        \enc([D]) = \enc_{N}([D]) \quad \text{if} \quad [D] \in [\D_N]
        %E([(\vx_1, y_1), \ldots, (\vx_m, y_m)])
        %= \sum_{i=1}^m \phi(y_i) k(\vardot, x_i)
    \end{equation}
    is a homeomorphism.
    %Denote this inverse by $\enc^{-1}$, where $\enc^{-1}(f) = \enc_n^{-1}(f)$ if $f \in \Hb_n$.
    %Then the restrictions $E|_{[\D'_m]}\colon [\D'_m] \to \Hb_m$ and $E^{-1}|_{\Hb_m}\colon\Hb_m \to [\D'_m]$ are continuous for all $m \in [N]$. 
\end{lemma}
\begin{proof}
    To begin with, we show that
    $(\Hb_{N})_{N=0}^{N\ss{max}}$ are pairwise disjoint.
    % From repeated application of \cref{lem:compact_injection,lem:closed_injection}, we find that every restriction
    % \begin{equation}
    %     E|_{[\D'_m]} \colon [\D'_m] \to \Hb_m
    % \end{equation}
    % is invertible, continuous, and has a continuous inverse.
    % The result then follows by showing that $(\Hb_m)_{m=1}^N$ are closed and pairwise disjoint.
    %$[\D'_m]$ is closed and $(E|_{[\D'_m]})^{-1}$ is continuous, so
    %\begin{equation}
    %    \Hb_m = ((E|_{[\D'_m]})^{-1})^{-1}([\D'_m])
    %\end{equation}
    %is closed, since it is the preimage of a closed set under a continuous function. \aykfnote{This argument is a bit subtle. Does this prove that $\Hb_m$ is closed under the subset topology (which of course it is) or are we proving that it is closed as a subset of $\Hb^{K+1}$ (which is what we actually want to show)?}
    % Recall that $\enc_n$ is injective for every $n \in [N]$.
    % Hence, to demonstrate that $\enc$ is injective it remains to show that $(\Hb_n)_{n=1}^N$ are pairwise disjoint.
    To this end, suppose that
    \begin{equation}
        \sum_{n=1}^{N} \phi(y_n) k(\vardot, x_n)
        = \sum_{n=1}^{N'} \phi(y'_n) k(\vardot, x'_n)
    \end{equation}
    for $N \neq N'$.
    Then, by arguments like in the proof of \cref{lem:compact_injection},
    \begin{equation}
        \sum_{n=1}^N \phi(y_n)
        = \sum_{n=1}^{N'} \phi(y'_n).
    \end{equation}
    Since $\phi_1(\vardot)=1$, this gives $N = N'$, which is a contradiction.

    Second, we show that $\enc$ is a homeomorphism.
    Note that
    $(\Hb_{N})_{N=0}^{N\ss{max}}$ are closed and pairwise disjoint,
    and that $([\D'_N])_{N=0}^{N\ss{max}}$ are also closed and pairwise disjoint.
    By \cref{lem:closed_injection}, every
    \begin{equation}
        \enc|_{[\D'_N]} \colon [\D'_N] \to \Hb_N
    \end{equation}
    is a homeomorphism.
    Therefore,
    \begin{equation}
        \enc \colon \union_{N=0}^{N\ss{max}} [\D'_N] \to  \union_{N=0}^{N\ss{max}} \Hb_{N}
    \end{equation}
    is also a homeomorphism:
    invertibility follows from disjointness,
    continuity of $\enc$ follows from the pasting lemma \parencite[Theorem 18.3;][]{Munkres:2000:Topology},
    and continuity of $\enc^{-1}$ similarly follows from the pasting lemma.
\end{proof}

% \begin{lemma} \label{lemma:total_continuity}
%     Let $\Phi\colon [\D'_{\le N}] \to C_b(\mathcal{X}, \mathcal{Y})$ be a map from $[\D'_{\le N}]$ to $C_b(\mathcal{X}, \mathcal{Y})$, the space of continuous bounded functions from $\mathcal{X}$ to $\mathcal{Y}$, such that every restriction $\Phi|_{[\D'_m]}$ is continuous, and 
%     let $E$ be from \cref{lemma:encoding_varying_size}.
%     Then
%     \begin{equation}
%         \Phi \comp E^{-1}\colon \Hb_{\le N} \to C_b(\mathcal{X}, \mathcal{Y})
%     \end{equation}
%     is continuous.
% \end{lemma}
% \begin{proof}
%     Recall that, due to \cref{lem:compact_injection}, for every $m \in [N]$, $E^{-1}_{m}$ is continuous and has image $[\D'_m]$.
%     By the continuity of $\Phi|_{[\D'_m]}$, then $\Phi|_{[\D'_m]} \comp E^{-1}_{m}$ is continuous for every $m \in [N]$.
%     Since $\Phi \comp E^{-1}|_{\Hb_m} = \Phi|_{[\D'_m]} \comp E^{-1}_{m}$ for all $m \in [N]$, we have that $\Phi \comp E^{-1}|_{\Hb_m}$ is continuous for all $m \in [N]$.
%     Therefore, as $\Hb_m$ is closed in $\Hb_{\le N}$ for every $m \in [N]$, the pasting lemma \parencite{munkres1974topology} yields that $\Phi \comp E^{-1}$ is continuous.
%     % By continuity of $\Phi$ and $E^{-1}$, every restriction $\Phi \comp E^{-1}|_{\Hb_m}$ is continuous.
%     % We show that $\Phi \comp E^{-1}$ is continuous by showing that every convergent sequence in $\Hb_{\le N}$ is eventually contained in some $\Hb_m$ and appealing to continuity of $\Phi \comp E^{-1}|_{\Hb_m}$.
%     % For suppose that $(f^{(n)})_{n=1}^\infty \sub \Hb_{\le N}$ converging to some $f \in \Hb_{\le N}$ visits $\Hb_m$ and $\Hb_{m'}$ with $m \neq m'$ infinitely often.
%     % Then we can extract subsequences $(f^{(n_k)})_{k=1}^\infty \sub \Hb_{m}$ and $(f^{(n_{k'})})_{k'=1}^\infty \sub \Hb_{m'}$, which must both be convergent to $f$.
%     % Since $\Hb_m$ and $\Hb_{m'}$ are closed and disjoint, this is a contradiction.
% \end{proof}
%

\restatement{../../chapters/reprtheorems/statements/conv_deep_set.tex}
\begin{proof}
    Our proof follows the strategy used by \textcite{Zaheer:2017:Deep_Sets}.
    If $\pi$ is of the stated form, then it is continuous and translation equivariant as a composition of two continuous and translation-equivariant functions.
    Note that continuity of $\enc$ follows from \cref{lem:encoding_varying_size},
    and that translation equivariance of $\enc$ is easily verified.
    Conversely, suppose that $\pi \colon [\D'] \to Z$ is continuous and translation equivariant.
    Let $\enc$ be of the stated form.
    By \cref{lem:encoding_varying_size}, $\enc$ is a homeomorphism.
    Set $\dec = \pi \comp \enc^{-1}$.
    Then $\enc$ is continuous,
    and $\dec$ is continuous as a composition of two continuous functions.
    Moreover, $\enc$ is translation equivariant by construction, and $\dec$ is translation equivariant as a composition of two translation equivariant functions.
\end{proof}

\section{Proofs for Section \ref{\xrprefix{sec:repr_theorems:conv_deep_sets_dte}}}
\label{sec:proofs_repr_theorems:conv_deep_sets_dte}

\begingroup
    \newcommand{\maybeprefix}[1]{\xrprefix{#1}}
    \restatement{../../chapters/reprtheorems/statements/conv_deep_set_dte.tex}
\endgroup
\begin{proof}[Addendum to proof.]
    The proof in \cref{\xrprefix{sec:repr_theorems:conv_deep_sets_dte}} is nearly complete.
    The only thing which remains is to show that the extension
    $
        \overline{\pi} \colon A \times \set{\T_{\vtau} c : \vtau \in \X \times \X} \to Z
    $
    is continuous.
    To this end, consider a sequence $(a_i, \T_{\vtau_i} c)_{i \ge 1}$ convergent to $(a, \T_{\vtau} c)$.
    Let $\ve_\parallel = (1, 1) / \sqrt{2}$ and $\ve_\perp = (1, -1) / \sqrt{2}$.
    Denote $\vtau_{i,\parallel} = \lra{\vtau_i, \ve_{\parallel}} \ve_{\parallel}$
    and $\vtau_{i,\perp} = \lra{\vtau_i, \ve_{\perp}} \ve_{\perp}$.
    Then
    \begin{equation} \label{eq:overline_pi_to_pi}
        \overline{\pi}(a_i, \T_{\vtau_i} c)
        \overset{\smash{\text{(i)}}}{=} \overline{\pi}(a_i, \T_{\vtau_{i, \perp}} c)
        \overset{\smash{\text{(ii)}}}{=} \T_{\vtau_{i, \perp}} \overline{\pi}(\T_{-\vtau_{i,\perp}}a_i, c)
        \overset{\smash{\text{(iii)}}}{=} \T_{\vtau_{i, \perp}} \pi(\T_{-\vtau_{i,\perp}}a_i, c)
        \hspace{-5pt}
    \end{equation}
    using in (i) that $c$ is diagonally translation invariant, in (ii) that $\pi$ is TE, and in (iii) the definition of $\overline{\pi}$.
    Because $c$ is anti-diagonal discriminating, $(\vtau_{i,\perp})_{i \ge 1}$ is convergent.
    Note that $(\vtau, a) \mapsto \T_{-\vtau} a$
    and $(\vtau, z) \mapsto \T_{\vtau} z$ are continuous because $A$ and $Z$ are topological $(\X\times\X)$-translation spaces.
    Therefore, using continuity of $\pi$,
    \begin{align}
        \lim_{i \to \infty} \overline{\pi}(a_i, \T_{\vtau_i} c)
        = \T_{\vtau_{\perp}} \pi(\T_{-\vtau_{\perp}}a_i, c)
        = \overline{\pi}(a, \T_{\vtau} c)
    \end{align}
    where the latter equality reverses the steps in \eqref{eq:overline_pi_to_pi}.
    % \begin{align}
    %     \to \T_{\vtau_{\perp}} \pi(\T_{-\vtau_{\perp}}a, c) \\
    %     &= \T_{\vtau_{\perp}} \overline{\pi}(\T_{-\vtau_{\perp}}a, c) \\
    %     &= \overline{\pi}(a, \T_{\vtau_{\perp}} c) \\
    %     &= \overline{\pi}(a, \T_{\vtau} c) \\
    % \end{align}
\end{proof}


\end{document}
