\documentclass[12pt, twoside]{report}
\usepackage{libertine}

% General packages:
\usepackage[utf8]{inputenc}    % Input encoding
\usepackage[T1]{fontenc}       % Font encoding
\usepackage[british]{babel}    % Naming of figures and such
\usepackage[british]{isodate}  % Formatting of dates
\cleanlookdateon               % Show dates cleanly.
%\usepackage{xurl}              % Allow line breaks anywhere in URLs.
\usepackage{minted}            % Needs to be loaded before `csquotes`.
\usepackage[
    citestyle    = authoryear,
    bibstyle     = authoryear,
    giveninits   = true,
    uniquename   = init,
    uniquelist   = false,
    sortlocale   = en_GB,
    backend      = biber,
    backref      = true,
    maxbibnames  = 100,
    maxcitenames = 1,
    dashed       = false,
    % Needed to get sorting right for tussenvoegsels.
    %sortcites    = true,       
    ]{biblatex}                % Bibliography
\usepackage[
    style=american
    ]{csquotes}                % Required for BiBLaTeX
\usepackage{import}            % Importing subdocuments
\usepackage{standalone}        % Compilable subdocuments
% `setspace` needs to be loaded before `hyperref`.
\usepackage{setspace}          % Set line spacing
\usepackage{hyperref}          % Clickable references
\usepackage{microtype}         % Nice typography
\usepackage{makeidx}           % Make an index
\usepackage{silence}           % Silence warnings and errors
\WarningFilter{remreset}{The remreset package}
\WarningFilter*[parskip]{latex}{Command}

% Set line spacing.
\setstretch{1.25}

% Page layout:
\usepackage{tocloft}         % Control ToC
\usepackage[
    a4paper,
    bottom     = 1.1in,
    top        = 1.1in,
    left       = 1.1in,
    right      = 1.1in,
    headheight = 15pt,
    footskip   = 1.5\baselineskip
    ]{geometry}              % Margins
\usepackage{fancyhdr}        % Header
\usepackage{lastpage}        % Page numbers in footer
\usepackage{setspace}        % Line spacing
\usepackage{ragged2e}        % Better line endings
\ActivateWarningFilters[parskip]
\usepackage[
    parfill
    ]{parskip}               % Newlines instead of indentation
\DeactivateWarningFilters[parskip]
\usepackage{titlesec}        % Size of sections
\usepackage{titlecaps}       % Automatic capitalisation of titles
\usepackage[
    hang,
    bottom
    ]{footmisc}              % Configure footnotes
\usepackage{afterpage}       % Include stuff after the current page
\usepackage{placeins}        % Things like `\FloatBarrier`
\usepackage{pdflscape}       % Turn pages sideways
\usepackage{emptypage}       % Remove pagenumbers on empty pages

% Utility:
\usepackage[table]{xcolor}   % Colours
\usepackage[
    separate-uncertainty=true,
    per-mode=symbol
    ]{siunitx}               % Display units
\usepackage{enumitem}        % Better enumeration
\usepackage{xifthen}         % If statements
\usepackage{xpatch}          % Patch things
\usepackage{lipsum}          % Dummy text
\usepackage{listings}        % Listings
\usepackage{xparse}          % Better arguments for \newcommand
\usepackage{xfrac}           % Better fractions
\usepackage{etoolbox}        % Patch stuff
\usepackage[
    notquote
    ]{hanging}               % Hanging paragraphs
\usepackage{scalerel}        % Scale objects
\usepackage{soul}            % Highlighting

% Figures:
\usepackage{pgfplots}        % Plots
\pgfplotsset{compat=newest}
\usepackage{tikz}            % TikZ figures
\usepackage{float}           % Control floating of figures
\usepackage{multirow}        % Cells spanning multiple rows
\usepackage{graphicx}        % Graphics
\usepackage{caption}         % Subfigures and captions
\usepackage{subcaption}      % Subfigures and captions
\usepackage{booktabs}        % Nice looking tables
\usepackage{tabularx}        % Extended functionality for tables

% Math:
\usepackage{amsmath}         % Math
\usepackage{amssymb}         % Math symbols
\usepackage{mathtools}       % Math tools
\usepackage{amsthm}          % Theorems
\usepackage{thmtools}        % More math theorems
\usepackage{bm}              % Bold math symbols
\usepackage{bbm}             % More bold math symbols
\usepackage{cancel}          % Cancel equations
\usepackage{upgreek}         % Upright greek symbols
\usepackage[
    artemisia
    ]{textgreek}             % Greek symbols in text

% This package should be loaded last.
\usepackage[
    noabbrev,
    capitalize,
    nameinlink]{cleveref}    % Automatic referencing

% Add a comma in the cite style.
\renewcommand*{\nameyeardelim}{\addcomma\space}

% Make sure that every `\citeyear` also prints the disambiguating `extradate`.
\DeclareCiteCommand{\citeyearlabel}
    {\usebibmacro{prenote}}
    {\printfield{year}\printfield{extradate}}
    {\multicitedelim}
    {\usebibmacro{postnote}}
\let\citeyearold\citeyear
\let\citeyear\citeyearlabel
\let\citeyearnolabel\citeyearold

% Define a standard for citing within theorem statements.
\newcommand{\theoremcite}[1]{\citeauthor{#1}, \citeyear{#1}}

\newcommand{\fulltextcite}[1]{%
    \AtNextCite{\AtEachCitekey{\defcounter{maxnames}{999}}}%
    \textcite{#1}%
}
\newcommand{\fullparencite}[1]{%
    \AtNextCite{\AtEachCitekey{\defcounter{maxnames}{999}}}%
    \parencite{#1}%
}

% Remove the colon after "In" in the bibliograhpy.
\renewbibmacro*{in:}{\bibstring{in} }

% Tune the electronic print field.
\DeclareFieldFormat{eprint}{%
  \iffieldundef{eprinttype}
    {Electronic print}
    {\thefield{eprinttype}}%
  \addcolon\space
  \ifhyperref
    {\url{#1}}
    {\nolinkurl{#1}}%
  \iffieldundef{eprintclass}
    {}
    {\addspace\mkbibparens{\thefield{eprintclass}}}\printtext{.}}

% Tune URL.
\DeclareFieldFormat{url}{\mkbibacro{URL}\addcolon\space\url{#1}\printtext{.}}

% Tune DOI.
\DeclareFieldFormat{doi}{%
  \mkbibacro{DOI}\addcolon\space
  \ifhyperref
    {\href{https://doi.org/#1}{\nolinkurl{#1}}}
    {\nolinkurl{#1}}\printtext{.}}

% In the below, we will redefine `\finentry` to not print a period if
%` pageref` exists. The period is then added back by `pageref`, but before the
% parentheses. Problems occur when `urldate` is also defined. In that case, 
% the extra period by `pageref` will add a period after `urldate`s
% parentheses, which we do not want. We use the toggle `printperiod` to detect
% this case.

% Tune "visited on".
\DeclareFieldFormat{urldate}{\mkbibparens{\bibstring{urlseen}\space#1\printtext{.}}}
\newtoggle{printperiod}
\toggletrue{printperiod}
\AtBeginBibliography{\renewbibmacro*{urldate}{\printurldate\togglefalse{printperiod}}}

% Tune "cited on".
\renewbibmacro*{finentry}{\iflistundef{pageref}{}{\renewcommand{\finentrypunct}{}}\finentry}
\renewbibmacro*{pageref}{%
  \iflistundef{pageref}
    {}
    {\iftoggle{printperiod}{\setunit{\adddot\addspace}}{}\toggletrue{printperiod}\printtext[parens]{%
       \ifnumgreater{\value{pageref}}{1}
         {\bibstring{backrefpages}\ppspace}
         {\bibstring{backrefpage}\ppspace}%
          \printlist[pageref][-\value{listtotal}]{pageref}.}}}

% Add Oxford comma.
\DefineBibliographyExtras{british}{\def\finalandcomma{\addcomma}}

% Don't abbreviate the backreferences.
\DefineBibliographyStrings{english}{%
    backrefpage = {Cited on page},
    backrefpages = {Cited on pages}
}

% Get tussenvoegsels right. This is not a perfect solution, because the
% in-text citations will be sorted as if the tussenvoegsel is part of the
% last name.
\makeatletter
\AtBeginDocument{\toggletrue{blx@useprefix}}
\AtBeginBibliography{\togglefalse{blx@useprefix}}
\makeatother

% Make an index.
\makeindex

% Show the page number in the bottom center.
\fancypagestyle{plain}{
    \fancyhf{}                          % Clear all header and footers.
    \renewcommand{\headrulewidth}{0pt}  % Remove the header rule.
    \cfoot{\thepage}                    % Show page number in the center.
}
\pagestyle{plain}

% Chapter styling from https://texblog.org/2012/07/03/fancy-latex-chapter-styles/
\newcommand{\hsp}{\hspace{20pt}}
\titleformat{\chapter}[hang]
    {\Huge\bfseries}
    {\thechapter\hsp\textcolor{gray75}{|}\hsp}
    {0pt}
    {\Huge\bfseries}
\titleformat{\section}
    {\normalfont\Large\bfseries\raggedright}
    {\thesection}
    {1em}
    {}
\titleformat{\subsection}
    {\normalfont\large\bfseries\raggedright}
    {\thesubsection}
    {1em}
    {}

% Set spacing after chapter equal to spacing right before section. Then
% things look nicely aligned.
\titlespacing{\chapter}{0pt}{3.5ex}{3.5ex}

% Define the styling of paragraphs.
\renewcommand{\paragraph}[1]{\textbf{#1.}}

% Redefine abstract environment.
\renewenvironment{abstract}
    {\begingroup
    \setstretch{1.25}
    \begin{center}
        \textbf{Abstract}
    \end{center}}
    {\par
    \endgroup}

% Kill any use of the \cite command.
\renewcommand{\cite}[1]{\PackageError{thesis}{use either parencite or authorcite}{}}

% Configure captions.
\WarningFilter{caption}{Unused \captionsetup}
\captionsetup[table]{font=small}
\captionsetup[figure]{font=small}

% Shortcuts for writing.
\newcommand{\ie}{\textit{i.e.}}
\newcommand{\Ie}{\textit{I.e.}}
\newcommand{\eg}{\textit{e.g.}}
\newcommand{\Eg}{\textit{E.g.}}
\newcommand{\cf}{\textit{c.f.}}
\newcommand{\Cf}{\textit{C.f.}}

% \ifempty command:
\newcommand{\ifempty}[3]{\ifthenelse{\equal{#1}{}}{#2}{#3}}

% Style footnotes.
\setlength{\skip\footins}{\baselineskip}
\setlength{\footnotesep}{.75\baselineskip}
\renewcommand{\footnotelayout}{\setstretch{1.0}}
\setlength{\footnotemargin}{1em}

% Typewriter font:
\renewcommand*{\ttdefault}{pcr}  % Courier
\newcommand{\code}[1]{{\small\texttt{#1}}}

% Define some colours.
\definecolor{darkblue} {rgb} {0.0 , 0.0 , 0.65}
\definecolor{darkred}  {rgb} {0.80, 0.0 , 0.0 }
\definecolor{redaccent}{HTML}{E64C66}
\definecolor{darkgreen}{rgb} {0.0 , 0.50, 0.0 }
\definecolor{gray75}   {gray}{0.75}

% Define shortcuts for colours.
\newcommand{\red}[1]{{\color{red} #1}}
\newcommand{\blue}[1]{{\color{blue} #1}}
\newcommand{\green}[1]{{\color{green} #1}}
\newcommand{\orange}[1]{{\color{orange} #1}}
\newcommand{\darkred}[1]{{\color{darkred} #1}}
\newcommand{\darkblue}[1]{{\color{darkblue} #1}}
\newcommand{\darkgreen}[1]{{\color{darkgreen} #1}}
\newcommand{\magenta}[1]{{\color{magenta} #1}}
\newcommand{\grey}[1]{{\color{gray} #1}}

% Use colours to define checkmarks and crosses.
\usepackage{pifont}
\newcommand{\xmark}{\text{\ding{55}}}
\newcommand{\cmark}{\text{\ding{51}}}
\newcommand{\good}{\darkgreen{$\checkmark$}}
\newcommand{\mediumgood}{\orange{$\checkmark$}}
\newcommand{\mediumbad}{\orange{\xmark}}
\newcommand{\bad}{\darkred{\xmark}}

% Load TikZ libraries.
\usetikzlibrary{
    calc,
    positioning,
    fit,
    tikzmark,
    arrows.meta,
    shapes,
    decorations.pathreplacing,
    intersections,
    through
}
% Graphical models:
\tikzset{
    line/.style = {
        thick,
        ->,
        > = {
            Triangle[length=1.5mm, width=1.5mm]
        }
    },
    arrow/.style = {
        line
    },
    % Invisible node:
    hidden node/.style = {
        circle,
        minimum size = 1cm,
        draw = white,
        thick
    },
    % Latent variable:
    latent node/.style = {
        hidden node,
        draw = black,
    },
    % Latent variable:
    factor node/.style = {
        hidden node,
        rectangle,
        draw = black,
    },
    % Observed variable:
    observed node/.style = {
        latent node,
        fill = gray!15
    },
    % Plate:
    plate/.style = {
        draw,
        label={[anchor=north west]south west:#1},
        rounded corners=2pt,
        shape=rectangle,
        inner sep=10pt,
        thick
    }
}

% Circled number:
\newcommand{\ballnumber}[1]{%
    \tikz[baseline=(n.base)] \node[circle,fill=.,inner sep=1pt,text=white] (n) {\normalshape\bfseries\footnotesize #1};%
}
\newcommand{\itemballnumber}[1]{%
    \raisebox{.5pt}{\ballnumber{#1}}%
}

% Style enumerate and lists. Do not use `label=(\arabic*)` because equation
% numbering already uses the same style.
\setenumerate{topsep=.25\baselineskip, itemsep=0pt}
\setlist{topsep=.25\baselineskip, itemsep=0pt}
% \setlist[enumerate]{
%     label=(\arabic*),
%     itemsep=-0.25\baselineskip,
%     topsep=0.25\baselineskip,
%     after={\vspace*{0.25\baselineskip}}
% }
% \setlist[enumerate,2]{
%     topsep=-0.25\baselineskip,
%     itemsep=0pt,
%     label=(\alph*)
% }
% \setlist[enumerate,3]{
%     topsep=-0.25\baselineskip,
%     itemsep=0pt,
%     label=(\roman*)
% }
% \setlist[itemize]{
%     label=\textbullet,
%     itemsep=-0.25\baselineskip,
%     topsep=0.25\baselineskip
% }
% \setlist[itemize,2]{
%     topsep=-0.25\baselineskip,
%     itemsep=0pt
% }
% \setlist[itemize,3]{
%     topsep=-0.25\baselineskip,
%     itemsep=0pt
% }

% Set the default float placement correctly.
\floatplacement{figure}{tbp}
\floatplacement{table}{tbp}

% Set spacing between figures and text.
\setlength{\textfloatsep}{30pt plus 1.0pt minus 2.0pt}
\setlength{\floatsep}{30pt plus 1.0pt minus 2.0pt}
\setlength{\intextsep}{30pt plus 1.0pt minus 2.0pt}

% Adjust line spacing in captions.
\captionsetup{font={stretch=1.25}}

% Newline for in a title.
\newcommand{\tnl}{\texorpdfstring{\\}{}}

% Define outline tools.
\newcommand{\outline}[1]{
    \parbox{\linewidth}{
        \setstretch{1.0}
        \color{darkgreen}
        #1
    }\vspace{\parskip}
}
\newlist{outlinelist}{enumerate}{1}
\setlist[outlinelist]{
    label=\arabic*.,
    noitemsep,
    topsep=0pt,
    parsep=0pt,
    partopsep=0pt
}
\newcommand{\note}[1]{{\color{darkred}#1}}

% Left-justified text in tabularx environment:
\newcolumntype{L}{>{\RaggedRight\arraybackslash}X}

% Hyperlink setup.
\hypersetup{
    colorlinks,
    citecolor = black,
    filecolor = black,
    linkcolor = black,
    urlcolor  = black
}

% Landscape figures:
\newcommand{\landscapefloat}[1]{
    \afterpage{
        \begin{landscape}
            #1
        \end{landscape}
    }
}

% Hanging full citation:
\newcommand{\hangcite}[1]{\hangpara{\bibhang}{1}\AtNextCite{\defcounter{maxnames}{99}}\fullcite{#1}}

% Define \charfusion. From https://tex.stackexchange.com/a/52673.
\makeatletter
\def\moverlay{\mathpalette\mov@rlay}
\def\mov@rlay#1#2{\leavevmode\vtop{%
   \baselineskip\z@skip \lineskiplimit-\maxdimen
   \ialign{\hfil$\m@th#1##$\hfil\cr#2\crcr}}}
    \newcommand{\charfusion}[3][\mathord]{
        #1{\ifx#1\mathop\vphantom{#2}\fi
        \mathpalette\mov@rlay{#2\cr#3}}
        \ifx#1\mathop\expandafter\displaylimits\fi
    }
\makeatother

% Sha symbol.
%   Source: https://tex.stackexchange.com/questions/124738/i-just-want-to-write-sha-without-ruining-everything
\DeclareFontFamily{U}{wncy}{}
\DeclareFontShape{U}{wncy}{m}{n}{<->wncyr10}{}
\DeclareSymbolFont{mcy}{U}{wncy}{m}{n}
\DeclareMathSymbol{\Sha}{\mathord}{mcy}{"58}

% Define a bigger \cdot.
%   Source: https://tex.stackexchange.com/questions/235118/making-a-thicker-cdot-for-dot-product-that-is-thinner-than-bullet
\makeatletter
\newcommand*\bigcdot{\mathpalette\bigcdot@{.5}}
\newcommand*\bigcdot@[2]{\mathbin{\vcenter{\hbox{\scalebox{#2}{$\m@th#1\bullet$}}}}}
\makeatother

% Bold math symbols from text greeks:
\newcommand{\mathbfup}[1]{\mathord{\textnormal{\textbf{#1}}}}

% Bold characters:
\renewcommand{\vec}[1]{\boldsymbol{#1}}
\newcommand{\vecu}[1]{\hat{\vec{#1}}}
\newcommand{\mat}[1]{\vec{#1}}

% Predefined matrices and vectors:
\newcommand{\vnull}{\mathbf{0}}

\newcommand{\va}{\mathbf{a}}
\newcommand{\vb}{\mathbf{b}}
\newcommand{\vc}{\mathbf{c}}
\newcommand{\vd}{\mathbf{d}}
\newcommand{\ve}{\mathbf{e}}
\newcommand{\vf}{\mathbf{f}}
\newcommand{\vg}{\mathbf{g}}
\newcommand{\vh}{\mathbf{h}}
\newcommand{\vi}{\mathbf{i}}
\newcommand{\vj}{\mathbf{j}}
\newcommand{\vk}{\mathbf{k}}
\newcommand{\vl}{\mathbf{l}}
\newcommand{\vm}{\mathbf{m}}
\newcommand{\vn}{\mathbf{n}}
\newcommand{\vo}{\mathbf{o}}
\newcommand{\vp}{\mathbf{p}}
\newcommand{\vq}{\mathbf{q}}
\newcommand{\vr}{\mathbf{r}}
\newcommand{\vs}{\mathbf{s}}
\newcommand{\vt}{\mathbf{t}}
\newcommand{\vu}{\mathbf{u}}
\newcommand{\vv}{\mathbf{v}}
\newcommand{\vw}{\mathbf{w}}
\newcommand{\vx}{\mathbf{x}}
\newcommand{\vy}{\mathbf{y}}
\newcommand{\vz}{\mathbf{z}}

% Must be enclosed in another group.
\newcommand{\vmu}{{\mathbfup{\textmu}}}
\newcommand{\vtheta}{{\bm{\uptheta}}}
\newcommand{\vphi}{{\mathbfup{\textphi}}}
\newcommand{\vtau}{{\mathbfup{\texttau}}}
\newcommand{\vep}{{\mathbfup{\textepsilon}}}
\newcommand{\vth}{{\mathbfup{\texttheta}}}

\newcommand{\mA}{\mathbf{A}}
\newcommand{\mB}{\mathbf{B}}
\newcommand{\mC}{\mathbf{C}}
\newcommand{\mD}{\mathbf{D}}
\newcommand{\mE}{\mathbf{E}}
\newcommand{\mF}{\mathbf{F}}
\newcommand{\mG}{\mathbf{G}}
\newcommand{\mH}{\mathbf{H}}
\newcommand{\mI}{\mathbf{I}}
\newcommand{\mJ}{\mathbf{J}}
\newcommand{\mK}{\mathbf{K}}
\newcommand{\mL}{\mathbf{L}}
\newcommand{\mM}{\mathbf{M}}
\newcommand{\mN}{\mathbf{N}}
\newcommand{\mO}{\mathbf{O}}
\newcommand{\mP}{\mathbf{P}}
\newcommand{\mQ}{\mathbf{Q}}
\newcommand{\mR}{\mathbf{R}}
\newcommand{\mS}{\mathbf{S}}
\newcommand{\mT}{\mathbf{T}}
\newcommand{\mU}{\mathbf{U}}
\newcommand{\mV}{\mathbf{V}}
\newcommand{\mW}{\mathbf{W}}
\newcommand{\mX}{\mathbf{X}}
\newcommand{\mY}{\mathbf{Y}}
\newcommand{\mZ}{\mathbf{Z}}

\newcommand{\mSigma}{\mathbfup{\textSigma}}

% Letters commonly used in blackboard bold font:
\newcommand{\Ab}{\mathbb{A}}
\newcommand{\Bb}{\mathbb{B}}
\newcommand{\Db}{\mathbb{D}}
\newcommand{\C}{\mathbb{C}}
\newcommand{\E}{\mathbb{E}}
\newcommand{\Hb}{\mathbb{H}}
\newcommand{\Jb}{\mathbb{J}}
\newcommand{\K}{\mathbb{K}}
\newcommand{\Lb}{\mathbb{L}}
\newcommand{\Mb}{\mathbb{M}}
\newcommand{\N}{\mathbb{N}}
\renewcommand{\P}{\mathbb{P}}
\newcommand{\Q}{\mathbb{Q}}
\newcommand{\R}{\mathbb{R}}
\newcommand{\eR}{\overline{\mathbb{R}}}
\newcommand{\Sb}{\mathbb{S}}
\newcommand{\Tb}{\mathbb{T}}
\newcommand{\V}{\mathbb{V}}
\newcommand{\Z}{\mathbb{Z}}

% Letters commonly used in calligraphic font:
\newcommand{\A}{\mathcal{A}}
\newcommand{\B}{\mathcal{B}}
\newcommand{\Cc}{\mathcal{C}}
\newcommand{\D}{\mathcal{D}}
\newcommand{\Ec}{\mathcal{E}}
\newcommand{\F}{\mathcal{F}}
\newcommand{\G}{\mathcal{G}}
\let\Haccent\H
\renewcommand{\H}{\mathcal{H}}
\newcommand{\J}{\mathcal{J}}
\newcommand{\Jrat}{\mathcal{J}_{\text{rat}}}
\newcommand{\Kc}{\mathcal{K}}
\renewcommand{\L}{\mathcal{L}}
\newcommand{\Li}{\mathcal{L}{}^1}
\newcommand{\eLi}{\overline{\mathcal{L}}{}^1}
\newcommand{\m}{\text{m}}
\newcommand{\M}{\mathcal{M}}
\newcommand{\eM}{\overline{\mathcal{M}}}
\newcommand{\Nc}{\mathcal{N}}
\newcommand{\Oc}{\mathcal{O}}
\newcommand{\Pc}{\mathcal{P}}
\newcommand{\Qc}{\mathcal{Q}}
\newcommand{\Rc}{\mathcal{R}}
\renewcommand{\S}{\mathcal{S}}
\newcommand{\Tc}{\mathcal{T}}
\newcommand{\U}{\mathcal{U}}
\newcommand{\Vc}{\mathcal{V}}
\newcommand{\W}{\mathcal{W}}
\newcommand{\X}{\mathcal{X}}
\newcommand{\Y}{\mathcal{Y}}
\newcommand{\Zc}{\mathcal{Z}}

% Letters sometimes used with a wide tilde:
\newcommand{\tpi}{\widetilde{\pi}}
\newcommand{\ttau}{\widetilde{\tau}}
\newcommand{\tvtau}{\widetilde{\vtau}}
\newcommand{\tD}{\widetilde{\D}}
\newcommand{\tX}{\widetilde{\X}}
\newcommand{\tI}{\widetilde{I}}
\newcommand{\tQc}{\widetilde{\Qc}}

% Letters commonly used in sans serif font:
\newcommand{\T}{\text{\normalshape\textsf{T}}}
\newcommand{\Ht}{\text{\normalshape\textsf{H}}}
\renewcommand{\c}{\text{\normalshape\textsf{c}}}

% Symbols:
\newcommand{\es}{\varnothing}
\newcommand{\e}{\varepsilon}
\newcommand{\sub}{\subseteq}
\renewcommand{\d}{\partial}
\renewcommand{\th}{\theta}
\newcommand{\Th}{\Theta}

% Substitute l for \ell throughout.
%   Source: https://tex.stackexchange.com/questions/1975/substituting-character-l-with-ell-throughout-math-mode
\mathcode`l="8000
\begingroup
\makeatletter
\lccode`\~=`\l
\DeclareMathSymbol{\lsb@l}{\mathalpha}{letters}{`l}
\lowercase{\gdef~{\ifnum\the\mathgroup=\m@ne \ell \else \lsb@l \fi}}%
\endgroup

% Convergence symbols:
\newcommand{\oto}[1]{\overset{#1}{\to}}
\newcommand{\uto}[1]{\underset{#1}{\to}}
\newcommand{\outo}[2]{\overset{#1}{\underset{#2}{\to}}}
\newcommand{\Lto}[1]{\oto{\L^{#1}}}
\newcommand{\weakto}{\rightharpoonup}
\newcommand{\distto}{\oto{\text{d}}}
\newcommand{\weakstarto}{\overset{\ast}{\rightharpoonup}}
% Big-O notation:
\renewcommand{\O}{O}

% Equality symbols:
\newcommand{\disteq}{\overset{\text{\normalshape d}}{=}}

% Operators:
\DeclareMathOperator*{\argmax}{arg\,max}
\DeclareMathOperator*{\argmin}{arg\,min}
\let\max\relax\DeclareMathOperator*{\max}{max}
\let\min\relax\DeclareMathOperator*{\min}{min}
\let\sup\relax\DeclareMathOperator*{\sup}{sup}
\let\inf\relax\DeclareMathOperator*{\inf}{inf\vphantom{p}}
\let\limsup\relax\DeclareMathOperator*{\limsup}{lim\,sup}
\let\liminf\relax\DeclareMathOperator*{\liminf}{lim\,inf\vphantom{p}}
\DeclareMathOperator*{\esssup}{ess\,sup}
\DeclareMathOperator*{\essinf}{ess\,inf}
\newcommand{\AC}{\operatorname{AC}}
\newcommand{\atan}{\operatorname{atan}}
\newcommand{\Breg}{\operatorname{D}}
\newcommand{\BV}{\operatorname{BV}}
\newcommand{\card}{\#}
\newcommand{\cconv}{\circledast}
\newcommand{\chol}{\operatorname{chol}}
\newcommand{\ch}{\operatorname{ch}}
\newcommand{\cliques}{\operatorname{cliques}}
\newcommand{\cl}{\operatorname{cl}}
\newcommand{\col}{\operatorname{col}}
\newcommand{\comp}{\circ}
\newcommand{\contains}{\supseteq}
\newcommand{\conv}{\ast}
\newcommand{\cov}{\operatorname{cov}}
\newcommand{\var}{\operatorname{var}}
\renewcommand{\deg}{\operatorname{deg}}
\renewcommand{\det}{\operatorname{det}}
\newcommand{\diag}{\operatorname{diag}}
\renewcommand{\dim}{\operatorname{dim}}
\newcommand{\diam}{\operatorname{diam}}
\renewcommand{\div}{\operatorname{div}}
\newcommand{\dist}{\operatorname{dist}}
\newcommand{\dom}{\operatorname{dom}}
\newcommand{\dotcup}{\charfusion[\mathbin]{\cup}{\cdot}}
\newcommand{\dotunion}{\charfusion[\mathop]{\bigcup}{\cdot}}
\newcommand{\dsum}{\oplus}
\newcommand{\Dsum}{\bigoplus}
\newcommand{\erf}{\operatorname{erf}}
\newcommand{\epi}{\operatorname{epi}}
\newcommand{\had}{\odot}
\newcommand{\Hell}{\operatorname{D}_{\text{Hell}}}
\newcommand{\hypo}{\operatorname{hypo}}
\newcommand{\im}{\operatorname{im}}
\renewcommand{\Im}{\operatorname{Im}}
\newcommand{\ind}{\mathbbm{1}}
\newcommand{\intersection}{\bigcap}
\newcommand{\join}{\lor}
\newcommand{\KL}{\operatorname{KL}}
\newcommand{\KOT}{\operatorname{KOT}}
\newcommand{\kron}{\otimes}
\newcommand{\len}{\operatorname{len}}
\newcommand{\Lip}{\operatorname{Lip}}
\newcommand{\meet}{\land}
\newcommand{\MOT}{\operatorname{MOT}}
\newcommand{\nullspace}{\operatorname{null}}
\newcommand{\Per}{\operatorname{Per}}
\newcommand{\push}{_\#}
\newcommand{\poly}{\operatorname{poly}}
\newcommand{\rank}{\operatorname{rank}}
\renewcommand{\Re}{\operatorname{Re}}
\newcommand{\row}{\operatorname{row}}
\newcommand{\schur}{\operatorname{Schur}}
\newcommand{\sign}{\operatorname{sign}}
\newcommand{\sinth}{\operatorname{\sin\!\text{-}\Theta}}
\newcommand{\Sinth}{\operatorname{\text{Sin-}\Theta}}
\newcommand{\supp}{\operatorname{supp}}
\newcommand{\symdiff}{\bigtriangleup}
\newcommand{\Tan}{\operatorname{Tan}}
\newcommand{\tensor}{\otimes}
\newcommand{\TV}{\operatorname{TV}}
\newcommand{\tr}{\operatorname{tr}}
\newcommand{\union}{\bigcup}
\newcommand{\vecspan}{\operatorname{span}}
\newcommand{\vect}{\operatorname{vec}}
\newcommand{\vol}{\operatorname{vol}}

% Probability distributions:
\newcommand{\Ber}{\operatorname{Ber}}
\newcommand{\Beta}{\operatorname{Beta}}
\newcommand{\Bin}{\operatorname{Bin}}
\newcommand{\BM}{\operatorname{BM}}
\newcommand{\Cat}{\operatorname{Cat}}
\newcommand{\Cauchy}{\operatorname{Cauchy}}
\newcommand{\Dir}{\operatorname{Dir}}
\newcommand{\Exp}{\operatorname{Exp}}
\newcommand{\Gam}{\operatorname{Gamma}}
\newcommand{\Geom}{\operatorname{Geom}}
\newcommand{\GP}{\mathcal{GP}}
\newcommand{\InvWishart}{\mathcal{W}^{-1}}
\newcommand{\Laplace}{\operatorname{Laplace}}
\newcommand{\LogNormal}{\log\mathcal{N}}
\newcommand{\Mult}{\operatorname{Mult}}
\newcommand{\NegBin}{\operatorname{NegBin}}
\newcommand{\Normal}{\mathcal{N}}
\newcommand{\Poisson}{\operatorname{Poisson}}
\newcommand{\Rad}{\operatorname{Rad}}
\newcommand{\Unif}{\operatorname{Unif}}
\newcommand{\Wishart}{\mathcal{W}}
\newcommand{\WN}{\operatorname{WN}}

\newcommand{\simiid}{\overset{\text{i.i.d.}}{\sim}}

% Probability commands:
\newcommand{\Var}{\V}
\newcommand{\Lik}{\L}
\newcommand{\iidsim}{\overset{\text{\tiny{i.i.d.}}}{\sim}}

% Special math commands:
\newcommand{\cond}{\,|\,}                  % Conditioning
\newcommand{\middlecond}{\,\middle|\,}     % Conditioning between a \left and \right
\newcommand{\divsep}{\,\|\,}               % Separator in divergences
\newcommand{\middledivsep}{\,\middle\|\,}  % Separator in divergences ebtween a \left and a \right

\newcommand{\sd}[1]{\mathrm{d} #1}         % Straight 'd'
\newcommand{\isd}[1]{\, \mathrm{d} #1}     % Straight 'd' in integral (with spacing)
% Straight 'd' in integral with spacing and absolute value around it
\newcommand{\asd}[1]{\, |\mathrm{d} #1|}

\newcommand{\idf}{\text{\textsf{id}}}            % Identity function
\newcommand{\sce}{\text{\sc{e}}}                 % Scientific notation
\newcommand{\vardot}{\,\bigcdot\,}               % Variable dot
\let\ssorig\ss
\renewcommand{\ss}[1]{_{\text{\normalshape #1}}} % Text subscripts
\newcommand{\us}[1]{^{\text{\normalshape #1}}}   % Text superscripts

% Hard to type words:
\newcommand{\cadlag}{c\`adl\`ag}
\newcommand{\Cadlag}{C\`adl\`ag}
\newcommand{\ito}{It\^o}
\newcommand{\Ito}{\ito}
\newcommand{\levy}{L\`evy}
\newcommand{\Levy}{\levy}

% Subscripts
\newcommand{\loc}{_{\text{loc}}}
\newcommand{\lip}{_{\text{Lip}}}

% Half rectangle:
\newcommand{\halfrect}[2]{[\![#1,#2)\hspace{-1pt}\!)}
\newcommand{\Rint}{\text{(R)}\int}

% Paired delimiters:
\DeclarePairedDelimiter\parens{(}{)}             % Parentheses
\DeclarePairedDelimiter\sbrac{[}{]}              % Square brackets
\DeclarePairedDelimiter\cbrac{\{}{\}}            % Curly braces
\DeclarePairedDelimiter\set{\{}{\}}              % Set
\DeclarePairedDelimiter\lra{\langle}{\rangle}    % Angle brackets
\DeclarePairedDelimiter\dlra
    {\langle\!\langle}{\rangle\!\rangle}         % Double angle brackets
\DeclarePairedDelimiter\floor{\lfloor}{\rfloor}  % Floor
\DeclarePairedDelimiter\ceil{\lceil}{\rceil}     % Ceil
\DeclarePairedDelimiter\norm{\|}{\|}             % Norm
\DeclarePairedDelimiter\abs{|}{|}                % Absolute value

% Shortcuts for angle brackets:
\newcommand{\la}{\langle}
\newcommand{\ra}{\rangle}
\newcommand{\lla}{\left\langle}
\newcommand{\rra}{\right\rangle}

% Other commands:
\newcommand{\resp}[1]{[#1]}                  % Respective statement
\newcommand{\bs}{\textbackslash}             % Blackslash in text
% Use \cref in a title.
\newcommand{\creftitle}[2]{\texorpdfstring{\cref{#2}}{#1 \ref{#2}}}
\newcommand{\mcheckthis}{{}^{[\checkmark]}}  % Checkmark in math
\newcommand{\checkthis}{$\mcheckthis$}       % Checkmark in text

% Compatibility with old commands:
\newcommand{\id}[1]{\sd{#1}}
\newcommand{\sargmax}[1]{\argmin_{#1}}
\newcommand{\sargmin}[1]{\argmax_{#1}}
\newcommand{\lrset}[1]{\set*{#1}}
\newcommand{\middleCond}{\middlecond}
\newcommand{\rel}[2]{($(#1) + (#2)$)}
\newcommand{\blackLinks}{}
\newcommand{\Ac}{\A}

\newcommand{\notheorems}{}  % We'll configure our own numbering below.
% Patch \listoftheorems.
%   Source: https://tex.stackexchange.com/questions/249963/remove-repeated-theorem-in-the-list-of-theorems
\makeatletter
\patchcmd\thmt@mklistcmd
    {\thmt@thmname}
    {\check@optarg{\thmt@thmname}}
    {}{}
\patchcmd\thmt@mklistcmd
    {\thmt@thmname\ifx}
    {\check@optarg{\thmt@thmname}\ifx}
    {}{}
\protected\def\check@optarg#1{%
    \@ifnextchar\thmtformatoptarg\@secondoftwo{#1}%
}
\makeatother

% Define lists of things commands.
\let\oldlistoftheorems\listoftheorems
\newcommand{\listofmodels}{
    \renewcommand{\listtheoremname}{List of Models}
    \oldlistoftheorems[ignoreall, show={model}]
}
\newcommand{\listofstatements}{
    \renewcommand{\listtheoremname}{List of Mathematical Statements}
    \oldlistoftheorems[ignoreall, show={theorem,corollary,proposition,lemma,fact}]
}
\renewcommand{\listoftheorems}{
    \renewcommand{\listtheoremname}{List of Theorems}
    \oldlistoftheorems[ignoreall, show={theorem}]
}

% Define environments.
\newlength{\thmtopsep}\setlength{\thmtopsep}{\topsep}
\newlength{\thmbotsep}\setlength{\thmbotsep}{\topsep}
\newtheoremstyle{theoremstyle}
    {\thmtopsep}{\thmbotsep}
    {}           % Body font
    {}           % Indent amount
    {\bfseries}  % Theorem head font
    {.}          % Punctuation after theorem head
    {.5em}       % Space after theorem head
    {}           % Theorem head spec
\theoremstyle{theoremstyle}

\ifcsname notheorems\endcsname
\else
    \newtheorem{theorem}{Theorem}[section]
    \newtheorem{proposition}{Proposition}[section]
    \newtheorem{corollary}{Corollary}[section]
    \newtheorem{fact}{Fact}[section]
    \newtheorem{lemma}{Lemma}[section]

    \newtheorem{assumption}{Assumption}[section]
    \newtheorem{definition}{Definition}[section]
    \newtheorem{question}{Question}[section]
    \newtheorem{example}{Example}[section]
    \newtheorem{model}{Model}[section]
    \newtheorem{remark}{Remark}[section]
\fi

% Set referencing formats.
\crefname{assumption}{Assumption}{Assumptions}
\Crefname{assumption}{Assumption}{Assumptions}
\crefname{corollary}{Corollary}{Corollaries}
\Crefname{corollary}{Corollary}{Corollaries}
\crefname{definition}{Definition}{Definitions}
\Crefname{definition}{Definition}{Definitions}
\crefname{example}{Example}{Examples}
\Crefname{example}{Example}{Examples}
\crefname{fact}{Fact}{Facts}
\Crefname{fact}{Fact}{Facts}
\crefname{lemma}{Lemma}{Lemmas}
\Crefname{lemma}{Lemma}{Lemmas}
\crefname{model}{Model}{Models}
\Crefname{model}{Model}{Models}
\crefname{proposition}{Proposition}{Propositions}
\Crefname{proposition}{Proposition}{Propositions}
\crefname{question}{Question}{Questions}
\Crefname{question}{Question}{Questions}
\crefname{remark}{Remark}{Remarks}
\Crefname{remark}{Remark}{Remarks}
\crefname{theorem}{Theorem}{Theorems}
\Crefname{theorem}{Theorem}{Theorems}

% Referentiable list items in environments
\newlist{asslist}{enumerate}{1}
\setlist[asslist]{
    ref=\theassumption.(\arabic*),
    label=(\arabic*),
    % topsep=0pt,
}
\crefname{asslisti}{Assumption}{Assumptions}
\Crefname{asslisti}{Assumption}{Assumptions}
\newlist{corlist}{enumerate}{1}
\setlist[corlist]{
    ref=\thecorollary.(\arabic*),
    label=(\arabic*),
    % topsep=0pt,
}
\crefname{corlisti}{Corollary}{Corollaries}
\Crefname{corlisti}{Corollary}{Corollaries}
\newlist{deflist}{enumerate}{1}
\setlist[deflist]{
    ref=\thedefinition.(\arabic*),
    label=(\arabic*),
    % topsep=0pt,
}
\crefname{deflisti}{Definition}{Definitions}
\Crefname{deflisti}{Definition}{Definitions}
\newlist{exlist}{enumerate}{1}
\setlist[exlist]{
    ref=\theexample.(\arabic*),
    label=(\arabic*),
    % topsep=0pt,
}
\crefname{exlisti}{Example}{Examples}
\Crefname{exlisti}{Example}{Examples}
\newlist{factlist}{enumerate}{1}
\setlist[factlist]{
    ref=\thefact.(\arabic*),
    label=(\arabic*),
    % topsep=0pt,
}
\crefname{factlisti}{Fact}{Facts}
\Crefname{factlisti}{Fact}{Facts}
\newlist{lemlist}{enumerate}{1}
\setlist[lemlist]{
    ref=\thelemma.(\arabic*),
    label=(\arabic*),
    % topsep=0pt,
}
\crefname{lemlisti}{Lemma}{Lemmas}
\Crefname{lemlisti}{Lemma}{Lemmas}
\newlist{modlist}{enumerate}{1}
\setlist[modlist]{
    ref=\themodel.(\arabic*),
    label=(\arabic*),
    % topsep=0pt,
}
\crefname{modlisti}{Model}{Models}
\Crefname{modlisti}{Model}{Models}
\newlist{proplist}{enumerate}{1}
\setlist[proplist]{
    ref=\theproposition.(\arabic*),
    label=(\arabic*),
    % topsep=0pt,
}
\crefname{proplisti}{Proposition}{Propositions}
\Crefname{proplisti}{Proposition}{Propositions}
\newlist{qlist}{enumerate}{1}
\setlist[qlist]{
    ref=\theremark.(\arabic*),
    label=(\arabic*),
    % topsep=0pt,
}
\crefname{qlisti}{Question}{Questions}
\Crefname{qlisti}{Question}{Questions}
\newlist{remlist}{enumerate}{1}
\setlist[remlist]{
    ref=\theremark.(\arabic*),
    label=(\arabic*),
    % topsep=0pt,
}
\crefname{remlisti}{Remark}{Remarks}
\Crefname{remlisti}{Remark}{Remarks}
\newlist{thmlist}{enumerate}{1}
\setlist[thmlist]{
    ref=\thetheorem.(\arabic*),
    label=(\arabic*),
    % topsep=0pt,
}
\crefname{thmlisti}{Theorem}{Theorems}
\Crefname{thmlisti}{Theorem}{Theorems}

% Reference numbers in the list.
\newcommand{\listnum}[1]{(#1)}
\newcommand{\listimp}[2]{\listnum{#1} $\Rightarrow$ \listnum{#2}}
\newcommand{\listeq}[2]{\listnum{#1} $\Leftrightarrow$ \listnum{#2}}

% Backward compatibility:
\newcommand{\listimpl}[2]{\listnum{#1} $\Rightarrow$ \listnum{#2}:}


% We actually will use coloured links.
\definecolor{aquamarine}{HTML}{218274}
\hypersetup{
    colorlinks,
    citecolor = aquamarine,
    filecolor = aquamarine,
    linkcolor = aquamarine,
    urlcolor  = aquamarine,
}

% Use italic font in bodies for clarity.
\newtheoremstyle{theoremstyle}
    {\thmtopsep}{\thmbotsep}
    {\itshape}   % Body font
    {}           % Indent amount
    {\bfseries}  % Theorem head font
    {.}          % Punctuation after theorem head
    {.5em}       % Space after theorem head
    {}           % Theorem head spec
\theoremstyle{theoremstyle}

% Set numbering of theorems right.
\newtheorem{theorem}{Theorem}[chapter]
\newtheorem{proposition}[theorem]{Proposition}
\newtheorem{corollary}[theorem]{Corollary}
\newtheorem{fact}[theorem]{Fact}
\newtheorem{lemma}[theorem]{Lemma}

\newtheorem{assumption}[theorem]{Assumption}
\newtheorem{definition}[theorem]{Definition}
\newtheorem{procedure}[theorem]{Procedure}

\newtheorem{question}[theorem]{Question}
\newtheorem{example}[theorem]{Example}
\newtheorem{model}[theorem]{Model}
\newtheorem{remark}[theorem]{Remark}

% Hack `\listoftheorem` to remove the title.
\makeatletter
\renewcommand\listoftheorems[1][]{%
    \begingroup
    \setlisttheoremstyle{#1}%
    \let\listfigurename\listtheoremname
    \def\contentsline##1{%
        \csname thmt@contentsline@##1\endcsname{##1}%
    }%
    \@for\thmt@envname:=\thmt@allenvs\do{%
        \thmtlo@newentry
    }%
    \let\thref@starttoc\@starttoc
    \def\@starttoc##1{\thref@starttoc{loe}}%
    \@fileswfalse
    \AtEndDocument{%
        \if@filesw
        \@ifundefined{tf@loe}{%
            \expandafter\newwrite\csname tf@loe\endcsname
            \immediate\openout \csname tf@loe\endcsname \jobname.loe\relax
        }{}%
        \fi
    }%
    \@starttoc{lof}
    \endgroup
}
\makeatother

% Fix spacing around proofs.
\xpatchcmd{\proof}{\topsep6\p@\@plus6\p@\relax}{}{}{}
\BeforeBeginEnvironment{proof}{\vspace{-0.5em}}
\AfterEndEnvironment{proof}{\vspace{-0.5em}}

% Commands specific for thesis:
\newcommand{\AR}{\operatorname{AR}}
\newcommand{\enc}{\mathsf{enc}}
\newcommand{\dec}{\mathsf{dec}}
\newcommand{\mult}{\operatorname{mult}}

\newcommand{\missingcite}{\note{(CITATION MISSING)}}

% Listings:
\usepackage{tcolorbox}
\tcbuselibrary{minted,skins,breakable}

\definecolor{solarized-light-bg}{HTML}{fdf6e3}
\definecolor{solarized-light-fg}{HTML}{586e75}
\newtcblisting{pythoncode}[2]{
    listing engine = minted,
    listing only,
    minted style = solarized-light,
    minted language = python,
    minted options = {
        fontsize = #1,
        escapeinside = ||,
        mathescape = true,
        highlightlines = #2,
        highlightcolor = red,
    },
    colback = solarized-light-bg,
    colframe = solarized-light-bg,
    toprule = 0pt,
    left = 5pt,
    left = 5pt,
    leftrule = 0pt,
    rightrule = 0pt,
    bottomrule = 0pt,
    arc = 0pt,
    frame hidden,
    breakable,
}
% Disable italics.
\AtBeginEnvironment{pythoncode}{\let\itshape\relax}

% Override the textwriter font.
%\usepackage[scaled=0.8]{beramono}
\usepackage[scaled=0.95]{inconsolata}
\renewcommand{\code}[1]{\colorbox{solarized-light-bg}{\small\color{solarized-light-fg}\texttt{#1}}}

% We never want footnotes to break across pages.
\interfootnotelinepenalty=10000

% Allow restatable environments.
\newenvironment{manual}[3][]{%
    \def\savedarg{#2}%
    \expandafter\renewcommand\csname the#2\endcsname{#3}%
    \ifempty{#1}{\csname #2\endcsname}{\csname #2\endcsname[#1]}%
}{%
    \csname end\savedarg\endcsname%
    \addtocounter{\savedarg}{-1}%
}
\newcommand{\statement}[1]{
    \begingroup
        \subimport{}{#1}
        \ifempty{\statementoption}{
            \csname\statementtype\endcsname
        }{
            \expandafter\csname\statementtype\endcsname[\statementoption]%
        }
        \label{\statementlabel}
        \statementcontent
        \csname end\statementtype\endcsname
    \endgroup
}
\newcommand{\restatement}[1]{
    \begingroup
        \subimport{}{#1}
        \ifempty{\statementoption}{
            \manual{\statementtype}{\ref{\xrprefix{\statementlabel}}}
        }{
            \manual[\statementoption]{\statementtype}{\ref{\xrprefix{\statementlabel}}}
        }
        \newcommand{\insiderestatement}{}
        \renewcommand{\label}[1]{}
        \statementcontent
        \endmanual
    \endgroup
}

%\newcommand{\ballnumber}[1]{\tikz[baseline=(myanchor.base)] \node[circle,fill=.,inner sep=1pt] (myanchor) {\color{-.}\normalshape\bfseries\footnotesize #1};}

\newcommand{\highlight}{\textcolor{}{}}

\addbibresource{../../bibliography.bib}

\usepackage{xr}
\externaldocument[xr-]{../../main}
\newcommand{\xrprefix}[1]{xr-#1}

\begin{document}

\chapter
    {Proofs for Chapter \ref{\xrprefix{chap:predmap}}}
\label{app:proofs_predmap}

\section{Proofs for Section \ref{\xrprefix{sec:predmap:motivation}}}
\label{sec:proofs_predmap:motivation}

%Before proving \cref{\xrprefix{prop:posterior_prediction_map_regularity:part_one}}, we require a lemma about the normalising constant $Z$, which was defined in \eqref{\xrprefix{def:posterior_prediction_map}}.

Recall the definition of the posterior prediction map in \cref{\xrprefix{def:posterior_prediction_map_formal}}.
Henceforth, for all $D \in \D$, denote the Radon--Nikodym derivative of $\pi_f(D)$ with respect to $\mu_f$ by $\pi'(D)$:
\begin{equation}
    \pi'(D) = \frac
        {r(\vy - f(\vx))}
        {Z(\vx, \vy)}
\end{equation}
where $D = (\vx, \vy)$,
$r(\vx) = \exp({-\tfrac{1}{2\sigma_f^2}\norm{\vx}_2^2})$,
and $Z(\vx, \vy) = \E_f[r(\vy - f(\vx))]$.

\begin{lemma} \label{lem:inf_sup_Z}
    Let $(D_i)_{i \ge 1} \sub \D$ be a bounded sequence and denote $D_i = (\vx_i, \vy_i)$.
    Then
    \begin{equation}
        0
        < \inf_{i \ge 1}\, Z(\vx_i, \vy_i)
        \le \sup_{i \ge 1}\, Z(\vx_i, \vy_i)
        < \infty.
    \end{equation}
\end{lemma}
\begin{proof}
    That the supremum is finite follows from that $r(\vx) \le 1$.
    For the infimum,
    let $M > 0$, and consider the event $A = \set{f : \norm{f(\vx_i)}_\infty \le M}$.
    Then
    \begin{equation}
        \E_f[e^{-\tfrac1{2\sigma_f^2}\norm{\vy_i - f(\vx_i)}_2^2}]
        \ge \E_f[\ind_A e^{-\tfrac1{\sigma_f^2}(\norm{\vy_i}_2^2 + \norm{f(\vx_i)}_2^2)}]
        \ge \P(A)\, e^{-\tfrac1{\sigma_f^2}\norm{\vy_i}_2^2 - \tfrac{1}{\sigma^2} N M^2}.
    \end{equation}
    To lower bound the probability, denote $V = \sup_{x \in \X}\, \norm{f(x)}^2_{L^2}$.
    Note that $V < \infty$ by \cref{\xrprefix{assum:f_regularity}}.
    Estimate
    \begin{equation}
        \P(A)
        = 1 - \P(A^\c) \vphantom{\sup_{x \in \X}}
        \ge 1 - \vphantom{\frac NN} N\,\sup_{x \in \X}\, \P(\abs{f(x)} \ge M)
        \ge 1 - \frac{N V}{M^2}.
    \end{equation}
    Set $M = \sqrt{2 N V}$ to find $\P(A) \ge \tfrac12$.
    Hence
    \begin{equation} \label{lem:inf_sup_Z:lower_bound}
        \E_f[e^{-\tfrac1{2\sigma_f^2}\norm{\vy_i - f(\vx_i)}_2^2}]
        \ge \tfrac12 e^{-\tfrac1{\sigma_f^2}\norm{\vy_i}_2^2 - \tfrac{2}{\sigma_f^2} VN^2}
    \end{equation}
    We conclude that the infimum is non-zero: $(\norm{\vy_i}^2_2)_{i \ge 1}$ is bounded, because $(D_i)_{i \ge 1}$ is bounded.
\end{proof}

\restatement{../../chapters/predmap/statements/regularity_part_one.tex}
\begin{proof}
    \listnum{1}:
    Let $D \in \D$ and denote $D = (\vx, \vy)$.
    Then use that
    \begin{equation}
        \norm{\vardot}_{L^p(\pi_f(D))}
        = \E_f[\pi'(D)(f)(\vardot)^p]^{1/p}
        \le \frac{1}{(Z(\vx, \vy))^{1/p}} \norm{\vardot}_{L^p},
    \end{equation}
    and we already noted that $Z(\vx, \vy) > 0$.

    \listnum{2}:
    Let $D_i \to D$ and let $L\colon C_b(\X, \Y) \to \R$ be a continuous and bounded function.
    To begin with, $(\vx, \vy) \mapsto Z(\vx, \vy)$ is continuous by pathwise continuity of $f$ in combination with bounded convergence, and $(\vx, \vy) \mapsto r(\vy - f(\vx))$ is almost surely continuous also by pathwise continuity of $f$.
    Hence, for almost every $f$, $(D, f) \mapsto \pi'(D)(f)$ is continuous in $D$.
    In addition, $(D_i)_{i \ge 1}$ is bounded, so
    $\sup_{i \ge 1,\, f \in \Y^\X} \pi'(D_i)(f) < \infty$
    by \cref{lem:inf_sup_Z}.
    Therefore, by bounded convergence,
    \begin{align}
        \lim_{i \to \infty} \E_{\pi_f(D_i)}[L(f)]
        &= \lim_{i \to \infty} \E_{f}[\pi'(D_i)(f)L(f)] \\
        &= \vphantom{\lim_{i \to \infty}}\E_{f}[ \pi'(D)(f)L(f)] \\
        &= \E_{\pi_f(D)}[L(f)].
    \end{align}
    Since $L$ was arbitrary, we conclude that $\pi_f(D_i) \weakto \pi_f(D)$.
\end{proof}


\section{Proofs for Section \ref{\xrprefix{sec:predmap:neural_process_objective}}}
\label{sec:proofs_predmap:neural_process_objective}


\restatement{../../chapters/predmap/statements/neural_process_objective_well-defined.tex}
\begin{proof}
    Let $(\pi, \sigma) \in \overline{\M}$ and
    $D \in \D$.
    We have $\pi_f(D) \in \Pc$, so there exists a measurable $A_1 \sub \tI$ such that $\vx \mapsto P_\vx^{\sigma_f} \pi_f(D)$ is weakly continuous at all $\vx \in A_1$ and $\int_{A_1} p(\vx) \isd \vx = 1$.
    Similarly, $\pi(D) \in \Pc$, so there exists another measurable $A_2 \sub \tI$ such that $\vx \mapsto P_\vx^{\sigma} \pi(D)$ is weakly continuous at all $\vx \in A_2$ and $\int_{A_2} p(\vx) \isd \vx = 1$.
    Set $A = A_1 \cap A_2$.
    Then still $\int_A p(\vx) \isd \vx = 1$.
    Denote $h\colon \tI \to \R \cup \set{\infty}$, $h(\vx) =\KL(P_\vx^{\sigma_f} \pi_f(D), P_\vx^{\sigma} \pi(D))$.
    Let $\tau$ be all open sets of $\tI$.
    By weak lower semi-continuity of the Kullback--Leibler divergence \parencite{Posner:1975:Random_Coding_Strategies_for_Minimum}, the restriction $h|_A$ of $h$ to $A$ is a lower semi-continuous function with respect to the subspace topology $(A, A \cap \tau)$.
    In particular, $h|_A$ is measurable with respect to the Borel $\sigma$-algebra on $(A, A \cap \tau)$, which is equal to $\sigma(A \cap \tau) = A \cap \sigma(\tau) = A \cap \B(\tI)$.
    Since $A$ has probability one, this suggests that $h$ is also measurable, which we now argue.
    Let $y \in \R$.
    Decompose
    \begin{equation}
        \set{\vx \in \tI : h(\vx) \le y}
        =
            \set{\vx \in A : h(\vx) \le y}
            \cup \set{\vx \in \tI \setminus A : h(\vx) \le y}.
    \end{equation}
    By $A \cap \B(I)$--measurability of $h|_A$, $\set{\vx \in A : h(\vx) \le y} \in A \cap \B(\tI) \sub \B(\tI)$.
    In addition, $\set{\vx \in \tI \setminus A : h(\vx) \le y} \sub \tI \setminus A$, so it is measurable by completeness of $p(\vx)$ (\cref{\xrprefix{assum:full_support_and_complete}}).
    Therefore, $\set{\vx \in \tI : h(\vx) \le y}$ is measurable.
    Since $y \in \R$ was arbitary, $h$ is measurable.
    We conclude that the expectation $\E_{p(\vx)}[\KL(P_\vx^{\sigma_f} \pi_f(D), P_\vx^{\sigma} \pi(D))]$ is well defined.

    By \cref{\xrprefix{prop:posterior_prediction_map_is_continuous}}, the posterior prediction map $\pi_f$ is continuous.
    Note that $\pi$ is also continuous.
    Then, for any $\vx \in I$ and $\sigma > 0$, $D \mapsto (P_\vx^{\sigma_f} \pi_f(D), P_\vx^{\sigma} \pi(D))$ is weakly continuous, so $D \mapsto \KL(P_\vx^{\sigma_f} \pi_f(D), P_\vx^{\sigma} \pi(D))$ is lower semi-continuous by weak lower semi-continuity of the Kullback--Leibler divergence \parencite{Posner:1975:Random_Coding_Strategies_for_Minimum}.
    By Fatou's lemma, $D \mapsto \E_{p(\vx)}[\KL(P_\vx^{\sigma_f} \pi_f(D), P_\vx^{\sigma} \pi(D))]$ is then also lower semi-continuous.
    In particular, it is measurable, which means that $\E_{p(D)p(\vx)}[\KL(P_\vx^{\sigma_f} \pi_f(D), P_\vx^{\sigma} \pi(D))]$
    is well defined.
\end{proof}

\restatement{../../chapters/predmap/statements/neural_process_objective_lsc.tex}
\begin{proof}
    Consider $(\pi, \sigma) \mapsto P^\sigma_\vx \pi(D)$.
    This function is weakly continuous;
    recall that the topology on $\overline{\M}$ is defined in
    \cref{\xrprefix{def:noisy_prediction_maps}}.
    The result then follows from Fatou's lemma in combination with weak lower semi-continuity of the Kullback--Leibler divergence \parencite{Posner:1975:Random_Coding_Strategies_for_Minimum}.
\end{proof}

\begin{lemma} \label{lem:equality_dense}
    Consider $\mu_1, \mu_2 \in \Pc\ss{c}$.
    Let $I' \sub I$ be dense.
    Then $P^{\sigma_1}_\vx \mu_1 = P^{\sigma_2}_\vx \mu_2$ for all $\vx \in I'$ if and only if $\mu_1 = \mu_2$ and $\sigma_1 = \sigma_2$.
\end{lemma}
\begin{proof}
    The ``if'' is clear, so we show the ``only if''.
    Let $f^{(1)} \sim \mu_1$ and $f^{(2)} \sim \mu_2$.
    Fix some $\vx \in I$, and let $(\vx^{(\ell)}_1, \ldots, \vx^{(\ell)}_\ell)_{\ell \ge 1} \sub I$ be a sequence of tuples of vectors such that
    (1) every concatenation $\vx^{(\ell)}_1 \oplus \cdots \oplus \vx^{(\ell)}_\ell$ is in $I'$; and
    (2) as $m \to \infty$, it holds that $\sup\, \abs{\vx - \vx^{(\ell)}_k} \to 0$, where the supremum is over $\ell \ge m$ and $k = m, \ldots, \ell$.
    To construct such a sequence, take any sequence $(\vz_{\ell})_{\ell \ge 1} \sub I$ convergent to $\vx$ and use density of $I'$ to find, for every $\ell \ge 1$, an element $\vx^{(\ell)}_1 \oplus \cdots \oplus \vx^{(\ell)}_\ell \in I'$ which is at most distance $1/\ell$ from $\vz_1 \oplus \cdots \oplus \vz_\ell$.

    For all $k \ge 1$,
        let $\vep\us{(1)}_k \sim \Normal(\vnull, \sigma_1^2 \mI)$
        and $\vep\us{(2)}_k \sim \Normal(\vnull, \sigma_2^2 \mI)$.
    Set
    \begin{equation}
        \vy^{(1)}_\ell = \frac1\ell\sum_{k=1}^\ell f\us{(1)}(\vx^{(\ell)}_k) + \frac1\ell\sum^\ell_{k=1}\vep\us{(1)}_k,
        \quad
        \vy^{(2)}_\ell = \frac1\ell\sum_{k=1}^\ell f\us{(2)}(\vx^{(\ell)}_k) + \frac1\ell\sum^\ell_{k=1}\vep\us{(2)}_k.
    \end{equation}
    By almost sure uniform continuity of $f\us{(1)}$ and $f\us{(2)}$ ($\X$ is compact) and the strong law of large numbers,%
    \begin{equation}
        \vy_\ell\us{(1)} \to f\us{(1)}(\vx)
        \quad \text{and} \quad
        \vy_\ell\us{(2)} \to f\us{(2)}(\vx)
    \end{equation}
    almost surely and therefore weakly.
    But, by assumption, 
    \begin{equation}
        P^{\sigma_1}_{\vx^{(\ell)}_1 \oplus \ldots \oplus \vx_\ell^{(\ell)}} \mu_1
         = P^{\sigma_2}_{\vx^{(\ell)}_1 \oplus \ldots \oplus \vx_\ell^{(\ell)}} \mu_2
         \quad\text{for all $\ell \ge 1$},
    \end{equation}
    so $\vy\us{(1)}_\ell \disteq \vy\us{(2)}_\ell$ for all $\ell \ge 1$.
    Therefore, since weak limits are unique, $f\us{(1)}(\vx) \disteq f\us{(2)}(\vx)$.
    Since $\vx \in I$ was arbitrary, we conclude that $f\us{(1)} \disteq f\us{(2)}$.

    To find that also $\sigma_1 = \sigma_2$, repeat the above construction with $\vx$ instead fixed to some $\tilde\vx \in I'$.
    Denote $\tilde\vy\us{(1)} = f\us{(1)}(\tilde\vx) + \tilde\vep\us{(1)}$
    and $\tilde\vy\us{(2)} = f\us{(2)}(\tilde\vx) + \tilde\vep\us{(2)}$.
    Then
    \begin{equation}
        (\vy_\ell\us{(1)}, {\tilde\vy}\us{(1)})
        \to (f\us{(1)}(\tilde\vx), {\tilde\vy}\us{(1)})
        \quad\text{and}\quad
        (\vy_\ell\us{(2)}, \tilde\vy\us{(2)})
        \to (f\us{(2)}(\tilde\vx), {\tilde\vy}\us{(2)})
    \end{equation}
    weakly
    and $(\vy_\ell\us{(1)}, \tilde\vy\us{(1)}) \disteq (\vy_\ell\us{(2)}, \tilde\vy\us{(2)})$ for all $\ell \ge 1$,
    so $(f\us{(1)}(\tilde\vx), \tilde\vy\us{(1)}) \disteq (f\us{(2)}(\tilde\vx), \tilde\vy\us{(2)})$.
    Finally apply $(\vf, \vy) \mapsto \vy - \vf$ to find that $\tilde\vep\us{(1)} \disteq \tilde\vep\us{(2)}$, so $\sigma_1 = \sigma_2$.
\end{proof}

\restatement{../../chapters/predmap/statements/neural_process_objective_minimiser.tex}
\begin{proof}
    It is clear that $\L\ss{NP}(\pi_f, \sigma_f) = 0$.
    For the converse, assume that $\L\ss{NP}(\pi, \sigma) = 0$ for some $(\pi, \sigma) \in \overline{M}\ss{c}$.
    Then there exists a measurable $A \sub \tD$ such that $\int_A p(D) \isd D = 1$ and 
    \begin{equation}
        \E_{p(\vx)}[\KL(P^{\sigma_f}_\vx \pi_f(D), P^{\sigma}_\vx \pi(D)] = 0
        \quad\text{for all $D \in A$}.
    \end{equation}
    Fix such a $D \in A$.
    Then there exists a measurable $B \sub \tI$ such that $\int_{B} p(\vx) \isd \vx = 1$ and $P^{\sigma_f}_\vx \pi_f(D)= P^\sigma_\vx \pi(D)$ for all $\vx \in B$.
    But $B$ is dense in $\tI$, because $p(\vx)$ assigns positive probability to every open set (\cref{\xrprefix{assum:full_support_and_complete}}), which means that $B$ is also dense in $I$.
    Therefore, by \cref{lem:equality_dense}, $\pi_f(D) = \pi(D)$ and $\sigma_f = \sigma$.
    Similarly, $A$ is dense in $\tD$, because $p(D)$ assigns positive probability to every open set.
    Since $\pi_f$ and $\pi$ are two continuous functions that agree on a dense subset of $\tD$, $\pi_f$ and $\pi$ must agree on all of $\tD$.
\end{proof}

\section{Proofs for Section \ref{\xrprefix{sec:predmap:np_approximations}}}
\label{sec:proofs_predmap:np_approximations}


\restatement{../../chapters/predmap/statements/gnpa_characterisation.tex}
\begin{proof}
    Let $\Pc^N_\lambda$ be the collection of distributions on $\R^N$
    that (a) have a density with respect to the Lebesgue measure and (b) have a covariance matrix which is strictly positive definite.
    Let $\Pc\ss{$\lambda$,G}^N \sub \Pc_\lambda^N$ be subcollection of distributions which are Gaussian.
    Consider $\mu \in \Pc_\lambda^N$ and $\nu \in \Pc\ss{$\lambda$,G}^N$ such that
    $\inf_{\nu\ss{G} \in \Pc\ss{$\lambda$,G}^N} \KL(\mu, \nu\ss{G}) < \infty$.
    Define
    \begin{equation}
        \text{G}(\mu, \nu) =% \begin{cases}
            \KL(\mu, \nu) - \inf_{\nu\ss{G} \in \Pc\ss{$\lambda$,G}^N} \KL(\mu, \nu\ss{G})
        %     &\text{if } \inf_{\nu\ss{G} \in \Pc\ss{$\lambda$,G}^N} \KL(\mu, \nu\ss{G}) < \infty, \\
        %     \infty &\text{otherwise}.
        % \end{cases}
    \end{equation}
    Note that $\text{G}(\mu, \nu) \ge 0$, because $\nu \in \Pc\ss{$\lambda$,G}^N$.
    Moreover, a computation shows that $\text{G}(\mu, \nu) = \KL(\Normal(\mu), \nu)$
    where we denote $\Normal(\mu) = \Normal(\vmu, \mSigma)$ with $\vmu$ the mean vector of $\mu$ and $\mSigma$ the covariance matrix of $\mu$.
    Therefore,
    \begin{equation}
        \KL(\mu, \nu)
        =
        \text{G}(\mu, \nu)
        + \inf_{\nu\ss{G} \in \Pc\ss{$\lambda$,G}^N} \KL(\mu, \nu\ss{G})
    \end{equation}
 and $\text{G}(\mu, \nu) = 0$ if and only if $\nu = \Normal(\mu)$.

    Consider $(\pi, \sigma) \in \Qc\ss{G}$.
    Let $\vx \in \tI$ and $D \in \tD$.
    Note that $P^{\sigma_f}_\vx \pi_f(D) \in \Pc_\lambda^N$ and $P_\vx^\sigma \pi(D) \in \Pc\ss{$\lambda$,G}^N$.
    Moreover, using that $\sigma_f > 0$, $\inf_{\nu\ss{G} \in \Pc\ss{$\lambda$,G}^N} \KL(P_\vx^{\sigma_f} \pi_f(D), \nu\ss{G}) < \infty$, so
    \begin{align}
        &\KL(P^{\sigma_f}_\vx \pi_f(D), P^{\sigma}_\vx \pi(D)) \nonumber \\
        &\qquad= \text{G}(P^{\sigma_f}_\vx \pi_f(D), P^{\sigma}_\vx \pi(D))
        + \inf_{\nu\ss{G} \in \Pc\ss{$\lambda$,G}^N} \KL(P^{\sigma_f}_\vx \pi_f(D), \nu\ss{G}).
    \end{align}
    We therefore have the decomposition
    \begin{align}
        &\E_{p(D)p(\vx)}[\KL(P^{\sigma_f}_\vx \pi_f(D), P^{\sigma}_\vx \pi(D))] \\
        &\qquad=
        \underbracket[1pt]{\E_{p(D)p(\vx)}[
            \vphantom{\inf_{\nu\ss{G} \in \Pc\ss{G}^N}}
            \text{G}(P^{\sigma_f}_\vx \pi_f(D), P^{\sigma}_\vx \pi(D))
        ]}_{\text{(i)}}
        +
        \underbracket[1pt]{\E_{p(D)p(\vx)}[
            \inf_{\nu\ss{G} \in \Pc\ss{$\lambda$,G}^N} \KL(P^{\sigma_f}_\vx \pi_f(D), \nu\ss{G})
        ]}_{\text{(ii)}}.
         \nonumber
    \end{align}
    It is clear that $\text{(ii)} \le \inf_{\Qc\ss{G}} \L\ss{NP} < \infty$, so (ii) is finite.
    Hence, $(\pi, \sigma)$ minimises $\L\ss{NP}$ over $\Qc\ss{G}$ if and only if $(\pi, \sigma)$ minimises (i) over $\Qc\ss{G}$.

    Suppose that (i) is zero.
    Then there exists a measurable $A \sub \tD$ such that $\int_A p(D) \isd D = 1$ and
    \begin{equation}
        \E_{p(\vx)}[\text{G}(P^{\sigma_f}_\vx \pi_f(D), P^{\sigma}_\vx \pi(D))] = 0
        \quad \text{for all $D \in A$}.
    \end{equation}
    Fix such a $D \in A$.
    Then there exists a measurable $B \sub \tI$ such that $\int_{B} p(\vx) \isd \vx = 1$ and%
    \begin{equation}
        \text{G}(P^{\sigma_f}_\vx \pi_f(D), P^{\sigma}_\vx \pi(D)) = 0
        \quad \text{for all $\vx \in B$}.
    \end{equation}
    Hence, %by our earlier observation about $\text{G}$, % this is equivalent to
    $
        P^{\sigma}_\vx \pi(D)
        = \Normal(P^{\sigma_f}_\vx \pi_f(D))
        % \quad \text{for all $\vx \in B$}.
    $ for all $\vx \in B$, which means that
    %This condition, in turn, is equivalent to
    \begin{align}
        \label{eq:gaussian_np_approximation:mean_condition}
        m_\pi(D)(x') &= m_f(D)(x') \\
               % &\text{for all $x \in \set{x_1, \ldots, x_{\abs{\vx}}}$,} \\
        \label{eq:gaussian_np_approximation:kernel_condition}
        k_\pi(D)(x', y') + \sigma^2 \ind(x' = y') &= k_f(D)(x', y') + \sigma^2_f \ind(x' = y')
               % &\text{for all $x, y \in \set{x_1, \ldots, x_{\abs{\vx}}}$}.
    \end{align}
    for all $\vx \in B$ and $x', y' \in \set{x_1, \ldots, x_{\abs{\vx}}}$.
    But $B$ is dense in $\tI$, because $p(\vx)$ assigns positive probability to every open set (\cref{\xrprefix{assum:full_support_and_complete}}). 
    Additionally, by assumption, $\tI$ is dense in $\X^N$ for some $N \ge 2$.
    Therefore, \eqref{eq:gaussian_np_approximation:mean_condition} and \eqref{eq:gaussian_np_approximation:kernel_condition} hold for a collection of $(x',y')$ dense in $\X^2$.
    The condition that $N \ge 2$ is necessary, otherwise always $x' = y'$, so \eqref{eq:gaussian_np_approximation:kernel_condition} would not necessarily hold for $x' \neq y'$.
    Taking appropriate limits, using that $m_\pi$, $m_f$, $k_\pi$, and $k_f$ are continuous, we hence find that
    \begin{equation}
        m_\pi|_{B} = m_f|_{B},
        \qquad
        k_\pi|_{B} = k_f|_{B},
        \qquad
        \text{and}
        \qquad
        \sigma = \sigma_f.
    \end{equation}
    Finally, using density of $B$ in $\tD$ in combination with continuity of $\pi_f$ and $\pi$ gives
    \begin{equation} \label{eq:gnpa_characterisation:desired_condition}
        m_\pi|_{\tD} = m_f|_{\tD},
        \qquad
        k_\pi|_{\tD} = k_f|_{\tD},
        \qquad
        \text{and}
        \qquad
        \sigma = \sigma_f,
    \end{equation}
    which is the desired condition.
%
    Conversely, suppose that \eqref{eq:gnpa_characterisation:desired_condition} is satisfied.
    Then \eqref{eq:gaussian_np_approximation:mean_condition} and \eqref{eq:gaussian_np_approximation:kernel_condition} are satisfied for all $x', y' \in \X$ and $D \in \tD$, so (i) is zero.
\end{proof}

\restatement{../../chapters/predmap/statements/cnpa_characterisation.tex}
\begin{proof}
    The proof goes exactly like the proof of \cref{\xrprefix{prop:gnpa_characterisation}}, with the following modifications.
    Additionally assume that the distributions in $\Pc\ss{$\lambda$,G}^N$ factorise, \ie~that the dimensions are independent.
    Then $\text{G}(\mu, \nu) = 0$ if and only if $\nu = \mathcal{N}\ss{d}(\mu)$
    where $\Nc\ss{d}(\mu) = \Nc(\vmu, d(\mSigma))$ with $d(\mSigma)$ the diagonal matrix with diagonal equal to $\diag(\mSigma)$.
    The key difference with respect to the proof of \cref{\xrprefix{prop:gnpa_characterisation}} is that $k_\pi(D)$ in \eqref{eq:gaussian_np_approximation:mean_condition} and \eqref{eq:gaussian_np_approximation:kernel_condition} is now only evaluated at $x' = y'$, which yields a condition on the variance map.
    Because the variance map is a function of one argument,
    rather than requiring $N \ge 2$, the requirement $N \ge 1$ suffices.
    Since always $x' = y'$,
    the indicator functions in \eqref{eq:gaussian_np_approximation:mean_condition} and \eqref{eq:gaussian_np_approximation:kernel_condition} cannot be used any more to find $\sigma = \sigma_f$;
    we now only find $v_\pi|_{\tD} + \sigma^2 = v_f|_{\tD} + \sigma_f^2$.
\end{proof}

\begingroup
    \newcommand{\possibleprefix}[1]{\xrprefix{#1}}
    \restatement{../../chapters/predmap/statements/gnpa_regularity.tex}
\endgroup
\begin{proof}
    Let $D \in \tD$.
    By \cref{\xrprefix{prop:gnpa_characterisation}}, $m_{\pi\ss{G}}(D) = m_f(D)$ and $k_{\pi\ss{G}}(D) = k_f(D)$.
    Therefore, for any $x, y \in \X$,
    \begin{equation}
        \norm{f(x) - f(y)}_{L^2(\pi\ss{G}(D))}
        = \norm{f(x) - f(y)}_{L^2(\pi_f(D))}.
    \end{equation}
    The result then follows from $\norm{\vardot}_{L^2(\pi_f(D))} \le \norm{\vardot}_{L^p(\pi_f(D))}$ ($p \ge 2$; see \cref{\xrprefix{assum:f_regularity}}) in combination with \cref{\xrprefix{prop:posterior_prediction_map_Holder}}.
\end{proof}


\section{Proofs for Section \ref{\xrprefix{sec:predmap:consistency}}}
\label{sec:proofs_predmap:consistency}

\begin{lemma} \label{lem:local_Lipschitzness_r}
    \begin{equation}
        \abs{r(\vx_1) - r(\vx_2)} \le \frac{1}{\underline{\sigma}}\norm{\vx_1 - \vx_2}_2.
    \end{equation}
\end{lemma}
\begin{proof}
    For $x_1, x_2 \in \R$, using the fundamental theorem of calculus,
    \begin{equation}
        \abs{e^{-\frac12 x_1^2} - e^{-\frac12 x_2^2}} = \abs*{
            \int_{x_1}^{x_2} x e^{-\frac12 x^2} \isd x
        }
        \le \norm{x \mapsto x e^{-\frac12 x^2}}_\infty \abs{x_1 - x_2},
    \end{equation}
    and $\norm{x \mapsto x e^{-\frac12 x^2}}_\infty \le 1$.
    Set $x_1 = \frac1{\sigma}\norm{\vx_1}_2$ and $x_2 = \frac1{\sigma}\norm{\vx_2}_2$,
    so
    \begin{equation}
        \abs{r(\vx_1) - r(\vx_2)}
        \le \frac{1}{\sigma} \abs{\norm{\vx_1}_2 - \norm{\vx_2}_2}
        \le \frac{1}{\underline{\sigma}} \norm{\vx_1 - \vx_2}_2,
    \end{equation}
    which concludes.  
\end{proof}

\begin{lemma} \label{lem:bounds_Z}
    There exist universal $c_\ell$ and $c_u$ such that, for all $(\vx, \vy) \in \tD$,
    \begin{equation}
        0
        < c_\ell
        \le Z(\vx, \vy)
        \le c_u
        < \infty.
    \end{equation}
    In particular, $\frac12 e^{-\frac{1}{\underline{\sigma}^2} B_{\tD}^2 (1 + 2 B_f^2)} \le c_\ell$ and $c_u \le 1$.
\end{lemma}
\begin{proof}
    The proof is similar to \cref{lem:inf_sup_Z}.
    For the lower bound on $c_\ell$, see \eqref{lem:inf_sup_Z:lower_bound}.
\end{proof}

For $f \in C(\X, \Y)$, define the \emph{modulus of continuity} by
\begin{equation}
    \omega_f(h)
    = \sup_{x, y \in \X \,:\, \abs{x - y} < h} \abs{f(x) - f(y)}.
\end{equation}

\begin{proposition}
    Under \cref{\xrprefix{assum:f_regularity}},
    \begin{equation}
        \P(\omega_f(h) \ge \e) \le \frac{c_{\beta, p}}{\e} h^{\beta - \frac{1}{p}}
    \end{equation}
    where $c_{\beta, p} > 0$ is a constant only depending on $c$, $\beta$, and $p$.
\end{proposition}
\begin{proof}
    This follows directly from \cref{prop:uniform_control_modulus_of_continuity}.
    \Cref{prop:uniform_control_modulus_of_continuity} is a stronger result which we will prove later.
\end{proof}

\begin{proposition} \label{prop:Lipschitzness_Z}
    There exists a universal $c_Z > 0$ such that, for all $D_1, D_2 \in \D$,
    \begin{equation}
        \abs{Z(\vx_1, \vy_1) - Z(\vx_2, \vy_2)}
        \le c_Z\, d_\D(D_1, D_2)^{\frac12(\beta - \frac1p)}
        \quad\text{whenever}\quad
        d_\D(D_1,D_2) < 1
    \end{equation}
    where $D_1 = (\vx_1, \vy_1)$ and $D_2 = (\vx_2, \vy_2)$.
\end{proposition}
\begin{proof}
    Suppose that $D_1, D_2 \in \D^N$.
    Consider the event $A = \set{f : 
        \omega_f(\norm{\vx_1 - \vx_2}_\infty) < \e
    }$.
    We have
    \begin{equation}
        \abs{Z(\vx_1, \vy_1) - Z(\vx_2, \vy_2)}
         \le 
            \abs{\E_f[
                (
                    r(\vy_1 - f(\vx_1)) 
                    - r(\vy_2 - f(\vx_2)) 
                ) \ind_A
            ]}
            + 2\, \P(A^\c).
    \end{equation}
    We bound the two terms separately, starting with the second term:
    \begin{equation}
        \P(A^\c)
        =
            \P(\omega_f(\norm{\vx_1 - \vx_2}_\infty) \ge \e)
            % + \P(\norm{f(\vx_1)}_\infty \ge M)
            % + \P(\norm{f(\vx_2)}_\infty \ge M) \nonumber \\
        \le 
        \frac{c_{\beta, p}}{\e} \norm{\vx_1 - \vx_2}_\infty^{\beta - \frac{1}{p}}.
        % + \frac{2N}{M} \sup_{x \in \X} \, \norm{f(x)}_{L^1}.
    \end{equation}
    For the first term, simply take a supremum and use \cref{lem:local_Lipschitzness_r}:
    \begin{align}
        &\abs{\E_f[
            \ind_A
            (
                r(\vy_1 - f(\vx_1)) 
                - r(\vy_2 - f(\vx_2)) 
            ) 
        ]}
        \vphantom{\frac{}{\underline{\sigma}}}
        \nonumber \\
        &\qquad \le 
            \vphantom{\frac{1}{\underline{\sigma}}}
            \sup_{f \in A} \, 
                \abs{
                    r(\vy_1 - f(\vx_1)) 
                    - r(\vy_2 - f(\vx_2)) 
                } \\
        &\qquad \le 
            \frac{1}{\underline{\sigma}}\, \sup_{f \in A} \, 
                (\norm{f(\vx_1) - f(\vx_2)}_2 + \norm{\vy_1 - \vy_2}_2) \\
        &\qquad \le 
            \frac{1}{\underline{\sigma}}\, \sup_{f \in A} \, 
                (\sqrt{N}\omega_f(\norm{\vx_1 - \vx_2}_\infty) + \norm{\vy_1 - \vy_2}_2) \\
        &\qquad \le 
            \frac{1}{\underline{\sigma}}\,
                (\sqrt{N} \e + \norm{\vy_1 - \vy_2}_2),
    \end{align}
    where the last inequality follows from that $\omega_f(\norm{\vx_1 - \vx_2}_\infty) \le \e$ for all $f \in A$.
    Putting the bounds for the two terms together, we have
    \begin{equation}
        \abs{Z(\vx_1, \vy_1) - Z(\vx_2, \vy_2)}
        \le 
            \frac{2 c_{\beta, p}}{\e} \norm{\vx_1 - \vx_2}_\infty^{\beta - \frac{1}{p}}
            + \frac{1}{\underline{\sigma}}(\sqrt{N}\e + \norm{\vy_1 - \vy_2}_2).
            %+ \frac{4N}{M} \sup_{x \in \X} \, \norm{f(x)}_{L^1}.
    \end{equation}
    This holds for any choice of $\e > 0$, so we may also take an infimum over $\e > 0$.
    For $a,b \ge 0$, note that the infimum of $a/\e + b$ over $\e > 0$ is given by $2 \sqrt{a b}$.
    Therefore,
    \begin{equation}
        \abs{Z(\vx_1, \vy_1) - Z(\vx_2, \vy_2)}
        \le 
            \frac{2\sqrt{2}}{\sqrt{\underline{\sigma}}}
            (\sqrt{N} c_{\beta,p})^{\frac12}
            \norm{\vx_1 - \vx_2}_\infty^{\frac12(\beta - \frac1p)}
            + \frac{1}{\underline{\sigma}}\norm{\vy_1 - \vy_2}_2.
    \end{equation}
    Finally note that $\norm{\vx_1 - \vx_2}_\infty < 1$ and $\norm{\vy_1 - \vy_2}_2 < 1$ whenever $d_\D(D_1, D_2) <1$ to find the result.
\end{proof}

\begin{proposition} \label{prop:Lipschitzness_functional}
    Let $D_1, D_2 \in \D_N \cap \tD$ and denote $D_1 = (\vx_1, \vy_1)$ and $D_2 = (\vx_2, \vy_2)$.
    Let $F\colon C(\X, \Y) \to \R$ be a continuous functional such that, for some $\gamma' > 0$, $B_F = \norm{F(f)}_{L^{1+\gamma'}} < \infty$.
    Then there exists a function $L\colon[0,\infty) \to [0, \infty)$ such that
    \begin{equation}
        \abs{\E_{\pi(D_1)}[F(f)] - \E_{\pi(D_2)}[F(f)]}
        \le L(d(D_1, D_2)).
    \end{equation}
    Moreover, $L$ only depends on the universal parameters, $\gamma'$, and $B_F$; and $L$ goes to zero as $d(D_1, D_2)$ goes to zero.
\end{proposition}
\begin{proof}
    Consider the event $A = \set{f : 
        \omega_f(\norm{\vx_1 - \vx_2}_\infty) < \e
    }$.
    Denote $g(\vx, \vy) = r(\vy - f(\vx)) / Z(\vx, \vy)$ and
    note that $g(\vx_1, \vy_1) \le 1 /c_\ell$ and $g(\vx_2, \vy_2) \le 1/c_\ell$, where $c_\ell$ is defined in \cref{lem:bounds_Z}.
    Then
    \begin{align}
        &\abs{\E_{\pi(D_1)}[F(f)] - \E_{\pi(D_2)}[F(f)]} \nonumber \\
        &\qquad = \vphantom{\frac11} \abs{\E_f\sbrac{
            \parens{
                g(\vx_1, \vy_1)
                - g(\vx_2, \vy_2)
            }
            F(f)
        }} \\
        &\qquad \le \abs{\E_f\sbrac{
            \ind_A\parens{
                g(\vx_1, \vy_1)
                - g(\vx_2, \vy_2)
            }
            F(f)
        }} + \frac2{c_\ell} \E_f[\ind_{A^\c}\abs{F(f)}].
    \end{align}
    We bound the two terms separately, starting with the second term:
    \begin{equation}
        \E[\ind_{A^\c} \abs{F(f)}]
        \le
            \norm{F(f)}_{L^{1 + \gamma'}}
            \P(A^\c)^{\frac{\gamma'}{1 + \gamma'}} 
        \le
            \norm{F(f)}_{L^{1 + \gamma'}}
            \sbrac*{
                \frac{c_{\beta, p}}{\e} \norm{\vx_1 - \vx_2}_\infty^{\beta - \frac{1}{p}}
            }^{\smash{\frac{\gamma'}{1 + \gamma'}}}.
    \end{equation}
    For the first term, observe that, for any two functions $f$ and $g$ such that $0 < c_\ell \le g$ and $f, g \le c_u < \infty$,
    \begin{equation}
        \abs*{
            \frac{f(x_1)}{g(x_1)}
            -
            \frac{f(x_2)}{g(x_2)}
        }
        \le \frac{c_u}{c_\ell^2}\parens*{
            \abs{f(x_1) - f(x_2)}
            + \abs{g(x_1) - g(x_2)}
        }.
    \end{equation}
    Therefore, taking a supremum,
    \begin{align}
        &\abs{\E_f\sbrac{
            \ind_A\parens{
                g(\vx_1, \vy_1)
                - g(\vx_2, \vy_2)
            }
            F(f)
        }} \\
        &\qquad\le 
            \frac{c_u}{c_\ell^2}
            \,
            \norm{F(f)}_{L^1}
            \,
            \sup_{f \in A}
            \,\big[
                \abs{
                    r(\vy_1 - f(\vx_1))
                    - r(\vy_2 - f(\vx_2))
                }
                + \abs{Z(\vx_1, \vy_1) - Z(\vx_2, \vy_2)}
            \big]. \nonumber
    \end{align}
    Here, using \cref{lem:local_Lipschitzness_r,prop:Lipschitzness_Z},
    \begin{align}
        \abs{
            r(\vy_1 - f(\vx_1))
            - r(\vy_2 - f(\vx_2))
        }
        &\le \frac{1}{\underline{\sigma}}(
            \norm{f(\vx_1) - f(\vx_2)}_2
            + \norm{\vy_1 - \vy_2}_2
        ) \\
        &\le \frac{1}{\underline{\sigma}}(
            \sqrt{N}\omega_f(\norm{\vx_1 - \vx_2}_\infty)
            + \norm{\vy_1 - \vy_2}_2
        ), \\
        \abs{Z(\vx_1, \vy_1) - Z(\vx_2, \vy_2)}
        &\le \vphantom{\frac{1}{}} L_Z(d(D_1, D_2)),
    \end{align}
    so, using that $\omega_f(\norm{\vx_1 - \vx_2}_\infty) \le \e$ for $f \in A$,
    \begin{align}
        &\abs{\E_f\sbrac{
            \ind_A\parens{
                g(\vx_1, \vy_1)
                - g(\vx_2, \vy_2)
            }
            F(f)
        }} \nonumber \\
        &\qquad
            \le 
            \frac{c_u}{c_\ell^2}
            \,\norm{F(f)}_{L^1}
            \,\sbrac*{
                \frac{1}{\underline{\sigma}}(
                    \sqrt{N}\e
                    + \norm{\vy_1 - \vy_2}_2
                )
                + L_Z(d(D_1, D_2))
            }.
    \end{align}
    Putting the bounds for the two terms together, we have
    \begin{align}
        &\abs{\E_{\pi(D_1)}[F(f)] - \E_{\pi(D_2)}[F(f)]} \nonumber \\
        &\qquad
            \le
            \frac{2}{c_\ell}
            \,\norm{F(f)}_{L^{1 + \gamma'}}
            \,\sbrac*{
                \frac{c_{\beta, p}}{\e} \norm{\vx_1 - \vx_2}_\infty^{\beta - \frac{1}{p}}
            }^{\frac{\gamma'}{1 + \gamma'}}
            \nonumber \\
        &\qquad\phantom{\le}\;
            + \frac{c_u}{c_\ell^2}\,\norm{F(f)}_{L^1}
            \,\sbrac*{
                \frac{1}{\underline{\sigma}}(
                    \sqrt{N}\e
                    + \norm{\vy_1 - \vy_2}_2
                )
                + L_Z(d(D_1, D_2))
            },
    \end{align}
    which we simplify to
    \begin{align}
        &\frac{c_\ell}{2}\, \norm{F(f)}^{-1}_{L^{1 + \gamma'}}\abs{\E_{\pi(D_1)}[F(f)] - \E_{\pi(D_2)}[F(f)]} \\
        &\qquad
            \le
            \sbrac*{
                \frac{c_{\beta, p}}{\e} \norm{\vx_1 - \vx_2}_\infty^{\beta - \frac{1}{p}}
            }^{\frac{\gamma'}{1 + \gamma'}}
            + \frac{c_u}{c_\ell}
            \,\sbrac*{
                \frac{1}{\underline{\sigma}}(
                    \sqrt{N}\e
                    + \norm{\vy_1 - \vy_2}_2
                )
                + L_Z(d(D_1, D_2))
            }. \nonumber
    \end{align}
    This holds for any choice of $\e > 0$,
    so we may also take an infimum over $\e > 0$.
    Temporarily denote $\rho = \frac{\gamma'}{1 + \gamma'}$.
    For $a, b \ge 0$ and $\rho > 0$, note that the infimum of $(a/\e)^\rho + b \e$ over $\e > 0$ is given by
    $c_\rho (ab)^{\frac{\rho}{1+\rho}}$ with $c_\rho = \rho^{\frac1{1+\rho}} + \rho^{-\frac{\rho}{1+\rho}}$.
    Therefore,
    \begin{align}
        &\frac{c_\ell}{2} \norm{F(f)}^{-1}_{L^{1 + \gamma'}}\abs{\E_{\pi(D_1)}[F(f)] - \E_{\pi(D_2)}[F(f)]} \\
        &\qquad
            \le
            c_\rho \sbrac*{
                \frac{c_u}{c_\ell}
                \frac{\smash{\sqrt{N}}}{\underline{\sigma}}
                c_{\beta, p} \norm{\vx_1 - \vx_2}_\infty^{\beta - \frac{1}{p}}
            }^{\frac{\gamma'}{1 + 2 \gamma'}}
            + \frac{c_u}{c_\ell}
            \,\sbrac*{
                \frac{1}{\underline{\sigma}} \norm{\vy_1 - \vy_2}_2
                + L_Z(d(D_1, D_2))
            }, \nonumber
    \end{align}
    noting that $\frac{\rho}{1+ \rho}=\frac{\gamma'}{1+2\gamma'}$.
\end{proof}


\restatement{../../chapters/predmap/statements/regularity_part_two.tex}   
\begin{proof}
    \listnum{1}:
    For the mean, choose some $x \in \X$, apply \cref{prop:Lipschitzness_functional} with $F(f) = f(x)$ and $\gamma' = 1 + \gamma$, and verify that
    \begin{equation}
        \norm{f(x)}_{L^{1 + \gamma'}}
        \le \sup_{x \in \X} \, \norm{f(x)}_{L^{2 + \gamma}}
        = B_f,
    \end{equation}
    which only depends on the universal parameters.
    For the covariance, choose two $x, y \in \X$, apply \cref{prop:Lipschitzness_functional} with $F(f) = f(x)f(y)$ and $\gamma' = \frac12 \gamma$, and verify that 
    \begin{equation}
        \norm{f(x) f(y)}_{L^{1 + \gamma'}}
        \le
            \norm{f(x)}_{L^{2 + 2\gamma'}}
            \norm{f(y)}_{L^{2 + 2\gamma'}}
        \le \sup_{x \in \X}\, \norm{f(x)}^2_{L^{2 + \gamma}}
        = B^2_f
    \end{equation}
    which again only depends on the universal parameters.
    
    \listnum{2}:
    We will repeatedly use the observation that, for any $q \ge 1$ and $D \in \tD$,
    \begin{equation}
        \norm{\vardot}_{L^q(\pi(D))} \le \frac{1}{c_\ell^{1/q}} \norm{\vardot}_{L^q}
    \end{equation}
    where $c_\ell$ is from \cref{lem:bounds_Z}.
    We first prove the inequality for the mean:
    \begin{align}
        &\abs{\E_{\pi(D_1)}[f(x_1)] - \E_{\pi(D_2)}[f(x_2)]} \nonumber \\
        &\qquad\le
            \abs{\E_{\pi(D_1)}[f(x_1)] - \E_{\pi(D_2)}[f(x_1)]}
            + \abs{\E_{\pi(D_2)}[f(x_1)] - \E_{\pi(D_2)}[f(x_2)]} \\
        &\qquad\le
            L_m(d(D_1, D_2))
            + \norm{f(x_1) - f(x_2)}_{L^1(\pi(D_2))} \\
        &\qquad\le
            L_m(d(D_1, D_2))
            + \frac{c}{c_\ell} \abs{x_1 - x_2}^\beta
    \end{align}
    where the second inequality follows from part \listnum{2} and the third inequality from the observation combined with \cref{\xrprefix{assum:f_regularity}}.
    For the covariance,
    \begin{align}
        &\abs{\E_{\pi(D_1)}[f(x_1)f(y_1)] - \E_{\pi(D_2)}[f(x_2)f(y_2)]} \nonumber \\
        &\qquad\le
            \abs{\E_{\pi(D_1)}[f(x_1)f(y_1)] - \E_{\pi(D_2)}[f(x_1)f(y_1)]} \nonumber \\
        &\qquad\;\qquad\qquad
            + \abs{\E_{\pi(D_2)}[f(x_1)f(y_1)] - \E_{\pi(D_2)}[f(x_1)f(y_2)]} \nonumber \\
        &\qquad\;\qquad\qquad
            + \abs{\E_{\pi(D_2)}[f(x_1)f(y_2)] - \E_{\pi(D_2)}[f(x_2)f(y_2)]}.
    \end{align}
    From part \listnum{2}, we have
    \begin{equation}
        \abs{\E_{\pi(D_1)}[f(x_1)f(y_1)] - \E_{\pi(D_2)}[f(x_1)f(y_1)]}
        \le L_k(d(D_1, D_2)).
    \end{equation}
    We also have
    \begin{align}
        \abs{\E_{\pi(D_2)}[f(x_1)(f(y_1) - f(y_2))]}
        &\le \vphantom{\frac{}{c_\ell}}\norm{f(x_1)}_{L^2(\pi(D_2))}\norm{f(y_1) - f(y_2)}_{L^2(\pi(D_2))} \\
        &\le \vphantom{\frac{B_f}{}}\frac1{c_\ell}
            \norm{f(x_1)}_{L^2} \norm{f(y_1) - f(y_2)}_{L^2} \\
        &\le B_f \frac{c}{c_\ell}
            \abs{y_1 - y_2}^\beta.
    \end{align}
    Therefore,
    \begin{align}
        &\abs{\E_{\pi(D_1)}[f(x_1)f(y_1)] - \E_{\pi(D_2)}[f(x_2)f(y_2)]} \nonumber \\
        &\qquad\le
            L_k(d(D_1, D_2))
            + B_f \frac{c}{c_\ell} \abs{x_1 - x_2}^\beta
            + B_f \frac{c}{c_\ell} \abs{y_1 - y_2}^\beta,
    \end{align}
    which concludes.
\end{proof}

\restatement{../../chapters/predmap/statements/cnpa_consistency.tex}
\begin{proof}
    The result follows directly from \cref{\xrprefix{prop:cnpa_characterisation}} in combination with Theorem 5.41 by \textcite{Vaart:1998:Asymptotic_Statistics}.
    It remains to verify the conditions for Theorem 5.4.1.
    Convergence in the metric on $\widetilde\Qc\ss{G,MF}$ (see \eqref{\xrprefix{eq:metric_Q_G_MF}}) implies convergence of all means and variances, which is stronger than the required continuity condition.
    For the integrability condition, combine the following three facts.
    First, by assumption, the maximum target set size is bounded.
    Second, the universal H\"older condition implies that $\sup_{\pi \in \widetilde\Qc\ss{G,MF},\,D \in \tD,\, x \in \X} \E_{\pi(D)}[f^2(x)] < \infty$, which bounds all means and covariances.
    Third, by \cref{\xrprefix{assum:noise_boundedness}} and the following paragraph, every noise variance $\sigma$ is in $[\underline{\sigma}, \overline{\sigma}]$. 
\end{proof}

\restatement{../../chapters/predmap/statements/gnpa_consistency.tex}
\begin{proof}
    The proof goes exactly like the proof of \cref{\xrprefix{prop:cnpa_consistency}}.
\end{proof}

\begin{proposition} \label{prop:uniform_control_modulus_of_continuity}
    Consider $(\mu_i)_{i\ge1} \sub \mathcal{P}\ss{c}$.
    Suppose there exist $p > 0$, $\beta > \tfrac1p$, a constant $c > 0$, and a radius $r > 0$ such that
    \begin{equation}
        \sup_{i \ge 1}\,
        \norm{f(x) - f(y)}_{L^p(\mu_i)} \le c \abs{x - y}^\beta
        \quad\text{whenever}\quad
        \abs{x - y} < r.
    \end{equation}
    Then, for all $\e > 0$,
    \begin{equation}
        \sup_{i \ge 1}\, \mu_i(\omega_f(h) \ge \e)
        \le \frac{c}{\e} \frac{2}{1 - 2^{-(\beta - \frac1p)}} h^{\beta - \frac1p}.
    \end{equation}
\end{proposition}
\begin{proof}
    Choose $k \in \N$ such that $2^{-(k + 1)} \le h \le 2^{-k}$.
    The proof strategy mimics the proof of Theorem 4.2.1 by \textcite{Norris:2018:Advanced_Probability}.
    Let $i \in \N$.
    For $n \in \N$, denote $\Db_n = \set{0, 2^{-n}, 2 \cdot 2^{-n}, \ldots, 1}$.
    Consider $f \in C([0, 1], \mathcal{Y})$.
    Set
    \begin{equation}
        K_n = \sup_{x \in \Db_n \setminus \set{1}} = \abs{f(x + 2^{-n}) - f(x)}.
    \end{equation}
    Overestimate the supremum by a sum and use $L^{p}$-H\"older continuity of $\mu_i$:
    \begin{equation}
         \E_{\mu_i}[K_n^p]
         \le c \sum_{i=0}^{2^n - 1} \E_{\mu_i}[\abs{f(x + 2^{-n}) - f(x)}^p]
         \le c \sum_{i=0}^{2^n - 1} 2^{-\beta p n}
         = 2^{-(\beta p - 1) n}.
    \end{equation}
    For any $x, y \in \union_{n=1}^\infty \Db_n$ such that $x < y < x + 2^{-k}$, note that the interval $[x, y)$ is the finite, disjoint union of intervals $[r, r + 2^{-n})$ with $r \in \Db_n$ for $n \ge k + 1$ where no three intervals have the same length \parencite[proof of Theorem 4.2.1;][]{Norris:2018:Advanced_Probability}.
    Therefore, for such $x$ and $y$, by continuity of $f$,
    \begin{equation}
        \omega_f(2^{-k})
        = \sup_{x, y \in \union_{n=1}^\infty \Db_n: \abs{x - y}< 2^{-k}}\abs{f(x) - f(y)}
        \le 2\sum_{n=k+1}^\infty K_n.
    \end{equation}
    Hence,
    \begin{equation}
        \E_{\mu_i}\sbrac{\omega_f(2^{-k})}
        % \le 2\sum_{n=k+1}^\infty \E_{\mu_i}[K_n]
        \le 2\sum_{n=k+1}^\infty \E_{\mu_i}[K_n^p]^{\frac1p}
        \le 2c\sum_{n=k+1}^\infty 2^{-(\beta - \frac1p)n}
        = c\frac{2^{1-(\beta - \frac1p)(k + 1)}}{1 - 2^{-(\beta - \frac1p)}},
        %= c_{p,q} 2^{-(\beta - \frac1p)k},
    \end{equation}
    using that $\beta - \frac1p > 0$,
    where
    is some constant $c_{p, q}$ that depends on $p$, $\beta$ and $c$. 
    Then, for any $\e > 0$, by Markov's inequality,
    \begin{equation}
        \sup_{i \ge 1}\, \mu_i(\omega_f(h) \ge \e)
        \le \sup_{i \ge 1}\, \mu_i(\omega_f(2^{-k}) \ge \e)
        % \le \frac{c_{p,q}}{\e} 2^{-(\beta - \frac1p)k}
        \le \frac{2c}{\e} \frac{2^{-(\beta - \frac1p)k}}{2^{\beta - \frac1p} - 1}
        \le \frac{2c}{\e} \frac{(2h)^{\beta - \frac1p}}{2^{\beta - \frac1p} - 1}
        %\le \frac{c_{p,q}}{\e} (2h)^{\beta - \frac1p},
    \end{equation}
    which proves the result.
\end{proof}



\end{document}
