\documentclass[12pt, twoside]{report}
\usepackage{libertine}

% General packages:
\usepackage[utf8]{inputenc}    % Input encoding
\usepackage[T1]{fontenc}       % Font encoding
\usepackage[british]{babel}    % Naming of figures and such
\usepackage[british]{isodate}  % Formatting of dates
\cleanlookdateon               % Show dates cleanly.
%\usepackage{xurl}              % Allow line breaks anywhere in URLs.
\usepackage{minted}            % Needs to be loaded before `csquotes`.
\usepackage[
    citestyle    = authoryear,
    bibstyle     = authoryear,
    giveninits   = true,
    uniquename   = init,
    uniquelist   = false,
    sortlocale   = en_GB,
    backend      = biber,
    backref      = true,
    maxbibnames  = 100,
    maxcitenames = 1,
    dashed       = false,
    % Needed to get sorting right for tussenvoegsels.
    %sortcites    = true,       
    ]{biblatex}                % Bibliography
\usepackage[
    style=american
    ]{csquotes}                % Required for BiBLaTeX
\usepackage{import}            % Importing subdocuments
\usepackage{standalone}        % Compilable subdocuments
% `setspace` needs to be loaded before `hyperref`.
\usepackage{setspace}          % Set line spacing
\usepackage{hyperref}          % Clickable references
\usepackage{microtype}         % Nice typography
\usepackage{makeidx}           % Make an index
\usepackage{silence}           % Silence warnings and errors
\WarningFilter{remreset}{The remreset package}
\WarningFilter*[parskip]{latex}{Command}

% Set line spacing.
\setstretch{1.25}

% Page layout:
\usepackage{tocloft}         % Control ToC
\usepackage[
    a4paper,
    bottom     = 1.1in,
    top        = 1.1in,
    left       = 1.1in,
    right      = 1.1in,
    headheight = 15pt,
    footskip   = 1.5\baselineskip
    ]{geometry}              % Margins
\usepackage{fancyhdr}        % Header
\usepackage{lastpage}        % Page numbers in footer
\usepackage{setspace}        % Line spacing
\usepackage{ragged2e}        % Better line endings
\ActivateWarningFilters[parskip]
\usepackage[
    parfill
    ]{parskip}               % Newlines instead of indentation
\DeactivateWarningFilters[parskip]
\usepackage{titlesec}        % Size of sections
\usepackage{titlecaps}       % Automatic capitalisation of titles
\usepackage[
    hang,
    bottom
    ]{footmisc}              % Configure footnotes
\usepackage{afterpage}       % Include stuff after the current page
\usepackage{placeins}        % Things like `\FloatBarrier`
\usepackage{pdflscape}       % Turn pages sideways
\usepackage{emptypage}       % Remove pagenumbers on empty pages

% Utility:
\usepackage[table]{xcolor}   % Colours
\usepackage[
    separate-uncertainty=true,
    per-mode=symbol
    ]{siunitx}               % Display units
\usepackage{enumitem}        % Better enumeration
\usepackage{xifthen}         % If statements
\usepackage{xpatch}          % Patch things
\usepackage{lipsum}          % Dummy text
\usepackage{listings}        % Listings
\usepackage{xparse}          % Better arguments for \newcommand
\usepackage{xfrac}           % Better fractions
\usepackage{etoolbox}        % Patch stuff
\usepackage[
    notquote
    ]{hanging}               % Hanging paragraphs
\usepackage{scalerel}        % Scale objects
\usepackage{soul}            % Highlighting

% Figures:
\usepackage{pgfplots}        % Plots
\pgfplotsset{compat=newest}
\usepackage{tikz}            % TikZ figures
\usepackage{float}           % Control floating of figures
\usepackage{multirow}        % Cells spanning multiple rows
\usepackage{graphicx}        % Graphics
\usepackage{caption}         % Subfigures and captions
\usepackage{subcaption}      % Subfigures and captions
\usepackage{booktabs}        % Nice looking tables
\usepackage{tabularx}        % Extended functionality for tables

% Math:
\usepackage{amsmath}         % Math
\usepackage{amssymb}         % Math symbols
\usepackage{mathtools}       % Math tools
\usepackage{amsthm}          % Theorems
\usepackage{thmtools}        % More math theorems
\usepackage{bm}              % Bold math symbols
\usepackage{bbm}             % More bold math symbols
\usepackage{cancel}          % Cancel equations
\usepackage{upgreek}         % Upright greek symbols
\usepackage[
    artemisia
    ]{textgreek}             % Greek symbols in text

% This package should be loaded last.
\usepackage[
    noabbrev,
    capitalize,
    nameinlink]{cleveref}    % Automatic referencing

% Add a comma in the cite style.
\renewcommand*{\nameyeardelim}{\addcomma\space}

% Make sure that every `\citeyear` also prints the disambiguating `extradate`.
\DeclareCiteCommand{\citeyearlabel}
    {\usebibmacro{prenote}}
    {\printfield{year}\printfield{extradate}}
    {\multicitedelim}
    {\usebibmacro{postnote}}
\let\citeyearold\citeyear
\let\citeyear\citeyearlabel
\let\citeyearnolabel\citeyearold

% Define a standard for citing within theorem statements.
\newcommand{\theoremcite}[1]{\citeauthor{#1}, \citeyear{#1}}

\newcommand{\fulltextcite}[1]{%
    \AtNextCite{\AtEachCitekey{\defcounter{maxnames}{999}}}%
    \textcite{#1}%
}
\newcommand{\fullparencite}[1]{%
    \AtNextCite{\AtEachCitekey{\defcounter{maxnames}{999}}}%
    \parencite{#1}%
}

% Remove the colon after "In" in the bibliograhpy.
\renewbibmacro*{in:}{\bibstring{in} }

% Tune the electronic print field.
\DeclareFieldFormat{eprint}{%
  \iffieldundef{eprinttype}
    {Electronic print}
    {\thefield{eprinttype}}%
  \addcolon\space
  \ifhyperref
    {\url{#1}}
    {\nolinkurl{#1}}%
  \iffieldundef{eprintclass}
    {}
    {\addspace\mkbibparens{\thefield{eprintclass}}}\printtext{.}}

% Tune URL.
\DeclareFieldFormat{url}{\mkbibacro{URL}\addcolon\space\url{#1}\printtext{.}}

% Tune DOI.
\DeclareFieldFormat{doi}{%
  \mkbibacro{DOI}\addcolon\space
  \ifhyperref
    {\href{https://doi.org/#1}{\nolinkurl{#1}}}
    {\nolinkurl{#1}}\printtext{.}}

% In the below, we will redefine `\finentry` to not print a period if
%` pageref` exists. The period is then added back by `pageref`, but before the
% parentheses. Problems occur when `urldate` is also defined. In that case, 
% the extra period by `pageref` will add a period after `urldate`s
% parentheses, which we do not want. We use the toggle `printperiod` to detect
% this case.

% Tune "visited on".
\DeclareFieldFormat{urldate}{\mkbibparens{\bibstring{urlseen}\space#1\printtext{.}}}
\newtoggle{printperiod}
\toggletrue{printperiod}
\AtBeginBibliography{\renewbibmacro*{urldate}{\printurldate\togglefalse{printperiod}}}

% Tune "cited on".
\renewbibmacro*{finentry}{\iflistundef{pageref}{}{\renewcommand{\finentrypunct}{}}\finentry}
\renewbibmacro*{pageref}{%
  \iflistundef{pageref}
    {}
    {\iftoggle{printperiod}{\setunit{\adddot\addspace}}{}\toggletrue{printperiod}\printtext[parens]{%
       \ifnumgreater{\value{pageref}}{1}
         {\bibstring{backrefpages}\ppspace}
         {\bibstring{backrefpage}\ppspace}%
          \printlist[pageref][-\value{listtotal}]{pageref}.}}}

% Add Oxford comma.
\DefineBibliographyExtras{british}{\def\finalandcomma{\addcomma}}

% Don't abbreviate the backreferences.
\DefineBibliographyStrings{english}{%
    backrefpage = {Cited on page},
    backrefpages = {Cited on pages}
}

% Get tussenvoegsels right. This is not a perfect solution, because the
% in-text citations will be sorted as if the tussenvoegsel is part of the
% last name.
\makeatletter
\AtBeginDocument{\toggletrue{blx@useprefix}}
\AtBeginBibliography{\togglefalse{blx@useprefix}}
\makeatother

% Make an index.
\makeindex

% Show the page number in the bottom center.
\fancypagestyle{plain}{
    \fancyhf{}                          % Clear all header and footers.
    \renewcommand{\headrulewidth}{0pt}  % Remove the header rule.
    \cfoot{\thepage}                    % Show page number in the center.
}
\pagestyle{plain}

% Chapter styling from https://texblog.org/2012/07/03/fancy-latex-chapter-styles/
\newcommand{\hsp}{\hspace{20pt}}
\titleformat{\chapter}[hang]
    {\Huge\bfseries}
    {\thechapter\hsp\textcolor{gray75}{|}\hsp}
    {0pt}
    {\Huge\bfseries}
\titleformat{\section}
    {\normalfont\Large\bfseries\raggedright}
    {\thesection}
    {1em}
    {}
\titleformat{\subsection}
    {\normalfont\large\bfseries\raggedright}
    {\thesubsection}
    {1em}
    {}

% Set spacing after chapter equal to spacing right before section. Then
% things look nicely aligned.
\titlespacing{\chapter}{0pt}{3.5ex}{3.5ex}

% Define the styling of paragraphs.
\renewcommand{\paragraph}[1]{\textbf{#1.}}

% Redefine abstract environment.
\renewenvironment{abstract}
    {\begingroup
    \setstretch{1.25}
    \begin{center}
        \textbf{Abstract}
    \end{center}}
    {\par
    \endgroup}

% Kill any use of the \cite command.
\renewcommand{\cite}[1]{\PackageError{thesis}{use either parencite or authorcite}{}}

% Configure captions.
\WarningFilter{caption}{Unused \captionsetup}
\captionsetup[table]{font=small}
\captionsetup[figure]{font=small}

% Shortcuts for writing.
\newcommand{\ie}{\textit{i.e.}}
\newcommand{\Ie}{\textit{I.e.}}
\newcommand{\eg}{\textit{e.g.}}
\newcommand{\Eg}{\textit{E.g.}}
\newcommand{\cf}{\textit{c.f.}}
\newcommand{\Cf}{\textit{C.f.}}

% \ifempty command:
\newcommand{\ifempty}[3]{\ifthenelse{\equal{#1}{}}{#2}{#3}}

% Style footnotes.
\setlength{\skip\footins}{\baselineskip}
\setlength{\footnotesep}{.75\baselineskip}
\renewcommand{\footnotelayout}{\setstretch{1.0}}
\setlength{\footnotemargin}{1em}

% Typewriter font:
\renewcommand*{\ttdefault}{pcr}  % Courier
\newcommand{\code}[1]{{\small\texttt{#1}}}

% Define some colours.
\definecolor{darkblue} {rgb} {0.0 , 0.0 , 0.65}
\definecolor{darkred}  {rgb} {0.80, 0.0 , 0.0 }
\definecolor{redaccent}{HTML}{E64C66}
\definecolor{darkgreen}{rgb} {0.0 , 0.50, 0.0 }
\definecolor{gray75}   {gray}{0.75}

% Define shortcuts for colours.
\newcommand{\red}[1]{{\color{red} #1}}
\newcommand{\blue}[1]{{\color{blue} #1}}
\newcommand{\green}[1]{{\color{green} #1}}
\newcommand{\orange}[1]{{\color{orange} #1}}
\newcommand{\darkred}[1]{{\color{darkred} #1}}
\newcommand{\darkblue}[1]{{\color{darkblue} #1}}
\newcommand{\darkgreen}[1]{{\color{darkgreen} #1}}
\newcommand{\magenta}[1]{{\color{magenta} #1}}
\newcommand{\grey}[1]{{\color{gray} #1}}

% Use colours to define checkmarks and crosses.
\usepackage{pifont}
\newcommand{\xmark}{\text{\ding{55}}}
\newcommand{\cmark}{\text{\ding{51}}}
\newcommand{\good}{\darkgreen{$\checkmark$}}
\newcommand{\mediumgood}{\orange{$\checkmark$}}
\newcommand{\mediumbad}{\orange{\xmark}}
\newcommand{\bad}{\darkred{\xmark}}

% Load TikZ libraries.
\usetikzlibrary{
    calc,
    positioning,
    fit,
    tikzmark,
    arrows.meta,
    shapes,
    decorations.pathreplacing,
    intersections,
    through
}
% Graphical models:
\tikzset{
    line/.style = {
        thick,
        ->,
        > = {
            Triangle[length=1.5mm, width=1.5mm]
        }
    },
    arrow/.style = {
        line
    },
    % Invisible node:
    hidden node/.style = {
        circle,
        minimum size = 1cm,
        draw = white,
        thick
    },
    % Latent variable:
    latent node/.style = {
        hidden node,
        draw = black,
    },
    % Latent variable:
    factor node/.style = {
        hidden node,
        rectangle,
        draw = black,
    },
    % Observed variable:
    observed node/.style = {
        latent node,
        fill = gray!15
    },
    % Plate:
    plate/.style = {
        draw,
        label={[anchor=north west]south west:#1},
        rounded corners=2pt,
        shape=rectangle,
        inner sep=10pt,
        thick
    }
}

% Circled number:
\newcommand{\ballnumber}[1]{%
    \tikz[baseline=(n.base)] \node[circle,fill=.,inner sep=1pt,text=white] (n) {\normalshape\bfseries\footnotesize #1};%
}
\newcommand{\itemballnumber}[1]{%
    \raisebox{.5pt}{\ballnumber{#1}}%
}

% Style enumerate and lists. Do not use `label=(\arabic*)` because equation
% numbering already uses the same style.
\setenumerate{topsep=.25\baselineskip, itemsep=0pt}
\setlist{topsep=.25\baselineskip, itemsep=0pt}
% \setlist[enumerate]{
%     label=(\arabic*),
%     itemsep=-0.25\baselineskip,
%     topsep=0.25\baselineskip,
%     after={\vspace*{0.25\baselineskip}}
% }
% \setlist[enumerate,2]{
%     topsep=-0.25\baselineskip,
%     itemsep=0pt,
%     label=(\alph*)
% }
% \setlist[enumerate,3]{
%     topsep=-0.25\baselineskip,
%     itemsep=0pt,
%     label=(\roman*)
% }
% \setlist[itemize]{
%     label=\textbullet,
%     itemsep=-0.25\baselineskip,
%     topsep=0.25\baselineskip
% }
% \setlist[itemize,2]{
%     topsep=-0.25\baselineskip,
%     itemsep=0pt
% }
% \setlist[itemize,3]{
%     topsep=-0.25\baselineskip,
%     itemsep=0pt
% }

% Set the default float placement correctly.
\floatplacement{figure}{tbp}
\floatplacement{table}{tbp}

% Set spacing between figures and text.
\setlength{\textfloatsep}{30pt plus 1.0pt minus 2.0pt}
\setlength{\floatsep}{30pt plus 1.0pt minus 2.0pt}
\setlength{\intextsep}{30pt plus 1.0pt minus 2.0pt}

% Adjust line spacing in captions.
\captionsetup{font={stretch=1.25}}

% Newline for in a title.
\newcommand{\tnl}{\texorpdfstring{\\}{}}

% Define outline tools.
\newcommand{\outline}[1]{
    \parbox{\linewidth}{
        \setstretch{1.0}
        \color{darkgreen}
        #1
    }\vspace{\parskip}
}
\newlist{outlinelist}{enumerate}{1}
\setlist[outlinelist]{
    label=\arabic*.,
    noitemsep,
    topsep=0pt,
    parsep=0pt,
    partopsep=0pt
}
\newcommand{\note}[1]{{\color{darkred}#1}}

% Left-justified text in tabularx environment:
\newcolumntype{L}{>{\RaggedRight\arraybackslash}X}

% Hyperlink setup.
\hypersetup{
    colorlinks,
    citecolor = black,
    filecolor = black,
    linkcolor = black,
    urlcolor  = black
}

% Landscape figures:
\newcommand{\landscapefloat}[1]{
    \afterpage{
        \begin{landscape}
            #1
        \end{landscape}
    }
}

% Hanging full citation:
\newcommand{\hangcite}[1]{\hangpara{\bibhang}{1}\AtNextCite{\defcounter{maxnames}{99}}\fullcite{#1}}

% Define \charfusion. From https://tex.stackexchange.com/a/52673.
\makeatletter
\def\moverlay{\mathpalette\mov@rlay}
\def\mov@rlay#1#2{\leavevmode\vtop{%
   \baselineskip\z@skip \lineskiplimit-\maxdimen
   \ialign{\hfil$\m@th#1##$\hfil\cr#2\crcr}}}
    \newcommand{\charfusion}[3][\mathord]{
        #1{\ifx#1\mathop\vphantom{#2}\fi
        \mathpalette\mov@rlay{#2\cr#3}}
        \ifx#1\mathop\expandafter\displaylimits\fi
    }
\makeatother

% Sha symbol.
%   Source: https://tex.stackexchange.com/questions/124738/i-just-want-to-write-sha-without-ruining-everything
\DeclareFontFamily{U}{wncy}{}
\DeclareFontShape{U}{wncy}{m}{n}{<->wncyr10}{}
\DeclareSymbolFont{mcy}{U}{wncy}{m}{n}
\DeclareMathSymbol{\Sha}{\mathord}{mcy}{"58}

% Define a bigger \cdot.
%   Source: https://tex.stackexchange.com/questions/235118/making-a-thicker-cdot-for-dot-product-that-is-thinner-than-bullet
\makeatletter
\newcommand*\bigcdot{\mathpalette\bigcdot@{.5}}
\newcommand*\bigcdot@[2]{\mathbin{\vcenter{\hbox{\scalebox{#2}{$\m@th#1\bullet$}}}}}
\makeatother

% Bold math symbols from text greeks:
\newcommand{\mathbfup}[1]{\mathord{\textnormal{\textbf{#1}}}}

% Bold characters:
\renewcommand{\vec}[1]{\boldsymbol{#1}}
\newcommand{\vecu}[1]{\hat{\vec{#1}}}
\newcommand{\mat}[1]{\vec{#1}}

% Predefined matrices and vectors:
\newcommand{\vnull}{\mathbf{0}}

\newcommand{\va}{\mathbf{a}}
\newcommand{\vb}{\mathbf{b}}
\newcommand{\vc}{\mathbf{c}}
\newcommand{\vd}{\mathbf{d}}
\newcommand{\ve}{\mathbf{e}}
\newcommand{\vf}{\mathbf{f}}
\newcommand{\vg}{\mathbf{g}}
\newcommand{\vh}{\mathbf{h}}
\newcommand{\vi}{\mathbf{i}}
\newcommand{\vj}{\mathbf{j}}
\newcommand{\vk}{\mathbf{k}}
\newcommand{\vl}{\mathbf{l}}
\newcommand{\vm}{\mathbf{m}}
\newcommand{\vn}{\mathbf{n}}
\newcommand{\vo}{\mathbf{o}}
\newcommand{\vp}{\mathbf{p}}
\newcommand{\vq}{\mathbf{q}}
\newcommand{\vr}{\mathbf{r}}
\newcommand{\vs}{\mathbf{s}}
\newcommand{\vt}{\mathbf{t}}
\newcommand{\vu}{\mathbf{u}}
\newcommand{\vv}{\mathbf{v}}
\newcommand{\vw}{\mathbf{w}}
\newcommand{\vx}{\mathbf{x}}
\newcommand{\vy}{\mathbf{y}}
\newcommand{\vz}{\mathbf{z}}

% Must be enclosed in another group.
\newcommand{\vmu}{{\mathbfup{\textmu}}}
\newcommand{\vtheta}{{\bm{\uptheta}}}
\newcommand{\vphi}{{\mathbfup{\textphi}}}
\newcommand{\vtau}{{\mathbfup{\texttau}}}
\newcommand{\vep}{{\mathbfup{\textepsilon}}}
\newcommand{\vth}{{\mathbfup{\texttheta}}}

\newcommand{\mA}{\mathbf{A}}
\newcommand{\mB}{\mathbf{B}}
\newcommand{\mC}{\mathbf{C}}
\newcommand{\mD}{\mathbf{D}}
\newcommand{\mE}{\mathbf{E}}
\newcommand{\mF}{\mathbf{F}}
\newcommand{\mG}{\mathbf{G}}
\newcommand{\mH}{\mathbf{H}}
\newcommand{\mI}{\mathbf{I}}
\newcommand{\mJ}{\mathbf{J}}
\newcommand{\mK}{\mathbf{K}}
\newcommand{\mL}{\mathbf{L}}
\newcommand{\mM}{\mathbf{M}}
\newcommand{\mN}{\mathbf{N}}
\newcommand{\mO}{\mathbf{O}}
\newcommand{\mP}{\mathbf{P}}
\newcommand{\mQ}{\mathbf{Q}}
\newcommand{\mR}{\mathbf{R}}
\newcommand{\mS}{\mathbf{S}}
\newcommand{\mT}{\mathbf{T}}
\newcommand{\mU}{\mathbf{U}}
\newcommand{\mV}{\mathbf{V}}
\newcommand{\mW}{\mathbf{W}}
\newcommand{\mX}{\mathbf{X}}
\newcommand{\mY}{\mathbf{Y}}
\newcommand{\mZ}{\mathbf{Z}}

\newcommand{\mSigma}{\mathbfup{\textSigma}}

% Letters commonly used in blackboard bold font:
\newcommand{\Ab}{\mathbb{A}}
\newcommand{\Bb}{\mathbb{B}}
\newcommand{\Db}{\mathbb{D}}
\newcommand{\C}{\mathbb{C}}
\newcommand{\E}{\mathbb{E}}
\newcommand{\Hb}{\mathbb{H}}
\newcommand{\Jb}{\mathbb{J}}
\newcommand{\K}{\mathbb{K}}
\newcommand{\Lb}{\mathbb{L}}
\newcommand{\Mb}{\mathbb{M}}
\newcommand{\N}{\mathbb{N}}
\renewcommand{\P}{\mathbb{P}}
\newcommand{\Q}{\mathbb{Q}}
\newcommand{\R}{\mathbb{R}}
\newcommand{\eR}{\overline{\mathbb{R}}}
\newcommand{\Sb}{\mathbb{S}}
\newcommand{\Tb}{\mathbb{T}}
\newcommand{\V}{\mathbb{V}}
\newcommand{\Z}{\mathbb{Z}}

% Letters commonly used in calligraphic font:
\newcommand{\A}{\mathcal{A}}
\newcommand{\B}{\mathcal{B}}
\newcommand{\Cc}{\mathcal{C}}
\newcommand{\D}{\mathcal{D}}
\newcommand{\Ec}{\mathcal{E}}
\newcommand{\F}{\mathcal{F}}
\newcommand{\G}{\mathcal{G}}
\let\Haccent\H
\renewcommand{\H}{\mathcal{H}}
\newcommand{\J}{\mathcal{J}}
\newcommand{\Jrat}{\mathcal{J}_{\text{rat}}}
\newcommand{\Kc}{\mathcal{K}}
\renewcommand{\L}{\mathcal{L}}
\newcommand{\Li}{\mathcal{L}{}^1}
\newcommand{\eLi}{\overline{\mathcal{L}}{}^1}
\newcommand{\m}{\text{m}}
\newcommand{\M}{\mathcal{M}}
\newcommand{\eM}{\overline{\mathcal{M}}}
\newcommand{\Nc}{\mathcal{N}}
\newcommand{\Oc}{\mathcal{O}}
\newcommand{\Pc}{\mathcal{P}}
\newcommand{\Qc}{\mathcal{Q}}
\newcommand{\Rc}{\mathcal{R}}
\renewcommand{\S}{\mathcal{S}}
\newcommand{\Tc}{\mathcal{T}}
\newcommand{\U}{\mathcal{U}}
\newcommand{\Vc}{\mathcal{V}}
\newcommand{\W}{\mathcal{W}}
\newcommand{\X}{\mathcal{X}}
\newcommand{\Y}{\mathcal{Y}}
\newcommand{\Zc}{\mathcal{Z}}

% Letters sometimes used with a wide tilde:
\newcommand{\tpi}{\widetilde{\pi}}
\newcommand{\ttau}{\widetilde{\tau}}
\newcommand{\tvtau}{\widetilde{\vtau}}
\newcommand{\tD}{\widetilde{\D}}
\newcommand{\tX}{\widetilde{\X}}
\newcommand{\tI}{\widetilde{I}}
\newcommand{\tQc}{\widetilde{\Qc}}

% Letters commonly used in sans serif font:
\newcommand{\T}{\text{\normalshape\textsf{T}}}
\newcommand{\Ht}{\text{\normalshape\textsf{H}}}
\renewcommand{\c}{\text{\normalshape\textsf{c}}}

% Symbols:
\newcommand{\es}{\varnothing}
\newcommand{\e}{\varepsilon}
\newcommand{\sub}{\subseteq}
\renewcommand{\d}{\partial}
\renewcommand{\th}{\theta}
\newcommand{\Th}{\Theta}

% Substitute l for \ell throughout.
%   Source: https://tex.stackexchange.com/questions/1975/substituting-character-l-with-ell-throughout-math-mode
\mathcode`l="8000
\begingroup
\makeatletter
\lccode`\~=`\l
\DeclareMathSymbol{\lsb@l}{\mathalpha}{letters}{`l}
\lowercase{\gdef~{\ifnum\the\mathgroup=\m@ne \ell \else \lsb@l \fi}}%
\endgroup

% Convergence symbols:
\newcommand{\oto}[1]{\overset{#1}{\to}}
\newcommand{\uto}[1]{\underset{#1}{\to}}
\newcommand{\outo}[2]{\overset{#1}{\underset{#2}{\to}}}
\newcommand{\Lto}[1]{\oto{\L^{#1}}}
\newcommand{\weakto}{\rightharpoonup}
\newcommand{\distto}{\oto{\text{d}}}
\newcommand{\weakstarto}{\overset{\ast}{\rightharpoonup}}
% Big-O notation:
\renewcommand{\O}{O}

% Equality symbols:
\newcommand{\disteq}{\overset{\text{\normalshape d}}{=}}

% Operators:
\DeclareMathOperator*{\argmax}{arg\,max}
\DeclareMathOperator*{\argmin}{arg\,min}
\let\max\relax\DeclareMathOperator*{\max}{max}
\let\min\relax\DeclareMathOperator*{\min}{min}
\let\sup\relax\DeclareMathOperator*{\sup}{sup}
\let\inf\relax\DeclareMathOperator*{\inf}{inf\vphantom{p}}
\let\limsup\relax\DeclareMathOperator*{\limsup}{lim\,sup}
\let\liminf\relax\DeclareMathOperator*{\liminf}{lim\,inf\vphantom{p}}
\DeclareMathOperator*{\esssup}{ess\,sup}
\DeclareMathOperator*{\essinf}{ess\,inf}
\newcommand{\AC}{\operatorname{AC}}
\newcommand{\atan}{\operatorname{atan}}
\newcommand{\Breg}{\operatorname{D}}
\newcommand{\BV}{\operatorname{BV}}
\newcommand{\card}{\#}
\newcommand{\cconv}{\circledast}
\newcommand{\chol}{\operatorname{chol}}
\newcommand{\ch}{\operatorname{ch}}
\newcommand{\cliques}{\operatorname{cliques}}
\newcommand{\cl}{\operatorname{cl}}
\newcommand{\col}{\operatorname{col}}
\newcommand{\comp}{\circ}
\newcommand{\contains}{\supseteq}
\newcommand{\conv}{\ast}
\newcommand{\cov}{\operatorname{cov}}
\newcommand{\var}{\operatorname{var}}
\renewcommand{\deg}{\operatorname{deg}}
\renewcommand{\det}{\operatorname{det}}
\newcommand{\diag}{\operatorname{diag}}
\renewcommand{\dim}{\operatorname{dim}}
\newcommand{\diam}{\operatorname{diam}}
\renewcommand{\div}{\operatorname{div}}
\newcommand{\dist}{\operatorname{dist}}
\newcommand{\dom}{\operatorname{dom}}
\newcommand{\dotcup}{\charfusion[\mathbin]{\cup}{\cdot}}
\newcommand{\dotunion}{\charfusion[\mathop]{\bigcup}{\cdot}}
\newcommand{\dsum}{\oplus}
\newcommand{\Dsum}{\bigoplus}
\newcommand{\erf}{\operatorname{erf}}
\newcommand{\epi}{\operatorname{epi}}
\newcommand{\had}{\odot}
\newcommand{\Hell}{\operatorname{D}_{\text{Hell}}}
\newcommand{\hypo}{\operatorname{hypo}}
\newcommand{\im}{\operatorname{im}}
\renewcommand{\Im}{\operatorname{Im}}
\newcommand{\ind}{\mathbbm{1}}
\newcommand{\intersection}{\bigcap}
\newcommand{\join}{\lor}
\newcommand{\KL}{\operatorname{KL}}
\newcommand{\KOT}{\operatorname{KOT}}
\newcommand{\kron}{\otimes}
\newcommand{\len}{\operatorname{len}}
\newcommand{\Lip}{\operatorname{Lip}}
\newcommand{\meet}{\land}
\newcommand{\MOT}{\operatorname{MOT}}
\newcommand{\nullspace}{\operatorname{null}}
\newcommand{\Per}{\operatorname{Per}}
\newcommand{\push}{_\#}
\newcommand{\poly}{\operatorname{poly}}
\newcommand{\rank}{\operatorname{rank}}
\renewcommand{\Re}{\operatorname{Re}}
\newcommand{\row}{\operatorname{row}}
\newcommand{\schur}{\operatorname{Schur}}
\newcommand{\sign}{\operatorname{sign}}
\newcommand{\sinth}{\operatorname{\sin\!\text{-}\Theta}}
\newcommand{\Sinth}{\operatorname{\text{Sin-}\Theta}}
\newcommand{\supp}{\operatorname{supp}}
\newcommand{\symdiff}{\bigtriangleup}
\newcommand{\Tan}{\operatorname{Tan}}
\newcommand{\tensor}{\otimes}
\newcommand{\TV}{\operatorname{TV}}
\newcommand{\tr}{\operatorname{tr}}
\newcommand{\union}{\bigcup}
\newcommand{\vecspan}{\operatorname{span}}
\newcommand{\vect}{\operatorname{vec}}
\newcommand{\vol}{\operatorname{vol}}

% Probability distributions:
\newcommand{\Ber}{\operatorname{Ber}}
\newcommand{\Beta}{\operatorname{Beta}}
\newcommand{\Bin}{\operatorname{Bin}}
\newcommand{\BM}{\operatorname{BM}}
\newcommand{\Cat}{\operatorname{Cat}}
\newcommand{\Cauchy}{\operatorname{Cauchy}}
\newcommand{\Dir}{\operatorname{Dir}}
\newcommand{\Exp}{\operatorname{Exp}}
\newcommand{\Gam}{\operatorname{Gamma}}
\newcommand{\Geom}{\operatorname{Geom}}
\newcommand{\GP}{\mathcal{GP}}
\newcommand{\InvWishart}{\mathcal{W}^{-1}}
\newcommand{\Laplace}{\operatorname{Laplace}}
\newcommand{\LogNormal}{\log\mathcal{N}}
\newcommand{\Mult}{\operatorname{Mult}}
\newcommand{\NegBin}{\operatorname{NegBin}}
\newcommand{\Normal}{\mathcal{N}}
\newcommand{\Poisson}{\operatorname{Poisson}}
\newcommand{\Rad}{\operatorname{Rad}}
\newcommand{\Unif}{\operatorname{Unif}}
\newcommand{\Wishart}{\mathcal{W}}
\newcommand{\WN}{\operatorname{WN}}

\newcommand{\simiid}{\overset{\text{i.i.d.}}{\sim}}

% Probability commands:
\newcommand{\Var}{\V}
\newcommand{\Lik}{\L}
\newcommand{\iidsim}{\overset{\text{\tiny{i.i.d.}}}{\sim}}

% Special math commands:
\newcommand{\cond}{\,|\,}                  % Conditioning
\newcommand{\middlecond}{\,\middle|\,}     % Conditioning between a \left and \right
\newcommand{\divsep}{\,\|\,}               % Separator in divergences
\newcommand{\middledivsep}{\,\middle\|\,}  % Separator in divergences ebtween a \left and a \right

\newcommand{\sd}[1]{\mathrm{d} #1}         % Straight 'd'
\newcommand{\isd}[1]{\, \mathrm{d} #1}     % Straight 'd' in integral (with spacing)
% Straight 'd' in integral with spacing and absolute value around it
\newcommand{\asd}[1]{\, |\mathrm{d} #1|}

\newcommand{\idf}{\text{\textsf{id}}}            % Identity function
\newcommand{\sce}{\text{\sc{e}}}                 % Scientific notation
\newcommand{\vardot}{\,\bigcdot\,}               % Variable dot
\let\ssorig\ss
\renewcommand{\ss}[1]{_{\text{\normalshape #1}}} % Text subscripts
\newcommand{\us}[1]{^{\text{\normalshape #1}}}   % Text superscripts

% Hard to type words:
\newcommand{\cadlag}{c\`adl\`ag}
\newcommand{\Cadlag}{C\`adl\`ag}
\newcommand{\ito}{It\^o}
\newcommand{\Ito}{\ito}
\newcommand{\levy}{L\`evy}
\newcommand{\Levy}{\levy}

% Subscripts
\newcommand{\loc}{_{\text{loc}}}
\newcommand{\lip}{_{\text{Lip}}}

% Half rectangle:
\newcommand{\halfrect}[2]{[\![#1,#2)\hspace{-1pt}\!)}
\newcommand{\Rint}{\text{(R)}\int}

% Paired delimiters:
\DeclarePairedDelimiter\parens{(}{)}             % Parentheses
\DeclarePairedDelimiter\sbrac{[}{]}              % Square brackets
\DeclarePairedDelimiter\cbrac{\{}{\}}            % Curly braces
\DeclarePairedDelimiter\set{\{}{\}}              % Set
\DeclarePairedDelimiter\lra{\langle}{\rangle}    % Angle brackets
\DeclarePairedDelimiter\dlra
    {\langle\!\langle}{\rangle\!\rangle}         % Double angle brackets
\DeclarePairedDelimiter\floor{\lfloor}{\rfloor}  % Floor
\DeclarePairedDelimiter\ceil{\lceil}{\rceil}     % Ceil
\DeclarePairedDelimiter\norm{\|}{\|}             % Norm
\DeclarePairedDelimiter\abs{|}{|}                % Absolute value

% Shortcuts for angle brackets:
\newcommand{\la}{\langle}
\newcommand{\ra}{\rangle}
\newcommand{\lla}{\left\langle}
\newcommand{\rra}{\right\rangle}

% Other commands:
\newcommand{\resp}[1]{[#1]}                  % Respective statement
\newcommand{\bs}{\textbackslash}             % Blackslash in text
% Use \cref in a title.
\newcommand{\creftitle}[2]{\texorpdfstring{\cref{#2}}{#1 \ref{#2}}}
\newcommand{\mcheckthis}{{}^{[\checkmark]}}  % Checkmark in math
\newcommand{\checkthis}{$\mcheckthis$}       % Checkmark in text

% Compatibility with old commands:
\newcommand{\id}[1]{\sd{#1}}
\newcommand{\sargmax}[1]{\argmin_{#1}}
\newcommand{\sargmin}[1]{\argmax_{#1}}
\newcommand{\lrset}[1]{\set*{#1}}
\newcommand{\middleCond}{\middlecond}
\newcommand{\rel}[2]{($(#1) + (#2)$)}
\newcommand{\blackLinks}{}
\newcommand{\Ac}{\A}

\newcommand{\notheorems}{}  % We'll configure our own numbering below.
% Patch \listoftheorems.
%   Source: https://tex.stackexchange.com/questions/249963/remove-repeated-theorem-in-the-list-of-theorems
\makeatletter
\patchcmd\thmt@mklistcmd
    {\thmt@thmname}
    {\check@optarg{\thmt@thmname}}
    {}{}
\patchcmd\thmt@mklistcmd
    {\thmt@thmname\ifx}
    {\check@optarg{\thmt@thmname}\ifx}
    {}{}
\protected\def\check@optarg#1{%
    \@ifnextchar\thmtformatoptarg\@secondoftwo{#1}%
}
\makeatother

% Define lists of things commands.
\let\oldlistoftheorems\listoftheorems
\newcommand{\listofmodels}{
    \renewcommand{\listtheoremname}{List of Models}
    \oldlistoftheorems[ignoreall, show={model}]
}
\newcommand{\listofstatements}{
    \renewcommand{\listtheoremname}{List of Mathematical Statements}
    \oldlistoftheorems[ignoreall, show={theorem,corollary,proposition,lemma,fact}]
}
\renewcommand{\listoftheorems}{
    \renewcommand{\listtheoremname}{List of Theorems}
    \oldlistoftheorems[ignoreall, show={theorem}]
}

% Define environments.
\newlength{\thmtopsep}\setlength{\thmtopsep}{\topsep}
\newlength{\thmbotsep}\setlength{\thmbotsep}{\topsep}
\newtheoremstyle{theoremstyle}
    {\thmtopsep}{\thmbotsep}
    {}           % Body font
    {}           % Indent amount
    {\bfseries}  % Theorem head font
    {.}          % Punctuation after theorem head
    {.5em}       % Space after theorem head
    {}           % Theorem head spec
\theoremstyle{theoremstyle}

\ifcsname notheorems\endcsname
\else
    \newtheorem{theorem}{Theorem}[section]
    \newtheorem{proposition}{Proposition}[section]
    \newtheorem{corollary}{Corollary}[section]
    \newtheorem{fact}{Fact}[section]
    \newtheorem{lemma}{Lemma}[section]

    \newtheorem{assumption}{Assumption}[section]
    \newtheorem{definition}{Definition}[section]
    \newtheorem{question}{Question}[section]
    \newtheorem{example}{Example}[section]
    \newtheorem{model}{Model}[section]
    \newtheorem{remark}{Remark}[section]
\fi

% Set referencing formats.
\crefname{assumption}{Assumption}{Assumptions}
\Crefname{assumption}{Assumption}{Assumptions}
\crefname{corollary}{Corollary}{Corollaries}
\Crefname{corollary}{Corollary}{Corollaries}
\crefname{definition}{Definition}{Definitions}
\Crefname{definition}{Definition}{Definitions}
\crefname{example}{Example}{Examples}
\Crefname{example}{Example}{Examples}
\crefname{fact}{Fact}{Facts}
\Crefname{fact}{Fact}{Facts}
\crefname{lemma}{Lemma}{Lemmas}
\Crefname{lemma}{Lemma}{Lemmas}
\crefname{model}{Model}{Models}
\Crefname{model}{Model}{Models}
\crefname{proposition}{Proposition}{Propositions}
\Crefname{proposition}{Proposition}{Propositions}
\crefname{question}{Question}{Questions}
\Crefname{question}{Question}{Questions}
\crefname{remark}{Remark}{Remarks}
\Crefname{remark}{Remark}{Remarks}
\crefname{theorem}{Theorem}{Theorems}
\Crefname{theorem}{Theorem}{Theorems}

% Referentiable list items in environments
\newlist{asslist}{enumerate}{1}
\setlist[asslist]{
    ref=\theassumption.(\arabic*),
    label=(\arabic*),
    % topsep=0pt,
}
\crefname{asslisti}{Assumption}{Assumptions}
\Crefname{asslisti}{Assumption}{Assumptions}
\newlist{corlist}{enumerate}{1}
\setlist[corlist]{
    ref=\thecorollary.(\arabic*),
    label=(\arabic*),
    % topsep=0pt,
}
\crefname{corlisti}{Corollary}{Corollaries}
\Crefname{corlisti}{Corollary}{Corollaries}
\newlist{deflist}{enumerate}{1}
\setlist[deflist]{
    ref=\thedefinition.(\arabic*),
    label=(\arabic*),
    % topsep=0pt,
}
\crefname{deflisti}{Definition}{Definitions}
\Crefname{deflisti}{Definition}{Definitions}
\newlist{exlist}{enumerate}{1}
\setlist[exlist]{
    ref=\theexample.(\arabic*),
    label=(\arabic*),
    % topsep=0pt,
}
\crefname{exlisti}{Example}{Examples}
\Crefname{exlisti}{Example}{Examples}
\newlist{factlist}{enumerate}{1}
\setlist[factlist]{
    ref=\thefact.(\arabic*),
    label=(\arabic*),
    % topsep=0pt,
}
\crefname{factlisti}{Fact}{Facts}
\Crefname{factlisti}{Fact}{Facts}
\newlist{lemlist}{enumerate}{1}
\setlist[lemlist]{
    ref=\thelemma.(\arabic*),
    label=(\arabic*),
    % topsep=0pt,
}
\crefname{lemlisti}{Lemma}{Lemmas}
\Crefname{lemlisti}{Lemma}{Lemmas}
\newlist{modlist}{enumerate}{1}
\setlist[modlist]{
    ref=\themodel.(\arabic*),
    label=(\arabic*),
    % topsep=0pt,
}
\crefname{modlisti}{Model}{Models}
\Crefname{modlisti}{Model}{Models}
\newlist{proplist}{enumerate}{1}
\setlist[proplist]{
    ref=\theproposition.(\arabic*),
    label=(\arabic*),
    % topsep=0pt,
}
\crefname{proplisti}{Proposition}{Propositions}
\Crefname{proplisti}{Proposition}{Propositions}
\newlist{qlist}{enumerate}{1}
\setlist[qlist]{
    ref=\theremark.(\arabic*),
    label=(\arabic*),
    % topsep=0pt,
}
\crefname{qlisti}{Question}{Questions}
\Crefname{qlisti}{Question}{Questions}
\newlist{remlist}{enumerate}{1}
\setlist[remlist]{
    ref=\theremark.(\arabic*),
    label=(\arabic*),
    % topsep=0pt,
}
\crefname{remlisti}{Remark}{Remarks}
\Crefname{remlisti}{Remark}{Remarks}
\newlist{thmlist}{enumerate}{1}
\setlist[thmlist]{
    ref=\thetheorem.(\arabic*),
    label=(\arabic*),
    % topsep=0pt,
}
\crefname{thmlisti}{Theorem}{Theorems}
\Crefname{thmlisti}{Theorem}{Theorems}

% Reference numbers in the list.
\newcommand{\listnum}[1]{(#1)}
\newcommand{\listimp}[2]{\listnum{#1} $\Rightarrow$ \listnum{#2}}
\newcommand{\listeq}[2]{\listnum{#1} $\Leftrightarrow$ \listnum{#2}}

% Backward compatibility:
\newcommand{\listimpl}[2]{\listnum{#1} $\Rightarrow$ \listnum{#2}:}


% We actually will use coloured links.
\definecolor{aquamarine}{HTML}{218274}
\hypersetup{
    colorlinks,
    citecolor = aquamarine,
    filecolor = aquamarine,
    linkcolor = aquamarine,
    urlcolor  = aquamarine,
}

% Use italic font in bodies for clarity.
\newtheoremstyle{theoremstyle}
    {\thmtopsep}{\thmbotsep}
    {\itshape}   % Body font
    {}           % Indent amount
    {\bfseries}  % Theorem head font
    {.}          % Punctuation after theorem head
    {.5em}       % Space after theorem head
    {}           % Theorem head spec
\theoremstyle{theoremstyle}

% Set numbering of theorems right.
\newtheorem{theorem}{Theorem}[chapter]
\newtheorem{proposition}[theorem]{Proposition}
\newtheorem{corollary}[theorem]{Corollary}
\newtheorem{fact}[theorem]{Fact}
\newtheorem{lemma}[theorem]{Lemma}

\newtheorem{assumption}[theorem]{Assumption}
\newtheorem{definition}[theorem]{Definition}
\newtheorem{procedure}[theorem]{Procedure}

\newtheorem{question}[theorem]{Question}
\newtheorem{example}[theorem]{Example}
\newtheorem{model}[theorem]{Model}
\newtheorem{remark}[theorem]{Remark}

% Hack `\listoftheorem` to remove the title.
\makeatletter
\renewcommand\listoftheorems[1][]{%
    \begingroup
    \setlisttheoremstyle{#1}%
    \let\listfigurename\listtheoremname
    \def\contentsline##1{%
        \csname thmt@contentsline@##1\endcsname{##1}%
    }%
    \@for\thmt@envname:=\thmt@allenvs\do{%
        \thmtlo@newentry
    }%
    \let\thref@starttoc\@starttoc
    \def\@starttoc##1{\thref@starttoc{loe}}%
    \@fileswfalse
    \AtEndDocument{%
        \if@filesw
        \@ifundefined{tf@loe}{%
            \expandafter\newwrite\csname tf@loe\endcsname
            \immediate\openout \csname tf@loe\endcsname \jobname.loe\relax
        }{}%
        \fi
    }%
    \@starttoc{lof}
    \endgroup
}
\makeatother

% Fix spacing around proofs.
\xpatchcmd{\proof}{\topsep6\p@\@plus6\p@\relax}{}{}{}
\BeforeBeginEnvironment{proof}{\vspace{-0.5em}}
\AfterEndEnvironment{proof}{\vspace{-0.5em}}

% Commands specific for thesis:
\newcommand{\AR}{\operatorname{AR}}
\newcommand{\enc}{\mathsf{enc}}
\newcommand{\dec}{\mathsf{dec}}
\newcommand{\mult}{\operatorname{mult}}

\newcommand{\missingcite}{\note{(CITATION MISSING)}}

% Listings:
\usepackage{tcolorbox}
\tcbuselibrary{minted,skins,breakable}

\definecolor{solarized-light-bg}{HTML}{fdf6e3}
\definecolor{solarized-light-fg}{HTML}{586e75}
\newtcblisting{pythoncode}[2]{
    listing engine = minted,
    listing only,
    minted style = solarized-light,
    minted language = python,
    minted options = {
        fontsize = #1,
        escapeinside = ||,
        mathescape = true,
        highlightlines = #2,
        highlightcolor = red,
    },
    colback = solarized-light-bg,
    colframe = solarized-light-bg,
    toprule = 0pt,
    left = 5pt,
    left = 5pt,
    leftrule = 0pt,
    rightrule = 0pt,
    bottomrule = 0pt,
    arc = 0pt,
    frame hidden,
    breakable,
}
% Disable italics.
\AtBeginEnvironment{pythoncode}{\let\itshape\relax}

% Override the textwriter font.
%\usepackage[scaled=0.8]{beramono}
\usepackage[scaled=0.95]{inconsolata}
\renewcommand{\code}[1]{\colorbox{solarized-light-bg}{\small\color{solarized-light-fg}\texttt{#1}}}

% We never want footnotes to break across pages.
\interfootnotelinepenalty=10000

% Allow restatable environments.
\newenvironment{manual}[3][]{%
    \def\savedarg{#2}%
    \expandafter\renewcommand\csname the#2\endcsname{#3}%
    \ifempty{#1}{\csname #2\endcsname}{\csname #2\endcsname[#1]}%
}{%
    \csname end\savedarg\endcsname%
    \addtocounter{\savedarg}{-1}%
}
\newcommand{\statement}[1]{
    \begingroup
        \subimport{}{#1}
        \ifempty{\statementoption}{
            \csname\statementtype\endcsname
        }{
            \expandafter\csname\statementtype\endcsname[\statementoption]%
        }
        \label{\statementlabel}
        \statementcontent
        \csname end\statementtype\endcsname
    \endgroup
}
\newcommand{\restatement}[1]{
    \begingroup
        \subimport{}{#1}
        \ifempty{\statementoption}{
            \manual{\statementtype}{\ref{\xrprefix{\statementlabel}}}
        }{
            \manual[\statementoption]{\statementtype}{\ref{\xrprefix{\statementlabel}}}
        }
        \newcommand{\insiderestatement}{}
        \renewcommand{\label}[1]{}
        \statementcontent
        \endmanual
    \endgroup
}

%\newcommand{\ballnumber}[1]{\tikz[baseline=(myanchor.base)] \node[circle,fill=.,inner sep=1pt] (myanchor) {\color{-.}\normalshape\bfseries\footnotesize #1};}

\newcommand{\highlight}{\textcolor{}{}}

\addbibresource{../../bibliography.bib}

\usepackage{xr}
\externaldocument[xr-]{../../main}
\newcommand{\xrprefix}[1]{xr-#1}

\begin{document}

\chapter{Prediction Map Approximation}
\label{chap:predmap}

\paragraph{Abstract}
This chapter presents a theoretical framework called \emph{prediction map approximation} to
rigorously analyse neural processes.
The starting point is the definition of the so-called \emph{posterior prediction map}.
The posterior prediction map is the map from a data set to the posterior distribution of a stochastic process given that data set.
It is the main object of interest.
The premise of the framework is that neural processes form various classes of approximations of the posterior prediction map.

\paragraph{Outline}
In \cref{sec:predmap:motivation}, we motivate the key elements of the framework.
Afterwards, in \cref{sec:predmap:prelims}, we introduce the main technical concepts, setting us up for more formal analysis.
%The remainder of the sections will reintroduce neural process, but now more formally.
In \cref{sec:predmap:neural_process_objective}, we formally introduce and analyse the \emph{neural processes objective}.
As the name suggests, this is the objective that neural processes aim to optimise.
Then, in \cref{sec:predmap:np_approximations}, we define \emph{neural process approximations}.
A neural process approximation precisely defines what a class of neural processes targets.
Finally, in \cref{sec:predmap:consistency}, we analyse how a neural process approximation is estimated in practice.
We engage in a discussion about convergence.

\paragraph{Attributions and relationship to prior work}
Prediction maps and translation equivariance of prediction were first considered by \fulltextcite{Foong:2020:Meta-Learning_Stationary_Stochastic_Process_Prediction} and later further analysed by \fulltextcite{Bruinsma:2021:The_Gaussian_Neural_Process}.
Although the analysis of this chapter has not been published in its current form,
the analysis borrows considerably from \textcite{Bruinsma:2021:The_Gaussian_Neural_Process}.
\looseness=-1
The analysis by \textcite{Bruinsma:2021:The_Gaussian_Neural_Process} was primarily
conducted by the author and James Requeima and checked by Jonathan Gordon and Andrew Y.\ K.\ Foong.
All work was supervised by Richard E.\ Turner.

\section{Introduction}
\label{sec:predmap:motivation}

In this chapter, we develop a theoretical framework to rigorously analyse neural processes. 
Building on the idea of a \emph{prediction map} (\cref{\xrprefix{def:prediction_map}}),
this framework is called \emph{prediction map approximation}.
The primary goal of this chapter is an attempt to reduce the gap between the theory and practice of neural processes:
to formally present neural processes to the more theoretically minded audience, and to bring subtle but important theoretical issues to the attention of the more practically oriented.

In the current literature, theoretical analyses of meta-learning use a variety of techniques to study generalisation of meta-learning algorithms:
\textcite{Baxter:1998:Theoretical_Models_of_Learning_to_Learn,
Baxter:2000:A_Model_of_Inductive_Bias_Learning}
use classical learning-theoretic techniques;
\textcite{Maurer:2005:Algorithmic_Stability_and_Meta-Learning}
investigates algorithmic stability;
\textcite{Pentina:2014:A_PAC-Bayesian_Bound_for_Lifelong,
Alquier:2017:Regret_Bounds_for_Lifelong_Learning,
Amit:2018:Meta-Learning_by_Adjusting_Priors_Based,
Yin:2020:Meta-Learning_Without_Memorization,
Liu:2021:PAC-Bayes_Bounds_for_Meta-Learning_With,
Rothfuss:2021:PACOH_Bayes-Optimal_Meta-Learning_With_PAC-Guarantees}
use PAC-Bayes bounds;
\textcite{Farid:2021:Generalization_Bounds_for_Meta-Learning_via}
combine algorithmic stability with PAC-Bayes bounds;
\textcite{Jose:2021:Information-Theoretic_Generalization_Bounds_for_Meta-Learning,
Chen:2021:Generalization_Bounds_for_Meta-Learning_An,
Rezazadeh:2021:Conditional_Mutual_Information-Based_Generalization_Bound}
use bounds based on mutual information, which are similar to PAC-Bayes bounds; and
\textcite{Khodak:2019:Provable_Guarantees_for_Gradient-Based_Meta-Learning,
Denevi:2019:Online-Within-Online_Meta-Learning}
apply tools from convex analysis.
Although the theoretical definitions that we propose in this chapter could be used to engage in similar analyses for neural processes,
we will not get that far.
We will spend our effort on establishing a solid theoretical grounding for neural processes.

\index{ground truth}
Henceforth, assume the setting of a meta-learning problem as presented in \cref{\xrprefix{sec:nps:introduction}}.
To motivate the theoretical framework, the starting point is a reasonable generative model for the meta--data set $(D_m)_{m=1}^M$.
Specifically, we will assume that the tasks $(D_m)_{m=1}^M$ are independently and identically distributedly (i.i.d.) generated from some noise-corrupted ground-truth $\Y$-valued stochastic process $f$ on $\X$ with law $p(f)$.
That is, for $m = 1, \ldots, M$, independently and identically sample
\begin{align}
    f_m &\sim p(f), \\
    \vy\us{(c)}_m \cond f_m,\, \vx\us{(c)}_m, \vep\us{(c)}_m &= f_m(\vx\us{(c)}_m) + \vep\us{(c)}_m
    \quad\text{with}\quad\vep\us{(c)}_m \sim \Normal(\vnull, \sigma^2_f \mI), \\
    \vy\us{(t)}_m \cond f_m,\, \vx\us{(t)}_m, \vep\us{(t)}_m &= f_m(\vx\us{(t)}_m) + \vep\us{(t)}_m
    \quad\text{with}\quad\vep\us{(t)}_m \sim \Normal(\vnull, \sigma^2_f \mI),
\end{align}
where $\sigma_f > 0$ is the standard deviation of the observation noise.
Importantly, in this setup, the realisation of the ground-truth process is \emph{different for every task}.
In fact, the realisations $(f_m)_{m=1}^M$ are independently and identically drawn from $p(f)$.
That all $f_m$ have the same distribution may seem restrictive, because in practice tasks can be very different, like recognising different bird species.
This can be modelled by letting $p(f)$ be a mixture distribution. 
For example, roll a biased $K$-faced die, and then draw $f$ from one of $K$ different stochastic processes.
We hence see that the assumption of identical distributions is not necessarily so restrictive,
which leaves us with the assumption that all $(f_m)_{m=1}^M$ are drawn independently.
This independence assumption might or might not be applicable to a particular application,
but it is an assumption that we make to simplify the analysis.

Assume that the context inputs $(\vx\us{(c)}_m)_{m=1}^M$ are sampled i.i.d.\ from some distribution $p(\vx\us{(c)})$ and that the target inputs $(\vx\us{(t)}_m)_{m=1}^M$ are sampled i.i.d.\ from possibly a different distribution $p(\vx\us{(t)})$.
The distributions $p(\vy\us{(c)} \cond \vx\us{(c)})$ and $p(\vx\us{(c)})$ define a distribution over context sets $p(D\us{(c)})$.
Henceforth, to simplify the notation, we will drop the scripts $\vardot_m$, $\vardot\us{(c)}$, and $\vardot\us{(t)}$ and denote a context set by $D \sim p(D)$, the ground-truth stochastic process by $f \sim p(f)$, the target inputs by $\vx \sim p(\vx)$, and the target outputs by $\vy \sim p(\vy \cond \vx)$.

Meta-learning algorithms aim to make the best possible prediction for a target set given a context set.
For a meta--data set sampled from the generative model, the best possible prediction is given by the posterior over a target set given the context set.
More specifically, for a context set $D$ and target inputs $\vx$, the desired prediction for the target outputs $\vy$ is given by
$p(f(\vx) + \vep \cond \vx, D)$.
We more concisely define this solution with the \emph{posterior prediction map}.

\index{posterior prediction map}
\begin{definition}[Posterior prediction map]
    \label{def:posterior_prediction_map}
    The \emph{posterior prediction map} is defined by
    \begin{equation}
        \pi_f \colon \D \to \Pc, \quad
        \pi_f(D) = p(f \cond D).
    \end{equation}
\end{definition}

In words, the posterior prediction map maps a data set to the prediction for that data set by the ground-truth stochastic process.
For a $\Y$-valued stochastic process $g$ on $\X$ with law $\mu$ and target inputs $\vx$,
 denote the law of $g(\vx)$ by $P_\vx \mu$.
In addition, for noise standard deviation $\sigma > 0$, the law of $g(\vx)$ plus additive Gaussian noise with variance $\sigma^2$ denote by $P^\sigma_\vx \mu$:
\begin{equation}
    g(\vx) + \vep \sim P^\sigma_\vx \mu
    \quad\text{with}\quad
    \vep \sim \Normal(\vnull, \sigma^2 \mI).
\end{equation}
\looseness=-1
Note that $P^\sigma_\vx \mu$ always has a density with respect to the Lebesgue measure given by $\vx \mapsto \E_g[\Normal(\vy \cond g(\vx), \sigma^2 \mI)]$.
With this notation, to solve the meta-learning problem, for a context set $D$ and target inputs $\vx$, the desired prediction for the target outputs $\vy$ is given by $P^{\sigma_f}_\vx \pi_f(D)$.

The central premise of the prediction map framework is that a neural process aims to approximate the posterior prediction map $\pi_f$.
This means that, for every data set $D \in \D$, the neural process attempts to approximate $\pi_f(D)$.
Let us elaborate on this simple, but important observation.

What a neural process is \emph{not} doing is finding an approximation of the ground-truth stochastic process $f$, and then using this approximate prior to approximate any posterior $p(f \cond D)$.
Instead, a neural process aims to \emph{directly approximate} the posteriors $p(f \cond D)$ without approximating the prior as an intermediate step.
This distinction is best understood by comparing the empirical neural process objective (\cref{\xrprefix{def:empirical_neural_process_objective}}) proposed by \textcite{Garnelo:2018:Conditional_Neural_Processes} to the usual maximum likelihood objective.
Let $D\us{(c)}$ denote some context set, let $D\us{(t)}$ denote some target set, and let $\theta$ be the parameters of some model.
Whereas the usual application of maximum-likelihood estimation would maximise $\log p_\theta(D\us{(c)} , D\us{(t)})$, neural processes maximise only $\log p_\theta(D\us{(t)} \cond D\us{(c)})$.
As the following application of the product rule illustrates, this means that neural processes give up modelling the context data $D\us{(c)}$:\index{neural process objective}
\begin{equation}
    \underbracket[1pt]{
        \log p_\theta(D\us{(c)}, D\us{(t)})
    }_{\mathclap{\substack{
        \\[0.5\baselineskip]
        \displaystyle\text{\small usual maximum-likelihood}\vphantom{\text{\small p}}
        \\[0.5\baselineskip]
        \displaystyle\text{\small objective}
    }}}
    \hspace{15pt}=\hspace{15pt}
    \underbracket[1pt]{
        \log p_\theta(D\us{(t)} \cond D\us{(c)})
    }_{\mathclap{\substack{
        \\[0.5\baselineskip]
        \displaystyle\text{\small neural process}
        \\[0.5\baselineskip]
        \displaystyle\text{\small objective}
    }}}
    \hspace{15pt}+\hspace{15pt}
    \underbracket[1pt]{
        \log p_\theta(D\us{(c)})
    }_{\mathclap{\substack{
        \\[0.5\baselineskip]
        \displaystyle\text{\small what neural processes}
        \\[0.5\baselineskip]
        \displaystyle\text{\small give up}
    }}}
\end{equation}
To see why giving up modelling the context data $D\us{(c)}$ can be helpful,
consider the ground-truth stochastic process given by $f = h_0$ with probability $\tfrac12$ and $f = h_1$ otherwise, where $h_0, h_1 \colon \X \to \Y$ are fixed deterministic functions.
By construction, this prior $p(f)$ is bimodal, allowing only one of two possible realisations.
A posterior, on the other hand, can be much simpler: if the observed data $D$ is able to roughly pin down whether $f$ is $h_0$ or $h_1$, then $p(f \cond D)$ will be a roughly unimodal distribution centred around $h_0$ or $h_1$.
We therefore see that directly approximating a posterior $p(f \cond D)$ can be a simpler problem than trying to model the prior $p(f)$.
But we can make an even stronger case by considering what happens if we do approximate $p(f)$ and then use this approximate prior to approximate posteriors $p(f \cond D)$.
Suppose that we approximate $p(f)$ with a Gaussian process.
%In the limit of infinite data,
One reasonable approximation is given by the Gaussian process with mean function $x \mapsto \E_{p(f)}[f(x)]$ and covariance function $(x, y) \mapsto \cov_{p(f)}(f(x), f(y))$\footnote{
    Specifically, this approximation is the \emph{Gaussian neural process approximation} (GNPA) defined in \cref{def:gnpa} and characterised in \cref{prop:gnpa_characterisation}.
    Intuitively, the GNPA constructs the \emph{moment-matched Gaussian process} by taking the mean function and covariance function of the non-Gaussian process.
    See \cref{sec:predmap:np_approximations} for a more detailed discussion.
}:
$f \approx (\tfrac12 + \tfrac12 Z)h_1 + (\tfrac12 - \tfrac12 Z) h_2$ with $Z \sim N(0, 1)$\footnote{
    Write $f = B h_1 + (1 - B) h_2$ with $B \sim \Ber(\tfrac12)$.
    Then $f - \E[f] = (B - \tfrac12) h_1 - (B - \tfrac12) h_2$, so%
    \begin{align}
        &\E[(f(x) - \E[f(x)])(f(y) - \E[f(y)]) \nonumber \\
        &\qquad= \E[
            ((B - \tfrac12) h_1(x) - (B - \tfrac12) h_2(x))
            ((B - \tfrac12) h_1(y) - (B - \tfrac12) h_2(y))
        ] \\
        &\qquad = \label{eq:outer_product_covariance}
            \tfrac14 (h_1(x) - h_2(x))
            (h_1(y) - h_2(y)), 
    \end{align}
    noting that $\E[(B - \frac12)^2] = \tfrac12 \cdot\tfrac14 + \tfrac12 \cdot \tfrac14 =\tfrac14 = \tfrac12\cdot\tfrac12$.
    The covariance function in \eqref{eq:outer_product_covariance} is the covariance function of $\tfrac12 Z(h_1 - h_2)$ with $Z \sim \Normal(0, 1)$,
    so $f \approx \E[f] + \tfrac12 Z(h_1 - h_2) = (\tfrac12 + \tfrac12 Z) h_1 + (\tfrac12 - \tfrac12 Z) h_2$.
}.
This Gaussian process will cover not just $h_0$ and $h_1$, but it will cover all affine combinations of $h_0$ and $h_1$.
In words, this approximate prior will not just consider $h_0$ and $h_1$ as possible outcomes, but it will also consider all affine combinations of $h_0$ and $h_1$ as possible outcomes!
Since this range of affine combinations is a much bigger collection than just $\set{h_0, h_1}$, approximations of posteriors formed by using this approximate prior will have strongly inflated uncertainty and might give rise to realisations like $0.7 h_0 + 0.3 h_1$, which can be totally unlike $h_0$ or $h_1$.
By giving up modelling the prior using the context set $D\us{(c)}$ and directly approximating the posterior without approximating the prior as an intermediate step, neural processes attempt to circumvent this issue.

To approximate the posterior prediction map $\pi_f$,
the framework will define a loss function $\L(\pi, \sigma)$ which measures how well a candidate approximation $\pi$ and associated noise $\sigma > 0$ approximate the posterior prediction map $\pi_f$ and true noise $\sigma_f$.
The particular loss function that we choose is called the \emph{neural process objective} $\L\ss{NP}$:\index{neural process objective}
\begin{equation}
    \L\ss{NP}(\pi, \sigma)
    = \E_{p(D)p(\vx)}[\KL(P^{\sigma_f}_\vx \pi_f(D), P^{\sigma}_\vx \pi(D))].
\end{equation}
This loss function is called the neural process objective because a Monte Carlo approximation over the tasks $(D_m)_{m=1}^M$
recovers the \emph{empirical} neural process objective (\cref{\xrprefix{def:empirical_neural_process_objective}}) up to a constant that does not depend on $(\pi, \sigma)$.
Consistent with the notation previously introduced for $q$, let $q_\theta(\vardot\cond \vx, D)$ denote the density of $P^\sigma_\vx \pi(D)$ with respect to the Lebesgue measure.
Then\index{empirical neural process objective}
\begin{align}
    \L\ss{NP}(\pi, \sigma)
    &= -\E_{p(D)p(\vx)p(\vy)}[\log q_{\theta}(\vy \cond \vx, D)]
    - \E_{p(D)p(\vx)}[\Hb(P^{\sigma_f}_\vx \pi_f(D))] \label{eq:decomposition_NP_objective} \\
    &\approx -\frac1M \sum_{m=1}^M \log q_{\theta}(\vy_m\us{(t)} \cond \vx\us{(t)}_m, D\us{(c)}_m)
    + \text{constant indep.\ of $(\pi, \sigma)$}
    \label{eq:neural_process_objective_next_to_KL_objective}
\end{align}
which agrees with \cref{\xrprefix{def:empirical_neural_process_objective}}.
In \eqref{eq:decomposition_NP_objective}, $\Hb(P^{\sigma_f}_\vx \pi_f(D))$ denotes the differential entropy of the probability measure $P^{\sigma_f}_\vx \pi_f(D)$.
The neural process objective raises three important points of concern, which we address in turn in the next sections:
\begin{enumerate}
    \item
        The neural process objective $\L\ss{NP}$ is defined by taking an expectation of the function
        $(D, \vx) \mapsto \KL(P^{\sigma_f}_\vx \pi_f(D), P^{\sigma}_\vx \pi(D))$
        over $p(D)p(\vx)$.
        However, it is not at all clear whether this function is even measurable.
        For example, measurability of this function depends on the regularity of $D \mapsto \pi_f(D)$ and $D \mapsto \pi(D)$, which thus far remain unaddressed.
        In \cref{sec:predmap:neural_process_objective},
        we address these issues and carefully define the neural process objective $\L\ss{NP}$.
    \item 
        The neural process objective $\L\ss{NP}$ depends on $p(D)$ and $p(\vx)$.
        Therefore, if we minimise $\L\ss{NP}$ over a potentially restricted collection of prediction maps,
        then the minimiser depends on $p(D)$ and $p(\vx)$.
        In \cref{sec:predmap:np_approximations}, we define two classes of minimisers of $\L\ss{NP}$.
        We characterise these minimisers to determine how they depend on $p(D)$ and $p(\vx)$.
    \item 
        Even though \eqref{eq:neural_process_objective_next_to_KL_objective} is a Monte Carlo approximation of $\L\ss{NP}$, a minimiser of \eqref{eq:neural_process_objective_next_to_KL_objective} need not necessarily converge to a minimiser of $\L\ss{NP}$.
        In \cref{sec:predmap:consistency}, we will determine conditions under which a minimiser of \eqref{eq:neural_process_objective_next_to_KL_objective} converges to a minimiser of $\L\ss{NP}$.
\end{enumerate}

Having motivated the setup of the framework, we proceed to definitions that will form the technical foundation.

\section{Technical Preliminaries}
\label{sec:predmap:prelims}

In \cref{\xrprefix{sec:nps:introduction}}, we defined $\D_N$ as the collection of all data sets of size $N$ and $\D$ as the collection of all data sets of finite size.
It will be convenient to establish similar notation for inputs.
Let $I_N = \X^N$ be the collection of all $N$ inputs, and let $I = \union_{N=1}^\infty I_N$ be the collection of all finite collections of inputs.
\index{metric!data sets}
For a vector $\vx$, recall that $\abs{\vx}$ denotes the dimensionality of $\vx$.
Endow $\D$ with the metric
$d_\D(D_1, D_2) = \norm{\vx_1 - \vx_2}_2 + \norm{\vy_1 - \vy_2}_2$
if $\abs{\vx_1} = \abs{\vx_2}$ and $\infty$ otherwise.
\index{metric!inputs}
Similarly, endow $I$ with the metric
$d_I(\vx_1, \vx_2) = \norm{\vx_1 - \vx_2}_2$
if $\abs{\vx_1} = \abs{\vx_2}$ and $\infty$ otherwise.
Allowing $d_\D$ and $d_I$ to attain the value infinity for arguments of different dimensionality is nothing to worry about.
It simply means that the metric spaces associated to $d_\D$ and $d_I$ naturally break down into disjoint unions of metric spaces of fixed dimensionalities.
Alternatively, we could separately define metrics for each of these fixed-dimensional spaces, but allowing $d_\D$ and $d_I$ to attain infinity is a more concise and arguably more elegant construction.

Whereas in the motivating section the distribution $p(D)$ was induced by randomly sampling context inputs and passing these to a sample of the ground-truth stochastic process $f$,
the prediction map approximation framework more simply assumes that we are just given some distribution $p(D)$.
In some sense, this means that $p(D)$ is \emph{decoupled} from the ground-truth stochastic process $f$.
Similarly, the framework assumes also that we are just given some distribution over $p(\vx)$.
We now define these distributions.

\index{data sets of interest}
\index{target inputs of interest}
Let $\tD \sub \D$ be a collection of \emph{data sets of interest},
and let $\tI \sub I$ be a collection of \emph{target inputs of interest}.
The choices $\tD = \D$ and $\tI = I$ are allowed.
The reason for considering subsets of $\D$ is practical:
we most likely do not have context sets $(D\us{(c)})_{m=1}^M$ that span all possible data sets.
Rather, $(D\us{(c)})_{m=1}^M$ might visit only a particular subspace of $\D$,
and this can now be modelled by constraining $\tD$.
Similarly, we most likely will not sample target inputs of arbitrary size, which can be modelled by constraining $\tI$.
Although $\tI$ and $\tD$ may be chosen arbitrarily,
the approximation properties of neural processes will depend on $\tI$ and $\tD$.

For a choice of $\tD$ and $\tI$,
let $p(D)$ be any distribution on the Borel space of $\tD$ and let $p(\vx)$ be any distribution on the Borel space of $\tI$.
This means that $D \sim p(D)$ is a random element of $\tD$ and $\vx \sim p(\vx)$ is a random element of $\tI$.
We will require two technical conditions on $p(\vx)$ and $p(D)$.

\begin{assumption} \label{assum:full_support_and_complete}
    Assume that $p(\vx)$ and $p(D)$ assign positive probability to every open set.
    Moreover, assume that $p(\vx)$ and $p(D)$ are complete, possibly by completing the probability spaces.
\end{assumption}

A technical subtelty that the framework will have to deal with is that conditional neural processes do \emph{not} define continuous processes.
Namely, conditional neural processes define only marginal statistics and do not specify any dependencies between different function values.
An example of such a prediction is the stochastic process $f$ such that $f(\vx) \sim \Normal(\vnull, \mI)$ for all $\vx \in I$.
\index{Kolmogorov's extension theorem}
By Kolmogorov's extension theorem \parencite[Theorem 12.1.2;][]{Dudley:2002:Real_Analysis_and_Probability}, this construction defines a perfectly valid stochastic process.
The problem, however, is that the lack of regularity requires the process to be defined on the sample space $\Y^\X$ with the associated cylindrical $\sigma$-algebra,
and this $\sigma$-algebra contains too few measurable sets to be useful.
Namely, a set is in the cylindrical $\sigma$-algebra if and only if it depends on only countably many values of $f$;
that is, if it depends on $f(x)$ with $x \in T \sub \X$ where $T$ is countable.
But the space $\Y^\X$ lacks any kind of regularity, so, even for basic properties like continuity, we must put a condition on $f(x)$ for \emph{all} $x \in \X$.
As a consequence, the resulting probability space is not able to express, for example, whether $f$ is continuous or not.

Even though many conditional neural processes are defined on inexpressive probability spaces, we still require some kind of regularity.
For inputs $\vx$, let $\mu_\vx$ denote the distribution of $f(\vx)$.
Instead of assuming continuity of $f$, we will assume that $\vx \mapsto \mu_\vx$ is weakly continuous:
if $\vx_i \to \vx$, then $\mu_{\vx_i} \weakto \mu_{\vx}$ where $\weakto$ denotes convergence in the weak topology.
Note that this condition is automatically satisfied if $f$ is a continuous process, which means that the condition is weaker than pathwise continuity.
Unfortunately, this assumption is still too strong, which we illustrate with an example.
Consider the previous example of $f$ such that $f(\vx) \sim \Normal(\vnull,\mI)$ for all $\vx \in I$;
that is, like conditional neural processes, $f$ does not model dependencies.
Since $\cov(f(1/n), f(0)) \not\to \cov(f(0), f(0))$, $(x, y) \mapsto \cov(f(x), f(y))$ is not continuous at $(x, y) = (0, 0)$.
Consequently, $\vx \mapsto \mu_\vx$ is not weakly continuous at inputs $\vx$ with repeated elements.
Fortunately, this observation is inconsequential,
because the collection of all $\vx$ with repeated elements is a $p(\vx)$-null set whenever, \eg, $p(\vx)$ has a density with respect to the Lebesgue measure.
This motivates the following two definitions.

\index{stochastic process!regularity}
\begin{definition}[Regular stochastic process]
    Call a $\Y$-valued stochastic processes $\mu$ on $\X$ \emph{$p(\vx)$-regular} if $\vx \mapsto \mu_\vx$ is  weakly continuous $p(\vx)$-almost everywhere. 
    That is, $\mu$ is $p(\vx)$-regular if there exists a measurable $A \sub \tI$ such that $\int_A p(\vx) \isd \vx = 1$ and $\vx \mapsto \mu_\vx$ is weakly continuous at all $\vx \in A$.
    Denote the collection of all $p(\vx)$-regular processes by $\Pc$.
\end{definition}

\index{stochastic process!regularity}
\begin{definition}[Continuous stochastic process]
    Let $\Pc\ss{c}$ be the collection of all $\Y$-valued stochastic processes on $\X$ which are continuous.
    Note that $\Pc\ss{c} \sub \Pc$.
\end{definition}

Note that the definition of $\Pc$ depends on our choice of $p(\vx)$, but $\Pc\ss{c}$ does not.
%
Whereas we will assume $p(\vx)$-regularity for predictions of neural processes, for the ground-truth stochastic process $f$ we will require a slightly stronger assumption.
In particular, we will assume that $f$ is H\"older continuous with respect to the $L^p$-norm.

\index{ground truth!regularity}
\begin{assumption}[Regularity of ground-truth stochastic process] \label{assum:f_regularity}
    There exist an exponent $p\ge 2$, H\"older exponent $\beta \in (\tfrac1p, 1]$, constant $c > 0$, and radius $r > 0$ such that
    \begin{equation} \label{eq:f_regularity:Holder_condition}
        \norm{f(x) - f(y)}_{L^p} \le c \abs{x - y}^\beta
        \quad\text{whenever}\quad
        \abs{x - y} < r.
    \end{equation}
    In addition, assume that $f(x) \in L^p$ for all $x \in \X$.
\end{assumption}

By the triangle inequality, it holds that $\norm{f(x)}_{L^p} \le \norm{f(x) - f(y)}_{L^p} + \norm{f(y)}_{L^p}$.
Therefore, by \eqref{eq:f_regularity:Holder_condition}, the additional assumption that $f(x) \in L^p$ for \emph{all} $x \in \X$ is true if $f(y) \in L^p$ is true for \emph{some} $y \in \X$.

\Cref{assum:f_regularity} is a fairly standard assumption, except that we require the exponent $p$ at least be two.
For example, \eqref{eq:f_regularity:Holder_condition} is the assumption in Kolmogorov's continuity criterion, which says that $L^p$-H\"older continuity 
implies pathwise H\"older continuity by giving up $\frac1p$ in the exponent \parencite[Section 4.2, Theorem 1.4.2;][]{Norris:2018:Advanced_Probability}.
In particular, it implies that $f$ is a continuous process, so $f \in \Pc\ss{c}$.
For example, if $f$ is a zero-mean stationary Gaussian process with covariance function $k\colon \X \to \R$, then
\cref{assum:f_regularity} is satisfied if
\begin{equation} \label{eq:limit_condition}
    \lim_{r \to 0} \frac{k(0) - k(r)}{r^{2/p - \e}} = 0
\end{equation}
for some $\e > 0$.
Choosing $p = 4$ and $\e = \tfrac14$ works for many kernels.
More generally, if $k$ is H\"older continuous with any H\"older exponent, then we can always find large enough $p \ge 2$ and small enough $\e > 0$ such that \eqref{eq:limit_condition} is satisfied.

The prediction map framework is centred around approximating the posterior prediction map $\pi_f$ with another simpler, more tractable prediction map $\pi$.
Let $\mu_f$ denote the law of the ground-truth stochastic process $f$.

\index{posterior prediction map!formal}
\begin{definition}[Posterior prediction map, formal]
    \label{def:posterior_prediction_map_formal}
    Let $\sigma_f > 0$ be some observation noise.
    Then define the posterior prediction map $\pi_f\colon \D \to \Pc$
    by the following Radon--Nikodym derivatives:
    for all $D \in \D$,
    \begin{equation} \label{eq:density_posterior_prediction_map}
        \frac{\sd \pi_f(D)}{\sd \mu_f}(f)
        =
        \frac
            {\Normal(\vy \cond f(\vx), \sigma_f^2 \mI)}
            {\E_f[\Normal(\vy \cond f(\vx), \sigma_f^2 \mI)]}
        =
        \frac
            {r(\vy - f(\vx))}
            {Z(\vx, \vy)}
    \end{equation}
    where $D = (\vx, \vy)$,
    $r(\vx) = \exp({-\tfrac{1}{2\sigma_f^2}\norm{\vx}_2^2})$,
    and $Z(\vx, \vy) = \E_f[r(\vy - f(\vx))]$.
\end{definition}

Note that $Z(\vx, \vy) > 0$, for otherwise $\abs{f(x)} = \infty$ with positive probability for some $x \in \X$, contradicting \cref{assum:f_regularity}.
To approximate $\pi_f$, we must be concerned with regularity of $\pi_f$.
The following definition captures the basic form of regularity of a prediction map: continuity.
Recall that $\weakto$ denotes convergence in the weak topology.

\index{prediction map!continuity}
\begin{definition}[Continuous prediction map]
    \label{def:continuous_prediction_map}
    Call a prediction map $\pi\colon \D \to \Pc$ \emph{continuous} if
    $D_i \to D$
    implies that 
    $P_\vx \pi(D_i) \weakto P_\vx \pi(D)$ 
    for all inputs $\vx \in I$. 
    Denote the collection of all prediction maps $\D \to \Pc$ which are continuous by $\M$,
    and denote the collection of all prediction maps $\D \to \Pc\ss{c}$ which are continuous by $\M\ss{c}$.
    Note that $\M\ss{c} \sub \M$.
\end{definition}

The following proposition says that any posterior $\pi_f(D)$ satisfies the same $L^p$-H\"older and integrability condition as $f$ (\cref{assum:f_regularity}) and that $\pi_f$ is indeed continuous.

\index{posterior prediction map!regularity}
\statement{statements/regularity_part_one.tex}
\begin{proof}
    See \cref{\xrprefix{sec:proofs_predmap:motivation}}.
\end{proof}

We have setup the basic definitions of the theoretical framework.
We will now proceed with our formal analysis of neural processes.
Our first matter of concern is the neural process objective.

\section{The Neural Process Objective}
\label{sec:predmap:neural_process_objective}

In this section, we formally define the neural process objective $\L\ss{NP}$ (\cref{def:neural_process_objective}) and the space over which $\L\ss{NP}$ will be optimised (\cref{def:noisy_prediction_maps}).
Intuitively, the neural process objective $\L\ss{NP}$ is the infinite-sample limit of the \emph{empirical} neural process objective $\L_M$ (\cref{\xrprefix{def:empirical_neural_process_objective}}).
We will show that $\L\ss{NP}$ is lower semi-continuous (\cref{prop:neural_process_objective_is_lower_semi-continuous}) and that the posterior prediction map $\pi_f$ is the unique minimiser of $\L\ss{NP}$ (\cref{prop:neural_process_objective_minimiser}).
This presents the neural process objective $\L\ss{NP}$ as an appropriate objective that we can use to approximate the posterior prediction map $\pi_f$.
In this section and the following sections, all results assume \cref{assum:full_support_and_complete,assum:f_regularity}.

The goal of prediction map approximation is to approximate the posterior prediction map $\pi_f$ with a simpler, more tractable prediction map $\pi$.
The posterior prediction map $\pi_f$ is associated to some observation noise $\sigma_f$.
We will therefore also associate our approximation with some observation noise $\sigma > 0$.
This means that uncertainty in predictions by $\pi_f$ and $\pi$ are a sum of two components:
uncertainty about the ground-truth stochastic process $f$, represented by the variances of $\pi_f$ and $\pi$,
and observation noise, represented by $\sigma_f$ and $\sigma$.
These two uncertainties are fundamentally different,
because only the former can be reduced by observing more data.
We therefore call the former \emph{epistemic uncertainty} and the latter \emph{aleatoric uncertainty}.
\index{uncertainty!epistemic}
\index{uncertainty!aleatoric}
One potential issue is that an approximation $\pi$ of $\pi_f$ might conflate epistemic and aleatoric uncertainty.
We will return to this issue in \cref{sec:predmap:np_approximations}.

Since our approximation $\pi$ is associated to some observation noise $\sigma$, we call the tuple $(\pi, \sigma)$ a \emph{noisy prediction map}.
We now formalise the space of noisy prediction maps.

\index{prediction map!noisy}
\begin{definition}[Noisy prediction maps] \label{def:noisy_prediction_maps}
    Call $\overline{\M} = \M \times (0, \infty)$ the collection of \emph{noisy prediction maps}.
    Similarly define $\overline{\M}\ss{c} = \M\ss{c} \times (0, \infty)$.
    Note that $\overline{\M}\ss{c} \sub \overline{\M}$.
    Endow $\overline{\M}$ and $\overline{\M}\ss{c}$ with the following topology of pointwise convergence:
    $(\pi_i, \sigma_i)_{i \ge 1}$ converges to $(\pi, \sigma)$ whenever
    (1) $P_\vx \pi_i(D)\weakto P_\vx\pi(D)$ for all $\vx \in I$ and $D \in \D$ and
    (2) $\sigma_i\to \sigma$.
\end{definition}

Fox example, under \cref{def:noisy_prediction_maps}, a function $g \colon \overline{\M} \to \R$ on the space of noisy prediction maps is continuous if $P_\vx \pi_i(D) \weakto P_\vx\pi(D)$ for all $\vx \in I$ and $D \in \D$ and $\sigma_i \to \sigma$ imply that $g(\pi_i, \sigma_i) \to g(\pi, \sigma)$.
Having defined the space over which we will optimise the neural process objective,
we are now in a position to formally define $\L\ss{NP}$.

\index{neural process objective}
\begin{definition}[The neural process objective] \label{def:neural_process_objective}
    Call the function
    \begin{equation}
        \L\ss{NP}\colon\overline{\M} \to [0, \infty], \quad \L\ss{NP}(\pi, \sigma)
        = \E_{p(D)p(\vx)}[\KL(P_\vx^{\sigma_f} \pi_f(D), P_\vx^{\sigma} \pi(D))]
    \end{equation}
    the \emph{neural process objective}.
\end{definition}

It is not at all clear that the expectations in \cref{def:neural_process_objective} are well defined.
The following proposition addresses this.
\Cref{prop:neural_process_objective_is_well-defined} looks innocent, but its proof unfortunately is considerably technical.

\statement{statements/neural_process_objective_well-defined.tex}
\begin{proof}
    See \cref{\xrprefix{sec:proofs_predmap:neural_process_objective}}.
\end{proof}

We next show that $\L\ss{NP}$ is lower semi-continuous.
It is also true that $\L\ss{NP}$ is convex, which follows directly from convexity of the Kullback--Leibler divergence \parencite[Lemma 7.2;][]{Gray:2011:Entropy_and_Information_Theory} and linearity of expectation, but we will not need this property.

\statement{statements/neural_process_objective_lsc.tex}
\begin{proof}
    See \cref{\xrprefix{sec:proofs_predmap:neural_process_objective}}.
\end{proof}

\looseness=-1
We finally arrive at \cref{prop:neural_process_objective_minimiser}.
Modulo a few caveats, \cref{prop:neural_process_objective_minimiser} states that $(\pi_f, \sigma_f)$ is the unique minimiser of $\L\ss{NP}$ over all noisy prediction maps.
Unfortunately, minimising $\L\ss{NP}$ over all possible noisy prediction maps is not practically feasible.
Instead, we could hope to approximate $(\pi_f, \sigma_f)$ by minimising $\L\ss{NP}$ over a tractable, large-enough class of noisy prediction maps.
This is the approach that we will consider in the next section.

\statement{statements/neural_process_objective_minimiser.tex}
\begin{proof}
    See \cref{\xrprefix{sec:proofs_predmap:neural_process_objective}}.
\end{proof}

Note that \cref{prop:neural_process_objective_minimiser} minimises over $\overline{\M}\ss{c}$ rather than $\overline{\M}$.
This restriction is necessary, because otherwise some of the aleatoric noise $\sigma_f$ can be absorbed in the epistemic uncertainty, which means that the minimiser would not be unique.
Also note that \cref{prop:neural_process_objective_minimiser} requires $\tI$ to be dense in $I$.
In particular, this means that the target inputs of interest $\tI$ must consist of inputs of arbitrarily large size, that is, $\vx \in \X^N$ for arbitrarily large $N$.
Finally, if $\tD = \D$, then note that $(\pi_f, \sigma_f)$ is the unique minimiser of $\L\ss{NP}$.
On the other hand, if $\tD \subsetneq \D$, then
 a minimiser of $\L\ss{NP}$ recovers the posterior prediction map $\pi_f$ only on $\tD$.
This means that the collection of data sets of interest $\tD$ determines the approximation properties of minimisers of $\L\ss{NP}$.

\section{Neural Process Approximations}
\label{sec:predmap:np_approximations}
Having formally defined the neural process objective $\L\ss{NP}$ (\cref{def:neural_process_objective}), we turn our attention to the problem of approximating the posterior prediction map $(\pi_f, \sigma_f)$ with a simpler, more tractable prediction map $(\pi, \sigma)$.
We will form such an approximations by first defining a \emph{variational family} $\Qc \sub \overline{\M}$ (\cref{def:neural_process,def:cnp,def:gnp}).\index{variational family}
A variational family $\Qc$ is intended to consist of simpler, more tractable candidate approximations.
An approximation of $(\pi_f, \sigma_f)$ can then be obtained by
minimising the neural process objective $\L\ss{NP}$ over the variational family $\Qc$ (\cref{def:neural_process_approximation,def:cnpa,def:gnpa}).
Different variational families define different \emph{classes of neural processes} and we will call the associated minimisers \emph{neural process approximations}.\index{neural process approximation}
This means that neural process approximations formalise what different classes of neural processes precisely target.
Importantly, not all neural processes target the same approximation (\cref{prop:cnpa_characterisation,prop:gnpa_characterisation}).
Analysing the properties of neural process approximations is a way to understand the behaviour of neural processes in practice.

\begin{figure}[t]
    \centering
    \large
    \begin{tikzpicture}[
        scale=0.8,
        every arrow/.style={
            >={Triangle[length=2mm, width=2mm]}
        },
    ]
        \draw (0, 0) [rounded corners=10mm, thick] rectangle (12, 8);
        \node [anchor=south east, xshift=-10mm] at (12, 8) {$\overline{\M}$};
        \node (centre) at (4, 4) {};
        \node (start) at (5, 4) {};
        \node (end) at (10, 2) {};
        \draw [name path=Q, thick] (centre) circle (3);
        \path (centre) ++ (45:3) node [anchor=south west] {$\Qc$};
        \path [name path=line, thick] (start) -- (end);
        \draw [name intersections={of=Q and line}] (intersection-1) coordinate (int);
        \node [circle, fill, minimum width=6pt, inner sep=0pt, label={north:$\pi_M$}] (piM) at (start) {};
        \node [circle, fill, minimum width=6pt, inner sep=0pt, label={south east:$\pi_\infty$}] (piinfty) at (int) {};
        \node [circle, fill, minimum width=6pt, inner sep=0pt, label={north: $\pi_f$}] (pif) at (end) {};
        \draw [arrow, thick, dashed, ->] (piM) -- node [below, pos=0.5, sloped] {\small $M \to \infty$} (piinfty);
        \draw [arrow, thick, <-] ([yshift=-3pt]pif.south) |- ++ (-0.5, -0.5) node [anchor=east] {\normalsize what you want};
        \draw [arrow, thick, <-] ([xshift=-3pt]piM.west) -- ++ (-1, 0) node [align=center, anchor=east] {\normalsize what you \\[-0.5em] \normalsize compute};
        \draw [arrow, thick, <-] ([xshift=3pt, yshift=3pt]piinfty.north east) -- ++ (1, 1) node [yshift=-20pt, fill=white, align=center, anchor=south west] {\normalsize the neural process \\[-0.5em] \normalsize approximation \\[-0.25em] \footnotesize (the best you can do)};
    \end{tikzpicture}
    \small
    \caption
    [
        Connection between the post.\ pred.\ map and neural process approximations
    ]
    {
        Connection between the posterior prediction map $\pi_f$ (\cref{def:posterior_prediction_map}), 
        a neural process approximation $\pi_\infty$ (\cref{def:neural_process_approximation}),
        and what you compute in practice $\pi_M$.
        Shows the collection of all noisy prediction maps $\overline{\M}$ (\cref{def:noisy_prediction_maps}) and
        a variational family $\Qc$ of a neural process (\cref{def:neural_process}).
        For this variational family,
        shows a minimiser $\pi_M \in \argmin_\Qc \L_M$ of the empirical neural process objective $\L_M$ (\cref{\xrprefix{def:empirical_neural_process_objective}})
        and a minimiser $\pi_\infty \in \argmin_\Qc \L\ss{NP}$ of the neural process objective $\L\ss{NP}$ (\cref{\xrprefix{def:neural_process_objective}}).
        In the figure, distance is as measured by the neural process objective $\L\ss{NP}$.
        The idea is that what you compute in practice $\pi_M$ converges to the neural process approximation $\pi_\infty$ in the limit of infinite data $M \to \infty$.
    }
    \label{fig:overview_theory}
\end{figure}

\index{variational family}
\index{neural process}
\begin{definition}[Variational family, neural process]
    \label{def:neural_process}
    A \emph{variational family} $\Qc \sub \overline{\M}$ is a collection of noisy prediction maps.
    Variational families are also called \emph{neural processes}.
\end{definition}

\index{neural process approximation}
\begin{definition}[Neural process approximation]
    \label{def:neural_process_approximation}
    A \emph{neural process approximation} is the collection of minimisers of the neural process objective $\L\ss{NP}$ over a variational family.
\end{definition}

\Cref{fig:overview_theory} illustrates how all objects we have defined thus far connect.

The class of neural processes originally presented by \textcite{Garnelo:2018:Conditional_Neural_Processes} will be represented by the variational family $\Qc\ss{G,MF}$ called the collection of \emph{conditional neural processes} (CNPs; \cref{def:cnp}).
CNPs are prediction maps which map to Gaussian processes that \emph{do not} model dependencies between different function values.
This section also presents a new class of neural processes $\Qc\ss{G}$
called the collection of \emph{Gaussian neural processes} (GNPs; \cref{def:gnp}).
Contrary to CNPs, GNPs are prediction maps which map to Gaussian processes that \emph{do} model dependencies between different function values.
The following definitions set up the notation that we will use to define CNPs and GNPs.

\begin{definition}[Gaussian process]
    Let $\Pc\ss{G}$ be the collection of processes in $\Pc$ which are Gaussian.
\end{definition}

\index{prediction map!Gaussian}
\begin{definition}[Gaussian prediction map]
    Call a prediction map $\pi \colon \D \to \Pc\ss{G}$ \emph{Gaussian} if it maps to $\Pc\ss{G}$.
\end{definition}

\index{mean map}
\begin{definition}[Mean map]
    \label{def:mean_map}
    For any prediction map $\pi \colon \D \to \Pc$, not necessarily Gaussian, define the \emph{mean map} $m_\pi$ by
    \begin{equation}
        m_\pi\colon \D \to \Y^\X, \qquad
        m_\pi(D)= x \mapsto \E_{\pi(D)}[f(x)].
    \end{equation}
\end{definition}

\index{kernel map}
\begin{definition}[Kernel map]
    \label{def:kernel_map}
    For any prediction map $\pi \colon \D \to \Pc$, not necessarily Gaussian, define the \emph{kernel map} $k_\pi$ by
    \begin{equation}
        k_\pi\colon \D \to \R^{\X \times \X}, \qquad
        k_\pi(D) = (x, y) \mapsto \cov_{\pi(D)}(f(x), f(y)).
    \end{equation}
\end{definition}

\index{variance map}
\begin{definition}[Variance map]
    \label{def:variance_map}
    For any prediction map $\pi \colon \D \to \Pc$, not necessarily Gaussian, define the \emph{variance map} $v_\pi$ by
    \begin{equation}
        v_\pi\colon \D \to [0,\infty)^{\X \times \X}, \qquad
        v_\pi(D) = x \mapsto \var_{\pi(D)}(f(x)).
    \end{equation}
\end{definition}

If a prediction map $\pi_i$ is subscripted, for example by $i$, we more simply denote $m_i = m_{\pi_i}$, $k_i = k_{\pi_i}$, and $v_i = v_{\pi_i}$.
Similarly, for the ground-truth stochastic process $f$, we more simply denote $m_f = m_{\pi_f}$, $k_f = k_{\pi_f}$, and $v_f = v_{\pi_f}$.

We will now define the classes of conditional neural processes and Gaussian neural processes.
CNPs require the following technical condition, which we already alluded to in \cref{sec:predmap:prelims}.

\begin{assumption} \label{assum:repeated_elements_is_null_set}
    Assume that the collection of inputs $\vx \in I$ with repeated elements is a $p(\vx)$-null set.
\end{assumption}

\index{neural process!conditional}
\begin{definition}[Conditional neural process; CNP]
    \label{def:cnp}
    Let the collection of \emph{conditional neural processes} (CNPs) be
    \begin{equation}
        \Qc\ss{G,MF} = \set*{(\pi, \sigma) \in \overline{\M}
            \middlecond
            \begin{gathered}
                \text{\normalshape $\pi$ is Gaussian}, \\
                \text{\normalshape $k_\pi(D)(x, y) = 0$ for all $x \neq y$ and $D \in \D$}, \\
                \text{\normalshape $m_\pi(D)$ and $v_\pi(D)$ are continuous for all $D \in \D$} \\
            \end{gathered}
        }.
    \end{equation}
\end{definition}

Note that $\Qc\ss{G,MF}$ is a subset of $\overline{\M}$ rather than $\overline{\M}\ss{c}$, meaning that all prediction maps in $\Qc\ss{G,MF}$ map to $p(\vx)$-regular processes rather than continuous processes.
This generality is necessary, because assuming that $k_\pi(D)(x, y) = 0$ for all $x \neq y$ but allowing $k_\pi(D)(x, x) > 0$ can create discontinuities of $\vx \mapsto P_\vx \pi(D)$ at inputs $\vx$ with repeated elements.
Fortunately, this is allowed, because we assumed that the collection of all inputs $\vx$ with repeated elements is a $p(\vx)$-null set (\cref{assum:repeated_elements_is_null_set}).
It also means, however, that the mean maps $m_\pi$ and variance maps $v_\pi$ map to functions continuous only almost everywhere.
We strengthen this by assuming, in the definition of $\Qc\ss{G,MF}$, that $m_\pi$ and $v_\pi$ map to functions continuous everywhere.

The collection of conditional neural processes should be interpreted as the collection of prediction maps which map to Gaussian processes that \emph{do not} model dependencies between different function values.
\Cref{def:cnp} makes this precise by the condition on $k_\pi(D)$.
The ``MF'' in the subscript of $\Qc\ss{G,MF}$ stands for ``mean field'' to remind the reader of this.
A conditional neural process $(\pi, \sigma)$ is characterised by its mean map $m_\pi$, variance map $v_\pi$, and noise variance $\sigma$.

\index{neural process!Gaussian}
\begin{definition}[Gaussian neural process; GNP]
    \label{def:gnp}
    Let the collection of \emph{Gaussian neural processes} (GNPs) be
    \begin{equation}
        \Qc\ss{G} = \set{(\pi, \sigma) \in \overline{\M}\ss{c}
            :
            \begin{gathered}
                \text{\normalshape $\pi$ is Gaussian}
            \end{gathered}
        }.
    \end{equation}
\end{definition}

Note that $\Qc\ss{G}$ is a subset of $\overline{\M}\ss{c}$ rather than $\overline{\M}$, meaning that all prediction maps in $\Qc\ss{G}$ map to continuous Gaussian processes.
Consequently, for all $\pi \in \Qc\ss{G}$, the mean maps $m_\pi$ and kernel maps $k_\pi$ map to continuous functions.

The collection of Gaussian neural processes should be interpreted as the collection of prediction maps which map to Gaussian processes that \emph{do} model dependencies between different function values.
\Cref{def:gnp} makes this precise by allowing any continuous kernel map, which includes ones such that $k_\pi(D)(x, y) \neq 0$ for $x \neq y$;
compare this to \cref{def:cnp}, where $k_\pi(D)(x, y) = 0$ for all $x \neq y$.
A Gaussian neural process $(\pi, \sigma)$ is characterised by its mean map $m_\pi$, kernel map $k_\pi$, and noise variance $\sigma$.

Having defined two variational families $\Qc\ss{G,MF}$ and $\Qc\ss{G}$, we next define two classes of \emph{neural processes approximations} by minimising the neural process objective $\L\ss{NP}$ over these variational families.

\index{neural process approximation!conditional}
\begin{definition}[Conditional neural process approximation; CNPA]
    \label{def:cnpa}
    Let the collection of \emph{conditional neural process approximations} (CNPAs)
    be the minimisers of $\L\ss{NP}$ over $\Qc\ss{G,MF}$.
\end{definition}

\index{neural process approximation!Gaussian}
\begin{definition}[Gaussian neural process approximation; GNPA]
    \label{def:gnpa}
    Let the collection of \emph{Gaussian neural process approximations} (GNPAs)
    be the minimisers of $\L\ss{NP}$ over $\Qc\ss{G}$.
\end{definition}

\looseness=-1
The following propositions characterise CNPAs and GNPAs.
Whereas \cref{prop:neural_process_objective_minimiser} requires the assumption that $\tI$ is dense in $I$, GNPAs and CNPAs admit weaker assumptions on $\tI$.

\index{neural process approximation!characterisation}
\statement{statements/cnpa_characterisation.tex}
\begin{proof}
    See \cref{\xrprefix{sec:proofs_predmap:np_approximations}}.
\end{proof}

If $\tD = \D$, then $(\pi, \sigma) \in \Qc\ss{G,MF}$ is a CNPA if and only if $m_{\pi} = m_f$ and $v_{\pi} + \sigma^2 = v_f^2 + \sigma_f^2$.
This means that a CNPA is \emph{not} unique:
for every $\sigma \in (0, \sigma_f]$, choosing $v_{\pi}= v_f^2 + \sigma_f^2 - \sigma^2$ gives a CNPA.
On the other hand, if $\tD \subsetneq \D$, then a CNPA recovers $m_f$ and $v_f + \sigma^2$ only on $\tD$.
We therefore see that $\tD$ determines the approximation properties of CNPAs.

\Cref{prop:cnpa_characterisation} requires that $\tI$ is dense in $I_N$ for some $N \ge 1$.
This means that a CNPA can be found even if the target set size is only one.

\index{neural process approximation!characterisation}
\statement{statements/gnpa_characterisation.tex}
\begin{proof}
    See \cref{\xrprefix{sec:proofs_predmap:np_approximations}}.
\end{proof}

If $\tD = \D$,
then $(\pi, \sigma) \in \Qc\ss{G}$ is then a GNPA if and only if $m_{\pi} = m_f$, $k_{\pi} = k_f$, and $\sigma = \sigma_f$.
In particular, a GNPA is unique, so we may speak of \emph{the} GNPA.
On the other hand, if $\tD \subsetneq \D$, then a GNPA recovers $m_f$ and $k_f$ only on $\tD$.
We therefore see that $\tD$ also determines the approximation properties of GNPAs.

\Cref{prop:gnpa_characterisation} requires that $\tI$ is dense in $I_N$ for some $N \ge 2$.
Hence, to find a GNPA, one must consider target set sizes of at least two.
This is a stronger requirement than for the CNPA, where one can consider target set sizes of just one.

\index{moment matching}
If we interpret the mean map as the ``first moment'' of a prediction map and the kernel map as the ``second moment'', then the Gaussian neural process approximation \emph{moment matches the posterior prediction map}.
Such behaviour has previously been noted by \textcite{Ma:2018:Variational_Implicit_Processes}.
One might wonder whether moment matching the posterior prediction map yields a prediction map which maps to continuous processes.
As the following proposition shows, the GNPA inherits $L^p$-H\"older regularity from the posterior prediction map (\cref{assum:f_regularity}), but with $p = 2$.
In particular, this means that the GNPA indeed maps to continuous processes.

\begingroup
    \newcommand{\possibleprefix}[1]{#1}
    \statement{statements/gnpa_regularity.tex}
\endgroup

\begin{proof}
    See \cref{\xrprefix{sec:proofs_predmap:np_approximations}}.
\end{proof}

We have seen that the GNPA is unique, but CNPAs are not:
CNPAs define a range of solutions by absorbing part of the observation noise $\sigma_f$ in the variance map.
This highlights a fundamental distinction between Gaussian neural processes and conditional neural processes.
Since Gaussian neural processes model dependencies between function values, they are able to correctly separate epistemic and aleatoric uncertainty.
\index{uncertainty!separation}
Conditional neural processes, on the contrary, do not model these dependencies, and consequently they are unable to separate epistemic and aleatoric uncertainty;
they can only identify the sum of the two.
We will further compare GNPs and CNPs in \cref{\xrprefix{sec:convcnps:gnp}}.

\section{Consistency}
\label{sec:predmap:consistency}

\index{consistency!convergence}
In the previous section, we defined two approximations of the posterior prediction map $\pi_f$ by minimising the neural process objective $\L\ss{NP}$ over two different variational families $\Qc$ (\cref{def:cnp,def:gnp}).
Unfortunately, in practice, we cannot compute either of these approximations, because we do not have access to $\L\ss{NP}$.
Although we cannot compute $\L\ss{NP}$ exactly, we can compute a Monte Carlo approximation of it: \index{neural process objective!empirical}
\begin{equation} \label{eq:predmap:consistency:finite-sample_objective}
    \L_M(\pi, \sigma) = -\frac1M\sum_{m=1}^M \log q_\theta(\vy\us{(t)}_m \cond \vx\us{(t)}_m, D\us{(c)}_m) + \text{constant}.
\end{equation}
We earlier defined this Monte Carlo approximation as the \emph{empirical} neural process objective (\cref{\xrprefix{def:empirical_neural_process_objective}}),
which is the objective originally proposed by \textcite{Garnelo:2018:Conditional_Neural_Processes}.
To approach either of the approximations,
it seems reasonable that we could minimise $\L_M$ over the variational family $\Qc$, and hope that the minimiser converges to the minimiser of $\L\ss{NP}$ as $M \to \infty$, a property called \emph{consistency}.
See also the illustration in \cref{fig:overview_theory}.

\index{overfitting}
Sadly, for the variational families $\Qc\ss{G,MF}$ and $\Qc\ss{G}$ we introduced in \cref{def:cnp,def:gnp}, consistency does not hold.
To see this, simply note that \eqref{eq:predmap:consistency:finite-sample_objective} can be made arbitrarily small by setting $m_\pi(D\us{(c)}_m)(\vx_m\us{(t)}) = \vy_m\us{(t)}$ and making the variance sufficiently small.
We can always pick such $m_\pi$, because the variational families require $m_\pi$ to only be continuous and map to continuous functions, which means that we can arbitrarily choose the value of $m_\pi$ at any finite number of points.
What is essentially happening is that \eqref{eq:predmap:consistency:finite-sample_objective} is optimised over the collection of all continuous functions, which is too large, because a continuous function can always fit a data set exactly!
Consequently, our estimator will \emph{overfit}.

To combat overfitting, we must constrain our variational family $\Qc$ to a \emph{smaller} family $\widetilde{\Qc} \sub \Qc$ such that estimation with $\widetilde{\Qc}$ is consistent and, crucially, minimising $\L_M$ over $\widetilde{\Qc}$ still gives the desired neural process approximation.
As we will discuss shortly, a sufficient condition for $\widetilde\Qc$ to be appropriately small is that $\widetilde\Qc$ be \emph{compact} in an appropriate topology.\index{compactness}
In this section, we will show that there exist reasonable assumptions and a compact $\widetilde{\Qc}$ such that, for any ground-truth stochastic process satisfying these assumptions, the CNPA and GNPA are contained in $\widetilde{\Qc}$.
For the moment, let us very roughly compare $\Qc$ to $\R^n$. 
Then the sought-after neural process approximation would be some $z \in \R^n$.
An appropriate choice for $\widetilde{\Qc}$ is then apparent:
just take $\widetilde{\Qc}$ to be the equivalent of $[-M, M]^n$ for large enough $M > 0$.
Namely, $[-M, M]^n$ is compact and eventually captures all of $\R^n$ as $M \to \infty$.
While this idea is promising, the analogy is flawed:
$\Qc$ is \emph{infinite dimensional}, so, unlike $\R^n$, a sequence of compact sets which eventually capture all of $\Qc$ does not exist. 
In more technical terms, every \emph{infinite-dimensional} topological vector space is \emph{not locally compact} \parencite[Theorem 1.22;][]{Rudin:1991:Functional_Analysis}.
Therefore, it is not at all clear that we can find an appropriate $\widetilde{\Qc}$.
The main result of this section, however, is that we can (\cref{prop:posterior_prediction_map_regularity:part_two}),
leading us to consistency results for neural processes (\cref{prop:cnpa_consistency,prop:gnpa_consistency}).

To establish consistency of an estimator of a neural process approximation, we will follow \citeauthor{Wald:1949:Note_on_the_Consistency_of}'s (\citeyear{Wald:1949:Note_on_the_Consistency_of}) proof of consistency of maximum-likelihood estimators.
At the heart of \citeauthor{Wald:1949:Note_on_the_Consistency_of}'s approach lies \emph{compactness of the parameter space}.
As we revealed in the previous paragraph, in our setting, this means that we require \emph{compactness of the variational family $\Qc$}.
In particular, this means that we require a topology on $\Qc$, and this topology must be carefully selected.
For example, consider all mean maps associated to noisy prediction maps in $\Qc\ss{G}$.
Then a candidate topology would be the initial topology generated by $\set{m_\pi \mapsto m_\pi(D)(\vx) : D \in \D, \vx \in I}$.
This topology is problematic, because it admits ``too large'' compact sets.
As a result, optimising \eqref{eq:predmap:consistency:finite-sample_objective} over a set compact in this topology does not constrain the optimiser enough to avoid the problem of exactly fitting the data.
For example, by Tychonoff's theorem \parencite[Theorem 37.3;][]{Munkres:2000:Topology}, the collection of mean functions defined by all possible functions $\D \to [0, 1]^\X$ would be compact in this topology, but clearly we cannot optimise over this collection without running the risk of overfitting!

We therefore choose a topology on $\Qc\ss{G,MF}$ and $\Qc\ss{G}$ which does not admit ``too large'' compact sets.
Henceforth, in this section, we view mean maps more simply as functions $\D \times \X \to \Y$ rather than functions $\D \to \Y^\X$.
Similarly, we view kernel maps more simply as functions $\D \times \X \times \X \to \R$ and variance maps more simply as functions $\D \times \X \to [0, \infty)$.
For mean maps, define the supremum norm relative to $\tD$ as follows:%
\begin{equation}
    \norm{m_1 - m_2}_{\widetilde\infty} = \sup_{D \in \tD, x \in \X}\,\abs{m_{1}(D, x) - m_{2}(D, x)}.
\end{equation}
Similarly define this norm for kernel maps and variance maps.
For the class of conditional neural processes $\Qc\ss{G,MF}$, define the metric
\index{metric!CNPs}
\begin{equation} \label{eq:metric_Q_G_MF}
    d\ss{G,MF}((\pi_1, \sigma_1), (\pi_2, \sigma_2))
    = \norm{m_{1} - m_{2}}_{\widetilde\infty}
    + \norm{v_{1} - v_{2}}_{\widetilde\infty}
    + \abs{\sigma_1 - \sigma_2}.
\end{equation}
For the class of Gaussian neural processes $\Qc\ss{G}$, define the metric
\index{metric!GNPs}
\begin{equation} \label{eq:metric_Q_G}
    d\ss{G}((\pi_1, \sigma_1), (\pi_2, \sigma_2))
    = \norm{m_{1} - m_{2}}_{\widetilde\infty}
    + \norm{k_{1} - k_{2}}_{\widetilde\infty}
    + \abs{\sigma_1 - \sigma_2}.
\end{equation}
We will consider the topologies on $\Qc\ss{G,MF}$ and $\Qc\ss{G}$ induced by these metrics.
\index{Arzel\`a--Ascoli theorem}
Under these metrics, a characterisation of compactness is given by the Arzel\`a--Ascoli theorem \parencite[Theorem 7.25;][]{Rudin:1976:Principles_of_Mathematical_Analysis}:
 a subset of $\Qc$ of $\Qc\ss{G,MF}$ (respectively, $\Qc\ss{G}$) is compact if and only if it is closed and the associated mean maps and variance maps (respectively, kernel maps) are equicontinuous and uniformly bounded. 
Equicontinuity roughly means the following:
the collection of mean maps over which we optimise the finite-sample objective may only \emph{vary limitedly quickly}.
It is precisely this property that will disallow arbitrarily choosing the value at any finite number of points to exactly fit the data.

The remainder of this section proceeds as follows.
We first establish a number of reasonable conditions.
Based on these conditions, we define a compact subset $\widetilde{\Qc}$ of $\Qc\ss{G,MF}$ (respectively, $\Qc\ss{G}$).
We then prove that, for any ground-truth stochastic process satisfying these conditions, the CNPA (respectively, GNPA) is contained in $\widetilde{\Qc}$.
To establish these conditions, we will concoct pathological sequences which attempt to break equicontinuity.
In particular,
we want to avoid that the assumptions permit a sequence of data sets $(D_i)_{i \ge 1}$ and ground-truth stochastic processes $(f_i)_{i \ge 1}$ such that $m_{f_i}(D_i)$ becomes arbitrarily steeply sloped at any input.
We will take this as a guiding criterion. 

To begin with, for $n \ge 1$, consider $D_n = \set{(-1, -n), (1, n)}$.
Then $m_f(D_n)$ will approximately interpolate $(-1, -n)$ and $(1, n)$, so the slope of $m_f(D)$ at $x = 0$ is approximately $2n$.
Hence, as $n \to \infty$, the slope of $m_f(D_n)$ at $x = 0$ becomes arbitrarily large, contradicting equicontinuity.
To prevent this from happening, we must assume that the magnitude of the observations is bounded.
 
Similarly, consider $D_n = \set{(-2^{0}, -1),\ldots, (-2^{-n}, -1), (2^{-n}, 1), \ldots, (2^0, 1)}$.
Then $m_f(D)$ will approximately interpolate $(-2^{-n}, -1)$ and $(2^{-n}, 1)$.
As $n \to \infty$, the sheer number of observations will overwhelm the prior,
so $m_f(D)$ will interpolate $(-2^{-n}, -1)$ and $(2^{-n}, 1)$ more and more closely.
Hence, as $n \to \infty$, the slope of $m_f(D)$ at $x = 0$ will again become arbitrarily large.
We must therefore also assume that the number of observations is bounded.

\index{data sets of interest}
\begin{assumption}[Boundedness of context sets]
    \label{assum:context_boundedness}
    The collection of data sets of interest $\tD$ satisfies
    \begin{equation}
        B_{\tD} \coloneqq \sup\,\set{\abs{\vx} \lor \norm{\vy}_2 : (\vx, \vy) \in \tD} < \infty.
    \end{equation}
\end{assumption}

Instead of assuming that the magnitude of the observations is bounded, we could have assumed that $f$ is a bounded process.
Although this alternative assumption works and is arguably simpler, it is also more restrictive.
For example, it would exclude the common choice $f \sim \GP(0, e^{-\frac12(\vardot - \vardot)^2})$.
By assuming that only the context sets are bounded, whenever $D \in \tD$, we can still approximate $p(f \cond D)$ for unbounded $f$.

Finally, consider $D_n = \set{(-2^{-n}, -1), (2^{-n}, 1)}$ and a sequence of ground-truth stochastic processes $f_n$ with observation noises $\sigma_{f_n} = 2^{-n}$.
Then, as $n \to \infty$, the observation noise becomes smaller and smaller, so $m_{f_n}(D_n)$ will interpolate $(-2^{-n}, -1)$ and $(2^{-n}, 1)$ more and more closely, meaning that the slope of $m_{f_n}(D_n)$ at $x = 0$ will again become arbitrarily large.
We therefore require a universal lower bound on the allowed values for $\sigma_f$.

\begin{assumption}[Boundedness of noise] \label{assum:noise_boundedness}
    The observation noise $\sigma_f$ is larger than some universal lower bound $\underline{\sigma} > 0$.
\end{assumption}

In addition to a universal lower bound $\underline{\sigma}$, we will also assume that $\sigma_f \le \overline{\sigma} < \infty$ for some universal upper bound $\overline{\sigma}$.
Then $\sigma_f$ will be contained in the interval $[\underline{\sigma}, \overline{\sigma}]$, which indeed is compact.\footnote{
    This upper bound can likely be made redundant by considering the compactification $[\underline{\sigma}, \infty]$.
    See Example 5.16 by \textcite{Vaart:1998:Asymptotic_Statistics} for an illustration of how \citeauthor{Wald:1949:Note_on_the_Consistency_of}'s proof can be applied to a non-compact parameter space by ``adding $-\infty$ and $\infty$''.
    The upper bound is therefore mainly for technical convenience.
}

Our final assumption is a technical one, stating that we require a little more than the second moment to be uniformly bounded.
Note that, if $p > 2$, then this condition is implied by \cref{assum:f_regularity}.

\begin{assumption}[Boundedness of ground-truth stochastic process] \label{assum:f_boundedness}
    There exists a universal $\gamma > 0$ such that
    \begin{equation}
        B_f \coloneqq \sup_{x \in \X}\, \norm{f(x)}_{L^{2 + \gamma}} < \infty.
    \end{equation}
\end{assumption}

We collect the parameters of all assumptions thus far in a tuple $\mathrm{\textsf{U}}$ called the \emph{universal parameters}.

\index{universal parameters}
\begin{definition}[Universal parameters]
    The universal parameters $\mathrm{\textsf{U}}$ are the parameters defined in \cref{assum:f_regularity,assum:context_boundedness,assum:noise_boundedness,assum:f_boundedness}:
    \begin{equation}
        \mathrm{\textsf{U}} = (p, \beta, c, r, \gamma, B_f, B_{\tD}, \underline{\sigma}).
    \end{equation}
\end{definition}

Henceforth, when we say that an object exists universally, we mean that it depends only on the universal parameters and nothing else.
We now come to the first result of this section, which says that the mean map and the kernel map of the posterior prediction map satisfy a H\"older condition.
Crucially, the H\"older constants and exponents are universal, meaning that they only depend on the universal parameters.

\index{posterior prediction map!regularity}
\statement{statements/regularity_part_two.tex}
\begin{proof}
    See \cref{\xrprefix{sec:proofs_predmap:consistency}}.
\end{proof}

\Cref{prop:posterior_prediction_map_is_uniformly_continuous} states that $m_f$, $k_f$, and $v_f$ satisfy particular universal H\"older conditions.
Let $\mathsf{H}_m$, $\mathsf{H}_k$, and $\mathsf{H}_v$ be the collections of respectively all mean maps, kernel maps, and variance maps satisfying these universal H\"older conditions.
We now use these collections to define our desired compact variational family $\widetilde{\Qc}$.
Define the subfamilies $\widetilde\Qc\ss{G,MF} \sub \Qc\ss{G,MF}$ and $\widetilde\Qc\ss{G} \sub \Qc\ss{G}$ as follows:
\begin{align}
    \widetilde\Qc\ss{G,MF} &= \set*{
        (m_\pi, v_\pi, \sigma) \in \Qc\ss{G} :
            m_\pi \in \mathsf{H}_m,\,
            v_\pi \in \mathsf{H}_v,\,
            \sigma \in [\underline{\sigma}, \overline{\sigma}]
    }, \\
    \widetilde\Qc\ss{G} &= \set*{
        (m_\pi, k_\pi, \sigma) \in \Qc\ss{G} :
            m_\pi \in \mathsf{H}_m,\,
            k_\pi \in \mathsf{H}_k,\,
            \sigma \in [\underline{\sigma}, \overline{\sigma}]
    }.
\end{align}
By construction, $(m_f, v_f, \sigma_f) \in \widetilde\Qc\ss{G,MF}$ and $(m_f, k_f, \sigma_f) \in \widetilde\Qc\ss{G}$.
Crucially, by the H\"older conditions, the variational families $\widetilde\Qc\ss{G,MF}$ and $\widetilde\Qc\ss{G}$ are closed, equicontinuous, and uniformly bounded, which means that they are compact!
We have therefore arrived at our desired consistency result.

Recall the metric $d\ss{G,MF}$ on $\widetilde\Qc\ss{G,MF}$ defined in \eqref{eq:metric_Q_G_MF}
and the metric $d\ss{G}$ on $\widetilde\Qc\ss{G}$ defined in \eqref{eq:metric_Q_G}.
By $o_\P(1)$, we denote a random variable that converges to zero in probability as $M \to \infty$.
In the following statements, $o_\P(1)$ is models noise deriving stochastic optimisation procedures.

\index{neural process approximation!consistency}
\statement{statements/cnpa_consistency.tex}
\begin{proof}
    See \cref{\xrprefix{sec:proofs_predmap:consistency}}.
\end{proof}

\index{neural process approximation!consistency}
\statement{statements/gnpa_consistency.tex}
\begin{proof}
    See \cref{\xrprefix{sec:proofs_predmap:consistency}}.
\end{proof}

We end our discussion of consistency of neural processes with a remark on the mode of convergence for Gaussian neural processes.
Consider a GNPA $\pi \in \widetilde \Qc\ss{G}$ and
let $D \in \tD$ be some data set.
\Cref{prop:gnpa_consistency} gives convergence in $d\ss{G}$, which in turn implies that $P_\vx \pi_M(D) \weakto P_\vx \pi(D)$ for all $\vx \in I$.
For estimating expectations of functions which depend on only finitely many function values, \eg~$\E_{\pi(D)}[L(f(\vx))]$ where $L$ is continuous and bounded, this mode of convergence is appropriate.
However, we might want to estimate expectations of functions which depend on the entire realisation, like the position of the maximum: $\E_{\pi(D)}[\argmax_{x \in \X} f(x)]$.
To estimate such expectations, we require a stronger mode of convergence, namely $\pi_M(D) \weakto \pi(D)$.
Conveniently, again owing to the universal H\"older condition, \cref{\xrprefix{prop:uniform_control_modulus_of_continuity}} in \cref{\xrprefix{sec:proofs_predmap:consistency}} combined with Theorem 7.5 by \textcite{Billingsley:1999:Convergence_of_Probability_Measures} show that
 $P_\vx \pi_i(D) \weakto P_\vx \pi(D)$ for all $\vx \in I$ in fact implies that $\pi_i(D) \weakto \pi(D)$.

\section{Conclusion}

This chapter has presented a number of definitions and results collectively called \emph{prediction map approximation}.
This theoretical framework allowed us to formally define and analyse neural processes. 
We now summarise the main results and state the main takeaways.

Neural processes are various approximations of the \emph{posterior prediction map} (\cref{def:posterior_prediction_map}).
Although all neural processes attempt to approach the posterior prediction map, some classes of neural processes come closer than others.
In particular, in the limit of infinite data, different classes of neural processes approach different objects.
We call the object that a class of neural processes approaches a \emph{neural process approximation} (\cref{def:neural_process_approximation}).

The class of neural processes first proposed by \textcite{Garnelo:2018:Conditional_Neural_Processes} is the class of \emph{conditional neural processes} (\cref{def:cnp}).
Conditional neural processes have Gaussian predictions which \emph{do not} model dependencies.
In the limit of infinite data, conditional neural processes approach a \emph{conditional neural process approximation} (\cref{def:cnpa}).
A conditional neural process approximation recovers the mean map (\cref{def:mean_map}) of the posterior prediction map (\cref{prop:cnpa_characterisation}).
It also recovers the sum of the variance map (\cref{def:variance_map}) and the observation noise.
A conditional neural process approximation, however, cannot separate the two.
We therefore say that conditional neural processes fails to separate epistemic and aleatoric uncertainty.

In addition to conditional neural processes, we introduced the class of \emph{Gaussian neural processes} (\cref{def:gnp}).
Gaussian neural processes have Gaussian predictions which \emph{do} model dependencies.
In the limit of infinite data, Gaussian neural processes approach the \emph{Gaussian neural process approximation} (\cref{def:gnpa}).
The Gaussian neural process approximation recovers the mean map, kernel map (\cref{def:kernel_map}), and observation noise of the posterior prediction map (\cref{prop:gnpa_characterisation}).
In particular, Gaussian neural processes are able to separate epistemic and aleatoric uncertainty.
Because the Gaussian neural process approximation recovers the mean map and kernel map, we also say that the Gaussian neural process approximation \emph{moment matches} the posterior prediction map.

In practice, neural processes are learned by maximising the probability of the target sets under the predictions given the context sets \parencite{Garnelo:2018:Neural_Processes}.
We can consider this objective in the limit of infinite data.
We call this infinite-data objective the \emph{neural process objective} (\cref{def:neural_process_objective}).
Neural process approximations, the infinite-data limits of neural processes, are minimisers of the neural process objective (\cref{def:neural_process_approximation,def:cnpa,def:gnpa}).
To successfully deploy a neural process, we want what we compute in practice to converge to the minimiser of the neural process objective.
We call this property \emph{consistency}.

First, to achieve consistency, in the general case the target inputs must be of arbitrarily large size (\cref{prop:neural_process_objective_minimiser}).
This is not practical.
Conditional and Gaussian neural processes relax this requirement.
To approximate the Gaussian neural process approximation, it suffices to consider target inputs of size just $N$, as long as $N \ge 2$ (\cref{prop:gnpa_characterisation});
and to approximate a conditional neural processes approximation, one may even consider target inputs of size only one (\cref{prop:cnpa_characterisation}).

Second, to achieve consistency, we must prevent overfitting.
Overfitting happens if the neural process is arbitrarily flexible.
To prevent overfitting, the appropriate technical notion is that the variational family of the neural process must be \emph{compact} (\cref{sec:predmap:consistency}).
By making reasonable assumptions about the data and the ground-truth stochastic process (\cref{assum:context_boundedness,assum:noise_boundedness,assum:f_boundedness}), for conditional and Gaussian neural processes, compactness can reasonably be satisfied.
This, in turn, guarantees consistency of conditional and Gaussian neural processes (\cref{prop:cnpa_consistency,prop:gnpa_consistency}).

Our investigation shed light on some of the more subtle issues concerning neural processes.
However, many important questions still remain unsolved.
To begin with, although we established a consistency result for conditional and Gaussian neural processes, the more important question is how much data we need, quantitatively, to come sufficiently close to the neural process approximations.
Additionally,
the aleatoric noise has been Gaussian and homogeneous.
It would be more realistic to generalise this to non-Gaussian and heterogeneous noise.
As long as the noise distribution satisfies reasonable conditions and the parameters vary continuously with the input, this generalisation should work out.
Finally, in practice, we do not optimise over the compact family proposed by \cref{sec:predmap:consistency}, but use neural networks.
\looseness=-1
Although we will not hope to achieve finite-sample guarantees for neural networks, perhaps the proposed compact family can be better connected with practice.

\end{document}

