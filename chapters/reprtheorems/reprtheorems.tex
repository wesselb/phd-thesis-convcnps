\documentclass[12pt, twoside]{report}
\usepackage{libertine}

% General packages:
\usepackage[utf8]{inputenc}    % Input encoding
\usepackage[T1]{fontenc}       % Font encoding
\usepackage[british]{babel}    % Naming of figures and such
\usepackage[british]{isodate}  % Formatting of dates
\cleanlookdateon               % Show dates cleanly.
%\usepackage{xurl}              % Allow line breaks anywhere in URLs.
\usepackage{minted}            % Needs to be loaded before `csquotes`.
\usepackage[
    citestyle    = authoryear,
    bibstyle     = authoryear,
    giveninits   = true,
    uniquename   = init,
    uniquelist   = false,
    sortlocale   = en_GB,
    backend      = biber,
    backref      = true,
    maxbibnames  = 100,
    maxcitenames = 1,
    dashed       = false,
    % Needed to get sorting right for tussenvoegsels.
    %sortcites    = true,       
    ]{biblatex}                % Bibliography
\usepackage[
    style=american
    ]{csquotes}                % Required for BiBLaTeX
\usepackage{import}            % Importing subdocuments
\usepackage{standalone}        % Compilable subdocuments
% `setspace` needs to be loaded before `hyperref`.
\usepackage{setspace}          % Set line spacing
\usepackage{hyperref}          % Clickable references
\usepackage{microtype}         % Nice typography
\usepackage{makeidx}           % Make an index
\usepackage{silence}           % Silence warnings and errors
\WarningFilter{remreset}{The remreset package}
\WarningFilter*[parskip]{latex}{Command}

% Set line spacing.
\setstretch{1.25}

% Page layout:
\usepackage{tocloft}         % Control ToC
\usepackage[
    a4paper,
    bottom     = 1.1in,
    top        = 1.1in,
    left       = 1.1in,
    right      = 1.1in,
    headheight = 15pt,
    footskip   = 1.5\baselineskip
    ]{geometry}              % Margins
\usepackage{fancyhdr}        % Header
\usepackage{lastpage}        % Page numbers in footer
\usepackage{setspace}        % Line spacing
\usepackage{ragged2e}        % Better line endings
\ActivateWarningFilters[parskip]
\usepackage[
    parfill
    ]{parskip}               % Newlines instead of indentation
\DeactivateWarningFilters[parskip]
\usepackage{titlesec}        % Size of sections
\usepackage{titlecaps}       % Automatic capitalisation of titles
\usepackage[
    hang,
    bottom
    ]{footmisc}              % Configure footnotes
\usepackage{afterpage}       % Include stuff after the current page
\usepackage{placeins}        % Things like `\FloatBarrier`
\usepackage{pdflscape}       % Turn pages sideways
\usepackage{emptypage}       % Remove pagenumbers on empty pages

% Utility:
\usepackage[table]{xcolor}   % Colours
\usepackage[
    separate-uncertainty=true,
    per-mode=symbol
    ]{siunitx}               % Display units
\usepackage{enumitem}        % Better enumeration
\usepackage{xifthen}         % If statements
\usepackage{xpatch}          % Patch things
\usepackage{lipsum}          % Dummy text
\usepackage{listings}        % Listings
\usepackage{xparse}          % Better arguments for \newcommand
\usepackage{xfrac}           % Better fractions
\usepackage{etoolbox}        % Patch stuff
\usepackage[
    notquote
    ]{hanging}               % Hanging paragraphs
\usepackage{scalerel}        % Scale objects
\usepackage{soul}            % Highlighting

% Figures:
\usepackage{pgfplots}        % Plots
\pgfplotsset{compat=newest}
\usepackage{tikz}            % TikZ figures
\usepackage{float}           % Control floating of figures
\usepackage{multirow}        % Cells spanning multiple rows
\usepackage{graphicx}        % Graphics
\usepackage{caption}         % Subfigures and captions
\usepackage{subcaption}      % Subfigures and captions
\usepackage{booktabs}        % Nice looking tables
\usepackage{tabularx}        % Extended functionality for tables

% Math:
\usepackage{amsmath}         % Math
\usepackage{amssymb}         % Math symbols
\usepackage{mathtools}       % Math tools
\usepackage{amsthm}          % Theorems
\usepackage{thmtools}        % More math theorems
\usepackage{bm}              % Bold math symbols
\usepackage{bbm}             % More bold math symbols
\usepackage{cancel}          % Cancel equations
\usepackage{upgreek}         % Upright greek symbols
\usepackage[
    artemisia
    ]{textgreek}             % Greek symbols in text

% This package should be loaded last.
\usepackage[
    noabbrev,
    capitalize,
    nameinlink]{cleveref}    % Automatic referencing

% Add a comma in the cite style.
\renewcommand*{\nameyeardelim}{\addcomma\space}

% Make sure that every `\citeyear` also prints the disambiguating `extradate`.
\DeclareCiteCommand{\citeyearlabel}
    {\usebibmacro{prenote}}
    {\printfield{year}\printfield{extradate}}
    {\multicitedelim}
    {\usebibmacro{postnote}}
\let\citeyearold\citeyear
\let\citeyear\citeyearlabel
\let\citeyearnolabel\citeyearold

% Define a standard for citing within theorem statements.
\newcommand{\theoremcite}[1]{\citeauthor{#1}, \citeyear{#1}}

\newcommand{\fulltextcite}[1]{%
    \AtNextCite{\AtEachCitekey{\defcounter{maxnames}{999}}}%
    \textcite{#1}%
}
\newcommand{\fullparencite}[1]{%
    \AtNextCite{\AtEachCitekey{\defcounter{maxnames}{999}}}%
    \parencite{#1}%
}

% Remove the colon after "In" in the bibliograhpy.
\renewbibmacro*{in:}{\bibstring{in} }

% Tune the electronic print field.
\DeclareFieldFormat{eprint}{%
  \iffieldundef{eprinttype}
    {Electronic print}
    {\thefield{eprinttype}}%
  \addcolon\space
  \ifhyperref
    {\url{#1}}
    {\nolinkurl{#1}}%
  \iffieldundef{eprintclass}
    {}
    {\addspace\mkbibparens{\thefield{eprintclass}}}\printtext{.}}

% Tune URL.
\DeclareFieldFormat{url}{\mkbibacro{URL}\addcolon\space\url{#1}\printtext{.}}

% Tune DOI.
\DeclareFieldFormat{doi}{%
  \mkbibacro{DOI}\addcolon\space
  \ifhyperref
    {\href{https://doi.org/#1}{\nolinkurl{#1}}}
    {\nolinkurl{#1}}\printtext{.}}

% In the below, we will redefine `\finentry` to not print a period if
%` pageref` exists. The period is then added back by `pageref`, but before the
% parentheses. Problems occur when `urldate` is also defined. In that case, 
% the extra period by `pageref` will add a period after `urldate`s
% parentheses, which we do not want. We use the toggle `printperiod` to detect
% this case.

% Tune "visited on".
\DeclareFieldFormat{urldate}{\mkbibparens{\bibstring{urlseen}\space#1\printtext{.}}}
\newtoggle{printperiod}
\toggletrue{printperiod}
\AtBeginBibliography{\renewbibmacro*{urldate}{\printurldate\togglefalse{printperiod}}}

% Tune "cited on".
\renewbibmacro*{finentry}{\iflistundef{pageref}{}{\renewcommand{\finentrypunct}{}}\finentry}
\renewbibmacro*{pageref}{%
  \iflistundef{pageref}
    {}
    {\iftoggle{printperiod}{\setunit{\adddot\addspace}}{}\toggletrue{printperiod}\printtext[parens]{%
       \ifnumgreater{\value{pageref}}{1}
         {\bibstring{backrefpages}\ppspace}
         {\bibstring{backrefpage}\ppspace}%
          \printlist[pageref][-\value{listtotal}]{pageref}.}}}

% Add Oxford comma.
\DefineBibliographyExtras{british}{\def\finalandcomma{\addcomma}}

% Don't abbreviate the backreferences.
\DefineBibliographyStrings{english}{%
    backrefpage = {Cited on page},
    backrefpages = {Cited on pages}
}

% Get tussenvoegsels right. This is not a perfect solution, because the
% in-text citations will be sorted as if the tussenvoegsel is part of the
% last name.
\makeatletter
\AtBeginDocument{\toggletrue{blx@useprefix}}
\AtBeginBibliography{\togglefalse{blx@useprefix}}
\makeatother

% Make an index.
\makeindex

% Show the page number in the bottom center.
\fancypagestyle{plain}{
    \fancyhf{}                          % Clear all header and footers.
    \renewcommand{\headrulewidth}{0pt}  % Remove the header rule.
    \cfoot{\thepage}                    % Show page number in the center.
}
\pagestyle{plain}

% Chapter styling from https://texblog.org/2012/07/03/fancy-latex-chapter-styles/
\newcommand{\hsp}{\hspace{20pt}}
\titleformat{\chapter}[hang]
    {\Huge\bfseries}
    {\thechapter\hsp\textcolor{gray75}{|}\hsp}
    {0pt}
    {\Huge\bfseries}
\titleformat{\section}
    {\normalfont\Large\bfseries\raggedright}
    {\thesection}
    {1em}
    {}
\titleformat{\subsection}
    {\normalfont\large\bfseries\raggedright}
    {\thesubsection}
    {1em}
    {}

% Set spacing after chapter equal to spacing right before section. Then
% things look nicely aligned.
\titlespacing{\chapter}{0pt}{3.5ex}{3.5ex}

% Define the styling of paragraphs.
\renewcommand{\paragraph}[1]{\textbf{#1.}}

% Redefine abstract environment.
\renewenvironment{abstract}
    {\begingroup
    \setstretch{1.25}
    \begin{center}
        \textbf{Abstract}
    \end{center}}
    {\par
    \endgroup}

% Kill any use of the \cite command.
\renewcommand{\cite}[1]{\PackageError{thesis}{use either parencite or authorcite}{}}

% Configure captions.
\WarningFilter{caption}{Unused \captionsetup}
\captionsetup[table]{font=small}
\captionsetup[figure]{font=small}

% Shortcuts for writing.
\newcommand{\ie}{\textit{i.e.}}
\newcommand{\Ie}{\textit{I.e.}}
\newcommand{\eg}{\textit{e.g.}}
\newcommand{\Eg}{\textit{E.g.}}
\newcommand{\cf}{\textit{c.f.}}
\newcommand{\Cf}{\textit{C.f.}}

% \ifempty command:
\newcommand{\ifempty}[3]{\ifthenelse{\equal{#1}{}}{#2}{#3}}

% Style footnotes.
\setlength{\skip\footins}{\baselineskip}
\setlength{\footnotesep}{.75\baselineskip}
\renewcommand{\footnotelayout}{\setstretch{1.0}}
\setlength{\footnotemargin}{1em}

% Typewriter font:
\renewcommand*{\ttdefault}{pcr}  % Courier
\newcommand{\code}[1]{{\small\texttt{#1}}}

% Define some colours.
\definecolor{darkblue} {rgb} {0.0 , 0.0 , 0.65}
\definecolor{darkred}  {rgb} {0.80, 0.0 , 0.0 }
\definecolor{redaccent}{HTML}{E64C66}
\definecolor{darkgreen}{rgb} {0.0 , 0.50, 0.0 }
\definecolor{gray75}   {gray}{0.75}

% Define shortcuts for colours.
\newcommand{\red}[1]{{\color{red} #1}}
\newcommand{\blue}[1]{{\color{blue} #1}}
\newcommand{\green}[1]{{\color{green} #1}}
\newcommand{\orange}[1]{{\color{orange} #1}}
\newcommand{\darkred}[1]{{\color{darkred} #1}}
\newcommand{\darkblue}[1]{{\color{darkblue} #1}}
\newcommand{\darkgreen}[1]{{\color{darkgreen} #1}}
\newcommand{\magenta}[1]{{\color{magenta} #1}}
\newcommand{\grey}[1]{{\color{gray} #1}}

% Use colours to define checkmarks and crosses.
\usepackage{pifont}
\newcommand{\xmark}{\text{\ding{55}}}
\newcommand{\cmark}{\text{\ding{51}}}
\newcommand{\good}{\darkgreen{$\checkmark$}}
\newcommand{\mediumgood}{\orange{$\checkmark$}}
\newcommand{\mediumbad}{\orange{\xmark}}
\newcommand{\bad}{\darkred{\xmark}}

% Load TikZ libraries.
\usetikzlibrary{
    calc,
    positioning,
    fit,
    tikzmark,
    arrows.meta,
    shapes,
    decorations.pathreplacing,
    intersections,
    through
}
% Graphical models:
\tikzset{
    line/.style = {
        thick,
        ->,
        > = {
            Triangle[length=1.5mm, width=1.5mm]
        }
    },
    arrow/.style = {
        line
    },
    % Invisible node:
    hidden node/.style = {
        circle,
        minimum size = 1cm,
        draw = white,
        thick
    },
    % Latent variable:
    latent node/.style = {
        hidden node,
        draw = black,
    },
    % Latent variable:
    factor node/.style = {
        hidden node,
        rectangle,
        draw = black,
    },
    % Observed variable:
    observed node/.style = {
        latent node,
        fill = gray!15
    },
    % Plate:
    plate/.style = {
        draw,
        label={[anchor=north west]south west:#1},
        rounded corners=2pt,
        shape=rectangle,
        inner sep=10pt,
        thick
    }
}

% Circled number:
\newcommand{\ballnumber}[1]{%
    \tikz[baseline=(n.base)] \node[circle,fill=.,inner sep=1pt,text=white] (n) {\normalshape\bfseries\footnotesize #1};%
}
\newcommand{\itemballnumber}[1]{%
    \raisebox{.5pt}{\ballnumber{#1}}%
}

% Style enumerate and lists. Do not use `label=(\arabic*)` because equation
% numbering already uses the same style.
\setenumerate{topsep=.25\baselineskip, itemsep=0pt}
\setlist{topsep=.25\baselineskip, itemsep=0pt}
% \setlist[enumerate]{
%     label=(\arabic*),
%     itemsep=-0.25\baselineskip,
%     topsep=0.25\baselineskip,
%     after={\vspace*{0.25\baselineskip}}
% }
% \setlist[enumerate,2]{
%     topsep=-0.25\baselineskip,
%     itemsep=0pt,
%     label=(\alph*)
% }
% \setlist[enumerate,3]{
%     topsep=-0.25\baselineskip,
%     itemsep=0pt,
%     label=(\roman*)
% }
% \setlist[itemize]{
%     label=\textbullet,
%     itemsep=-0.25\baselineskip,
%     topsep=0.25\baselineskip
% }
% \setlist[itemize,2]{
%     topsep=-0.25\baselineskip,
%     itemsep=0pt
% }
% \setlist[itemize,3]{
%     topsep=-0.25\baselineskip,
%     itemsep=0pt
% }

% Set the default float placement correctly.
\floatplacement{figure}{tbp}
\floatplacement{table}{tbp}

% Set spacing between figures and text.
\setlength{\textfloatsep}{30pt plus 1.0pt minus 2.0pt}
\setlength{\floatsep}{30pt plus 1.0pt minus 2.0pt}
\setlength{\intextsep}{30pt plus 1.0pt minus 2.0pt}

% Adjust line spacing in captions.
\captionsetup{font={stretch=1.25}}

% Newline for in a title.
\newcommand{\tnl}{\texorpdfstring{\\}{}}

% Define outline tools.
\newcommand{\outline}[1]{
    \parbox{\linewidth}{
        \setstretch{1.0}
        \color{darkgreen}
        #1
    }\vspace{\parskip}
}
\newlist{outlinelist}{enumerate}{1}
\setlist[outlinelist]{
    label=\arabic*.,
    noitemsep,
    topsep=0pt,
    parsep=0pt,
    partopsep=0pt
}
\newcommand{\note}[1]{{\color{darkred}#1}}

% Left-justified text in tabularx environment:
\newcolumntype{L}{>{\RaggedRight\arraybackslash}X}

% Hyperlink setup.
\hypersetup{
    colorlinks,
    citecolor = black,
    filecolor = black,
    linkcolor = black,
    urlcolor  = black
}

% Landscape figures:
\newcommand{\landscapefloat}[1]{
    \afterpage{
        \begin{landscape}
            #1
        \end{landscape}
    }
}

% Hanging full citation:
\newcommand{\hangcite}[1]{\hangpara{\bibhang}{1}\AtNextCite{\defcounter{maxnames}{99}}\fullcite{#1}}

% Define \charfusion. From https://tex.stackexchange.com/a/52673.
\makeatletter
\def\moverlay{\mathpalette\mov@rlay}
\def\mov@rlay#1#2{\leavevmode\vtop{%
   \baselineskip\z@skip \lineskiplimit-\maxdimen
   \ialign{\hfil$\m@th#1##$\hfil\cr#2\crcr}}}
    \newcommand{\charfusion}[3][\mathord]{
        #1{\ifx#1\mathop\vphantom{#2}\fi
        \mathpalette\mov@rlay{#2\cr#3}}
        \ifx#1\mathop\expandafter\displaylimits\fi
    }
\makeatother

% Sha symbol.
%   Source: https://tex.stackexchange.com/questions/124738/i-just-want-to-write-sha-without-ruining-everything
\DeclareFontFamily{U}{wncy}{}
\DeclareFontShape{U}{wncy}{m}{n}{<->wncyr10}{}
\DeclareSymbolFont{mcy}{U}{wncy}{m}{n}
\DeclareMathSymbol{\Sha}{\mathord}{mcy}{"58}

% Define a bigger \cdot.
%   Source: https://tex.stackexchange.com/questions/235118/making-a-thicker-cdot-for-dot-product-that-is-thinner-than-bullet
\makeatletter
\newcommand*\bigcdot{\mathpalette\bigcdot@{.5}}
\newcommand*\bigcdot@[2]{\mathbin{\vcenter{\hbox{\scalebox{#2}{$\m@th#1\bullet$}}}}}
\makeatother

% Bold math symbols from text greeks:
\newcommand{\mathbfup}[1]{\mathord{\textnormal{\textbf{#1}}}}

% Bold characters:
\renewcommand{\vec}[1]{\boldsymbol{#1}}
\newcommand{\vecu}[1]{\hat{\vec{#1}}}
\newcommand{\mat}[1]{\vec{#1}}

% Predefined matrices and vectors:
\newcommand{\vnull}{\mathbf{0}}

\newcommand{\va}{\mathbf{a}}
\newcommand{\vb}{\mathbf{b}}
\newcommand{\vc}{\mathbf{c}}
\newcommand{\vd}{\mathbf{d}}
\newcommand{\ve}{\mathbf{e}}
\newcommand{\vf}{\mathbf{f}}
\newcommand{\vg}{\mathbf{g}}
\newcommand{\vh}{\mathbf{h}}
\newcommand{\vi}{\mathbf{i}}
\newcommand{\vj}{\mathbf{j}}
\newcommand{\vk}{\mathbf{k}}
\newcommand{\vl}{\mathbf{l}}
\newcommand{\vm}{\mathbf{m}}
\newcommand{\vn}{\mathbf{n}}
\newcommand{\vo}{\mathbf{o}}
\newcommand{\vp}{\mathbf{p}}
\newcommand{\vq}{\mathbf{q}}
\newcommand{\vr}{\mathbf{r}}
\newcommand{\vs}{\mathbf{s}}
\newcommand{\vt}{\mathbf{t}}
\newcommand{\vu}{\mathbf{u}}
\newcommand{\vv}{\mathbf{v}}
\newcommand{\vw}{\mathbf{w}}
\newcommand{\vx}{\mathbf{x}}
\newcommand{\vy}{\mathbf{y}}
\newcommand{\vz}{\mathbf{z}}

% Must be enclosed in another group.
\newcommand{\vmu}{{\mathbfup{\textmu}}}
\newcommand{\vtheta}{{\bm{\uptheta}}}
\newcommand{\vphi}{{\mathbfup{\textphi}}}
\newcommand{\vtau}{{\mathbfup{\texttau}}}
\newcommand{\vep}{{\mathbfup{\textepsilon}}}
\newcommand{\vth}{{\mathbfup{\texttheta}}}

\newcommand{\mA}{\mathbf{A}}
\newcommand{\mB}{\mathbf{B}}
\newcommand{\mC}{\mathbf{C}}
\newcommand{\mD}{\mathbf{D}}
\newcommand{\mE}{\mathbf{E}}
\newcommand{\mF}{\mathbf{F}}
\newcommand{\mG}{\mathbf{G}}
\newcommand{\mH}{\mathbf{H}}
\newcommand{\mI}{\mathbf{I}}
\newcommand{\mJ}{\mathbf{J}}
\newcommand{\mK}{\mathbf{K}}
\newcommand{\mL}{\mathbf{L}}
\newcommand{\mM}{\mathbf{M}}
\newcommand{\mN}{\mathbf{N}}
\newcommand{\mO}{\mathbf{O}}
\newcommand{\mP}{\mathbf{P}}
\newcommand{\mQ}{\mathbf{Q}}
\newcommand{\mR}{\mathbf{R}}
\newcommand{\mS}{\mathbf{S}}
\newcommand{\mT}{\mathbf{T}}
\newcommand{\mU}{\mathbf{U}}
\newcommand{\mV}{\mathbf{V}}
\newcommand{\mW}{\mathbf{W}}
\newcommand{\mX}{\mathbf{X}}
\newcommand{\mY}{\mathbf{Y}}
\newcommand{\mZ}{\mathbf{Z}}

\newcommand{\mSigma}{\mathbfup{\textSigma}}

% Letters commonly used in blackboard bold font:
\newcommand{\Ab}{\mathbb{A}}
\newcommand{\Bb}{\mathbb{B}}
\newcommand{\Db}{\mathbb{D}}
\newcommand{\C}{\mathbb{C}}
\newcommand{\E}{\mathbb{E}}
\newcommand{\Hb}{\mathbb{H}}
\newcommand{\Jb}{\mathbb{J}}
\newcommand{\K}{\mathbb{K}}
\newcommand{\Lb}{\mathbb{L}}
\newcommand{\Mb}{\mathbb{M}}
\newcommand{\N}{\mathbb{N}}
\renewcommand{\P}{\mathbb{P}}
\newcommand{\Q}{\mathbb{Q}}
\newcommand{\R}{\mathbb{R}}
\newcommand{\eR}{\overline{\mathbb{R}}}
\newcommand{\Sb}{\mathbb{S}}
\newcommand{\Tb}{\mathbb{T}}
\newcommand{\V}{\mathbb{V}}
\newcommand{\Z}{\mathbb{Z}}

% Letters commonly used in calligraphic font:
\newcommand{\A}{\mathcal{A}}
\newcommand{\B}{\mathcal{B}}
\newcommand{\Cc}{\mathcal{C}}
\newcommand{\D}{\mathcal{D}}
\newcommand{\Ec}{\mathcal{E}}
\newcommand{\F}{\mathcal{F}}
\newcommand{\G}{\mathcal{G}}
\let\Haccent\H
\renewcommand{\H}{\mathcal{H}}
\newcommand{\J}{\mathcal{J}}
\newcommand{\Jrat}{\mathcal{J}_{\text{rat}}}
\newcommand{\Kc}{\mathcal{K}}
\renewcommand{\L}{\mathcal{L}}
\newcommand{\Li}{\mathcal{L}{}^1}
\newcommand{\eLi}{\overline{\mathcal{L}}{}^1}
\newcommand{\m}{\text{m}}
\newcommand{\M}{\mathcal{M}}
\newcommand{\eM}{\overline{\mathcal{M}}}
\newcommand{\Nc}{\mathcal{N}}
\newcommand{\Oc}{\mathcal{O}}
\newcommand{\Pc}{\mathcal{P}}
\newcommand{\Qc}{\mathcal{Q}}
\newcommand{\Rc}{\mathcal{R}}
\renewcommand{\S}{\mathcal{S}}
\newcommand{\Tc}{\mathcal{T}}
\newcommand{\U}{\mathcal{U}}
\newcommand{\Vc}{\mathcal{V}}
\newcommand{\W}{\mathcal{W}}
\newcommand{\X}{\mathcal{X}}
\newcommand{\Y}{\mathcal{Y}}
\newcommand{\Zc}{\mathcal{Z}}

% Letters sometimes used with a wide tilde:
\newcommand{\tpi}{\widetilde{\pi}}
\newcommand{\ttau}{\widetilde{\tau}}
\newcommand{\tvtau}{\widetilde{\vtau}}
\newcommand{\tD}{\widetilde{\D}}
\newcommand{\tX}{\widetilde{\X}}
\newcommand{\tI}{\widetilde{I}}
\newcommand{\tQc}{\widetilde{\Qc}}

% Letters commonly used in sans serif font:
\newcommand{\T}{\text{\normalshape\textsf{T}}}
\newcommand{\Ht}{\text{\normalshape\textsf{H}}}
\renewcommand{\c}{\text{\normalshape\textsf{c}}}

% Symbols:
\newcommand{\es}{\varnothing}
\newcommand{\e}{\varepsilon}
\newcommand{\sub}{\subseteq}
\renewcommand{\d}{\partial}
\renewcommand{\th}{\theta}
\newcommand{\Th}{\Theta}

% Substitute l for \ell throughout.
%   Source: https://tex.stackexchange.com/questions/1975/substituting-character-l-with-ell-throughout-math-mode
\mathcode`l="8000
\begingroup
\makeatletter
\lccode`\~=`\l
\DeclareMathSymbol{\lsb@l}{\mathalpha}{letters}{`l}
\lowercase{\gdef~{\ifnum\the\mathgroup=\m@ne \ell \else \lsb@l \fi}}%
\endgroup

% Convergence symbols:
\newcommand{\oto}[1]{\overset{#1}{\to}}
\newcommand{\uto}[1]{\underset{#1}{\to}}
\newcommand{\outo}[2]{\overset{#1}{\underset{#2}{\to}}}
\newcommand{\Lto}[1]{\oto{\L^{#1}}}
\newcommand{\weakto}{\rightharpoonup}
\newcommand{\distto}{\oto{\text{d}}}
\newcommand{\weakstarto}{\overset{\ast}{\rightharpoonup}}
% Big-O notation:
\renewcommand{\O}{O}

% Equality symbols:
\newcommand{\disteq}{\overset{\text{\normalshape d}}{=}}

% Operators:
\DeclareMathOperator*{\argmax}{arg\,max}
\DeclareMathOperator*{\argmin}{arg\,min}
\let\max\relax\DeclareMathOperator*{\max}{max}
\let\min\relax\DeclareMathOperator*{\min}{min}
\let\sup\relax\DeclareMathOperator*{\sup}{sup}
\let\inf\relax\DeclareMathOperator*{\inf}{inf\vphantom{p}}
\let\limsup\relax\DeclareMathOperator*{\limsup}{lim\,sup}
\let\liminf\relax\DeclareMathOperator*{\liminf}{lim\,inf\vphantom{p}}
\DeclareMathOperator*{\esssup}{ess\,sup}
\DeclareMathOperator*{\essinf}{ess\,inf}
\newcommand{\AC}{\operatorname{AC}}
\newcommand{\atan}{\operatorname{atan}}
\newcommand{\Breg}{\operatorname{D}}
\newcommand{\BV}{\operatorname{BV}}
\newcommand{\card}{\#}
\newcommand{\cconv}{\circledast}
\newcommand{\chol}{\operatorname{chol}}
\newcommand{\ch}{\operatorname{ch}}
\newcommand{\cliques}{\operatorname{cliques}}
\newcommand{\cl}{\operatorname{cl}}
\newcommand{\col}{\operatorname{col}}
\newcommand{\comp}{\circ}
\newcommand{\contains}{\supseteq}
\newcommand{\conv}{\ast}
\newcommand{\cov}{\operatorname{cov}}
\newcommand{\var}{\operatorname{var}}
\renewcommand{\deg}{\operatorname{deg}}
\renewcommand{\det}{\operatorname{det}}
\newcommand{\diag}{\operatorname{diag}}
\renewcommand{\dim}{\operatorname{dim}}
\newcommand{\diam}{\operatorname{diam}}
\renewcommand{\div}{\operatorname{div}}
\newcommand{\dist}{\operatorname{dist}}
\newcommand{\dom}{\operatorname{dom}}
\newcommand{\dotcup}{\charfusion[\mathbin]{\cup}{\cdot}}
\newcommand{\dotunion}{\charfusion[\mathop]{\bigcup}{\cdot}}
\newcommand{\dsum}{\oplus}
\newcommand{\Dsum}{\bigoplus}
\newcommand{\erf}{\operatorname{erf}}
\newcommand{\epi}{\operatorname{epi}}
\newcommand{\had}{\odot}
\newcommand{\Hell}{\operatorname{D}_{\text{Hell}}}
\newcommand{\hypo}{\operatorname{hypo}}
\newcommand{\im}{\operatorname{im}}
\renewcommand{\Im}{\operatorname{Im}}
\newcommand{\ind}{\mathbbm{1}}
\newcommand{\intersection}{\bigcap}
\newcommand{\join}{\lor}
\newcommand{\KL}{\operatorname{KL}}
\newcommand{\KOT}{\operatorname{KOT}}
\newcommand{\kron}{\otimes}
\newcommand{\len}{\operatorname{len}}
\newcommand{\Lip}{\operatorname{Lip}}
\newcommand{\meet}{\land}
\newcommand{\MOT}{\operatorname{MOT}}
\newcommand{\nullspace}{\operatorname{null}}
\newcommand{\Per}{\operatorname{Per}}
\newcommand{\push}{_\#}
\newcommand{\poly}{\operatorname{poly}}
\newcommand{\rank}{\operatorname{rank}}
\renewcommand{\Re}{\operatorname{Re}}
\newcommand{\row}{\operatorname{row}}
\newcommand{\schur}{\operatorname{Schur}}
\newcommand{\sign}{\operatorname{sign}}
\newcommand{\sinth}{\operatorname{\sin\!\text{-}\Theta}}
\newcommand{\Sinth}{\operatorname{\text{Sin-}\Theta}}
\newcommand{\supp}{\operatorname{supp}}
\newcommand{\symdiff}{\bigtriangleup}
\newcommand{\Tan}{\operatorname{Tan}}
\newcommand{\tensor}{\otimes}
\newcommand{\TV}{\operatorname{TV}}
\newcommand{\tr}{\operatorname{tr}}
\newcommand{\union}{\bigcup}
\newcommand{\vecspan}{\operatorname{span}}
\newcommand{\vect}{\operatorname{vec}}
\newcommand{\vol}{\operatorname{vol}}

% Probability distributions:
\newcommand{\Ber}{\operatorname{Ber}}
\newcommand{\Beta}{\operatorname{Beta}}
\newcommand{\Bin}{\operatorname{Bin}}
\newcommand{\BM}{\operatorname{BM}}
\newcommand{\Cat}{\operatorname{Cat}}
\newcommand{\Cauchy}{\operatorname{Cauchy}}
\newcommand{\Dir}{\operatorname{Dir}}
\newcommand{\Exp}{\operatorname{Exp}}
\newcommand{\Gam}{\operatorname{Gamma}}
\newcommand{\Geom}{\operatorname{Geom}}
\newcommand{\GP}{\mathcal{GP}}
\newcommand{\InvWishart}{\mathcal{W}^{-1}}
\newcommand{\Laplace}{\operatorname{Laplace}}
\newcommand{\LogNormal}{\log\mathcal{N}}
\newcommand{\Mult}{\operatorname{Mult}}
\newcommand{\NegBin}{\operatorname{NegBin}}
\newcommand{\Normal}{\mathcal{N}}
\newcommand{\Poisson}{\operatorname{Poisson}}
\newcommand{\Rad}{\operatorname{Rad}}
\newcommand{\Unif}{\operatorname{Unif}}
\newcommand{\Wishart}{\mathcal{W}}
\newcommand{\WN}{\operatorname{WN}}

\newcommand{\simiid}{\overset{\text{i.i.d.}}{\sim}}

% Probability commands:
\newcommand{\Var}{\V}
\newcommand{\Lik}{\L}
\newcommand{\iidsim}{\overset{\text{\tiny{i.i.d.}}}{\sim}}

% Special math commands:
\newcommand{\cond}{\,|\,}                  % Conditioning
\newcommand{\middlecond}{\,\middle|\,}     % Conditioning between a \left and \right
\newcommand{\divsep}{\,\|\,}               % Separator in divergences
\newcommand{\middledivsep}{\,\middle\|\,}  % Separator in divergences ebtween a \left and a \right

\newcommand{\sd}[1]{\mathrm{d} #1}         % Straight 'd'
\newcommand{\isd}[1]{\, \mathrm{d} #1}     % Straight 'd' in integral (with spacing)
% Straight 'd' in integral with spacing and absolute value around it
\newcommand{\asd}[1]{\, |\mathrm{d} #1|}

\newcommand{\idf}{\text{\textsf{id}}}            % Identity function
\newcommand{\sce}{\text{\sc{e}}}                 % Scientific notation
\newcommand{\vardot}{\,\bigcdot\,}               % Variable dot
\let\ssorig\ss
\renewcommand{\ss}[1]{_{\text{\normalshape #1}}} % Text subscripts
\newcommand{\us}[1]{^{\text{\normalshape #1}}}   % Text superscripts

% Hard to type words:
\newcommand{\cadlag}{c\`adl\`ag}
\newcommand{\Cadlag}{C\`adl\`ag}
\newcommand{\ito}{It\^o}
\newcommand{\Ito}{\ito}
\newcommand{\levy}{L\`evy}
\newcommand{\Levy}{\levy}

% Subscripts
\newcommand{\loc}{_{\text{loc}}}
\newcommand{\lip}{_{\text{Lip}}}

% Half rectangle:
\newcommand{\halfrect}[2]{[\![#1,#2)\hspace{-1pt}\!)}
\newcommand{\Rint}{\text{(R)}\int}

% Paired delimiters:
\DeclarePairedDelimiter\parens{(}{)}             % Parentheses
\DeclarePairedDelimiter\sbrac{[}{]}              % Square brackets
\DeclarePairedDelimiter\cbrac{\{}{\}}            % Curly braces
\DeclarePairedDelimiter\set{\{}{\}}              % Set
\DeclarePairedDelimiter\lra{\langle}{\rangle}    % Angle brackets
\DeclarePairedDelimiter\dlra
    {\langle\!\langle}{\rangle\!\rangle}         % Double angle brackets
\DeclarePairedDelimiter\floor{\lfloor}{\rfloor}  % Floor
\DeclarePairedDelimiter\ceil{\lceil}{\rceil}     % Ceil
\DeclarePairedDelimiter\norm{\|}{\|}             % Norm
\DeclarePairedDelimiter\abs{|}{|}                % Absolute value

% Shortcuts for angle brackets:
\newcommand{\la}{\langle}
\newcommand{\ra}{\rangle}
\newcommand{\lla}{\left\langle}
\newcommand{\rra}{\right\rangle}

% Other commands:
\newcommand{\resp}[1]{[#1]}                  % Respective statement
\newcommand{\bs}{\textbackslash}             % Blackslash in text
% Use \cref in a title.
\newcommand{\creftitle}[2]{\texorpdfstring{\cref{#2}}{#1 \ref{#2}}}
\newcommand{\mcheckthis}{{}^{[\checkmark]}}  % Checkmark in math
\newcommand{\checkthis}{$\mcheckthis$}       % Checkmark in text

% Compatibility with old commands:
\newcommand{\id}[1]{\sd{#1}}
\newcommand{\sargmax}[1]{\argmin_{#1}}
\newcommand{\sargmin}[1]{\argmax_{#1}}
\newcommand{\lrset}[1]{\set*{#1}}
\newcommand{\middleCond}{\middlecond}
\newcommand{\rel}[2]{($(#1) + (#2)$)}
\newcommand{\blackLinks}{}
\newcommand{\Ac}{\A}

\newcommand{\notheorems}{}  % We'll configure our own numbering below.
% Patch \listoftheorems.
%   Source: https://tex.stackexchange.com/questions/249963/remove-repeated-theorem-in-the-list-of-theorems
\makeatletter
\patchcmd\thmt@mklistcmd
    {\thmt@thmname}
    {\check@optarg{\thmt@thmname}}
    {}{}
\patchcmd\thmt@mklistcmd
    {\thmt@thmname\ifx}
    {\check@optarg{\thmt@thmname}\ifx}
    {}{}
\protected\def\check@optarg#1{%
    \@ifnextchar\thmtformatoptarg\@secondoftwo{#1}%
}
\makeatother

% Define lists of things commands.
\let\oldlistoftheorems\listoftheorems
\newcommand{\listofmodels}{
    \renewcommand{\listtheoremname}{List of Models}
    \oldlistoftheorems[ignoreall, show={model}]
}
\newcommand{\listofstatements}{
    \renewcommand{\listtheoremname}{List of Mathematical Statements}
    \oldlistoftheorems[ignoreall, show={theorem,corollary,proposition,lemma,fact}]
}
\renewcommand{\listoftheorems}{
    \renewcommand{\listtheoremname}{List of Theorems}
    \oldlistoftheorems[ignoreall, show={theorem}]
}

% Define environments.
\newlength{\thmtopsep}\setlength{\thmtopsep}{\topsep}
\newlength{\thmbotsep}\setlength{\thmbotsep}{\topsep}
\newtheoremstyle{theoremstyle}
    {\thmtopsep}{\thmbotsep}
    {}           % Body font
    {}           % Indent amount
    {\bfseries}  % Theorem head font
    {.}          % Punctuation after theorem head
    {.5em}       % Space after theorem head
    {}           % Theorem head spec
\theoremstyle{theoremstyle}

\ifcsname notheorems\endcsname
\else
    \newtheorem{theorem}{Theorem}[section]
    \newtheorem{proposition}{Proposition}[section]
    \newtheorem{corollary}{Corollary}[section]
    \newtheorem{fact}{Fact}[section]
    \newtheorem{lemma}{Lemma}[section]

    \newtheorem{assumption}{Assumption}[section]
    \newtheorem{definition}{Definition}[section]
    \newtheorem{question}{Question}[section]
    \newtheorem{example}{Example}[section]
    \newtheorem{model}{Model}[section]
    \newtheorem{remark}{Remark}[section]
\fi

% Set referencing formats.
\crefname{assumption}{Assumption}{Assumptions}
\Crefname{assumption}{Assumption}{Assumptions}
\crefname{corollary}{Corollary}{Corollaries}
\Crefname{corollary}{Corollary}{Corollaries}
\crefname{definition}{Definition}{Definitions}
\Crefname{definition}{Definition}{Definitions}
\crefname{example}{Example}{Examples}
\Crefname{example}{Example}{Examples}
\crefname{fact}{Fact}{Facts}
\Crefname{fact}{Fact}{Facts}
\crefname{lemma}{Lemma}{Lemmas}
\Crefname{lemma}{Lemma}{Lemmas}
\crefname{model}{Model}{Models}
\Crefname{model}{Model}{Models}
\crefname{proposition}{Proposition}{Propositions}
\Crefname{proposition}{Proposition}{Propositions}
\crefname{question}{Question}{Questions}
\Crefname{question}{Question}{Questions}
\crefname{remark}{Remark}{Remarks}
\Crefname{remark}{Remark}{Remarks}
\crefname{theorem}{Theorem}{Theorems}
\Crefname{theorem}{Theorem}{Theorems}

% Referentiable list items in environments
\newlist{asslist}{enumerate}{1}
\setlist[asslist]{
    ref=\theassumption.(\arabic*),
    label=(\arabic*),
    % topsep=0pt,
}
\crefname{asslisti}{Assumption}{Assumptions}
\Crefname{asslisti}{Assumption}{Assumptions}
\newlist{corlist}{enumerate}{1}
\setlist[corlist]{
    ref=\thecorollary.(\arabic*),
    label=(\arabic*),
    % topsep=0pt,
}
\crefname{corlisti}{Corollary}{Corollaries}
\Crefname{corlisti}{Corollary}{Corollaries}
\newlist{deflist}{enumerate}{1}
\setlist[deflist]{
    ref=\thedefinition.(\arabic*),
    label=(\arabic*),
    % topsep=0pt,
}
\crefname{deflisti}{Definition}{Definitions}
\Crefname{deflisti}{Definition}{Definitions}
\newlist{exlist}{enumerate}{1}
\setlist[exlist]{
    ref=\theexample.(\arabic*),
    label=(\arabic*),
    % topsep=0pt,
}
\crefname{exlisti}{Example}{Examples}
\Crefname{exlisti}{Example}{Examples}
\newlist{factlist}{enumerate}{1}
\setlist[factlist]{
    ref=\thefact.(\arabic*),
    label=(\arabic*),
    % topsep=0pt,
}
\crefname{factlisti}{Fact}{Facts}
\Crefname{factlisti}{Fact}{Facts}
\newlist{lemlist}{enumerate}{1}
\setlist[lemlist]{
    ref=\thelemma.(\arabic*),
    label=(\arabic*),
    % topsep=0pt,
}
\crefname{lemlisti}{Lemma}{Lemmas}
\Crefname{lemlisti}{Lemma}{Lemmas}
\newlist{modlist}{enumerate}{1}
\setlist[modlist]{
    ref=\themodel.(\arabic*),
    label=(\arabic*),
    % topsep=0pt,
}
\crefname{modlisti}{Model}{Models}
\Crefname{modlisti}{Model}{Models}
\newlist{proplist}{enumerate}{1}
\setlist[proplist]{
    ref=\theproposition.(\arabic*),
    label=(\arabic*),
    % topsep=0pt,
}
\crefname{proplisti}{Proposition}{Propositions}
\Crefname{proplisti}{Proposition}{Propositions}
\newlist{qlist}{enumerate}{1}
\setlist[qlist]{
    ref=\theremark.(\arabic*),
    label=(\arabic*),
    % topsep=0pt,
}
\crefname{qlisti}{Question}{Questions}
\Crefname{qlisti}{Question}{Questions}
\newlist{remlist}{enumerate}{1}
\setlist[remlist]{
    ref=\theremark.(\arabic*),
    label=(\arabic*),
    % topsep=0pt,
}
\crefname{remlisti}{Remark}{Remarks}
\Crefname{remlisti}{Remark}{Remarks}
\newlist{thmlist}{enumerate}{1}
\setlist[thmlist]{
    ref=\thetheorem.(\arabic*),
    label=(\arabic*),
    % topsep=0pt,
}
\crefname{thmlisti}{Theorem}{Theorems}
\Crefname{thmlisti}{Theorem}{Theorems}

% Reference numbers in the list.
\newcommand{\listnum}[1]{(#1)}
\newcommand{\listimp}[2]{\listnum{#1} $\Rightarrow$ \listnum{#2}}
\newcommand{\listeq}[2]{\listnum{#1} $\Leftrightarrow$ \listnum{#2}}

% Backward compatibility:
\newcommand{\listimpl}[2]{\listnum{#1} $\Rightarrow$ \listnum{#2}:}


% We actually will use coloured links.
\definecolor{aquamarine}{HTML}{218274}
\hypersetup{
    colorlinks,
    citecolor = aquamarine,
    filecolor = aquamarine,
    linkcolor = aquamarine,
    urlcolor  = aquamarine,
}

% Use italic font in bodies for clarity.
\newtheoremstyle{theoremstyle}
    {\thmtopsep}{\thmbotsep}
    {\itshape}   % Body font
    {}           % Indent amount
    {\bfseries}  % Theorem head font
    {.}          % Punctuation after theorem head
    {.5em}       % Space after theorem head
    {}           % Theorem head spec
\theoremstyle{theoremstyle}

% Set numbering of theorems right.
\newtheorem{theorem}{Theorem}[chapter]
\newtheorem{proposition}[theorem]{Proposition}
\newtheorem{corollary}[theorem]{Corollary}
\newtheorem{fact}[theorem]{Fact}
\newtheorem{lemma}[theorem]{Lemma}

\newtheorem{assumption}[theorem]{Assumption}
\newtheorem{definition}[theorem]{Definition}
\newtheorem{procedure}[theorem]{Procedure}

\newtheorem{question}[theorem]{Question}
\newtheorem{example}[theorem]{Example}
\newtheorem{model}[theorem]{Model}
\newtheorem{remark}[theorem]{Remark}

% Hack `\listoftheorem` to remove the title.
\makeatletter
\renewcommand\listoftheorems[1][]{%
    \begingroup
    \setlisttheoremstyle{#1}%
    \let\listfigurename\listtheoremname
    \def\contentsline##1{%
        \csname thmt@contentsline@##1\endcsname{##1}%
    }%
    \@for\thmt@envname:=\thmt@allenvs\do{%
        \thmtlo@newentry
    }%
    \let\thref@starttoc\@starttoc
    \def\@starttoc##1{\thref@starttoc{loe}}%
    \@fileswfalse
    \AtEndDocument{%
        \if@filesw
        \@ifundefined{tf@loe}{%
            \expandafter\newwrite\csname tf@loe\endcsname
            \immediate\openout \csname tf@loe\endcsname \jobname.loe\relax
        }{}%
        \fi
    }%
    \@starttoc{lof}
    \endgroup
}
\makeatother

% Fix spacing around proofs.
\xpatchcmd{\proof}{\topsep6\p@\@plus6\p@\relax}{}{}{}
\BeforeBeginEnvironment{proof}{\vspace{-0.5em}}
\AfterEndEnvironment{proof}{\vspace{-0.5em}}

% Commands specific for thesis:
\newcommand{\AR}{\operatorname{AR}}
\newcommand{\enc}{\mathsf{enc}}
\newcommand{\dec}{\mathsf{dec}}
\newcommand{\mult}{\operatorname{mult}}

\newcommand{\missingcite}{\note{(CITATION MISSING)}}

% Listings:
\usepackage{tcolorbox}
\tcbuselibrary{minted,skins,breakable}

\definecolor{solarized-light-bg}{HTML}{fdf6e3}
\definecolor{solarized-light-fg}{HTML}{586e75}
\newtcblisting{pythoncode}[2]{
    listing engine = minted,
    listing only,
    minted style = solarized-light,
    minted language = python,
    minted options = {
        fontsize = #1,
        escapeinside = ||,
        mathescape = true,
        highlightlines = #2,
        highlightcolor = red,
    },
    colback = solarized-light-bg,
    colframe = solarized-light-bg,
    toprule = 0pt,
    left = 5pt,
    left = 5pt,
    leftrule = 0pt,
    rightrule = 0pt,
    bottomrule = 0pt,
    arc = 0pt,
    frame hidden,
    breakable,
}
% Disable italics.
\AtBeginEnvironment{pythoncode}{\let\itshape\relax}

% Override the textwriter font.
%\usepackage[scaled=0.8]{beramono}
\usepackage[scaled=0.95]{inconsolata}
\renewcommand{\code}[1]{\colorbox{solarized-light-bg}{\small\color{solarized-light-fg}\texttt{#1}}}

% We never want footnotes to break across pages.
\interfootnotelinepenalty=10000

% Allow restatable environments.
\newenvironment{manual}[3][]{%
    \def\savedarg{#2}%
    \expandafter\renewcommand\csname the#2\endcsname{#3}%
    \ifempty{#1}{\csname #2\endcsname}{\csname #2\endcsname[#1]}%
}{%
    \csname end\savedarg\endcsname%
    \addtocounter{\savedarg}{-1}%
}
\newcommand{\statement}[1]{
    \begingroup
        \subimport{}{#1}
        \ifempty{\statementoption}{
            \csname\statementtype\endcsname
        }{
            \expandafter\csname\statementtype\endcsname[\statementoption]%
        }
        \label{\statementlabel}
        \statementcontent
        \csname end\statementtype\endcsname
    \endgroup
}
\newcommand{\restatement}[1]{
    \begingroup
        \subimport{}{#1}
        \ifempty{\statementoption}{
            \manual{\statementtype}{\ref{\xrprefix{\statementlabel}}}
        }{
            \manual[\statementoption]{\statementtype}{\ref{\xrprefix{\statementlabel}}}
        }
        \newcommand{\insiderestatement}{}
        \renewcommand{\label}[1]{}
        \statementcontent
        \endmanual
    \endgroup
}

%\newcommand{\ballnumber}[1]{\tikz[baseline=(myanchor.base)] \node[circle,fill=.,inner sep=1pt] (myanchor) {\color{-.}\normalshape\bfseries\footnotesize #1};}

\newcommand{\highlight}{\textcolor{}{}}

\addbibresource{../../bibliography.bib}

\usepackage{xr}
\externaldocument[xr-]{../../main}
\newcommand{\xrprefix}[1]{xr-#1}

\begin{document}

\chapter{Representation Theorems}
\label{chap:repr_theorems}

\paragraph{Abstract}
In \cref{\xrprefix{sec:nps:anatomy}}, we saw that neural processes approach a meta-learning problem by parametrising a prediction map $\pi_\theta \colon \D \to \Qc$ using an \emph{encoder--decoder architecture}.
%This means that $\pi_\theta = \enc_\theta \comp \dec_\theta$ is the composition of an \emph{encoder} and \emph{decoder}.
In this chapter, we will see how encoder--decoder architectures can be motivated theoretically with \emph{representation theorems}.
In the context of neural processes, representation theorems say that prediction maps may be decomposed as encoder--decoder architectures without losing representational power.

\paragraph{Outline}
In \cref{sec:repr_theorems:introduction}, we introduce representation theorems;
and in \cref{sec:repr_theorems:functions}, we introduce functions on data sets.
Then, in \cref{sec:repr_theorems:deep_sets}, we review a basic but important representation theorem called
\emph{deep sets} \parencite{Zaheer:2017:Deep_Sets,Edwards:2017:Towards_a_Neural_Statistician,Wagstaff:2019:On_the_Limitations_of_Representing}.
In \cref{sec:repr_theorems:conv_deep_sets}, we present an extension of deep sets called \emph{convolutional deep sets}.
Convolutional deep sets incorporate \emph{translation equivariance}.
Finally, in \cref{sec:repr_theorems:conv_deep_sets_dte}, we generalise translation equivariance to \emph{diagonal translation equivariance} and present a generalisation of convolutional deep sets to this symmetry.

\paragraph{Attributions and relationship to prior work}
\Cref{thm:conv_deep_set} was first presented as Theorem 1 by \fulltextcite{Gordon:2020:Convolutional_Conditional_Neural_Processes}.
%although our presentation removes the limitation that $k$ be a non-negative function.
The proof for Theorem 1 was originally developed by the author and Jonathan Gordon and later improved by Andrew Y.\ K.\ Foong.
We furthermore acknowledge David Burt and Mark Rowland for reviewing the proof.
\Cref{thm:conv_deep_set_dte} was first presented as Theorem E.1 by \fulltextcite{Bruinsma:2021:The_Gaussian_Neural_Process}.
%although our presentation again removes the limitation that $k$ be a non-negative function.
The proof of Theorem E.1 was originally developed by the author and James Requeima and later checked by Jonathan Gordon and Andrew Y.\ K.\ Foong.
All work was supervised by Richard E.\ Turner.

\section{Introduction}
\label{sec:repr_theorems:introduction}

As we discussed in \cref{\xrprefix{sec:nps:anatomy}}, designing a neural process involves parametrising a prediction map $\pi_\theta \colon \D \to \Pc$.
In parametrising a prediction map, we argued that there are two main challenges.
First, data sets vary in number of data points, which means that $\pi_\theta$ must process inputs of varying dimensionality.
Second, data sets with different orderings of the data points are fundamentally the same, which means that $\pi_\theta(D)$ should not depend on an ordering of the elements in $D$.
\index{encoder}\index{decoder}\index{encoding}\index{encoder--decoder architecture}
We then argued that these challenges can be addressed by parametrising $\pi_\theta$ with an \emph{encoder--decoder architecture}.
Encoder--decoder architectures
decompose $\pi_\theta  = \dec_\theta \comp \enc_\theta$ into an \emph{encoder} and \emph{decoder}.
The encoder does not depend on the order of the elements and produces an \emph{encoding} that always has the same format. 
In \cref{sec:repr_theorems:deep_sets}, we will review deep sets, which is a \emph{representation theorem} proved by \textcite{Zaheer:2017:Deep_Sets} that motivates the encoder--decoder architecture on a theoretical level.

\index{representation theorem}
In the context of neural processes, the study of representation theorems more generally means the study of functions on data sets $\D \to Z$, where $Z$ is some codomain.
The supposition of this study is that we would like to efficiently represent such functions on a computer.
The main problem, as have seen, is that it is not at all clear how to do that, due to the structure of $\D$.
Representation theorems provide general decompositions of functions $\D \to Z$ without losing representational power.
Importantly, if the components of this decomposition can be implemented on a computer, then the theorems provide ways to generally implement functions $\D \to Z$ on a computer.
In particular, by applying representation theorems to prediction maps, we find ways to concretely parametrise and implement prediction maps on a computer.
The application of representation theorems to prediction maps will be what we focus on in the next chapter, \cref{\xrprefix{chap:convcnps}}.
In this chapter, we take a step back from the neural process setup and more generally consider the study of functions on data sets.
We will, however, remain to use the language of ``encoder'', ``encoding'' and ``decoder'' to smoothly transition back in the next chapter.

\index{universal approximation theorem}
In practice, the components of a representation theorem will be implemented with neural networks.
For the neural network to be able to approximate these components,
the components must be \emph{continuous} \parencite{Cybenko:1989:Approximation_by_Superpositions_of_a}.
For this reason, representation theorems always guarantee that their components are continuous.
Without the requirement of continuity, more powerful statements are possible, but these statements are of little practical utility \parencite[Section 3;][]{Wagstaff:2019:On_the_Limitations_of_Representing}.
Before proceeding to our study of representation theorems, we first more carefully define functions on data sets.

\section{Functions on Data Sets}
\label{sec:repr_theorems:functions}
For simplicity, like in the previous chapter, we will assume that the inputs and outputs are one-dimensional: $\X \sub \R$ and $\Y \sub \R$.
All results, however, may readily be extended to multidimensional inputs and outputs.

In our developments up to this point, a data set $D \in \D_N$ has been an element of $(\X \times \Y)^N$.
In particular, $D$ is associated to an ordering of the input--output pairs.
However, two data sets with different orderings of the data points are fundamentally the same.
To this end, we will consider data sets equal if their input--output pairs are equal up to a permutation, which we now formalise.
For a permutation $\sigma \in \Sb^N$ and vector $\vz \in \R^N$, denote
$
    \sigma \vz = (z_{\sigma(1)}, \ldots, z_{\sigma(N)})
$.
For a data set $D = (\vx, \vy) \in \D^N$, denote $\sigma D = (\sigma \vx, \sigma \vy)$.
Then define the equivalence relation $\sim$ by $D_1 \sim D_2$ whenever there exists a permutation $\sigma \in \Sb^N$ such that $\sigma D_1 = D_2$.
Denote the equivalence class of $D$ by $[D]$, and denote the collection of all equivalence classes of $\D_N$ by $[\D_N] = \D_N /\!\sim$.
An element $[D] \in [\D_N]$ represents a data set that does not depend on the ordering of the data points.
Like before, let $[\D] = \union_{N \ge 0} [\D_N]$ be the collection of all data sets of finite size, which includes the empty set $\es$.
In this chapter, we additionally let $[\D_{\le N}] = \union_{n=0}^N [\D_{n}]$ be the collection of all data sets of size at most $N$, again including the empty set $\es$.
More generally, for a subset $A \sub \D$, denote $[A] = \set{[D] : D \in A}$.
We then have the following definition.

\begin{definition}[Function on data sets]
    A \emph{function on data sets} $f$ is a function $f\colon[\D] \to Z$, for some codomain $Z$.
\end{definition}

An example of a function on data sets is the function which computes the average observed value:
\begin{equation}
    f\colon [\D] \to \R, \quad
    f([D]) = \sum_{n=1}^{\abs{\vx}} y_n.
\end{equation}
We will be concerned with \emph{continuity} of functions on data sets, which means that we must endow $[\D]$ with a topology.
We previously endowed $\D$ with the metric
\index{metric!data sets}
\begin{equation}
    d_\D(D_1, D_2) = \begin{cases}
        \norm{\vx_1 - \vx_2}_2 + \norm{\vy_1 - \vy_2}_2 & \text{if $\abs{\vx_1} = \abs{\vx_2}$}, \\
        \infty & \text{otherwise}.
    \end{cases}
\end{equation}
For a data set $D$, let $\abs{D}$ denote the number of data points.
Endow $[\D]$ with the following metric:
\begin{equation} \label{eq:metric_data_sets_infimum}
    d_{[\D]}([D_1], [D_2]) = \inf_{\sigma_2 \in \Sb^{\abs{D_2}}} d_\D(D_1, \sigma_2 D_2).
\end{equation}
We collect some basic results in the following proposition.
Call a set $A \sub \D$ \emph{permutation invariant} if $D \in A$ implies that $\sigma D \in A$ for all permutations $\sigma \in \Sb^{\abs{D}}$.

% \begin{proposition}~
%     \begin{proplist}
%         \item
%             The function $d_{[\D]}$ is a metric on $[\D]$.
%         \item
%             The topology on $[\D]$ induced by $d_{[\D]}$ coincides with the quotient topology.
%         \item
%             \label{prop:properties_metric_data_sets:open_implies_open}
%             If $A \sub \D$ is open, then $[A]$ is open.
%         \item
%             \label{prop:properties_metric_data_sets:closed_implies_closed}
%             If $A \sub \D$ is closed, then $[A]$ is closed.
%     \end{proplist}
% \end{proposition}
\statement{statements/properties.tex}
\begin{proof}
    See \cref{\xrprefix{sec:proofs_repr_theorems:functions}}.
\end{proof}

For \cref{prop:properties_metric_data_sets:open_implies_open,prop:properties_metric_data_sets:closed_implies_closed},
since the topology on $[\D]$ induced by $d_{[\D]}$ coincides with the quotient topology, $[A]$ is also respectively open and closed in the quotient topology.

Call a function $f\colon\D  \to Z$ permutation invariant if $f(\sigma D) = f(D)$ for all $\sigma \in \Sb^{\abs{D}}$.
We then have the following result.

% \begin{proposition}
%     \label{prop:functions_on_data_sets_connection}
%     Suppose that $f \colon \D \to Z$ is continuous and permutation invariant.
%     Then $[f] \colon [\D] \to Z$ defined by $[f]([D]) = f(D')$ for any $D'$ such that $D' = [D]$ is well defined and continuous.
%     Conversely, suppose that $f \colon [\D] \to Z$ is continuous.
%     Then $f\colon \D \to Z$ defined by $f(D) = [f]([D])$ is continuous and permutation invariant.
% \end{proposition}
\index{permutation invariance}
\statement{statements/equivalence.tex}
\begin{proof}
    See \cref{\xrprefix{sec:proofs_repr_theorems:functions}}.
\end{proof}

\Cref{prop:functions_on_data_sets_connection} says that it does not matter whether we study continuous functions on $\D$ which are permutation invariant or continuous functions on $[\D]$;
they are essentially the same.
Therefore, in this chapter, we will confine ourselves to continuous functions on $[\D]$ and will not be concerned with the notion of permutation invariance.

\Textcite{Qi:2017:PointNet_Deep_Learning_on_Point} investigate functions on sets rather than on data sets.
In this more general setting, \citeauthor{Qi:2017:PointNet_Deep_Learning_on_Point} use the Hausdorff distance to measure distance between sets.
In our case, the collection of all data sets $[\D]$ has more structure, because it is a quotient by a group of isometries.
This structure allows us to employ the infimum-based metric in \eqref{eq:metric_data_sets_infimum};
in general, such a definition might violate the triangle inequality.
In this light, \cref{prop:properties_metric_data_sets:same_topology} is a specialisation of Theorem 2.1.(ii) by \textcite{Cagliari:2015:The_Natural_Pseudo-Distance_as_A} to $[\D]$.

\section{Deep Sets}
\label{sec:repr_theorems:deep_sets}

The starting point of our study of functions on data sets is the following theorem by \textcite{Zaheer:2017:Deep_Sets}, which characterises functions on $[\D_N]$.

\index{deep set}
\begin{theorem}[Deep set, preliminary; \theoremcite{Zaheer:2017:Deep_Sets}] \label{thm:deep_set_preliminary}
    Assume that $\X \sub \R$ and $\Y \sub \R$ are compact.
    A map $\pi \colon [\D_N] \to Z$ is continuous if and only if it is of the form
    \begin{equation} \label{eq:deep_set_preliminary}
        \pi = \dec \comp \enc \qquad \text{where} \qquad
        \enc(D) = \sum_{\smash{(x, y) \in D}} \phi(x, y)
    \end{equation}
    with $\enc \colon [\D_N] \to \smash{\R^{2(N+1)}}$,
    $\phi\colon \X \times \Y \to \smash{\R^{2(N+1)}}$, and
    $\dec \colon \smash{\R^{2(N+1)}} \to Z$ continuous.
    Moreover, the choice of $\enc$ and $\phi$ do not depend on $\pi$.
\end{theorem}
\begin{proof}
    See the proof of Theorem 2 by \textcite{Zaheer:2017:Deep_Sets}.
\end{proof}

Thoughout this thesis, we will call functions of the form of \eqref{eq:deep_set_preliminary} \emph{deep sets}.
In terms of encoders, note that deep sets encode on two levels.
For every data point $(x, y) \in D$, first $\phi(x, y)$ produces an encoding of that data point.
These data-point-specific encodings are then summed over all data points in the data set, finally producing an encoding of the whole data set $D$.
Although \cref{thm:deep_set_preliminary} was originally presented by \textcite{Zaheer:2017:Deep_Sets}, in concurrent work \textcite{Edwards:2017:Towards_a_Neural_Statistician} employed a similar approach for a similar problem.
\Cref{thm:deep_set_preliminary} is also comparable to Theorem 1 by \textcite{Qi:2017:PointNet_Deep_Learning_on_Point}.
Theorem 1 by \textcite{Qi:2017:PointNet_Deep_Learning_on_Point} uses a maximum instead of a sum for $\enc$ and characterises function continuous with respect to a different topology.

To prove \cref{thm:deep_set_preliminary}, \citeauthor{Zaheer:2017:Deep_Sets} proceed by defining
\begin{equation} \label{eq:power_series}
    \phi(x, y) =
    (
        x^0, x^1, \ldots, x^{N + 1},
        y^0, y^1, \ldots, y^{N + 1}
    ).
\end{equation}
Indeed, $\phi(x, y)$ is of dimensionality $2(N + 1)$.
It is clear that $\enc$ is continuous.
The key step is to show that $\enc\colon[\D_N] \to \R^{2(N+1)}$ is also \emph{injective}, which means that
it always maps two different $[D_1] \in [\D_N]$ and $[D_2] \in [\D_N]$ to distinct encodings.
\citeauthor{Zaheer:2017:Deep_Sets} show this in their Lemma 4.
Having shown that $\phi$ is a continuous injection, it follows that it is a continuous bijection from onto its image $\enc([\D_N])$.
The fact that we now use is that all continuous bijections $A \to B$ with $A$ compact and $B$ Hausdorff have a continuous inverse \parencite[Theorem 26.6;][]{Munkres:2000:Topology}. 
Since $[\D_N]$ is compact, we may conclude that $\phi\colon [\D_N] \to \phi([\D_N])$ has a continuous inverse $\phi^{-1}$.
\Cref{thm:deep_set_preliminary} then follows by setting $\dec = \pi \comp \phi^{-1}$.

\Cref{thm:deep_set_preliminary} has an important limitation: 
it only applies to data sets of a single fixed size $N$.
\citeauthor{Wagstaff:2019:On_the_Limitations_of_Representing} later extended \cref{thm:deep_set_preliminary}, addressing this restriction:

\index{deep set}
\begin{theorem}[Deep set; \theoremcite{Zaheer:2017:Deep_Sets}; \theoremcite{Wagstaff:2019:On_the_Limitations_of_Representing}] \label{thm:deep_set}
    Assume that $\X \sub \R$ and $\Y \sub \R$ are compact.
    A map $\pi \colon [\D_{\le N}] \to Z$ is continuous if and only if it is of the form
    \begin{equation} \label{eq:deep_set}
        \pi = \dec \comp \enc \qquad \text{where} \qquad \enc(D) = \sum_{\smash{(x, y) \in D}} \phi(x, y)
    \end{equation}
    with $\enc \colon [\D_N] \to \smash{\R^{2N}}$,
    $\phi\colon \X \times \Y \to \smash{\R^{2N}}$, and
    $\dec \colon \smash{\R^{2(N+1)}} \to Z$ continuous.
    Moreover, the choice of $\enc$ and $\phi$ do not depend on $\pi$.
\end{theorem}
\begin{proof}
    See the proof of Theorem 4.4 by \textcite{Wagstaff:2019:On_the_Limitations_of_Representing}.
\end{proof}

Compared to \cref{thm:deep_set_preliminary}, there are two differences.
First, in \cref{thm:deep_set}, $\phi$ maps into $\R^{2N}$ rather than $\R^{2(N+1)}$.
\citeauthor{Wagstaff:2019:On_the_Limitations_of_Representing} make this improvement by noting that the elements $x^0$ and $y^0$ in \eqref{eq:power_series} are always one and can therefore be removed.
Second, in \cref{thm:deep_set}, the theorem applies to any function on $[\D_{\le N}]$ rather than $[\D_{N}]$.
This means that it applies to all data set of size at most $N$, addressing the aforementioned limitation of \cref{thm:deep_set_preliminary}.
\citeauthor{Wagstaff:2019:On_the_Limitations_of_Representing} make this improvement by designating a special element in $\R \setminus \X$.
Let us denote this special element by $\mathsf{missing} \in \R \setminus \X$.
Using $\mathsf{missing}$, any data set $D \in \D_{N'}$ with $N' < N$ can be made a data set of $N$ data points by appending the appropriate number of $\mathsf{missing}$s.
In this way, $\D_{\le N}$ can be embedded into the collection of data sets with $N$ data points with inputs in
$\X \cup \set{\mathsf{missing}}$ and outputs in $\Y \cup \set{\mathsf{missing}}$.
Since $\X \cup \set{\mathsf{missing}}$ and $\Y \cup \set{\mathsf{missing}}$ are still compact, \cref{thm:deep_set_preliminary} can be applied.

\Cref{thm:deep_set} is an important result, because it provides a way to concretely parametrise a map $\pi\colon[\D_{\le N}] \to Z$ on a computer, supposing that $Z$ is some Euclidean space. 
Namely, in \eqref{eq:deep_set}, the components $\phi$ and $\dec$ are just continuous functions between Euclidean spaces, which can therefore be approximated with multi-layer perceptrons \parencite{Cybenko:1989:Approximation_by_Superpositions_of_a}.\index{universal representation theorem}

\section{Convolutional Deep Sets}
\label{sec:repr_theorems:conv_deep_sets}

Although \cref{thm:deep_set} provides a way to generally implement a map $\pi\colon[\D_{\le N}] \to Z$, this implementation requires two multi-layer perceptrons: one for $\phi$, and one for $\dec$.
Although multi-layer perceptrons (MLPs) can approximate any continuous function \parencite{Cybenko:1989:Approximation_by_Superpositions_of_a},
in some domains, MLPs are so parameter inefficient that they cannot feasibly be used.
One such domain is images.
Consider a $100\times100$ image, consisting of three colour channels. 
Then a one-layer MLP mapping to an image of the same size would consist of nearly a billion parameters, which is on the order of the biggest models that can nowadays be trained \parencite{Brown:2020:Language_Models_Are_Few-Shot_Learners}.
Specifically for the domain of images, convolutional neural networks (CNNs) were developed \parencite{Fukushima:1982:Neocognitron_A_Self-Organizing_Neural_Network}, which are MLPs with certain weights set to zero.
In comparison, a one-layer CNN mapping a $100\times100$ image with three colour channels to an image of the same size may only consist of only 81 parameters.
CNNs achieve this parameter efficiency by incorporating a symmetry called \emph{translation equivariance}.\index{translation equivariance}
Recall that, in \cref{\xrprefix{def:translation},\xrprefix{def:translation_equivariance}} in \cref{\xrprefix{sec:nps:translation_equivariance}}, we defined translations and translation equivariance.
We also explained the intuition behind translation equivariance.
In this section, we present a translation-equivariant version of deep sets (\cref{thm:deep_set}) called \emph{convolutional deep sets} (\cref{thm:conv_deep_set}).
As the name suggests, convolutional deep sets are to deep sets what CNNs are to MLPs.
In particular, convolutional deep sets inherit the parameter efficiency of CNNs.

In this section, for simplicity, we still assume one-dimensional inputs and outputs,
and we emphasise again that all results readily generalise to higher-dimensional inputs and outputs.
For the notion of translation equivariance to be well defined, we require that $\X = \R$ entirely.
If $\X \subsetneq \R$, then $\T_\tau x$ could shift an input outside $\X$.
As an example, if $\X = [0, 1]$, then $\T_{\frac12} 1 = \tfrac32 \notin \X$.

Every CNN is a translation-equivariant MLP.
It turns out that every MLP which is translation equivariant is also a CNN \parencite{Kondor:2018:On_the_Generalization_of_Equivariance}.
This means that, amongst all MLPs, translation equivariance \emph{characterises} CNNs.
In a similar spirit, the main result of this section is a characterisation of functions on data sets which are translation equivariant.
Let us for a moment contemplate such a result.
Suppose that 
\begin{equation}
    \pi = \enc \comp \dec \qquad \text{where} \qquad \enc(D) = \sum_{(x, y) \in D} \phi(x, y)
\end{equation}
were translation equivariant: $\pi \comp \T_\tau = \T_\tau \comp \pi$ for all $\tau \in \X$.
To establish translation equivariance of $\pi$, one reasonable construction would be to ensure that $\enc$ and $\dec$ are translation equivariant.
If that were true, then, by associativity of function composition,%
\begin{equation}
    \dec \comp \enc \comp \T_\tau
    \overset{\text{($\enc$ is TE)}}{=} \dec \comp \T_\tau \comp \enc
    \overset{\text{($\dec$ is TE)}}{=} \T_\tau \comp \dec \comp \enc
\end{equation}
which would mean that $\pi$ is indeed translation equivariant.
But there is a big problem with this construction:
$\enc$ is a function mapping into $\R^{2N}$ (\cref{thm:deep_set}),
and, crucially, vectors in $\R^{2N}$ cannot naturally be translated by translations in $\X = \R$!
One way to translate a vector in $\R^{2N}$ is to shift along its elements, wrapping around the outermost elements:%
\begin{equation}
    \T_1 (z_1, z_2, \ldots, z_{2N-1}, z_{2N})
    = (z_{2N}, z_1, \ldots, z_{2N-2}, z_{2N - 1}).
\end{equation}
This, however, only works for integer-valued translations.
Another proposal might be to simply add the translation to all elements of the vector:
\begin{equation}
    \T_{\tau} (z_1, \ldots, z_{2N})
    = (z_1 + \tau, \ldots, z_{2N} + \tau).
\end{equation}
But then what would you do for two-dimensional inputs?
In general, there is no natural way to translate vectors in $\R^{2N}$ by $\X$-valued translations.
We therefore see that the Euclidean space $\R^{2N}$ in which the encoding lives is an obstacle for deriving a translation-equivariant version of deep sets.
Instead, we would like the encoding to live in a space on which translations can naturally act.
And the most natural such space is a space of \emph{functions}, which motivates the following idea.

\index{encoding!functional}
\begin{definition}[Functional encoding]
    \label{def:functional_encoding}
    Call an encoding \emph{functional} if the output of $\enc$ and hence the input of $\dec$ is a function on $\X$:
    \begin{equation}
        \enc\colon [\D] \to Z^\X, \qquad
        \dec\colon Z^\X \to C
    \end{equation}
    where $Z$ is the codomain of the functional encoding and $C$ the codomain of the decoder. 
\end{definition}

\looseness=-1
If the encoding were functional, then translation equivariance is a perfectly well-defined notion for the output of $\enc$ and the input of $\dec$.
Therefore, in that case, we could pursue the construction of building translation equivariance into $\pi$ by building it into $\enc$ and $\dec$.

Before we state the main result, we need a few definitions.
\index{translation space}\index{translation space!topological}
Call a space $A$ a \emph{translation space} if elements in $A$ can be translated.
Call a translation space $A$ \emph{topological} if $(\tau, a) \mapsto T_\tau a$ is continuous.
Call a subset $A' \subset A$ of a translation space $A$ \emph{closed under translations} if $a \in A'$ implies that $\T_\tau a \in A'$ for all $\tau \in \X$.
Call a function $k\colon \X \times \X \to \R$ \emph{(strictly) positive definite} if, for all $N \in \N$ and $\vx \in \X^N$, the matrix
\begin{equation}
    \begin{bmatrix}
        k(x_1, x_1) & \cdots & k(x_1, x_n) \\
        \vdots & \ddots & \vdots \\
        k(x_n, x_1) & \cdots & k(x_n, x_n)
    \end{bmatrix}
\end{equation}
is (strictly) positive definite.
Call a function $k \colon \X \to \R$ (strictly) positive definite if $(x, y) \mapsto k(x - y)$ is (strictly) positive definite. 
For a Hilbert space $(\Hb, \lra{\vardot,\vardot}_\Hb)$ of functions on $\X$, call a positive-definite function $k$ a \emph{reproducing kernel} if $k(\vardot, x) \in \Hb$ for all $x \in \X$ and $\lra{f, k(\vardot, x)}_\Hb = f(x)$ for all $f \in \Hb$ and $x \in \X$.
For every positive-definite function $k$, there exists one and only one Hilbert space of functions on $\X$ for which $k$ is a reproducing kernel \parencite[Moore--Aronszajn theorem; Section 2;][]{Aronszajn:1950:Theory_of_Reproducing_Kernels}.
We call this Hilbert space the \emph{reproducing kernel Hilbert space} (RKHS) of $k$.
Finally, we define the \emph{multiplicity} of a data set $[D] \in [\D]$, which counts how many times an input is repeated exactly.

\index{multiplicity}
\begin{definition}[Multiplicity of data set]
    For a data set $[D] \in [\D]$, let the \emph{multiplicity} of $[D]$ be the maximum number of times an input is repeated exactly.
    Denote the multiplicity of $[D]$ by $\mult\,[D]$.
    Then
    \begin{equation}
        \mult\,[D]
        = \sup\, \set{\abs{\set{i \in [N]: x_i = x}} :  x \in \set{x_1, \ldots, x_{N}}}
        \text{ where } D = (\vx, \vy).
    \end{equation}
    For a subset $[\D'] \sub [\D]$, define $\mult\,[\D'] = \sup_{[D] \in [\D']} \mult\,[D]$.
\end{definition}

% \begin{theorem}[Convolutional deep set]
%     \label{thm:conv_deep_set}
%     Let $\Y \sub \R$ be compact.
%     Suppose that $[\D'] \sub [\D]$ is closed, is closed under translations, has multiplicity $K$, and has maximum data set size $N < \infty$.
%     Let $k \colon \X \to \R$ be a continuous strictly-positive-definite function such that $k(0) = \sigma^2 > 0$ and $k(\tau) \to 0$ as $\abs{\tau} \to \infty$.
%     Denote the reproducing kernel Hilbert space of $k$ by $\Hb$.
%     Let $Z$ be a translation space.
%     Then a function $\pi \colon [\D'] \to Z$ is continuous and translation equivariant if and only if it is of the form
%     \begin{equation} \label{eq:conv_deep_set}
%         \pi = \dec \comp \enc
%         \quad\text{where}\quad
%         \enc(D) =
%             \sum_{\smash{(x, y) \in D}} \phi(y) k(\vardot - x)
%     \end{equation}
%     with $\enc \colon [\D'] \to \Hb'$ continuous and translation equivariant,
%     $\dec\colon \Hb' \to Z$ continuous and translation equivariant,
%     and $\phi(y) = (0, y^1, \ldots, y^{K})$.
%     Here $\Hb' = \enc([\D'])$ is a subspace of $\Hb^{K+1}$ which is closed and closed under translations.
% \end{theorem}
\index{convolutional deep set}
\statement{statements/conv_deep_set.tex}
\begin{proof}
    The proof proceeds like the proof for \cref{thm:deep_set}.
    We show that $\enc$ is injective, therefore bijective onto its image.
    In this case, however, the image of $\enc$ is not compact. 
    Instead, we use the structure of the RKHS $\Hb$ to prove continuity of the inverse of $\enc$.
    See \cref{\xrprefix{sec:proofs_repr_theorems:conv_deep_sets}}.
\end{proof}

\Cref{thm:conv_deep_set} is a characterisation of functions $\pi \colon [\D'] \to Z$ on data sets which are translation equivariant.
In the following paragraphs, we will carefully analyse various aspects of the theorem.

A restrictive assumption of \cref{thm:conv_deep_set} is that the domain $[\D']$ must be simultaneously closed and have multiplicity $K$.
For example, suppose that we choose $[\D']$ to be the data sets in $[\D_2]$ with distinct inputs.
Then $\mult\,[\D']=1$.
However, $[\D']$ is not closed: $[((1/n, 0), (0, 0))] \in [\D']$ for all $n \ge 1$, but
\begin{equation}
    \lim_{n \to \infty}[((1/n, 0), (0, 0))] = [((0, 0), (0, 0))],
\end{equation}
and $[((0, 0), (0, 0))]$ has multiplicity two, so it is not in $[\D']$.
%Here we slightly abuse notation by denoting sets with duplicate elements;
%it would be more correct to use \emph{multisets}, a generalisation of sets which allow duplicate elements.
A reasonable way to construct subsets of $[\D']$ which are simultaneously closed and have multiplicity $K$ is to fix some $\e > 0$, and to consider the data sets in $[\D']$ with at most $K$ inputs exactly repeated and all other inputs $\e$ or more apart.

Like \cref{thm:deep_set}, \cref{thm:conv_deep_set} also lets $\phi$ be a power expansion (see \eqref{eq:power_series}).
However, the highest power of $\phi$ in \cref{thm:deep_set} is dictated by the maximum data set size, whereas the highest power of $\phi$ in \cref{thm:conv_deep_set} is set by the multiplicity of $[\D']$.
In particular, if the multiplicity is low, then the highest power of $\phi$ in \cref{thm:conv_deep_set} is low, regardless of the maximum data sets size.

\looseness=-1
As we alluded to before, \cref{thm:conv_deep_set} constructs a translation-equivariant encoder and decoder by employing a functional encoding (\cref{def:functional_encoding}).
The particular function space that the encoding lives is the $(K+1)$-fold product of the reproducing kernel Hilbert space $\Hb$ of a continuous positive-definite kernel $k$.
We are free to choose $k$.
In \cref{\xrprefix{chap:convcnps}}, we will choose $k$ to be a simple Gaussian kernel: \index{kernel}
$k(\tau) = \exp(-\frac{1}{2\ell^2}\tau^2)$ for some length scale $\ell > 0$.

The decoder is defined only on a subspace $\Hb' \sub \Hb^{K+1}$ which is closed and closed under translations.
It would be desirable to continuously extend the decoder to all of $\Hb^{K+1}$ whilst preserving translation equivariance.
Continuously extending functions from a closed subspace to the entire space is classically done with the Tietze extension theorem \parencite[Theorem 35.1;][]{Munkres:2000:Topology}.
The Tietze extension theorem, however, assumes that $Z$ is a finite-dimensional Euclidean space and does not preserve translation equivariance.
In \citeyear{Gleason:1950:Spaces_With_a_Compact_Lie}, \citeauthor{Gleason:1950:Spaces_With_a_Compact_Lie} extended Tietze's theorem to preserve symmetries with respect to compact groups by averaging over the Haar measure.
The group of translations, unfortunately, is not compact, though it is locally compact.
In \citeyear{Dugundji:1951:An_Extension_of_Tietzes_Theorem},
\citeauthor{Dugundji:1951:An_Extension_of_Tietzes_Theorem} extended Tietze's theorem to any $Z$ which is a locally convex vector space.
Later \textcite{Jaworowski:1981:An_Equivariant_Extension_Theorem_and} and more recently \textcite{Feragen:2006:Characterization_of_Equivariant_ANEs} proposed extensions of Tietze's theorem which generalise in both directions.
We leave it to future work to determine whether an equivariant extension theorem can be applied to extend the decoder from $\Hb'$ to all of $\Hb^{K+1}$.

Finally, we compare the form of convolution deep sets (\cref{thm:conv_deep_set}) to the form deep sets (\cref{thm:deep_set}).
There are two important differences.
First, for deep sets, the encoder is implemented with an MLP, and the decoder is also implemented with an MLP.
For convolutional deep sets, the encoder only depends on a choice for $k$, like the Gaussian kernel.
It can therefore trivially be implemented and, in particular, it \emph{does not depend on any neural network}.
Additionally, the decoder is now a translation-equivariant map between translation spaces.
Crucially, such maps are exactly what CNNs approximate \parencite[Theorem 3.1;][]{Yarotsky:2022:Universal_Approximations_of_Invariant_Maps}!
Therefore, for convolutional deep sets, the decoder can be implemented with a CNN.
The second difference pertains to the choice of maximum data set size $N$.
For deep sets, the dimensionality of the encoding depends on $N$.
For convolutional deep sets, on the other hand, the encoding always lives in $\Hb^{K+1}$, regardless of the maximum data set size.
We could therefore imagine applying \cref{thm:conv_deep_set} to a sequence of $N$s going to infinity to obtain a version of convolution deep sets that holds for data sets of all sizes.
We leave such a construction for future work.

After introducing \cref{thm:conv_deep_set} \parencite{Gordon:2020:Convolutional_Conditional_Neural_Processes}, \textcite{Xu:2020:MetaFun_Meta-Learning_With_Iterative_Functional} also introduced functional representations in the context of meta-learning for the purpose of improving representational capacity.
Whereas \cref{thm:conv_deep_set} uses a functional encoding, \citeauthor{Xu:2020:MetaFun_Meta-Learning_With_Iterative_Functional} argue that the encoder in \cref{thm:conv_deep_set} is limited.
We, however, emphasise that the simplicity of the encoder is without loss of generality, because \cref{thm:conv_deep_set} applies to \emph{any} continuous function $[\D'] \to Z$ which is translation equivariant.
The simplicity of the encoder can therefore be seen as a feature rather than a downside.

\section{Diagonal Translation Equivariance}
\label{sec:repr_theorems:conv_deep_sets_dte}

\looseness=-1
Convolutional deep sets (\cref{thm:conv_deep_set}) characterise translation-equivariant functions on data sets.
In this section, we investigate a slightly more general class:
functions on data sets which are \emph{equivariant with respect to diagonal translations}.
Before stating the definition of \emph{diagonal translation equivariance} (DTE), we first explain where DTE comes from.

In \cref{\xrprefix{sec:predmap:np_approximations}}, 
we defined
\emph{mean maps} (\cref{\xrprefix{def:mean_map}}),
functions of the form $m \colon [\D] \to \Y^\X$,
and  \emph{kernel maps} (\cref{\xrprefix{def:kernel_map}}), functions of the form $k \colon [\D] \to \R^{\X \times \X}$.
In \cref{\xrprefix{sec:convcnps:gnp}}, we will encounter mean maps $m$ which are translation equivariant:
\index{translation equivariance!mean map}
\begin{equation} \label{eq:mean_map_TE}
    m \comp \T_\tau = \T_\tau \comp m \qquad \text{for all $\tau \in \X$}.
\end{equation}
\Cref{thm:conv_deep_set} can be used to characterise such mean maps $m$.
We will also encounter kernel maps $k$ which are translation equivariant.
Kernel maps, however, turn out to be translation equivariant in a slightly different sense:%
\index{translation equivariance!kernel map}
\begin{equation} \label{eq:kernel_map_TE}
    k \comp \T_{\tau} = \T_{(\tau, \tau)} \comp k \qquad \text{for all $\tau \in \X$}.
\end{equation}
Compared to \eqref{eq:mean_map_TE}, \eqref{eq:kernel_map_TE} has $\T_{(\tau, \tau)}$ on the right-hand side instead of $\T_\tau$, where $\T_{(\tau, \tau)}$ is a translation on $\X \times \X$.
Recall that
the codomain of $k$ consists of functions on $\X \times \X$, not on $\X$.
Because \eqref{eq:kernel_map_TE} has different symbols on both sides, we cannot directly apply \cref{thm:conv_deep_set}.
We will see that the symmetry in \eqref{eq:kernel_map_TE} is closely related to the notion of \emph{diagonal translation equivariance} (\cref{def:diagonal_translation_equivariance}), which we later define.
The main result of this section is a characterisation of functions which are \emph{diagonally translation equivariant} in the sense of \eqref{eq:kernel_map_TE}.

\index{convolutional deep set}
\begingroup
    \newcommand{\maybeprefix}[1]{#1}
    \statement{statements/conv_deep_set_dte.tex}
\endgroup

\index{kernel}
If $C$ is the space of functions $\X \times \X \to \R$ and $\X = \R$, then a simple and appropriate choice for $c$ is the Gaussian $c(\vx) = \exp(-\tfrac1{2\ell^2}\lra{\vx,\ve_\perp}^2)$ for some length scale $\ell > 0$.
Intuitively, this choice for $c$ measures the distance to the diagonal and decays based on that distance.

\begin{proof}


To prove \cref{thm:conv_deep_set_dte}, our approach will be to first rewrite \eqref{eq:conv_deep_set_dte_equivariance} to have $\T_{(\tau, \tau)}$ on both sides.
Once \eqref{eq:conv_deep_set_dte_equivariance} is in this form, we will apply \cref{thm:conv_deep_set} with $\X\times\X$ as the space of inputs.
Recall that \cref{thm:conv_deep_set} generalises to higher-dimensional inputs.

\newcommand{\dup}{\operatorname{dup}{}\!}%
\newcommand{\dedup}{\operatorname{dup}^{-1}\!{}}%
To rewrite \eqref{eq:kernel_map_TE}, we use a simple trick: \emph{duplicating the inputs}.
%Like $\D$ is the collection of all data sets with inputs in $\X$ and outputs in $\Y$, 
Let $\D\ss{dbl}$ be the collection of data sets with inputs in $\X\ss{dbl} \coloneqq \X \times \X$ and outputs in $\Y$.
The ``dbl'' in the subscript of $\D\ss{dbl}$ stands for ``\underline{d}ou\underline{bl}e'' and serves to remind the reader that there are double the number of inputs.
Let $\operatorname{dup}\colon \D \to \D\ss{dbl}$ be the function that maps a data set into $\D\ss{dbl}$ by duplicating the inputs.
More precisely, for $D = (\vx,\vy) \in \D_N$,%
\begin{equation} \label{eq:dup}
    \operatorname{dup}(D) = (
        ((x_1, x_1), y_1),
        \ldots
        ((x_N, x_N), y_N)
    ).
\end{equation}
In \eqref{eq:dup}, every input on the right-hand side has duplicated elements,
so we say that such data sets have duplicated inputs.
Let $\D\ss{dup} \subsetneq \D\ss{dbl}$ be the subset of $\D\ss{dbl}$ with duplicated inputs.
Note that $\D\ss{dbl}$ contains data sets with and without duplicated inputs.
Let $\operatorname{dup}^{-1}\colon \D\ss{dup} \to \D$ be right inverse of $\operatorname{dup}$:
$\operatorname{dup}^{-1}$ takes in a data set with duplicated inputs and recovers the original by \emph{deduplicating the inputs}.

With these definitions, we can relate $\T_\tau\colon \D \to \D$ to $\T_{(\tau, \tau)}\colon \D\ss{dbl} \to \D\ss{dbl}$:
\begin{equation} \label{eq:duplicate_T_tau}
    \T_\tau = \dedup \comp \T_{(\tau, \tau)} \comp \dup.
\end{equation}
Define $\pi\ss{dup} \coloneqq \pi \comp \dedup$,
and note that $\pi\ss{dup}\colon \D\ss{dup} \to Z$ operates on data sets with duplicated inputs.
Then, by substituting \eqref{eq:duplicate_T_tau} in \eqref{eq:conv_deep_set_dte_equivariance}, we find that\footnote{\setstretch{0.7}\setlength{\jot}{0pt}
To carefully argue this, note that
\begin{align}
    \T_{(\tau, \tau)} \comp \pi
    &= \pi \comp \T_{\tau}
        &&\text{(apply \eqref{eq:conv_deep_set_dte_equivariance})} \\
    &= \pi \comp (\dedup \comp \T_{(\tau, \tau)} \comp \dup)
        &&\text{(definitions of $\dup$ and $\dedup$)}\\
        &= (\pi \comp \dedup) \comp \T_{(\tau, \tau)} \comp \dup, \label{eq:dup_manipulation} \\[.5em]
    \T_{(\tau, \tau)} \comp \pi\ss{dup}
    &= \T_{(\tau, \tau)} \comp (\pi \comp \dedup)
        &&\text{(definition of $\pi\ss{dup}$)} \\
    &= (\T_{(\tau, \tau)} \comp \pi) \comp \dedup \\
    &= ((\pi \comp \dedup) \comp \T_{(\tau, \tau)} \comp \dup) \comp \dedup
        &&\text{(apply \eqref{eq:dup_manipulation})} \\
    &= (\pi\ss{dup} \comp \T_{(\tau, \tau)} \comp \dup) \comp \dedup
        &&\text{(definition of $\pi\ss{dup}$)} \\
    &= \pi\ss{dup} \comp \T_{(\tau, \tau)} \comp (\dup \comp \dedup) \\
    &= \pi\ss{dup} \comp \T_{(\tau, \tau)}.
        &&\text{($\dedup$ is right inverse of $\dup$)}
\end{align}
}
\begin{equation} \label{eq:pi_TE_lifted}
    \pi\ss{dup} \comp \T_{(\tau, \tau)} = \T_{(\tau, \tau)} \comp \pi\ss{dup}
    \qquad \text{for all $\tau \in \X$.}
\end{equation}
Crucially, \eqref{eq:pi_TE_lifted} has $\T_{(\tau,\tau)}$ on both sides, so we may try to characterise $\pi\ss{dup}$ by applying \cref{thm:conv_deep_set} with $\X \times \X$ as the space of inputs!
Having characterised $\pi\ss{dup}$, we also find a characterisation of $\pi = \pi\ss{dup} \comp \dup$, which is what we are ultimately after.
To apply \cref{thm:conv_deep_set} with $\X \times \X$ as the space of inputs, however, we require $\pi\ss{dup}$ to be $(\X \times \X)$-translation equivariant, that is, equivariant with respect to \emph{all} translations on $\X \times \X$:%
\begin{equation} \label{eq:translation_equivariance_lifted}
    \smash{
        \pi\ss{dup} \comp \T_{(\tau_1, \tau_2)} \overset{\text{?}}{=} \T_{(\tau_1, \tau_2)} \comp \pi\ss{dup}
    }
    \qquad \text{for all $\tau_1, \tau_2 \in \X$}.
\end{equation}
Unfortunately, \eqref{eq:pi_TE_lifted} says that $\pi\ss{dup}$ is only equivariant with respect to translations along the \emph{diagonal} of the space $\X \times \X$: translations of the form $\T_{(\tau_1, \tau_2)}$ where $\tau_1 = \tau_2$.
This motivates the general definition of \emph{diagonal translation equivariance}.

\index{translation equivariance!diagonal}
\begin{definition}[Diagonal translation equivariance; DTE]
    \label{def:diagonal_translation_equivariance}
    Consider a map $\pi \colon A \to B$ where $A$ and $B$ are $(\X \times \X)$-translation spaces.
    Then $\pi$ is \emph{equivariant with respect to diagonal translations} or \emph{diagonally translation equivariant} (DTE) if 
    \begin{equation}
        \pi \comp \T_{(\tau, \tau)} = \T_{(\tau, \tau)} \comp \pi
        \quad
        \text{for all $\tau \in \X$}.
    \end{equation}
\end{definition}

If a map $\pi$ is $(\X \times \X$)-translation equivariant, then it is also diagonally translation equivariant.
The class of DTE functions is therefore larger than the class of $(\X \times \X)$-TE functions.
Whereas the previous section characterised functions on data sets which are TE, this section characterises functions on data sets which are DTE.

Diagonal translation equivariance is intimately related to $(\X \times \X)$-translation equivariance.
In technical terms, the group of diagonal translations is a \emph{subgroup} of the group of $(\X \times \X)$-translations.
We will use this relationship to derive a characterisation of DTE functions from \cref{thm:conv_deep_set}.
The key idea is as follows.
Consider a DTE map $\pi \colon A \to B$ between $(\X \times \X)$-translation spaces $A$ and $B$.
If $\pi$ were also $(\X \times \X)$-TE, then we could apply \cref{thm:conv_deep_set}, and we would be done.
However, $\pi$ is not $(\X \times \X)$-TE; it is just DTE.
The idea is to \emph{make $\pi$ $(\X \times \X)$-TE} by \emph{defining} what happens for non-diagonal translations:
\begin{equation} \label{eq:redefinition_pi}
    \pi(\T_{(\tau_1, \tau_2)} a) \coloneqq \T_{(\tau_1, \tau_2)} \pi(a)
    \quad
    \text{for all $\tau_1, \tau_2 \in \X$, $\tau_1 \neq \tau_2$}.
\end{equation}
This \emph{almost} works.
The problem is that \eqref{eq:redefinition_pi} might clash with the original definition of $\pi$:
for some $a \in A$ and $\tau_1, \tau_2 \in \X$, $\T_{(\tau_1, \tau_2)} a$ might equal another $a' \in A$, and it might be that $\T_{(\tau_1, \tau_2)} \pi(a) \neq \pi(a')$.
Fortunately, this problem can be circumvented, as we will now explain.

Let $C$ be another $(\X \times \X)$-translation space, this time topological.
Let $\ve_\perp = (1, -1)$.
Consider $c \in C$ such that, for all sequences $(\vtau_i)_{i \ge 1} \sub \X\ss{dbl}$,
the sequence $(\lra{\vtau_i, \ve_\perp})_{i \ge 1}$ is convergent whenever  $(\T_{\vtau_i} c)_{i \ge 1}$ is convergent. 
In particular, then
$\T_{\vtau\ss{I}} c = \T_{\vtau\ss{II}} c$ for $\vtau\ss{I}, \vtau\ss{II} \in \X$ implies that
$\lra{\vtau\ss{I}, \ve_\perp} = \lra{\vtau\ss{II}, \ve_\perp}$.\footnote{
    Consider the sequence $(\vtau_i)_{i \ge 1} \sub \X$ defined by $\vtau_i = \vtau\ss{I}$ for $i$ odd and $\vtau_{i} = \vtau\ss{II}$ for $i$ even.
}
In words, by observing $c$, you can identify the \emph{anti-diagonal} component of a translation.
We call such $c$ \emph{anti-diagonal discriminating}.\index{anti-diagonal discriminating}
Moreover, assume that $c$ is \emph{diagonally translation invariant} (DTI), by which we mean that $\T_{(\tau,\tau)} c = c$ for all $\tau \in \X$.\index{translation invariance!diagonal}
Then define the map
\begin{equation}
    \overline{\pi}\colon A \times \set{c} \to B,
    \quad \overline{\pi}(a, c) = \pi(a).
\end{equation}
By DTE of $\pi$ and DTI of $c$, this map is still DTE:
for all $\tau \in \X$:
\begin{align}
    \overline{\pi}(\T_{(\tau, \tau)} a, \T_{(\tau, \tau)} c)
    &= \overline{\pi}(\T_{(\tau, \tau)} a, c) &&\text{($c$ is DTI)} \\
    &= \pi(\T_{(\tau, \tau)} a) &&\text{(definition of $\overline{\pi}$)} \\
    &= \T_{(\tau, \tau)} \pi(a) &&\text{($\pi$ is DTE)} \\
    &= T_{(\tau, \tau)}\overline{\pi}(a, c). &&\text{(definition of $\overline{\pi}$)}
\end{align}
However, crucially, $c \neq \T_{(\tau_1, \tau_2)} c$ for any $\tau_1, \tau_2 \in \X$ such that $\tau_1 \neq \tau_2$, because $c$ is anti-diagonal discriminating.
Therefore, $\overline{\pi}$ is undefined for any non-diagonal translation!
This means that \eqref{eq:redefinition_pi} now goes through, and we can finally apply \cref{thm:conv_deep_set}.

There is one final technical detail to verify, which is that the extension $\overline{\pi}$ needs to be continuous.
\Cref{\xrprefix{sec:proofs_repr_theorems:conv_deep_sets_dte}} verifies this detail.
\end{proof}

% Hitherto, to motivate the key idea, we considered an input space of the form $\X = \tX \times \tX$.
% We now make a shift in perspective: we will consider $\tX$ to really be the space that inputs of data sets live in.
% Conforming to this shift in perspective, we will relabel $\tX$ to $\X$, meaning that our discussing thus far pertained to diagonal translation equivariance as follows:
% \begin{equation}
%     \label{eq:diagonal_translation_equivariance_relabelled}
%     \pi \comp \T_{(\tau, \tau)} = \T_{(\tau, \tau)} \comp \pi
%     \quad
%     \text{for all $\tau \in \X$}.
% \end{equation}
% Let $[\tD] = \union_{N = 0}^\infty [(\tX \times \Y)^N]$ be the collection of all data sets of finite size with inputs in $\tX$.
% In comparison, data sets in $[\D]$ have inputs in $\X$.
% The main result of this section is a characterisation of functions $\pi\colon [\D] \to Z$ which are diagonally translation equivariant in the following sense:
% \begin{equation}
%     \label{eq:conv_deep_set_dte_sense}
%     \pi \comp \T_{\tau} = \T_{(\tau, \tau)} \comp \pi
%     \quad
%     \text{for all $\tau \in \X$}.
% \end{equation}
% By following the construction for translation-equivariant kernel maps $k$ from the beginning of this section, duplicating the inputs of data sets,
% \eqref{eq:conv_deep_set_dte_sense} becomes diagonal translation equivariance in the sense of \cref{def:diagonal_translation_equivariance}.
% We have thus proved the following theorem.

% \begin{theorem}[Convolutional deep set for DTE]
%     \label{thm:conv_deep_set_dte}
%     Let $\Y \sub \R$ be compact.
%     Suppose that $[\D'] \sub [\D] $ is
%     closed, is closed under translations, has multiplicity $K$, and has maximum data set size $N < \infty$.
%     Let $k \colon \X \times \X \to \R$ be a continuous strictly-positive-definite function such that $k(\vnull) = \sigma^2 > 0$ and $k(\vtau) \to 0$ as $\norm{\vtau}_2 \to \infty$.
%     Denote the reproducing kernel Hilbert space associated to $k$ by $\Hb$.
%     Let $Z$ be a $(\X\times\X)$-translation space.
%     Let $C$ be another $(\X \times \X)$-translation space, let $c \in C$ be diagonally translation invariant and anti-diagonal discriminating, and denote $C' = \set{\T_\vtau c : \vtau \in \X}$.
%     Then a function $\pi \colon [\D'] \to C$ is continuous and diagonally translation equivariant in the sense of \eqref{eq:conv_deep_set_dte_sense} if and only if it is of the form
%     \begin{equation}
%         \pi = \dec \comp \enc
%         \quad\text{where}\quad
%         \enc(D) =
%             \begin{bmatrix}
%                \sum_{(x, y) \in C} \phi(y) k(\vardot - (x, x)) \\
%                c
%            \end{bmatrix}
%     \end{equation}
%     with $\enc \colon [\D'] \to \Hb' \times C'$ continuous and translation equivariant,
%     $\dec\colon \Hb^{K + 1} \to Z$ continuous and translation equivariant,
%     and $\phi(y) = (0, y^1, \ldots, y^{K})$.
%     Here $\Hb' = \enc([\D'])$ is a subspace of $\Hb^{K+1}$ which is closed and closed under translations.
% \end{theorem}


% \section{Approximate Translation Equivariance}
% \label{sec:repr_theorems:nonstationary}

% \note{[Often stationarity not true, but is ``roughly'' stationary, so, to achieve parameter efficiency, want to have parametrisation similar to representations for stationary processes.]}
%
% For a closed subspace $V \sub \Hb$, let $P_V$ be the orthogonal projection onto $V$.
% Define
% \begin{equation}
%     \pi_V = \pi(P_V)
% \end{equation}
% Call $\pi\colon \Hb \to \Hb$ \emph{finitely approximable} if, for all $K > 0$,
% \begin{equation}
%     \lim_{n \to \infty} \,\inf_{V \sub \H\, \text{linear},\, \dim(V) \le n} \,\sup_{h \in \Hb,\, \norm{h}_\Hb \le K} \,\norm{\pi(h) - P_V\pi(P_Vh)}_{\Hb} = 0.
% \end{equation}
% \note{[Better name? Kinda like nuclear operators, but for nonlinear maps.]}
% For example, if $\pi$ is linear and compact, then true.
%
% Let $G$ be a separable, metrisable, compact group with Haar measure $\mu$ acting on $\X$.
% Fix some $\vx_0 \in \X$.
% Consider
% \begin{align}
%     \Hb &= \set*{f \in C(\X) : \int f^2(g \vx_0) \isd \mu(g) < 0} /\!\sim, \\
%     \lra{f, h}_\Hb &= \int f(g \vx_0) h(g \vx_0) \isd \mu(g),
% \end{align}
% where $\sim$ denotes the equivalence relation given by almost sure equality.
% Then it follows that $(\Hb, \lra{\vardot, \vardot}_\Hb)$ is a separable Hilbert space.
% \note{Brezis Theorem 4.13 says separable metric space means $L^2$ is separable.}
%
% \begin{theorem}
%     Consider a continuous map $\pi \colon \Hb \to \Hb$ which is finitely approximable.
%     Let $\e > 0$.
%     Then there exists an $n \in \N$, a $G$-equivariant map $T\colon \Hb^{n + 1} \to \Hb$, and orthonormal vectors $(e_i)_{i=1}^n \sub \Hb$ such that $\sup_{h \in \Hb}\,\norm{\pi(h) - T(h, e_1, \ldots, e_n)}_\Hb \le \e$.
% \end{theorem}

\section{Conclusion}
\label{sec:repr_theorems:conclusion}

\begin{table}[t]
    \centering
    \caption{
        Comparison of deep sets and convolutional deep sets
        \index{deep set!comparison}
        \index{convolutional deep set!comparison}
    }
    \label{tab:comparison_deep_sets}
    \small
    \begin{tabular}{lll}
        \toprule
        ~
            & Deep sets (Thms \ref{thm:deep_set_preliminary} and \ref{thm:deep_set})
            & Convolutional deep sets (Thms \ref{thm:conv_deep_set} and \ref{thm:conv_deep_set_dte}) \\ \midrule
        Model for
            & Functions $[\D] \to Z$
            & Functions $[\D] \to Z$ which are TE \\
        Encoder
            & Multi-layer perceptron
            & Given; no neural network \\
        Encoding
            & Vector in $\R^{2N}$
            & Function in RKHS (\cref{def:functional_encoding}) \\
        Decoder
            & Multi-layer perceptron
            & Convolutional neural network \\
        \bottomrule
    \end{tabular}
\end{table}

In this chapter, we studied theorems which characterise functions on data sets.
In the context of neural processes, we call such theorems \emph{representation theorems}.
The starting point was a theorem by \textcite{Zaheer:2017:Deep_Sets,Wagstaff:2019:On_the_Limitations_of_Representing} called deep sets (\cref{thm:deep_set_preliminary,thm:deep_set}).
Deep sets generally characterise functions on data sets.
In a deep set, the encoder and decoder can be implemented with MLPs.

The main contribution of this chapter is an extension of deep sets called \emph{convolutional deep sets} (\cref{thm:conv_deep_set,thm:conv_deep_set_dte}).
Convolutional deep sets generally characterise functions on data sets which are \emph{translation equivariant}.
In a convolutional deep set, the encoder is trivially implemented and only depends on a choice of a positive-definite function.
In particular, it does not depend on a neural network.
Additionally, the decoder is implemented with a convolutional neural network.
\Cref{tab:comparison_deep_sets} summarises the key differences between deep sets and convolutional deep sets.

Convolutional deep sets have two important limitations.
First, the approximation of the encoder relies on a convolutional neural network with convolutions of dimensionality equal to the dimensionality of $\X$ in the case of \cref{thm:conv_deep_set} and twice the dimensionality of $\X$ in the case of \cref{thm:conv_deep_set_dte}.
One, two, or three-dimensional convolutions have reasonable computational expense.
Convolutions of dimension four or higher, however, may be prohibitively expensive to run.
This is particularly a problem for \cref{thm:conv_deep_set_dte}, where the convolutions have dimensionality twice the dimensionality of $\X$.
\Cref{thm:conv_deep_set} can therefore only reasonably be applied to data sets with one, two, or three-dimensional inputs
and \cref{thm:conv_deep_set_dte} to data sets only with one-dimensional inputs.
Second, whereas \cref{thm:conv_deep_set,thm:conv_deep_set_dte} characterise functions which are translation equivariant, what we might really need is a characterisation of functions which are \emph{approximately translation equivariant}.
Ideally, such a result would ``interpolate'' between deep sets and convolutional deep sets, depending on how translation equivariant the function to represent would be.
\Textcite{Wilk:2018:Learning_Invariances_Using_the_Marginal} discuss approximate invariance in the setting of Gaussian processes.

Finally, whereas we introduced functional encodings (\cref{def:functional_encoding}) to derive a translation-equivariant version of deep sets, this construction can also be used to provide an alternative proof of deep sets.
Theorem 2.4 by \textcite{Gordon:2020:Advanced_in_Probabilistic_Meta-Learning} pursues this approach, which was developed by Jonathan Gordon in collaboration with Andrew Y.\ K.\ Foong.

\end{document}



